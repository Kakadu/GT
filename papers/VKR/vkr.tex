\documentclass
  [ a4paper        % (Predefined, but who knows...)
  , draft,         % Show bad things.
  , 12pt           % Font size.
  , pagesize,      % Writes the paper size at special areas in DVI or
                   % PDF file. Recommended for use.
  , parskip=half   % Paragraphs: noindent + gap.
  , numbers=enddot % Pointed numbers.
  , BCOR=5mm       % Binding size correction.
  , submission
  , copyright
  , creativecommons
  ]{eptcs}
\providecommand{\event}{ML 2018}  % Name of the event you are submitting to
% \usepackage{breakurl}             % Not needed if you use pdflatex only.
\usepackage{underscore}           % Only needed if you use pdflatex.

\usepackage{marginnote}


\usepackage[margin=1cm]{geometry}

\usepackage{booktabs}
\usepackage{amssymb}
\usepackage{amsmath}
\usepackage{mathrsfs}
\usepackage{mathtools}
\usepackage{multirow}
\usepackage{indentfirst}
\usepackage{verbatim}
\usepackage{amsmath, amssymb}
\usepackage{graphicx}
\usepackage{xcolor}
\usepackage{url}
\usepackage{stmaryrd}
\usepackage{xspace}
\usepackage{comment}
\usepackage{wrapfig}
\usepackage[caption=false]{subfig}
\usepackage{placeins}
\usepackage{tabularx}
\usepackage{ragged2e}
\usepackage{soul}
\usepackage{csquotes}
\usepackage{inconsolata}

\usepackage{polyglossia}   % Babel replacement for XeTeX
  \setdefaultlanguage[spelling=modern]{russian}
  \setotherlanguage{english}
\usepackage{fontspec}    % Provides an automatic and unified interface 
                         % for loading fonts.
\usepackage{xunicode}    % Generate Unicode chars from accented glyphs.
\usepackage{xltxtra}     % "Extras" for LaTeX users of XeTeX.
\usepackage{xecyr}       % Help with Russian.

%% Fonts
\defaultfontfeatures{Mapping=tex-text}
\setmainfont{CMU Serif}
\setsansfont{CMU Sans Serif}
\setmonofont{CMU Typewriter Text}

\usepackage[final]{listings}

\lstdefinelanguage{ocaml}{
keywords={@type, function, fun, let, in, match, with, when, class, type,
nonrec, object, method, of, rec, repeat, until, while, not, do, done, as, val, inherit, and,
new, module, sig, deriving, datatype, struct, if, then, else, open, private, virtual, include, success, failure,
lazy, assert, true, false, end},
sensitive=true,
commentstyle=\small\itshape\ttfamily,
keywordstyle=\ttfamily\bfseries, %\underbar,
identifierstyle=\ttfamily,
basewidth={0.5em,0.5em},
columns=fixed,
fontadjust=true,
literate={->}{{$\to$}}3 {===}{{$\equiv$}}1 {=/=}{{$\not\equiv$}}1 {|>}{{$\triangleright$}}3 {\\/}{{$\vee$}}2 {/\\}{{$\wedge$}}2 {>=}{{$\ge$}}1 {<=}{{$\le$}} 1,
morecomment=[s]{(*}{*)}
}

\lstset{
mathescape=true,
%basicstyle=\small,
identifierstyle=\ttfamily,
keywordstyle=\bfseries,
commentstyle=\scriptsize\rmfamily,
basewidth={0.5em,0.5em},
fontadjust=true,
language=ocaml
}

\newcommand{\cd}[1]{\texttt{#1}}
\newcommand{\inbr}[1]{\left<#1\right>}
%\pagestyle{plain}
\sloppy

\title{Generic Programming with Combinators and Objects йцукен\thanks{This work was partially supported by the grant 18-01-00380 from the Russian Foundation for Basic Research.}}

\author{Dmitry Kosarev
  \institute{St. Petersburg State University\\
    JetBrains Research \\
    St. Petersburg, Russia}
\email{Dmitrii.Kosarev@protonmail.ch}
\and
Dmitry Boulytchev
\institute{St. Petersburg State University\\
  JetBrains Research \\
  St. Petersburg, Russia}
\email{dboulytchev@math.spbu.ru}
}

\def\titlerunning{Generic Programming with Combinators and Objects}
\def\authorrunning{D.Kosarev, D.Boulytchev}

\begin{document}
\maketitle

\begin{abstract}
  We present a generic programming framework for \textsc{OCaml} which makes it possible to implement extensible
  transformations for a large scale of type definitions. Our framework makes use of object-oriented features
  of \textsc{OCaml}, utilising late binding to override the default behaviour of generated transformations. The
  support for polymorphic variant types complements the ability to describe composable data types with the
  ability to implement composable transformations.
  
  В данной работе представлен подход для языка программирования \textsc{OCaml}, который позволяет реализовывать расширяемые трансформации для различных видов определений типов. Этот подход
  использует объектно-ориентированные возможности \textsc{OCaml},
  а именно позднее связывание, чтобы изменять поведение по-умолчанию
  автоматически сгенерированных трансформаций. Поддержка полиморфных вариантных типов позволяет композиционально описывать  типы данных
  с возможностью реализовать композициональные преобразования.
\end{abstract}

% !TeX encoding = windows-1251
\section{��������}

��� ���������� ������������ ����������� �� �������������� ������ ����������������, �������� ������ ���������� ������������� ���������� �������������� ��� ��������� �������� ������. ��� ����������� �������������� ������ (��������, \scheme{} � ��� ��������) ������ ����� ����������� ���� �������-������������������, ���������� ��� ���� �������� ������. � ������, ����� ���������� ������� �������� ���������� � ����� (��������, \Scala{}), ����� �������� ��������� ������������������ �� ����� ����������, ��������� �� ������ ����� �� ����� ���������� (�����������) � �������� ����� ����� ������ �� ���� (���������). 
 ��� ���������� �������������� ������ ����������������, � ������ ���������� ������� ���������� � ����� ������ �� �������� �� ����� ���������� (��������, \ocaml{} � \haskell{}), ����������� ������ ��� ��������� 
\emph{���������� (generic) ����������������}~\cite{Yallop,PPXLib}: %��� ���� ����������� � ���, ��� 
�� ������ ����������  �� �������� ���� ������ ������������� ������������ ������ ��������������, ������������������ ��� ��������� ����� ������. 
���������� ���������������� �������� ����������� �������� ��� ���� ������, ������� ��������� ��� ������������ ������ ����������, ��� �������, ��������� ��� ����������������, � ����� ������������� ����� �� ��������� �������������, ����� ��������� ��������� � ��������� ������������� ������������.

��� ������������� ����������� ���������������� ����������������� ���� ����������� ���� ���������� ���������� �� ������ ���������� �������������� � ������� ���������������� � ������� ���������. 
%���� ������ ������ ���� ���������� � ����������� �������, ������, ������ ��� ����� ���� ������������. 
� ��������� ������� ����� ������� �� ����������, �
���� ������������ �� ������ ���������� ���������� ��������������, %(��������, ���� ��� ���������� ������� ������������ ��������� ������ ��������������� ������������ �� ������),
�� �� ����� ����������� ����� ���������. ��-������, �� ����� ����������� ����� ��� �������������� ��� ����� ����. � ���� ������ � ������������ ����� ���������� ���������, ��� ��� �������� ����� �������������� ������� ������������� ����������,
%������ (��������, �� ����������� �������������� ������ ����� ����������������), 
���� ���������� ��� ������������ ���������� ������������ �����������. ������ �������� ������� �������� �������� ���������� ������������ �������������� ������� ��� ������������� ������� ����������� ����������������. � ���� ������ ���������� ������ ������� ���� � ���������� ��������� ������������ �������� � ������������ �����������. 

��� �������� �� �������� ������� � ������������ ��������-��������������� ������ ����������������, ��� ������������ ����� ������� ��������������� \emph{������� �����������}. ���� ���� � �������������� ������ ���������������� ������������ ����� ���� �� ������� ���������� �������� ��������� ��������-���������������� ���������������� (\haskell{}), ���� ������, ��� ��������-��������������� ������ ������� ���������� � ������������ ������ � ��� �������, ����� �� ����������� �������~\cite{modules-vs-objects}. 
%�������, �������� ��������\footnote{research question 1} (\textit{��� �������� �� ������ ������}) �������� ����� ������� ������������� ��������-��������������� ���� � �������������� ���������������� �� \ocaml{} ���, ����� ������������ ��������� �������� ������� � ����������� ��� ������������� ������������ �� �������������� �����.
� ������ ������ ������������ ������������� �������������� � ������� �������� ����� �������, ��� ����������� �������������� �������� �������� ���, �������������� ��������� ����������� ��������������� ���� ������. 
%� �������� ��������� �� ����������������, ��� ����������� �������������� ���� ���������� �������, ����� ������������ �������������� �������� ��������\footnote{research question 2, ���� RQ � ��� ����� � evaluation} (��� ������������ ������� � 4� �����, RQ �� ����).

� ������ ����������� ������, ���������� �������������� ����� ������ � ������� ��������, ��� �������� ����������� ��������-���������������� ����������������. ��������� ��� ����������� ����������� �� ������������� ����������� ��������������, �� �������� �� ������� ������. ��� ���� ����������� �������� ����������� ����������� ����������������, ��� ��������� �������� ��������������, ������� �������� ��� ������������� ������������ �� \ocaml{} ������� � �����������. ��� ��������� ������� ���� �������������� ���������� \GT{}\footnote{\url{https://github.com/kakadu/GT/tree/ppx}} (\emph{Generic Transformers}) ��� ����� \ocaml{}, ��� ����������� ������ ��������� ����������� ��������������, ����������� ���, ��� ��������������� ����������� ������������ ����������� ����������������, � ����� ������ ��������� ������ ��������������.


% \section{Exposition}
\label{expo}

В этом разделе мы постепенно представим наш подход используя несколько примеров. 
Хотя изложение не предоставляет конкретных деталей и не может использоваться как точная спецификация,
мы здесь предоставляем основные составляющие решения и мотивацию, которая привела к ним.
Далее мы будет использовать следующее соглашение: будем обозначать $\inbr{\dots}$ представление некоторого понятия в конкретном синтаксисе языка \textsc{OCaml}. Например, ``$\inbr{f_t}$`` является обозначением конкретной функции индексированной типом  ``$f$'' for a type ``$t$''. 
В конкретном синтаксисе оно может быть выражено как ``\lstinline{f_t}'', но мы пока воздержимся от указания конкретной формы.

Начнем с простого примера. Рассмотрим такое объявление типа арифметических выражений:

\begin{lstlisting}
   type expr =
   | Const of int
   | Var   of string
   | Binop of string * expr * expr
\end{lstlisting}

Рекурсивная функция ``$\inbr{show_{expr}}$'' (наиболее естественный кандидат на реализацию)
преобразует выражение в строку: 

\begin{lstlisting}
   let rec $\inbr{show_{expr}}$ = function
   | Const  n        -> "Const " ^ string_of_int n
   | Var    x        -> "Var " ^ x
   | Binop (o, l, r) ->
      Printf.sprintf "Binop (%S, %s, %s)" o ($\inbr{show_{expr}}$ l) ($\inbr{show_{expr}}$ r)
\end{lstlisting}

Представление, возвращаемое ``$\inbr{show_{expr}}$'', сохраняет имена конструкторов. Оно может быть
полезно приотладке или сериализации. Однако, как правило, также требуется иное, ``красивое''(\emph{pretty-printed}) представление. 
В этом представлении выражение представляется в ``естественном синтаксе'' с использованием инфиксных операций и без имён 
конструкторов, где скобки расставлены только там, где они действительно нужны. Мы можем реализовать это преобразование 
очень просто:

\begin{lstlisting}
   let $\inbr{pretty_{expr}}$ e =
     let rec pretty_prio p = function
     | Const  n        -> string_of_int n
     | Var    x        -> x
     | Binop (o, l, r) ->
        let po = prio o in
        (if po <= p then br else id) @@
        pretty_prio po l ^ " " ^ o ^ " " ^ pretty_prio po r
     in
     pretty_prio min_int e
\end{lstlisting}

Здесь мы пользуемся функциями ``\lstinline{prio}'', ``\lstinline{br}'' и ``\lstinline{id}'', доступными из вне. Функция ``\lstinline{prio}''
возвращает приоритет бианрной операции, ``\lstinline{br}'' окружает свой аргумент скобками, а ``\lstinline{id}'' --- тождественная функция.
Дополнительная функция ``\lstinline{pretty_prio}'' принимает числовой параметр, который обозначает приоритет окружающей операции (если такая имеется). Если приоритет текущей опреации меньше или равен переданному, тогда выражение окружается скобками. Для простоты мы считаем, что все операции неассоциативны, но такой же шаблон кода может быть использован для поддержки ассоциативных операций.
На верхнем уровне мы передаем наименьшее возможное число как приоритеть, чтобы убедиться, что мы не получим скобок, окружающих выражение целиком 

Реализации этих двух функций имеют очень мало общего. Обе возращаеют строки, но вторая принимает дополнительный аргумент, и 
правые части сопоставления с образцом для соответвующих конструкторов различаются. Единственной общей частью является
сопоставление с образцом само по себе. Мы может извлечь его в отдельную функцию и параметризовать эту функцию множеством трансформаций, 
соответсвующих конструкторам:

\begin{lstlisting}
   let $\inbr{gcata_{expr}}$ $\omega$ $\iota$ = function
   | Const n         -> $\omega$#$\inbr{Const}$ $\iota$ n
   | Var   x         -> $\omega$#$\inbr{Var}$   $\iota$ x
   | Binop (o, l, r) -> $\omega$#$\inbr{Binop}$ $\iota$ o l r
\end{lstlisting}

Здесь мы представляем множетсво семантически связанных функций объектом. ``$\omega$'' -- это объект, где методы соответсвуют конструктором
один к одному. ``$\iota$'' представляет дополнительный параметр, который может использоваться функциями как, например, ``$\inbr{pretty_{expr}}$'' (и игнорироваться фукнциями на подобие ``$\inbr{show_{expr}}$'').

Упомянутая в начале фукция ``$\inbr{show_{expr}}$'' может быть выражена следующим образом\footnote{Для ясности понимания мы опустили некоторые аннотации типов, которые помогают этому листингу кода пройти проверку типов.}:

\begin{lstlisting}
   let rec $\inbr{show_{expr}}$ e = $\inbr{gcata_{expr}}$
     object
       method $\inbr{Const}$ _ n   = "Const " ^ string_of_int n
       method $\inbr{Var}$  $\enspace$   _ x   = "Var " ^ x
       method $\inbr{Binop}$ _ o l r =
         Printf.sprintf "Binop (%S, %s, %s)" o ($\inbr{show_{expr}}$ l) ($\inbr{show_{expr}}$ r)
     end
     ()
     e
\end{lstlisting}

И, разумеется, всё то же  самое применимо к функции $\inbr{pretty_{expr}}$.

Вы могли заметить, что оба объекта, необходимые для реализации этих функций, могут быть созданы с помощью общего виртуального класса:

\begin{lstlisting}
   class virtual [$\iota$, $\sigma$] $\inbr{expr}$ =
   object
     method virtual $\inbr{Const}$ : $\iota$ -> int -> $\sigma$
     method virtual $\inbr{Var}\enspace\;\;$ : $\iota$ -> string -> $\sigma$
     method virtual $\inbr{Binop}$ : $\iota$ -> string -> expr -> expr -> $\sigma$  
   end
\end{lstlisting}

Конкретный класс, представляющий преобразование будет наследоваться от этого общего предка. Чтобы иметь возможность 
вызывать рекурсивно данное преобразование, мы параметризуем класс функцией самотрансформации ``\lstinline{fself}'' 
(\emph{открытая рекурсия}). 
Написание в стиле открытой рекурсии необходимо для возможности поддержки полиморфных вариантных типов и рекурсивных определений.
Теперь мы сможем релаизовать логику распечатки в формат, удобный человеку, в изоляции, отдельно от фукнции ``красивой'' распечатки
 (обратите внимание на использование ``\lstinline{fself}''):

\begin{lstlisting}
   class $\inbr{pretty_{expr}}$ (fself : $\iota$ -> expr -> $\sigma$) =
   object inherit [int, string] $\inbr{expr}$ 
     method $\inbr{Const}$ p n = string_of_int n
     method $\inbr{Var}$ p x = x
     method $\inbr{Binop}$ p o l r =
       let po = prio o in
       (if po <= p then fun s -> "(" ^ s ^ ")" else fun s -> s) @@
       fself po l ^ " " ^ o ^ " " ^ fself po r
   end
\end{lstlisting}

Функция распечатки в удобный человеку формат может быть легко описана с использованием класса выше и функции обощенной 
трансформации\footnote{Так как имена функции и классов находятся в разных пространствах имен в \textsc{OCaml}, мы может 
использовать одно и то же имя для класса и функции трансформации.}:

\begin{lstlisting}
   let $\inbr{pretty_{expr}}$ e =
     let rec pretty_prio p e = $\inbr{gcata_{expr}}$ (new $\inbr{pretty_{expr}}$ pretty_prio) p e in
     pretty_prio min_int e
\end{lstlisting}

Также мы может избежать объявления вложенной функции с помощью комбинатора неподвижной точки ``\lstinline{fix}'':

\begin{lstlisting}
   let $\inbr{pretty_{expr}}$ e =
     fix (fun fself p e -> $\inbr{gcata_{expr}}$ (new $\inbr{pretty_{expr}}$ fself) p e) min_int e
\end{lstlisting}

Выше мы смогли выделить две общие чатси для двух существенно различных преобразований: функцию обобщенного обхода
(``$\inbr{gcata_{expr}}$'') и такой виртуальный класс (``$\inbr{expr}$''), что все трансформации можно представить как его экземпляры.
Но стоило ли это того? В действительности, в этом примере мы добились не очень большого переиспользования кода путём добавления
большого количества абстракций. Итоговый код получился по размеру даже больше исходного.

We argue that in this particular case the transformations were not general enough. In order to justify our approach we consider another, more optimistic scenario. It is
well-known, that many transformations can be represented (and for a good reason) using \emph{catamorphisms}, or ``folds''~\cite{Fold,Bananas,CalculatingFP}. Technically, to
implement regular catamorphism we would need to abstract the type ``\lstinline{expr}'' of itself to make it a proper functor, but for now we stick with a more
lightweight version:

\begin{lstlisting}
   class [$\iota$] $\inbr{fold_{expr}}$ (fself : $\iota$ -> expr -> $\iota$) =
   object inherit [$\iota$, $\iota$] $\inbr{expr}$ 
     method $\inbr{Const}$ i n = i
     method $\inbr{Var}$ i x = i
     method $\inbr{Binop}$ i o l r = fself (fself i l) r
   end
\end{lstlisting}

This implementation simply threads the argument ``\lstinline{i}'' through all nodes of an expression and returns it unchanged. This seems pretty useless at a first
glance. However, if we modify this default behaviour a little, we can obtain something useful:

\begin{lstlisting}
   let fv e =
     fix (fun fself i e ->
            $\inbr{gcata_{expr}}$ (object inherit [string list] $\inbr{fold_{expr}}$ fself
                         method $\inbr{Var}$ i x = x :: i
                       end) i e
         ) [] e
\end{lstlisting}

This function calculates the list of all free variables in an expression (as there can be no binders this is simply the list of all variables). Immediate object we
construct here inherits from the ``useless'' ``$\inbr{fold_{expr}}$'' and redefines only one method~--- for variables. All other code makes exactly what we need~---
``$\inbr{gcata_{expr}}$'' traverses the expression, and all other methods of transformation object accurately pass the list of variables through. So, we indeed
managed to implement some interesting transformation with a very small modification of existing code (provided that ``$\inbr{fold_{expr}}$'' class was already supplied).
To avoid the impression that we carefully prepared everything to implement this particular example we can show another one:

\begin{lstlisting}
   let height e =
     fix (fun fself i e ->
            $\inbr{gcata_{expr}}$ (object inherit [int] $\inbr{fold_{expr}}$ fself
                         method $\inbr{Binop}$ i _ l r = 1 + max (fself i l) (fself i r) 
                       end) i e
         ) 0 e
\end{lstlisting}

Now we calculated the height of an expression. We used the same ``$\inbr{fold_{expr}}$'' class as a base for another immediate object; we redefined the method for
binary operators, which now calculates the heights of both sub expressions, takes the maximum and adds one. 

Another commonly recognised generic feature is ``map'':

\begin{lstlisting}
   class $\inbr{map_{expr}}$ fself =
   object inherit [unit, expr] $\inbr{expr}$
     method $\inbr{Var}$ _ x = Var x
     method $\inbr{Const}$ _ n = Const n
     method $\inbr{Binop}$ _ o l r = Binop (o, fself () l, fself () r)
   end
\end{lstlisting}

Again, as type ``\lstinline{expr}'' is not a functor, all we can do with ``$\inbr{map_{expr}}$'' is copying. However, by inheriting from it we
can provide more transformations:

\begin{lstlisting}
   class simplify fself =
   object inherit $\inbr{map_{expr}}$ fself
     method $\inbr{Binop}$ _ o l r =
       match fself () l, fself () r with
       | Const l, Const r -> Const ((op o) l r)
       | l      , r       -> Binop (o, l, r)     
   end
\end{lstlisting}

This class performs a constant folding: if both arguments of a binary operator are reduced (by the same transformation) to constants, then in
performs the operation. The function ``\lstinline{op}'' is defined elsewhere; it returns an integer function for evaluating given binary operator. One more:


\begin{lstlisting}
   class substitute fself state =
   object inherit $\inbr{map_{expr}}$ fself
     method $\inbr{Var}$ _ x = Const (state x)  
   end
\end{lstlisting}

This one substitutes variables in an expression with their values in some state, represented as function ``\lstinline{state}''. Two last
classes can be seamlessly combined to construct an evaluator:

\begin{lstlisting}
   class eval fself state =
   object
     inherit substitute fself state
     inherit simplify   fself
   end

   let eval state e =
     fix (fun fself i e -> $\inbr{gcata_{expr}}$ (new eval fself state) i e) () e  
\end{lstlisting}

In all these examples we, starting from some very common generic feature, implemented all needed transformations with a very little efforts (modulo
the verbose \textsc{OCaml} syntax for objects and classes). In each case we needed to override only one method, and we used a single per-type generic
function. On the other hand we dealt with a very simple type~--- for example, it was not even polymorphic, and supporting polymorphism might have
its own issues. In the rest of the paper we show that, indeed, the sketch we presented here can be extended to a generic programming
framework, in which all the components can be synthesised from type definitions. In particular, our approach provides the full support for:

\begin{itemize}
\item Polymorphism.
\item Type constructor application.
\item Mutual recursion. While there is no problem with implementation of hard-coded generic transformations, the implementation of \emph{extensible} ones
  requires extra efforts.
\item Polymorphic variant types. It includes the seamless integration via class inheritance of all features
  for polymorphic variant types when these types are combined into the one.
\item Separate compilation: we can generate code from type definitions for a module separately with no lookup into
  modules this one depends on.
\item Encapsulation: we support module signatures, including abstract and private type declarations. Generic functions, implemented for
  abstract types, can be safely used outside the module, but can be neither modified nor used to ``peep'' at the internal structure of
  the type.  
\end{itemize}

We also address some performance issues~--- as one could notice, in all preceding examples we created a whole bunch of \emph{identical} objects during a
transformation (one per each node of a data structure); as we will see, this can be avoided via memoization. Finally, our framework provides a plugin system which can be
used to generate a number of useful transformations (like ``\lstinline{show}'', ``\lstinline{fold}'' or ``\lstinline{map}''). The plugin system is
extensible as well~--- end users can implement their own plugins with a very little amount of extra effort since a large part of their functionality (the traversal
function and virtual transformation class) is already supplied by the framework. 

% \section{Метод}
\label{sec:implementation}

\textcolor{red}{Длинновато, и лучше явно чтобы структура виднелась}

Наш метод применяет обобщенное программирование к аннотированным  типам данных и порождает во время компиляции следующие сущности.

\begin{itemize}
\item Базовый виртуальный класс (раздел~\ref{transtypes}), который используется как общий предок для всех преобразований, один на каждый тип. 

\item Некоторое количество конкретных классов, по одному на каждый вид требуемого преобразования.

\item Обобщенная функция преобразования (\emph{generic catamorphism}, \emph{gcata})~--- одна на каждый тип данных.

\item Структура данных \emph{typeinfo} (раздел~\ref{typeinfo}), которая сохраняет сущности, построенные на предыдущих шагах, для последующего использования.
\end{itemize}


При объявлении базового класса (раздел~\ref{transtypes})
мы объявляем некоторое количество виртуальных методов, по одному методу на каждый конструктор алгебраического типа. Некоторые другие подходы~\cite{Visitors} требуют объявления большего количества виртуальных методов, что несет в себе риск раскрытия слишком большого количества деталей реализации типа, в случае, если программист пожелал скрыть эти детали, сделав тип абстрактным. Объявление слишком узкого интерфейса для объектов несёт риск построения недостаточно выразительных преобразований. В разделе~\ref{sec:Evaluation} мы испытываем наш метод на некотором количестве примеров, чтобы показать, что наш способ проектирования интерфейсов объектов позволяет получать достаточно разнообразные преобразования.
%минимальный ограниченный набор методов, что позволяет раскрывать меньше деталей при написании преобразований 
%избежать утечек абстракции при преобразованиях 
%абстрактных типов данных.\footnote{Сказатть, что объявляются виртуальные методы, они потом реализуются в объектах и мы хотим минимальное количество этих методов. Сформулировать способ как мы выбираем минимальные методы} 


Это ограничение вызывает трудности при построении функций-преобразований и преобразований для взаимно-рекурсивных типов, которые преодолеваются в разделах~\ref{memofix} и~\ref{murec}, соответственно. Построенные функции объединяются, что позволяет эмулировать (раздел~\ref{typeinfo}) возможность написание функций, индексированных типами (ad hoc полиморфизм~\cite{cardelli}).

Отметим, что функции-преобразования, построенные в разделе~\ref{memofix}, имеют такие же типы, как и функции, получающиеся стандартными подходами к обобщенному программированию, и  поэтому будут удобны для использования разработчиками. 
%на основе построенных классов и объектов мы строим (раздел~\ref{memofix}) преобразования-функции, чтобы интерфейс, предоставляемый подходом, был стандартным для \ocaml{}. 

\subsection{Типы преобразований}
\label{transtypes}

%TODO: сказать прямее и точнее

Наш метод построения расширяемых преобразований предполагает представление преобразований с помощью классов и объектов, из которых можно получать функции-преобразования. Эти функции-преобразования для простых типов данных используются при построении классов для сложных типов данных, поэтому мы начнем с описания типов функций-преобразований.

Основываясь на идее описания катаморфизмов~\cite{Bananas} с помощью атрибутных 
грамматик~\cite{AGKnuth,ObjectAlgebrasAttribute,AGSwierstra} мы рассматриваем функции-преобразования следующего вида:

\[
\iota \to t \to \sigma,
\]

\noindent где $t$ -- это тип, значения которого мы преобразуем, $\iota$ и $\sigma$~--- типы \emph{наследуемых} и \emph{синтезируемых} атрибутов. 
Мы не будем использовать атрибутные грамматики для того, чтобы описывать алгоритмическую часть преобразований, мы только переиспользуем терминологию для описания типов. 

Если тип $t$ является параметрическим, то преобразование тоже будет параметрическим. Далее мы будет обозначать с помощью
$\left\{...\right\}$ множественное вхождение сущности в скобках. С помощью такой нотации мы сможем следующим образом описать обобщенную форму наших преобразований:

\[
  \left\{\iota_i \to \alpha_i \to \sigma_i\right\}\to\iota \to\left\{\alpha_i\right\}\;t \to \sigma
\]

\noindent Эта форма состоит из $n$ функций-преобразований типовых параметров и функции-преобразования непосредственно типа  $t$. Все используемые функции-преобразования действуют на соответствующие наследуемые атрибуты и возвращают синтезированные атрибуты.
%TODO: как оно соотносится с предыдущей формулой, сказать точнее
%Здесь $\iota_i\to\alpha_i\to\sigma_i$ является функцией-пре\-образованием для типового параметра $\alpha_i$. В общем, функции-преобразования значений типы $t$ действуют также на наследуемые атрибуты и возвращают синтезируемые атрибуты (!!!). 
Общий для всех преобразований класс-предок для $n$-параметрического типа будет иметь $3*(n+1)$ типовых параметров:

\begin{itemize}
\item тройка $\iota_i$, $\alpha_i$, $\sigma_i$ для каждого типового параметра $\alpha_i$, где $\iota_i$ и $\sigma_i$ --- это типовые переменные наследуемого и синтезированного атрибутов для преобразования  $\alpha_i$;
\item пара дополнительных типовых переменных $\iota$ и $\sigma$ для представления наследуемого и синтезированного атрибутов преобразуемого типа;
\item дополнительная типовая переменная $\varepsilon$, которая приравнивается к ``\lstinline|$\{\alpha_i\}$ t|'' для типов отличных от полиморфных вариантные, и приравнивается к \emph{открытому} типу ``\lstinline|[> $\{\alpha_i\}$ t]|'' для полиморфных вариантных типов (подробнее в разделе~\ref{pv}).
\end{itemize}

\noindent Например, если нам дан двупараметрический тип \lstinline{($\alpha$, $\beta$) t}, то заголовком общего класса-предка будет 

\begin{lstlisting}
class virtual [$\iota_\alpha\!$, $\!\alpha\!$, $\!\sigma_\alpha$, $\!\iota_\beta$, $\!\beta\!$, $\!\sigma_\beta\!$, $\!\iota\!$, $\!\varepsilon\!$, $\!\sigma\!$] $\inbr{t}$
\end{lstlisting}

Конкретные преобразования будут наследоваться от этого класса и, возможно, конкретизировать некоторые из типовых параметров.
Дополнительно конкретные классы получают несколько аргументов-функций:

\begin{itemize}
\item $n$ функций, преобразующих типовые параметры: \lstinline|f$_{\alpha_i}$ : $\iota_i$ -> $\alpha_i$ -> $\sigma_i$|;
\item функция для реализации открытой рекурсии: \lstinline|fself : $\iota$ -> $\varepsilon$ ->  $\sigma$|.
\end{itemize}

\noindent Например, для типа, упомянутого выше и преобразования ``\lstinline{show}'' заголовок конкретного класс будет выглядеть как

\begin{lstlisting}
class [$\alpha$, $\beta$, $\varepsilon$] $\inbr{show_t}$ 
  (f$_\alpha$     : unit -> $\alpha$ -> string)
  (f$_\beta$     : unit -> $\beta$ -> string)
  (fself : unit -> $\varepsilon$ -> string) =
object 
  inherit [ unit, $\alpha$, string
          , unit, $\beta$, string
          , unit, $\varepsilon$, string] $\inbr{t}$
  $\dots$
end 
\end{lstlisting}

Обратите внимание, что мы поддерживаем эти соглашения для всех типов, хотя для некоторых типов некоторые компоненты могут быть излишни, например, ``\lstinline{fself}''
нужен только для рекурсивных типов. Объяснение этому простое: если мы \emph{используем} некоторый тип
то мы в общем случае не знаем его определения. Следовательно, для поддержки раздельной компиляции интерфейсы всех сущностей должны иметь общую структуру.

Эта схема типизации выглядит очень многословной и неочевидной. Присутствует большое количество типовых параметров в которых легко запутаться.
Однако, пользователям понадобится разбираться с ними только если они будут реализовывать преобразование \emph{вручную} с нуля путем 
наследования от общего класса-предка.
В большинстве случаев преобразование реализуется путем небольшой специализации конкретного преобразования или используя систему плагинов. 
В первом случае многие типовые параметры будут уже специализированные (например, для  ``\lstinline{show}'' большинство типовых параметров конкретизируется в базовые типы), во втором система плагинов упрощает процесс правильной конкретизации типовых параметром. % (подробнее в разделе~\ref{plugins}).

Также необходимо описать сигнатуры методов общего класса. Метод для конструктора  ``\lstinline|C of a$_1$ * a$_2$ * ... * a$_k$|'' имеет следующую сигнатуру:

\begin{lstlisting}
method virtual $\inbr{C}$ 
  : $\iota$ -> $\varepsilon$ -> a$_1$ -> a$_2$ -> ... -> a$_k$ -> $\sigma$
\end{lstlisting}

\noindent Метод принимает не только наследуемый атрибут и аргументы, соответствующие конструктору, но и значение, которое сейчас преобразуется.

Наконец, мы опишем тип обобщенных функций преобразования. Тип слегка изменяется для случая полиморфных вариантных типов.

Для типа, не являющегося полиморфным вариантным типом, с именем ``\lstinline|$\{\alpha_i\}$ t|'' \emph{обобщенная функция преобразования} имеет следующий тип:

\begin{lstlisting}
val $\inbr{gcata_t}$ : [$\{\iota_{\alpha_i}$, $\!\!\alpha_i$, $\!\!\sigma_{\alpha_i}\}$, $\!\!\iota\!$, $\!\!\{\alpha_i\}$ t, $\sigma$] #$\inbr{t}$ 
                -> $\iota$ -> $\{\alpha_i\}$ t -> $\sigma$
\end{lstlisting}

Она принимает объект, представляющий преобразование, у которого типовые параметры, полученные путём наследования от базового класса, соответствующим образом конкретизированы, наследуемый атрибут, значение, которое будет преобразовано и возвращает синтезируемый атрибут.
Дополнительный параметр ``$\varepsilon$'' конкретизируется в обрабатываемый тип, а 
для полиморфных вариантных типов ---~в \emph{открытую}
версию типа:  ``\lstinline|[> $\!\!\{\alpha_i\}$ t]|''. 

Обобщенная функция преобразования позволяет запускать   преобразования, представленные как объекты, а также преобразования, расширенные путём наследования. Преобразования типов, используемых при объявлении данного, никаким явным образом не попадают в интерфейсы объектов. Такой дизайн был выбран для того, чтобы не предоставлять детали реализации типа в интерфейсе объекта-преобразования. Другие подходы, например~\cite{Visitors}, и реализуют обобщённую функцию преобразования, и преобразования типовых параметров в виде методов объектов.


\subsection{Комбинатор неподвижной точки и мемоизация}
\label{memofix}

В предыдущем разделе мы описали способ построения объектов, представляющих преобразования. Теперь необходимо воспользоваться только что построенными объектами, чтобы построить функции-преобразования. Их необходимо строить, чтобы наш подход с использованием обобщённого программирования предоставлял практикующему разработчику знакомый интерфейс.

Мы полагаемся  на открытую рекурсию: класс, реализующий конкретное преобразование принимает функцию-преобразование самого себя как параметр.
Чтобы создать такую функцию необходим комбинатор неподвижной точки. В  этом разделе
мы рассмотрим только простой такой комбинатор, а именно для одиночного объявления типа.
Во взаимно рекурсивном случае понадобится чуть более сложная реализация. %(подробнее в разделе~\ref{murec}).

%Мы напоминаем вам пример из раздела~\ref{sec:expo}:

Преобразование $tr$ для типа $t$, представленное с помощью функции $\inbr{tr_{t}}$, реализуется следующим способом:

\begin{lstlisting}
let $\inbr{tr_{t}}$ $\{f_i\}$ $\iota$ x =
  transform_gc $\inbr{gcata_t}$ (new $\inbr{tr_{t}}$ $\{f_i\}$) $\iota$ x
\end{lstlisting}
\noindent Оно определяется с помощью класса  $\inbr{tr_{t}}$, представляющего преобразование для типа $t$, и обобщенной функции преобразования типа $t$. 

%\begin{lstlisting}
%let transform t = transform_gc t.gcata
%\end{lstlisting}
\begin{lstlisting}
let transform_gc gcata make_obj $\iota$ x =
  let rec obj = lazy (make_obj fself)
  and fself $\iota$ x = 
    gcata (Lazy.force obj) $\iota$ x in
  fself $\iota$ x
\end{lstlisting}

\noindent В этой реализации также используется комбинатор неподвижной точки \lstinline{transform_gc}, объявляемый один раз для всех типов. Он нужна для того, чтобы передавать в класс, описанный с помощью открытой рекурсии, функцию-преобразование типа $t$, которую мы хотим описать. В реализации используется ленивое создание объекта, представляющего преобразование, чтобы избежать создания этого объекта при каждом рекурсивном вызове. Эта экономия возможна по причине того, что во время применения преобразования объект, представляющий преобразование, не изменяется.

%\begin{comment}
%\begin{lstlisting}
%let $\inbr{pretty_{expr}}$ i e = 
%  fix (fun fself -> 
%         $\inbr{gcata_{expr}}$ (new $\inbr{pretty_{expr}}$ fself))
%  i e
%\end{lstlisting}
%\end{comment}


%\begin{comment}
%Здесь присутствует лямбда абстракции, тело которой вычисляется всякий раз, когда вызывается \lstinline{fself}'' в классе преобразования (по сути, для каждого узла в дереве трансформируемого значения). Так как все объекты одинаковы, то их создание можно соптимизировать.
%
%Мы мемоизируем создания объекта, представляющего преобразование, с помощью ленивых вычислений. Для этого мы абстрагируем создание объекта в функцию, которая принимает
%аргумент ``\lstinline{fself}''. Реализация комбинатора неподвижной точки выглядит следующим образом:
%
%\begin{lstlisting}
%let fix gcata make_obj $\iota$ x =
%  let rec obj = lazy (make_obj fself)
%  and fself $\iota$ x = gcata (Lazy.force obj) $\iota$ x in
%  fself $\iota$ x
%\end{lstlisting}
%
%Этот комбинатор может использоваться для всех типов и не является генерируемым по типу данных. Теперь мы может немного исправить объявление функции ``\lstinline{transform}'':
%
%\begin{lstlisting}
%let transform typeinfo = fix typeinfo.gcata
%\end{lstlisting}
%
%С помощью этого определения пользователю не нужно использовать комбинатор неподвижной точки явно:

%\begin{lstlisting}
%let $\inbr{show_{expr}}$ e =
%  transform(expr) (fun fself -> new $\inbr{show_{expr}}$ fself) () e
%\end{lstlisting}

%\end{comment}

\subsection{Взаимно рекурсивные определения}
\label{murec}

В случае, если необходимо построить преобразование для группы взаимно-рекурсивных определений типов, то порожденный код несколько усложнится. Во-первых классы преобразований также начнут получать преобразования для других типов, объявленных взаимно-рекурсивно, в следствие использования представления с открытой рекурсией. Во-вторых, для данной группы будет нужен специальный комбинатор неподвижной точки, который осуществляет <<завязывание в узел>> преобразований одного вида для данной группы типов. В третьих, будет необходимо объявить дополнительные классы преобразований, специализированные для только что построенных преобразований других типов, чтобы финальный интерфейс классов был одинаков для взаимно-рекурсивных и обычных определений.



\subsection{Полиморфные вариантные типы}
\label{pv}

Мы считаем поддержку полиморфных вариантных типов~\cite{PolyVar,PolyVarReuse} важной частью нашей работы, так как она позволяет 
композиционально объявлять типы данных, а также строить композициональные преобразования.
Главным отличием между полиморфными вариантным типами и алгебраическими, является возможность
\emph{расширения} объявленных ранее полиморфных вариантных типов либо путём добавления новых конструкторов, либо комбинированием нескольких типов в один.

Нашей задачей является предоставление  \emph{бесшовной} интеграции с обобщенными возможностями. Если несколько полиморфных вариантных типов будет скомбинированы, то мы должны уметь получать все обобщенные возможности простым наследованием соответствующих типов.

Как мы сказали ранее (раздел~\ref{transtypes}), дополнительный параметр  ``$\varepsilon$'' вычисляется в открытую разновидность полиморфного вариантного типа. Следовательно, системой типов языка \OCaml{} разрешено использовать ту же функцию обобщенного преобразования для более \emph{широкого} типа\footnote{Мы воздерживаемся от использования термина ``подтип'' так как в \textsc{OCaml} вместо него используется \emph{row-полиморфизм}.}. 
Это может быть достигнуто специфической формой обобщенной функции-преобразования, которая производит ``открытие'':

\begin{lstlisting}
let $\inbr{gcata_t}$ $\omega$ $\iota$ subj =
  match subj with
  | C $\dots$ -> 
      $\omega$#$\inbr{C}$ $\iota$ (match subj with 
                 #t as subj -> subj) 
      $\dots$
  | $\dots$
\end{lstlisting}

Тонкостью является применение методов объекта, представляющего преобразование, к открытой разновидности типа, в то время как обобщенная функция-преобразование принимает замкнутый тип.

Если несколько полиморфных вариантных типов объединяются, то обобщенная функция-преобразование сопоставляет значение с образцами-типами и передает управление соответствующим обобщенными функциям преобразования.

\subsection{Сохранение информации о типе}
\label{typeinfo}

Последним этапом порождения кода, является объединение в общую структуру данных построенных функций, а именно, обобщенной функции-преобразования, комбинатора неподвижной точки, и реализаций специфичных преобразований. Это открывает возможности по написанию кода, индексированного типами, что в сочетании с использованием модулей неявно (modular impicits~\cite{ModularImplicits}), может открыть интересные перспективы в будущем. 

В примере ниже мы описываем функцию, которая принимает информацию о типе и вызывает с помощью неё два преобразования: \lstinline{show} и \lstinline{gmap}.
Особенностью данного примера является, то, что в качестве \lstinline{info} можно передать информацию о \emph{любом} типе, для которого реализованы два соответствующих преобразования.


\begin{lstlisting}
let f info = 
  ... 
  GT.show info ...
  GT.gmap info ...
\end{lstlisting}


\subsection{Ограничения}

\textcolor{red}{Лучше 1), 2) и т.д.}

Мы поддерживаем различные варианты объявлений типов в  \ocaml{} со следующими ограничениями:

\begin{enumerate}
\item только регулярные алгебраические типы данных; обобщённые~\cite{GADT} обрабатываются как обычные алгебраические типы;
\item ограничения на типы (constraints) не учитываются;
\item расширяемые алгебраические типы данных
(``\lstinline{..}''/``\lstinline{+=}'') не поддерживаются;
\item объекты, модули и типы с ключевым словом ``\lstinline{nonrec}'' не поддерживаются.
\end{enumerate}

\noindent Пункты 1-3 являются стандартными ограничениями для обобщённого программирования, они присутствуют также и в~\cite{ppxderiving}. Ограничение 4 возникает потому, что \ocaml{} не позволяет описывать классы с одинаковым именем в одной области видимости. Однако,  необходимость в ключевом слове ``\lstinline{nonrec}'' возникает редко, и мы планируем решить эту проблему в будущем.




% Дальше закомментировано









\begin{comment}
\subsection{Система плагинов}
\label{plugins}


\subsection{Взаимная рекурсия}
%\label{murec}

%Опущены 2 страницы про них

Полная поддержка взаимно рекурсивных определений типов требует дополнительных усилий.
Формально, создание всех необходимых сущностей может быть произведена также, как и для 
одиночного случая, но это может нарушить расширяемость получаемых преобразований.
Мы продемонстрируем это феномен в примере ниже. Рассмотрим определение типа


\begin{lstlisting}
type expr = $\dots$ | LocalDef of def * expr
and  def  = Def of string * expr
\end{lstlisting}

где мы опустили неважные части (переменные, бинарные операции и т.д.) в объявлении типа выражений. Довольно очевидно, что обобщённые функции преобразований для обоих типов могут  быть оставлены как они есть, так как они по сути просто перекладывают работы про выполнению преобразования на плечи методов объекта и не зависят от наличия рекурсии в определениях типов.

\begin{lstlisting}
let $\inbr{gcata_{expr}}$ $\omega$ $\iota$ = function
$\dots$
| LocalDef (d, e) as x -> $\omega$#$\inbr{LocalDef}$ $\iota$ x d e

let $\inbr{gcata_{def}}$ $\omega$ $\iota$ = function
| Def (s, e) as x -> $\omega$#$\inbr{Def}$ $\iota$ x s e
\end{lstlisting}

То же самое верно и для общего класса-предка. Однако, если мы начнем реализовывать конкретные преобразования, то нам понадобится преобразование значений 
типа ``\lstinline{expr}'' внутри класса для ``\lstinline{def}'', и наоборот. Это может быть сделано с помощью взаимно рекурсивных определений классов (мы опять же опускаем неважные части кода):

\begin{lstlisting}
class $\inbr{show_{expr}}$ fself = object 
  inherit [unit, _, string] $\inbr{expr}$ fself
  $\dots$
  method $\inbr{LocalDef}$ $\iota$ x d e =
    $\dots$ (fix $\inbr{gcata_{def}}$ (fun fself -> new $\inbr{show_{def}}$ fself) $\dots$) $\dots$
end
and $\inbr{show_{def}}$ fself = object 
  inherit [unit, _, string] $\inbr{def}$ fself
  method $\inbr{Def}$ $\iota$ x s e =
    $\dots$ (fix $\inbr{gcata_{expr}}$ (fun fself -> new $\inbr{show_{expr}}$ fself) $\dots$) $\dots$
end
\end{lstlisting}

Заметьте, что в обоих аргументах ``\lstinline{fix}'' мы создаем \emph{конкретные} классы  (``$\inbr{show_{def}}$'' и ``$\inbr{show_{expr}}$''). На первый взгляд, это должно работать как полагается. Строго говоря, это \emph{конкретное} преобразование действительно работает.
Но что случится, если нам понадобится переопределить поведение в классе 
 ``$\inbr{show_{expr}}$''? Согласно подходу, определенному выше, на необходимо отнаследоваться от ``$\inbr{show_{expr}}$'', переопределить некоторые метода и сконструировать функцию с помощью комбинатора неподвижной точки:

\begin{lstlisting}
class custom_show fself = object 
  inherit $\inbr{show_{expr}}$ fself
  method $\inbr{Const}$ $\iota$ x n = "a constant"
end

let custom_show e = 
  fix $\inbr{gcata_{expr}}$ (fun fself -> new custom_show fself) () e
\end{lstlisting}

А это не будет работать так, как мы ожидаем, потому мы не определили метод
``$\inbr{LocalDef}$'', который использует класс по умолчанию для типа  ``\lstinline{def}'', который в свою очередь пользуется классом по умолчанию для типа  ``\lstinline{expr}''.
Получается, что мы переопределили поведение только одной компоненты взаимно рекурсивного преобразования типов, а именно для типа ``\lstinline{expr}''. 
Все вхождения типа ``\lstinline{expr}'' в других типах всё ещё преобразуются стандартным образом. Чтобы исправить это поведение, нам придется повторить реализацию взаимно рекурсивных классов \emph{целиком}, что обесценивает всю идею расширяемости.

Наше решение проблемы снова полагается на идею открытой рекурсии. Вкратце, мы параметризируем конкретный класс преобразования трансформациями \emph{всех} типов, участвующих во взаимно рекурсивном определении типов.
так как эта параметризация нарушает соглашение об интерфейсах классов, нам придется объявить эти классы как дополнительные. Для нашего примера они будут выглядит вот так:

\begin{lstlisting}
class $\inbr{show\_stub_{expr}}$ $f_{expr}$ $f_{def}$ = object 
  inherit [unit, _, string] $\inbr{expr}$ $f_{expr}$
  $\dots$
  method $\inbr{LocalDef}$ $\iota$ x d e = $\dots$ ($f_{def}$ $\dots$) $\dots$
end

class $\inbr{show\_stub_{def}}$ $f_{expr}$ $f_{def}$ = object 
  inherit [unit, _, string] $\inbr{def}$ $f_{def}$
  method $\inbr{Def}$ $\iota$ x s e = $\dots$ ($f_{expr}$ $\dots$) $\dots$
end
\end{lstlisting}

Обратите внимание на отсутствие рекурсивных классов.

Затем мы сгенерируем комбинатор неподвижной точки для этого взаимно рекурсивного определения:

\begin{lstlisting}
let $\inbr{fix_{expr, def}}$ ($c_{expr}$, $c_{def}$) =
  let rec $t_{expr}$ $\iota$ x = $\inbr{gcata_{expr}}$ ($c_{expr}$ $t_{expr}$ $t_{def}$) $\iota$ x
  and $t_{def}$ $\iota$ x = $\inbr{gcata_{def}}$ ($c_{def}$ $t_{expr}$ $t_{def}$) $\iota$ x in
  ($t_{expr}$, $t_{def}$)
\end{lstlisting}

Здесь $c_{expr}$ и $c_{def}$ являются генераторами объектов, которые принимают как параметры функции преобразования всех типов, которые встречаются во взаимно рекурсивном определении. Обратите внимание, что тот же самый комбинатор неподвижной точки может использоваться для того, чтобы сконструировать любое конкретное преобразование для данного взаимно рекурсивного определения типов.

С этими дополнительными классами мы может сконструировать реализации по умолчанию для любого конкретного преобразования:

\begin{lstlisting}
let $\inbr{show_{expr}}$, $\inbr{show_{def}}$ =
  $\inbr{fix_{expr,def}}$ (new $\inbr{show\_stub_{expr}}$, new $\inbr{show\_stub_{def}}$) 
\end{lstlisting}

Эти преобразования по умолчанию, во-первых, должны сохраниться во всех структурах с информацией о типах для соответствующих типов, и во-вторых, используются для создания классов трансформацией, с ожидаемым интерфейсом:

\begin{lstlisting}
class $\inbr{show_{expr}}$ fself = object 
  inherit $\inbr{show\_stub_{expr}}$ fself $\inbr{show_{def}}$ 
end
class $\inbr{show_{def}}$ fself = object 
  inherit $\inbr{show\_stub_{def}}$ $\inbr{show_{expr}}$ fself 
end
\end{lstlisting}

Здесь мы снова сделали взаимно рекурсивные типы неотличимыми от простых (в терминах интерфейсов классов), что позволяет единообразным способом конструировать преобразования этих типов в файлах, где эти типы используются, но не объявлены.

С другой стороны, чтобы расширить имеющееся преобразование, теперь необходимо наследоваться от \emph{дополнительных} классов и использовать специальный комбинатор неподвижной точки.
Для нашего предыдущего неудачного случая преобразование выглядит почти также просто, как и для одиночного объявления типа:

\begin{lstlisting}
let custom_show, _ =
  $\inbr{fix_{expr,def}}$ ((fun $f_{expr}$ $f_{def}$ ->
                  object inherit $\inbr{show\_stub_{expr}}$ $f_{expr}$ $f_{def}$
                    method $\inbr{Const}$ $\iota$ x n = "a constant"
                  end),
                new $\inbr{show\_stub_{def}}$) 
\end{lstlisting}

В конкретной реализации библиотеки мы генерируем мемоизирующий комбинатор неподвижной точки, который следует тому же шаблону, который был описан в разделе ~\ref{memofix}. К тому же, мы сохраняем данный комбинатор в структуре с информацией о типе, чтобы для 
типа ``\lstinline{t}'' этот комбинатор мог быть использован с помощью выражение 
``\lstinline{fix(t)}''. Пользователям, однако, придется держать в уме, что тип является взаимно рекурсивным, чтобы воспользоваться комбинатором правильно.

Однако присутствует одна сложность с поддержкой взаимной рекурсии: мы полагаемся на то свойства, что добавление одной функции преобразования для типа  достаточно, чтобы реализовать открытую рекурсию. Однако, строго говоря, это не так. Например, рассмотрим следующее объявление типа:

\begin{lstlisting}
type ($\alpha$, $\beta$) a = A of $\alpha$ b * $\beta$ b
and  $\alpha$ b = X of ($\alpha$, $\alpha$) a
\end{lstlisting}

В аргументах конструктора ``\lstinline{A}'' мы имеем \emph{различные} параметризации типа ``\lstinline{b}'', и поэтому нам понадобятся \emph{две} функции~--- для``\lstinline{$\alpha$ b}'' и для ``\lstinline{$\beta$ b}''. Однако, тип ``\lstinline{a}'' не является регулярным~--- начав преобразование типа ``\lstinline{($\alpha$, $\beta$) a}'' мы придём к необходимости преобразования значений типов ``\lstinline{($\alpha$, $\alpha$) a}'' и ``\lstinline{($\beta$, $\beta$) a}''.

Следовательно, мы уже отсеяли такие объявления типов. Получается, что взаимно рекурсивные объявления типов являются \emph{существенными} в том смысле, что они не всегда могут быть разделены на два не взаимно рекурсивных определения, а именно, когда каждая пара типов взаимно достижима. Если мы заменим второе объявление типа, скажем, на

\begin{lstlisting}
...
and $\alpha$ b = int
\end{lstlisting}

то мы получим объявление типов, которое не поддерживается у нас. Однако, так как типы ``\lstinline{a}'' и ``\lstinline{b}''  \emph{не являются}
по сути взаимно рекурсивными, то всё определение типов может быть переписано, что уже позволит воспользоваться нашими наработками.



\subsection{Полиморфные вариантные типы}
\label{pv}

Мы считаем поддержку полиморфных вариантных типов~\cite{PolyVar,PolyVarReuse} важной части нашей работы, так как она открывает возможности 
композиционального определения структур данных с возможность объявления композициональных преобразований.
Главным отличием между полиморфными вариантным типами  и алгебраическими, является возможность 
\emph{расширения} объявленных ранее полиморфных вариантных типов путём добавление новых конструкторов или комбинированием нескольких типов в один. 

Нашей задачей является предоставление  \emph{бесшовной} интеграции с обобщенными возможностями. Когда несколько типов будет скомбинированы, мы должны получить все обобщенные возможности простым наследование соответствующих типов.

Как мы сказали ранее, дополнительный параметр  ``$\varepsilon$'' вычисляется в открытую разновидность полиморфного вариантного типа. Следовательно, должно быть разрешено использовать ту же функцию обобщенного преобразования до для более \emph{широкого} типа\footnote{Мы воздерживаемся от использования термина ``подтип'' так как в \textsc{OCaml} нет настоящего подтипирования.}. 
Это может быть достигнуто специфической формой обобщенной функции трансформации, которая производит ``открытие'':

\begin{lstlisting}
let $\inbr{gcata_t}$ $\omega$ $\iota$ subj =
  match subj with
  $\dots$
  | C $\dots$ -> $\omega$#$\inbr{C}$ $\iota$ (match subj with #t as subj -> subj) $\dots$
  $\dots$
\end{lstlisting}

Это выливается в применении методов объекта, представляющего преобразование, к открытой разновидности типа, в то время как обобщенная функция преобразования принимает замкнутый тип.

Если несколько полиморфных вариантных типов объединяются, то обобщенная функция преобразования сопоставляем значение с образцами-типами и передает управление соответствующим обобщенными функциям преобразования.



\end{comment}
% \section{Реализация и примеры}
\label{sec:Evaluation}

Представленный метод был реализован в библиотеке Generic Transformers\footnote{\url{https://github.com/Kakadu/GT/tree/v0.3.0}} (\GT). Библиотека  поддерживает два наиболее распространенных вида синтаксических расширений языка \ocaml{}: \textsc{PPX}~\cite{PPXLib} и \camlpfive~\cite{camlp5}. Библиотека \GT{} является расширяемой: к ней прилагается интерфейс для добавления пользовательских плагинов, реализующих порождение новых видов преобразований, а также набор стандартных преобразований, использующихся в других подобных библиотеках.


В этом разделе мы представим несколько примеров, реализованных с помощью нашего подхода. В них используются синтаксические расширения \camlpfive{}, но это же может быть реализовано также с использованием \PPX{}. Данные примеры продемонстрируют практическую полезность предлагаемого метода для полиморфных вариантных типов языка \OCaml{} и типов с неограниченной рекурсией, а также  помогут ответить на \emph{RQ1: <<Какие виды расширяемых преобразований можно реализовывать с нашим интерфейсом объектов-преобразований, а какие нет?>>}
%Эффективно ли  кодирование конструкторов один к одному в методы объектов, позволяет ли оно описывать достаточно разнообразные расширяемые преобразования?>>}.

%\textcolor{red}{Сложносочиненные предложения в RQ плохо.}

Предложенный в работе метод позволяет получать расширяемые преобразования путём непрямого вызова частей преобразования. Такой подход приносит некоторые потери производительности. В разделе ~\ref{sec:performance} мы оценим производительность четырех методов построения расширяемых преобразований, что поможет ответить на 
\emph{RQ2: <<Какова скорость выполнения расширяемых преобразований по сравнению с нерасширяемыми?>>}
%\textcolor{red}{Сказать про производительность}

\subsection{Рассмотренные примеры}

Сначала (раздел~\ref{sec:lists}) мы продемонстрируем совместимость 
нашего подхода для \emph{типов данных с неограниченной рекурсией}
(ключ компилятора \texttt{-rectypes}), 
реализовав
представление логических значений, использующихся в  библиотеке реляционного программирования \OCanren{}~\cite{OCanren}. %Особенностью данного примера является то, что в нём используется относительно редкий ключа компилятора \texttt{-rectypes} для объявления типов данных. Не всякое представление объектов-преобразований будет работать с типами данных, для которых необходим этот ключ компиляции.

Затем мы решим <<The Expression Problem>>~\cite{ExpressionProblem}
%, с помощью полиморфных вариантных типов и нашего подхода
 (раздел~\ref{sec:nameless}). Эта задача часто используется как ``лакмусовый тест'' для оценки подходов к обобщенному программированию~\cite{ObjectAlgebras,ALaCarte}. В литературе встречается различные подходы к решению этой задачи, но наша реализация данного примера интересна тем, что использует  и обобщенное программирование, и \emph{полиморфные вариантные типы} языка \OCaml{}.

Эти два примера демонстрируют использование предложенного метода с теми типами данных \OCaml{}, которые в данный момент не поддерживаются другими подходами к построению расширяемых преобразований (в частности, \visitors~\cite{Visitors}). 
%Также данные примеры помогут обосновать наш дизайн  интерфейса объектов, а именно ответить на вопрос: <<Эффективно ли  на практике кодирование конструкторов один к одному в методы объектов, позволяет ли оно описывать достаточно разнообразные расширяемые преобразования?>>.

%Убрать то, что справа:
%В нём мы будем отдельно описывать преобразования для различных частей языка, а потом объединять их с помощью наследования. Это пример также не может быть переписан с использованием \visitors{}, так как та не поддерживает полиморфные вариантные типы языка \ocaml{}.

В разделе~\ref{sec:irregular} мы обсудим работу с нерегулярными типами данных~\cite{irregular}, которые не поддерживаются нашим методом непосредственно, а требуют некоторого изменения способа объявления типов данных, чтобы наш метод был применим. %но поддерживаются при использовании \visitors{}.

% нерегулярными типами данных~\cite{irregular}
% вместо "нерегулярными~\cite{irregular} типами данных"
% потому что термин разрывается 

%Кроме того, перечисленные выше примеры важны для сравнения с подходом~\cite{Visitors}, который  в данный момент не поддерживает использование ключа \texttt{-rectypes} и полиморфные вариантные типы.

В разделе~\ref{sec:design} мы коснемся вопросов дизайна расширяемых преобразований, которые могут упростить или усложнить использование  разработчиками полученных преобразований. В заключительном разделе \ref{sec:performance} ---~вопросов производительности.

%TODO: вообще сравниваться непосредственно с Visitors не хорошо, будет выглядет как мелкое улучшение примеров.

%Может быть другого сорта мотивацию, сказать почему примеры важны (а в конце, "кроме того, примеры важны для сравнения с взиторами, потому что...)

%\subsection{Research Questions}
%Данные примеры помогут ответить на исследовательский вопрос:

%\textcolor{red} {Это не вопрос!!!} Также важно ответить на вопрос: <<Является ли предоставляемый интерфейс достаточно знакомым для практикующего разработчика на \ocaml{}?>>

%TODO: Тут сказать почему они важны

%\parbox{\textwidth}{
%\textcolor{blue} {Запихнуть RQ в предыдущий раздел}
%}

\subsection{Типизированные логические значения}
\label{sec:lists}

Этот пример появился во время работы над строго типизированным встроенным логическим предметно-ориентированным языком на основе \textsc{OCaml}~\cite{OCanren}. 
В нём одной из самых важных конструкций является унификация термов, содержащих свободные логические переменные. Работать с такими структурами данных сложно, а допустить ошибку --- легко. 
Типичным сценарием взаимодействия  
%между логическими и нелогическими (\textcolor{red}{ПЕРЕФРАЗИРОВАТЬ})  
% частями программ 
cо встроенным языком 
является 
создание так называемых \emph{целей вычислений} (goal), содержащих структуры данных со свободными логическими переменными.
Решением логической цели является подстановка переменных, правые части которой в идеальном случае не содержат свободных переменных. 
Чтобы сконструировать цель вычислений необходимо уметь систематически вводить логические переменные в типизированную структуру данных,  а для восстановления ответа -- систематически извлекать из представления, подходящего для работы с \OCanren{}, ответы в обыкновенном
%нелогическом(\textcolor{red}{ПЕРЕФРАЗИРОВАТЬ})  
представлении (т.е. без логических переменных).

Упрощенный тип для логических переменных может быть описан следующим образом:

\begin{lstlisting}
@type 'a logic =
| V     of int
| Value of 'a       with show
\end{lstlisting}
Логическое значение может быть либо свободной логической переменной (``\lstinline{V}'') или каким-то другим значением (``\lstinline{Value}''), которое не является свободной переменной, но потенциально может содержать свободные переменные.
\begin{comment}

Чтобы преобразовывать в и из логических значений, можно воспользоваться следующими функциями:

\begin{lstlisting}
let lift x = Value x

let reify  = function
| V     _ -> invalid_arg "Free variable"
| Value x -> x
\end{lstlisting}

Функция ``\lstinline{reify}'' бросает исключение для свободных переменных, так как в присутствии вхождений свободных переменных
логическое значение нельзя рассматривать как обыкновенную (нелогическую) структуру данных.
\end{comment}


Когда мы работем с логическими структурами данных, нам необходима возможность вставлять логические переменные в произвольные позиции.
Это означает, что мы должны использовать другой тип данных, подходящий для использования 
с точки зрения системы типов. Например,
для списков нам придется абстрагироваться от рекурсии, чтобы иметь возможность описать тип логических списков \lstinline{llist}\footnote{Этот способ применим только при использовании ключа компиляции \texttt{-rectypes}.}:

\begin{lstlisting}
type ('a, 'self) list_like = 
    | Nil 
    | Cons of 'a * 'self
type 'a list = ('a, 'a list) list_like
type 'a llist = 
    ('a, 'a llist) list_like logic
\end{lstlisting}
%которые будут иметь тип ``\lstinline{lexpr}'', объявленный как
%
%\begin{lstlisting}
%type expr' = Var of string logic | Const of int logic 
%           | Binop of lexpr * lexpr
%and  lexpr = expr' logic
%\end{lstlisting}

Если мы захотим, чтобы списки типа \lstinline{llist} без логических значений преобразовывались в строковое представление также, как списки типа \lstinline{list}, необходимо модифицировать преобразование типа \lstinline{logic} в строку, убрав название конструктора \lstinline{Value}:

\begin{lstlisting}
class ['a, 'self] my_show fa fself = 
object
  inherit ['a, 'self] $\inbr{show_{logic}}$ fa fself
  method c_Value () _ x = fa () x
end
\end{lstlisting}
В такой реализации преобразования логических значений, где мы изменили только один конструктор, мы можем объявить тип логических списков заново, и получить для него преобразование в строку, которое на списках без переменных работает так же, как и для типа \lstinline{list}.

Особенностью данного подхода является, во-первых, получение нового преобразования в строку путём изменения одного метода, а, во-вторых, способ объявления типов \lstinline{list} и \lstinline{llist}, который не удается переиспользовать при использовании подхода, предоставляемого \visitors{}.

%Нам также нужно реализовать две функции преобразования. Все эти определения представляют собой типичный пример однотипного (boilerplate) кода.
%
%С изпользованием нашего подхода решение почти полностью декларативно\footnote{При условии включения ключа компиляции \cd{-rectypes}}.
%Во-первых, мы абстрагируемся от интересующего нас типа, заменяя все его вхождения типовой переменной с не встречающимся ранее именем:

%\begin{lstlisting}
%@type ('string, 'int, 'expr) a_expr =
%| Var   of 'string
%| Const of 'int
%| Binop of 'string * 'expr * 'expr with show, gmap
%\end{lstlisting}
%
%Здесь мы абстрагировали тип от всего конкретного, но мы могли обойтись абстрагированием только от самого себя. Заметьте, что 
%мы воспользовались двумя видами обобщенных преобразований~--- ``\lstinline{show}'' и ``\lstinline{gmap}''. 
%Первое будет полезно для отладочных целей, а второе является необходимым для нашего решения.
%
%Теперь мы можем объявить логические и нелогические составляющие как специализации исходного типа:
%
%\begin{lstlisting}
%@type expr  = (string, int, expr) a_expr 
%  with show, gmap
%@type lexpr = (string logic, int logic, lexpr) a_expr logic 
%  with show, gmap
%\end{lstlisting}

%Обратите внимание, что ``новый'' тип ``\lstinline{expr}'' эквивалентен старому, следовательно, такое переписывание типов не нарушает существующий код.
%
%Наконец, определения функций преобразования воспользуются преобразованием, полученным с помощью плагина ``\lstinline{gmap}'', предоставляемого библиотекой:
%
%\begin{lstlisting}
%let rec to_logic   expr = gmap(a_expr) lift  lift  to_logic  expr
%let rec from_logic expr = gmap(a_expr) reify reify from_logic @@ 
%                           reify expr
%\end{lstlisting}
%
%Как вы видите, поддержка типовых операторов существенна для этого примера. В предыдущей реализации~\cite{TransformationObjects} типовые операторы не были поддержаны и их было не так просто добавить.

\subsection{Преобразование в безымянное представление}
\label{sec:nameless}

Полиморфные вариантные типы в языке \ocaml{} позволяют описывать структуры данных композиционально, статически типизировано и в разных модулях компоновки~\cite{PolyVarReuse}.
Целесообразно объявлять преобразования таких структур данных отдельно друг от друга. Задача конструирования преобразований для 
раздельно объявленных и строго типизированных компонент известна как ``проблема выражений'' (``The Expression Problem''~\cite{ExpressionProblem}).
%которая часто используется (\textcolor{red}{Убрать в 5.1}) как ``лакмусовый тест'' для оценки подходов к обобщенному программированию~\cite{ObjectAlgebras,ALaCarte}. 
В этом подразделе мы представим решение этой задачи в рамках нашего подхода. В качестве конкретной задачи мы реализуем преобразование $\lambda$-выражений в безымянное представление.

Во-первых, опишем часть языка выражений без связывающих конструкций:

\begin{lstlisting}
@type ('name, 'lam) lam = 
[ `App of 'lam * 'lam
| `Var of 'name
] with show
\end{lstlisting}

\noindent Выделение этого типа выглядит логично, так как 
кроме указанных двух конструкций, потенциально в языке могут появиться другие, которые будут связывать переменные 
($\lambda$-абстракции, \lstinline{let}-определения и т.д.). Комбинируя различные типы и преобразования этих типов, можно получать различные расширения деревьев абстрактного синтаксиса и преобразований для $\lambda$-выражений.
%, их с несвязывающими конструкциями, а также с ними самими, можно получать различные языки с согласованным поведением \textcolor{red}{ПЕРЕФРАЗИРОВАТЬ}.

Введенный выше тип ``\lstinline{lam}'' является полиморфным: первый параметр используется для представления имен или индексов %(или уровней) 
де Брёйна\footnote{Способ представления лямбда-выражений в безымянном виде предложенный де Брёйном в~\cite{deBruijn}.}, второй необходим для открытой рекурсии (здесь мы следуем  подходу к описанию расширяемых структур данных с помощью полиморфных 
вариантных типов~\cite{PolyVarReuse}).

Для данного типа преобразование в безымянное представление можно определить следующим образом:
%Рассмотрим как для такого типа должны выглядеть преобразование в безымянное представление, а именно, как должен выглядеть класс преобразования.
%Как должно выглядеть преобразование в безымянное представление для такого типа? А именно, как должен выглядеть класс преобразования? Это показано ниже:

\begin{lstlisting}
class ['lam, 'nless] lam_to_nameless
 (flam : string list -> 'lam -> 'nless) =
object
  inherit 
    [ string list, string, int
    , string list, 'lam, 'nless
    , string list, 'lam, 'nless] $\inbr{lam}$
  method $\inbr{App}$ env _ l r = 
    `App (flam env l, flam env r)
  method $\inbr{Var}$ env _ x   = `Var (index env x)
end
\end{lstlisting}

% TODO: Здесь у нас нет call-by-value, поэтому это нифига не интерпретатор

\noindent Здесь мы используем список строк для хранения подстановки переменных и  передаем его как наследуемый атрибут. Затем мы пользуемся функцией 
``\lstinline{index}'' чтобы найти строку в подстановке, т.е.  эта функция преобразует имя в индекс де Брёйна. 
Интересной частью преобразования является типизация общего класса предка ``$\inbr{lam}$''. 
Первая тройка параметров описывает преобразование первого типового параметра. Можно заметить, что мы преобразуем строки в числа используя подстановку.
Здесь типовая переменная ``\lstinline{'lam}'', 
%как мы знаем, 
приравнивается (раздел~\ref{pv}) открытой версии типа ``\lstinline{lam}''. %(ДОИСПРАВИТЬ)
Наконец, результат преобразования типизируется с помощью переменной ``\lstinline{'nless}'', введение которой необходимо для правильной реализации преобразования объединения типов.
%Так происходит именно так потому, что, как будет понятно позднее,  это будет действительно другой тип. (\textcolor{blue}{Сказать прямее, может даже лишнее предложение})
Так как второй типовый параметр обычно ссылается рекурсивно на себя, третья тройка типовых параметров совпадает со второй.

Давайте теперь добавим в язык связывающую конструкцию --- $\lambda$-абстракцию:

\begin{lstlisting}
@type ('name, 'lam) abs = 
  [ `Abs of 'name * 'lam ] with show
\end{lstlisting}

Те же самые рассуждения применимы и тут: мы пользуемся открытой рекурсией и параметризируем представление относительно имени.
Класс для преобразования будет выглядеть похожим образом:

\begin{lstlisting}
class ['lam, 'nless] abs_to_nameless
 (flam : string list -> 'lam -> 'nless) =
object
  inherit [string list, string, int
          , string list, 'lam, 'nless
          , string list, 'lam, 'nless] $\inbr{abs}$
  method $\inbr{Abs}$ env name term = 
    `Abs (flam (name :: env) term)
end
\end{lstlisting}

Заметьте, что метод ``$\inbr{Abs}$'' конструирует значения \emph{другого} типа, чем любая возможная параметризация типа ``\lstinline{abs}''. Действительно, безымянное представление типа не должно содержать никаких суррогатов имён.

Теперь мы можем объединить эти два типа, чтобы получить тип выражений со связывающими конструкциями.

\begin{lstlisting}
@type ('name, 'lam) term = 
  [ ('name, 'lam) lam 
  | ('name, 'lam) abs) ] with show
\end{lstlisting}

Представим два новых типа для именованного и безымянного представления\footnote{Для того чтобы эти определения типов скомпилировались, необходимо использовать ключ компиляции \cd{-rectypes}.}:

\begin{lstlisting}
@type named = (string, named) term 
  with show
@type nameless = 
  [ (int, nameless) lam | `Abs of nameless] 
  with show
\end{lstlisting}

Наконец, мы можем описать преобразование, которое превращает именованные термы в их безымянное представление:

\begin{lstlisting}
class to_nameless
(f : string list -> named -> nameless) = 
object
 inherit 
   [string list, named, nameless] $\inbr{named}$
 inherit 
   [named, nameless] lam_to_nameless f
 inherit 
   [named, nameless] abs_to_nameless f
end
\end{lstlisting}

Это преобразование получается путём наследования поределеннных выше компонент: общего класса для всех преобразований типа ``\lstinline{named}'' 
и двух конкретных преобразований его составляющих: 
``\lstinline{lam_to_nameless}'' и ``\lstinline{abs_to_nameless}''.
Функция-преобразование может быть получена стандартным способом:

\begin{lstlisting}
let to_nameless term =
  transform(named) 
    (fun fself -> new to_nameless fself) 
    [] 
    term
\end{lstlisting}

Только что мы построили реализацию преобразования типа, комбинируя реализации преобразований его составляющих. Эти  реализации могут быть раздельно скомпонованы, но вся система при этом останется строго типизированной. В этом примере демонстрируются возможности подхода по раздельному и модульному представлению преобразований с помощью объектов, а также возможности по использованию полиморфных вариантных типов языка \ocaml{}, которые не доступны в подходе \visitors{}.

\subsection{Нерегулярные типы данных}
\label{sec:irregular}

Основным достоинством подхода \visitors{} является поддержка нерегулярных типов данных с некоторой оговоркой: поддерживаются преобразования в так называемом ``полиморфном режиме''~\cite{Visitors}. Наш метод не позволяет построить преобразования для уже описанных нерегулярных типов данных. Однако, если разработчик проектирует типы с нуля, то у него есть возможность описать их так, чтобы они были регулярными и были совместимы с нашим подходом. 

Рассмотрим объявления нерегулярного типа данных  из работы~\cite{irregular}.

\begin{lstlisting}
type 'a tree = N | C of 'a * ('a * 'a) tree
\end{lstlisting}
\noindent Для этого типа метод на основе \GT{} не сможет построить преобразование, так в языке \ocaml{} не поддерживается нерегулярная типизация объектов. Необходимо переписать это тип, абстрагировавшись от вхождения типа \lstinline{'a * 'a}, и описать тип \lstinline{t}, и уже с помощью него описать  необходимый тип \lstinline{tree} (потребуется использование ключа компилятора \texttt{-rectypes}).
\begin{lstlisting}
type ('a, 'b) t = N | C of 'a * ('a, 'b)
type 'a tree = ('a, 'a * 'a) t 
\end{lstlisting}
\noindent Для этих двух типов предлагаемый метод уже сможет построить требуемые преобразования.

\subsection{Особенности дизайна}
\label{sec:design}

В данном разделе мы рассмотрим пример построения расширяемых преобразований с помощью \visitors~\cite{Visitors}, ещё одного подхода по построению раширяемых преобразований в \OCaml{} с помощью объектов.

\begin{lstlisting}
(struct 
  type 'a menu = ('a * int) list
  [@@deriving visitors 
    { variety = "map"
    ; name = "map_menu"
    ; polymorphic = true }]
  ...
end : sig 
  type 'a menu 
  val add_exn
    : 'a menu -> 'a -> int -> 'a menu
    
  class virtual ['c] map_menu : object ('c)
    constraint 'c = ...
    method private visit_int : 
      'env. 'env -> int -> int
    method visit_menu :
      ('env -> 'a -> 'b) -> 
      'env -> 'a menu -> 'b menu
    ...
end)
\end{lstlisting}

\noindent Вы видите тип данных \lstinline{'a menu}, аннотированный для использования \visitors{}, который реализует ресторанное меню как список блюд и цен. Предположим, разработчик решил скрыть детали реализации меню, объявив тип абстрактным и предоставив функцию добавления в меню, которая, например, в случае указания отрицательной цены, приводит к исключительной ситуации.

Мы хотим обратить внимание на несколько недостатков данной реализации, которые отсутствуют в методе, предложенном в данной работе.

Во-первых, объект-преобразование реализует приватный метод \lstinline{visit_int}, для преобразования чисел, который разрешается переопределять при наследовании. Это позволяет определять преобразования, которые нарушают внутреннюю целостность типа \lstinline{'a menu}, например, которые создают пункты меню с отрицательными ценами. Данный недостаток хорошо известен среди исследователей обобщенного программирования~\cite{SYB} для языка \haskell{}, и порицается~\cite{SafeHaskell}.

Во-вторых, класс-преобразование кодирует свой интерфейс в единственном типовом параметре с помощью ключевого слова \lstinline{constraint}. Преимуществом такого кодирования является некоторое сокращение порождаемого объема кода. Недостатком является невозможность породить тип объекта преобразования в файлах-интерфейсах языка \OCaml{}, что заставляет выписывать типы вручную. Стандартные~\cite{ppxderiving} подходы к обобщенному программированию, как и предложенный в данной работе метод, не страдают от этого недостатка.

В-третьих, порожденный с помощью \visitors{} объект-преобразование, является виртуальным классом, и поэтому не готов к немедленному использованию пользователем. Другие подходы к обобщенному программированию сразу предоставляет готовые к использованию функции-преобразования.

%\subsection{Метрики}

%\textcolor{red}{Объединить со следующим}

%Наши измерения показывают, что относительная производительность методов незначительно зависит от размера входных данных для преобразований, и может существенно разниться между различными видами преобразований.

\subsection{Производительность}
\label{sec:performance}

%При сравнении преобразований, реализованных в традиционном и расширяемом виде,
%следует ожидать от первых большей производительности, так как дополнительный слой абстракции вносит некоторые накладные расходы.
%
%\textcolor{red}{Что стоит ожидать не писать. Пункты тоже убрать, уместить в 1 абзац}

%Тут минимум объяснений про процедуру...

%Были проведены замеры бла-бла...

При замерах были использованы преобразования, построенные с использованием четырёх методов:
нерасширяемые преобразования записанные с помощью рекурсивных функций;
расширяемые преобразования на основе данной работы; 
а также два метода с использованием популярных библиотек.
%частично расширяемые преобразования, реализованные c помощью записей в стиле библиотеки \cd{ppx\_deriving\_morphism};
%расширяемые преобразования по методу \visitors{};

Для ответа на \textit{RQ2} осуществлялись замеры производительности различных преобразований. Непосредственно измерялось количество преобразований, которые удается осуществить за единицу времени, а затем нормировалось относительно базового уровня, в качестве которого был выбран \GT{}.

Измерения осуществлялись на машине с процессором i7-4790K и версии компилятора \texttt{4.10.1+flambda} с помощью одной из распространенных библиотек\footnote{\url{https://github.com/Chris00/ocaml-benchmark}} для оценки производительности. Реализация тестов находится в том же репозитории\footnote{\url{https://github.com/JetBrains-Research/GT/tree/0.4.0/bench}}, что и сам \GT{}.

Для замеров были выбраны два вида преобразований: копирование выражения и преобразование в текстовый формат.
Данные преобразования применялись к $\lambda$-выражениям, состоящим состоит из вложенных $\lambda$-абстракций.
Количество таких абстракций определяет размер выражения. Для тестирования использовались различные размеры от 100 до 1000.

Результаты указаны в таблице \ref{tab:caption}.

\subsection{Анализ результатов}
\emph{RQ1: <<Эффективно ли  на практике кодирование конструкторов один к одному в методы объектов, позволяет ли оно описывать достаточно разнообразные расширяемые преобразования?>>} Приведенные примеры показывают применимость  предлагаемого метода для типичных задач обобщенного программирования. Предлагаемый метод применим для полиморфных вариантных типов \OCaml{} и типов с неограниченной рекурсией, а \Visitors{} не применим. Однако, метод на основе \Visitors{} несколько удобнее для поддержки нерегулярных типов данных, а также \Visitors{} применимы для построения преобразований в стиле SYB\cite{SYB}, но это достигается нарушением принципов абстракции данных.

\emph{RQ2: <<Какова скорость выполнения расширяемых преобразований по сравнению с нерасширяемыми?>>} Сделанные измерения производительности показывают, что при построении расширяемых преобразований итоговая производительность уменьшается по сравнению с нерасширяемыми. Для первого  преобразования (копирование), где измеряется производительность обхода выражения без полезной нагрузки, эти накладные расходы значительны для любого проверенного метода реализации. Однако, для второго вида преобразований (форматирование) обход структуры $\lambda$-выражения занимает малую часть, и поэтому накладные расходы на использование расширяемых преобразований приемлемы. 

При реализации различных преобразований единственным различием между предлагаемым подходом и \Visitors{} находится в обобщенной функция преобразования. В \Visitors{} она реализована как метод объекта, а в \GT{}  виде отдельной функции, которая вызывается не напрямую, что влияет на производительность для некоторых видов преобразований.

Метод \cd{ppx\_deriving\_morphism} не использует представление в виде объектов, поэтому в нём нет накладных расходов на работу с таблицей виртуальных методов, и поэтому он показывает лучшую производительность чем преобразования на основе объектов. Но в нём, как и в \Visitors{}, не поддерживаются полиморфные вариантные типы \OCaml{}.

Стандартный метод построения преобразований напрямую вызывает код для обработки отдельных конструкторов, и поэтому показывает наилучшую производительность. Однако, такие преобразования не являются расширяемыми.




\begin{table*}[t]
  \centering
  \begin{tabular}{l cccc}
    \toprule
    \multirow{2}{*}{Вид преобразования}&  \multicolumn{3}{c}{Метод реализации и улучшение (\%)} \\\cline{2-4}
     &  \visitors & \PPXMorphism  & Стандарт \\\hline
    %\multirow{5}{*}
    {Копирование}  
%      & 300  & --  & 149 & 220  & 311 \\
%      & 500  & --  & 146 & 218  & 246 \\
%      & 700  & -- & 143 & 212  & 244 \\
%      & 900  & --  & 141 & 210  & 243 \\
     &  +(14-17)\% & +(120-131)\% & +(292-305)\% \\\hline
    %\multirow{5}{*}
    {Форматирование}  
%      & 300  & 0  & -- & 1  & 3 \\
%      & 500  & 0  & -- & 1  & 4 \\
%      & 700  & 1  & -- & 1  & 4 \\
%      & 900  & -- & 1  & 2  & 3 \\
     &  не значимо  & +(2-3)\% & +(3-7)\% \\ 
    \bottomrule
  \end{tabular}
\caption{}{Производительность преобразований, реализованных по-разному. Базовая линия -- реализация методом \GT{}. Промежуток в скобках -- изменение производительности (больше -- лучше). Конкретное значение в промежутке определяется размером входных данных для преобразования. Измерения показывают, что для вырожденного преобразования (копирование) накладные расходы существенны, для для содержательного -- приемлемы.
}
\label{tab:caption}
\end{table*}




\section{Related Works}
\label{sec:relatedworks}

В данной работе использованы одновременно и функциональные (комбинаторы), и объектно-ориентированные возможности языка \textsc{OCaml}. Можно найти связанные работы  одновременно и в области типизированного функционального и объектно-ориентированного программирования. Наиболее близкой, использующий язык \textsc{OCaml} и имеющей отношение к этой работе, библиотекой является \textsc{Visitors}~\cite{Visitors}, которая использует те же самые идеи, но принимает существенно другие дизайнерские решения. Детальное сравнение с \textsc{Visitors} вы найдете в конце данного раздела.

Во-первых, сущестсвует несколько билиотек для обощенного программирования для \textsc{OCaml}, которые используют полностью генеративный подход~\cite{Yallop,PPXLib}~--- все необходимые обобщенные функции для всех типов генерируются по-отдельности. Этот подход очень практичен до тех пор, пока набор предоставляемых трансформаций удовлетворяет всем нуждам. Однако, если это не так, необходимо расширить кодовую базу, реализовав все отсутсвующие функции заново
(с потенциально очень малым переиспользовыванием кода). К тому же, те функции,
которые получаются в результате, нерасширяемы. В нашем подходе, во-первых,
множетсво полезных обощенных функций может быть получено из сгенерированных. Во-вторых, чтобы получить полностью новый плагин, достаточно только ``интересные'' части, так как функции обхода и класс для объекта преобразования создает библиотека сама.

Несколько подходов для функционального обобщенного программирования используют 
\emph{представление типов}~\cite{Hinze}. В основе лежит идея разработки универсального представления для произвольного типа (для которого необходимо получить преобразование) и предоставления двух функций преобразования: в универсальное представление и обратно (в идеале образующих изоморфизм). Обобщенные функции преобразуют представление истинных типов данных, что позволяет реализовать все необходимые преобразования один раз. Функции трансляции в универсальное представление и обратно могут быть получены (полу)автоматически, используя такие особенности системы типов  как классы типов~\cite{Hinze,ALaCarte} и семейства типов~\cite{InstantGenerics} в языке Haskell, или  используя синтаксические расширения~\cite{GenericOCaml} в языке \textsc{OCaml}. Хотя некоторые из этих подходов позволяют модификацию (например, обработка некоторых типов особым образом) и поддерживают расширяемые типы, наш подход более гибок, так как позволяет модификацию на уровне отдельных конструкторов. К тому же, мы позволяем сосоществовать нескольким видам трансформаций для одного типа.

Другой подход был задействован в ``Scrap Your Boilerplate'', или SYB~\cite{SYB}, изначально разработанного для  \textsc{Haskell}. Он делает возможным реализовать трансформации,  которые обнаруживают вхождения конкретного типа в произвольной структуре данных. Поддерживаются два основных вида действий: \emph{запросы}, которые выбирают значения конкретного типа данных на основе критериев, заданных пользователем, и \emph{преобразования}, которые единообразно применяют преобразование, сохраняющее тип, в конкретной структуре данных. В последующих статьях этот подход был расширен для трансформаций, которые обходят пару структур данных одновременно~\cite{SYB1}, а также поддержкой расширения уже сущетвующих транформаций новыми случаями~\cite{SYB2}. Позднее, данных полход был реализован в других языках, включая \textsc{OCaml}~\cite{SYBOCaml,Staged}. В отличие от нашего случая, SYB позволяет применять трансформации к конкретным типам целиком, а не отдельным кострукторам. К тому же, вид получающихся трансформаций выглядит достаточно ограниченным. Также, потенциально, функции трансформации в SYB-стиле могут сломать барьер инкапсуляции, так как могут обнаруживать вхождения значений нужно типа в структуре данных \emph{произвольного} типа. Таким образом, поведение зависит от особенностей внутренней реализации структуры данных, даже от тех, что были скрыты при инкапсуляции. Это может привести, во-первых, к возможности нежелаемой обратной разработки (путём применения различных, чувствительных к типу, преобразований и анализа результатов) и, во-вторых, к ненадежности интерфейсов~--- после модификации структуры данных реализация обобщенной функции для \emph{старой} версии всё ещё может быть применена без статических или динамических ошибок, но с неправильным (или нежелательным) результатом.

Сущетсвует определенное сходство между нашим подходом и \emph{алгебрами объектов}~\cite{ObjectAlgebras}. Алгебры объектов были предложены как решение проблемы выражения (expression problem) в распространенных объектно-ориентированных языках  (\textsc{Java}, \textsc{C++}, \textsc{C\#}), которые не требуют продвинутых особенностей системы типов кроме наследования и шаблонов. В оригинальном представлении алгебры объектов были преподнесены как шаблон проектирования и реализации; в последующих работах изначальная идея была улучшена различными способами~\cite{ObjectAlgebrasAttribute,ObjectAlgebrasSYB}. При использовании алгебр объектов преобразуемая структура данных также кодируется с использованием идеи ``методы и варианты (конструкты) один к одному'', которая предоставляет расширяемость в обоих направлениях, а также ретроактивную реализацию. Однако, будучи  разработанной для совершенно другого языкового окружения, решение с использование алгебр объектов существенно отличается от нашего. Во-первых, с использованием алгебр объектов ``форма'' структуры данных должна быть представлена в виде обощенной функции, которая принимает конкретный экземпляр алгебры объектов как параметер (кодирование Чёрча для типов~\cite{Hinze}). Применяя данную функцию к различным реализациям алгебры объектов можно получать различные преобразования (например, распечатывание). Чтобы инстанциировать саму структуру данных нужно предоставить особый экземляр алгебры объектов~---~\emph{фабрику}. Однако, после инстанциации структура данных больше не может быть трансформирована обобщенным образом. Следовательно, алгебры объектов заставляют пользователя переключиться на представление данных с помощью функций, которое можеть быть, а может не быть удобно в зависимости от обстоятельств.  Наш же подход недеструктивно добавляет новую функциональность к уже знакомому миру алгебраических типов данных, сопоставления с образцом и рекурсивных функций. Обобщенные реализации преобразований полностью отделены от представления данных и пользователи могут свободно преобразовывать их стурктуры данных привычным способом  без потери возможности объявлять (и расширять) обощенные функции. Другой особенностью OCaml, в отличии от распространенных языков объектно-ориентированного программирования, является то, что для написания расширяемого кода в основном используются полиморфные вариантные типы, а не классы. Поддержка полиморфных вариантных типов для написания расширяемых типов данных требует нового подхода.


Итого, среди уже сущетсвующих библиотек для обощенного программирования для \textsc{OCaml} мы можем называть две, которые напопинают нашу:: \cd{ppx\_deriving}/\cd{ppx\_traverse} (часть \cd{ppxlib}~\cite{PPXLib}) и \textsc{Visitors}~\cite{Visitors}.

\cd{ppx\_deriving} является наипростейшим подходом: объявления типов данных отображаются один к одному в рекурсивные функции, представляющие конкретный вид преобразования. Это наиболее эффективная реализация (функции вызываются напрямую, без позднего связывания), но нерасширяемая. Если пользователю понадобится слегка модифицировать сгенерированную функцию, то он должен будет полность скопировать реализацию функции и изменить её. Количество работы по программированию нового преобразования может сущетсвенно увеличиться, если тип данных будет видоизменяться во время цикла разработки.

In \cd{ppx\_traverse} extensible transformations are represented as objects; unlike our case, method-per-type approach is used. In addition 
\cd{ppx\_traverse} does not make use of inherited attributes, thus some transformations like equality or comparison are not representable.

\textsc{Visitors}, on the other hand, explores a similar to ours object-oriented approach, in which many decisions, rejected by us, were taken (and vice versa). Here
we summarise the main differences:

\begin{itemize}
   \item \textsc{Visitors} is excessively object-oriented~--- in order to use it one needs to instantiate some object and call proper method. In our case as long as
     only predefined features are required one can use a more native combinatorial interface.
     
   \item \textsc{Visitors} implements a number of useful transformations in an \emph{ad-hoc} manner; in our case all transformations are instances of the
     same generic scheme. It is possible to combine different transformations via inheritance as long as the types of underlying scheme unify. We also argue, that
     in our framework the implementation of user-defined plugins is much easier.
     
   \item Following SYB, \textsc{Visitors} takes a type-discriminating route: for each type of interest (including the built-in ones) there is a dedicated
     transformation method in each object, representing a transformation. While this solution indeed adds some flexibility, we firmly oppose it, since it
     breaks the encapsulation: inspecting the methods of a transformation (which cannot be hidden in a module signature) one can retrieve some
     information about the implementation of encapsulated types. Even worse, the data structures of abstract types can be manipulated in an unprescribed
     manner using the public type-transforming interface.

   \item In our case the type parameters for transformation classes have to be specified by an end user. With \textsc{Visitors} this burden is offloaded to the
     compiler with the aid of some neat trick. However, this trick makes it impossible to use \textsc{Visitors} syntax extension in module signatures. There is no
     such problem in our case~--- our framework can be equally used in both implementation and interface files.

   \item \textsc{Visitors} in its current state\footnote{The latest available version is 20180513} does not support polymorphic variants.
   
   \item \textsc{GT} supports arbitrary type constructor applications but \textsc{Visitors} in its current state doesn't (both in monomorphic and polymorphic mode).
     For instance, the following example doesn't compile:
     
   \begin{lstlisting}
      type ('a,'b) alist = Nil | Cons of 'a * 'b
      [@@deriving visitors { variety = "map"; polymorphic = true }]

      type 'a list = ('a, 'a list) alist
      [@@deriving visitors { variety = "map"; polymorphic = false }]
   \end{lstlisting}
   
   Moreover, adding an extra construct doesn't solve the problem:
   
    \begin{lstlisting}
       type 'a list = L of ('a, 'a list) alist [@@unboxed]
       [@@deriving visitors { variety = "map"; polymorphic = false }]
    \end{lstlisting}
    
    There is also an issue with type aliases in polymorphic mode (monomorphic part of \textsc{Visitors} compiles successfully):
    
    \begin{lstlisting}
       type ('a,'b) t = Foo of 'a * 'b (* OK *)
       [@@deriving visitors { variety = "map"; polymorphic = true }]
       
       type 'a t2 = ('a, int) t
       [@@deriving visitors { variety = "map"; name="yyy"; polymorphic = true }]
    \end{lstlisting}
    
    The generated code can be fixed manually by removing explicit polymorphic type annotations from objects' methods, which leads to the code
    very similar to the one generated by \textsc{GT}. From these we can conclude that \textsc{GT} can be seen an a reimplementation of polymorphic
    mode of \textsc{Visitors} where more type declarations compile successfully.
    
\end{itemize}

% \section{Производительность}
\label{sec:performance}

При сравнении преобразований, реализованных в традиционном и расширяемом виде,
следует ожидать от первых большей производительности, так как дополнительный слой абстракции вносит некоторые накладные расходы.

При замерах использовались преобразования, реализованные четырьмя способами.
\begin{enumerate}
\item Нерасширяемые преобразования записанные традиционным способом с помощью рекурсивных функций. От них следует ожидать максимальную производительность.
\item Частично расширяемые преобразования, реализованные c помощью записей в стиле библиотеки \cd{ppx\_deriving\_morphism}. Этот  подход вносит накладные расходы на косвенный вызов преобразований подвыражений.% и неприменим, например, для полиморфных вариантых типов. 
\item Расширяемые преобразования в стиле \visitors{}, т.е. с использованием объектов. Вносит некоторые накладные расходы на косвенный вызов из-за таблицы виртуальных методов.
\item Метод из данной работы, сходный с предыдущим, но где обобщенная функция преобразования (раздел~\ref{transtypes}) реализована отдельно от объекта.
\end{enumerate}

Для замеров были выбраны два вида преобразований: копирование выражения и преобразование в текстовый формат.
Данные преобразования применялись к $\lambda$-выражению, которое состоит из некоторого количества $\lambda$-абстракций,  примененных к тождественной функции. Количество таких абстракций определяет размер выражения.

При замерах производительности были реализованы четыре метода, описанные выше. В таблице \ref{tab:caption} в процентах указана производительность методов относительного самого медленного (больше --- лучше), который обозначается прочерком.

Сделанные измерения показывают, что с добавлением возможностей видоизменения и объединения преобразований итоговая производительность уменьшается из-за накладных расходов. Для первого  преобразования, где измеряется производительность обхода выражения без полезной нагрузки,
эти накладных расходы значительны. Однако для второго вида преобразований обход структуры $\lambda$-выражения занимает малую часть, и поэтому накладные расходы на объявление расширяемых преобразований приемлемы.


\begin{table*}[t]
  \centering
  \begin{tabular}{l ccccc}
    \toprule
    \multirow{2}{*}{Вид преобразования}& \multirow{2}{*}{Размер} & \multicolumn{4}{c}{Метод реализации и улучшение (\%)} \\\cline{3-6}
     & & \GT & \visitors & \PPXMorphism  & Default \\\hline
    \multirow{5}{*}{Копирование}  
      & 300  & --  & 149 & 220  & 311 \\
      & 500  & --  & 146 & 218  & 246 \\
      & 700  & -- & 143 & 212  & 244 \\
      & 900  & --  & 141 & 210  & 243 \\
      & 1000 & --  & 140 & 205  & 233 \\\hline
    \multirow{5}{*}{Форматирование}  
      & 300  & 0  & -- & 1  & 3 \\
      & 500  & 0  & -- & 1  & 4 \\
      & 700  & 1  & -- & 1  & 4 \\
      & 900  & -- & 1  & 2  & 3 \\
      & 1000 & -- & 0  & 1  & 3 \\ 
    \bottomrule
  \end{tabular}
\caption{Caption below table.}
\label{tab:caption}
\end{table*}

\section{Заключение}
\label{sec:futurework}

В данной работе представлено подход на основе обобщенного программирования, который кодирует преобразования значений типов данных с помощью объектов, что позволяет видоизменять построенные преобразования, не описывая их заново.

Существует несколько возможны направлений для дальнейшего развития проекта. Во-первых, можно снижать накладные раходы на реализацию расширяемых преобразований 
%в данной работе мы не касались вопросов производительности. Мы представляем преобразования в очень обобщенном виде, с несколькими слоями косвенности. Очевидно, что преобразования, реализованные с помощью нашей библиотеки будут работать медленнее, чем написанные вручную. Мы предполагаем, что 
%производительность может быть улучшено 
с помощью, так называемого, staging~\cite{Staged} или, возможно, с помощью оптимизаций, специфичных для объектов.
%  \textcolor{blue}{Как-то сказать, что тормоза не важны и более-менее приемлемы. Может сделать раздел про производительность}

Другим важным направлением является поддержка большего разнообразия объявлений типов, а именно GADT и нерегулярных типов. Хотя уже сделаны некоторые наработки, получившиеся решение делает интерфейс всей библиотеки чересчур сложным даже для простых случаев.

Наконец, структура с информацией о типе, которую мы генерируем, может быть использована, чтобы сымитировать \emph{ad-hoc} полиморфизм, так как они содержит реализацию функций, индексированных типами. Это в сумме с недавно предложенными расширениями~\cite{ModularImplicits} может открыть интересные перспективы.



\nocite{*}
\bibliographystyle{eptcs}
\bibliography{vkr}
\end{document}
