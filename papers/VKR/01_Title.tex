% Год, город, название университета и факультета предопределены,
% но можно и поменять.
% Если англоязычная титульная страница не нужна, то ее можно просто удалить.
\filltitle{ru}{
    chair              = {},
    title              = {Обобщённое программирование с комбинаторами и объектами},
    % Здесь указывается тип работы. Возможные значения:
    %   coursework - Курсовая работа
    %   diploma - Отчёт по преддипломной практике
    %   master - Диплом магистр%а
    %   bachelor - Диплом бакалавра
    type               = {vkr},
    position           = {},
%     group              = 371,
    author             = {Косарев Дмитрий Сергеевич},
    supervisorPosition = {позиция},
    supervisor         = {научник},
    reviewerPosition   = {позиция},
    reviewer           = {рецензент},
    % chairHeadPosition  = {д.\,ф.-м.\,н., профессор},
    % chairHead          = {Хунта К.\,Х.},
%   university         = {Санкт-Петербургский Государственный Университет},
%   faculty            = {Математико-механический факультет},
%   city               = {Санкт-Петербург},
%   year               = {2013}
}
% \filltitle{en}{
%     chair              = {Department of Software Engineering},
%     title              = {Compositional precise Horn-based verification of heap-manipulating programs},
%     author             = {Yurii Kostyukov},
%     supervisorPosition = {Senior lecturer},
%     supervisor         = {Dmitry Mordvinov},
%     reviewerPosition   = {Software Developer at IntelliJ Labs Co. Ltd.},
%     reviewer           = {Dmitry Kosarev},
% }
\maketitle
