\documentclass[a4paper,twoside,11pt]{article}
\usepackage[utf8]{inputenc}
\usepackage[english,russian]{babel}
\usepackage{fancyhdr}
\usepackage{newprog1e}
\usepackage{amsfonts,amsmath,amssymb,amsthm}
\usepackage{graphicx}
\usepackage{hyperref}
\usepackage{csquotes}
\usepackage{comment}
\usepackage[final]{listings}
\usepackage{cmap}   % for copy-pastable text

\begin{comment}
\usepackage{tabularx}
\newcolumntype{Y}{>{\centering\arraybackslash}X}
\newcolumntype{L}[1]{>{\raggedright\let\newline\\\arraybackslash\hspace{0pt}}m{#1}}
\newcolumntype{C}[1]{>{\centering\let\newline\\\arraybackslash\hspace{0pt}}m{#1}}
\newcolumntype{R}[1]{>{\raggedleft\let\newline\\\arraybackslash\hspace{0pt}}m{#1}}
\end{comment}


\newcommand{\ocaml}{\textsc{OCaml}}
\newcommand{\haskell}{\textsc{Haskell}}
\newcommand{\visitors}{\textsc{Visitors}}
\newcommand{\ocanren}{\textsc{OCanren}}
\newcommand{\OCanren}{\ocanren}
\newcommand{\scheme}{\textsc{Scheme}}
\newcommand{\Scala}{\textsc{Scala}}
\newcommand{\GT}{\textsc{GT}}
\newcommand{\PPX}{\textsc{PPX}}
\newcommand{\camlpfive}{\textsc{Camlp5}}


\lstdefinelanguage{ocaml}{
keywords={@type, function, fun, let, in, match, with, when, class, type,
nonrec, object, method, of, rec, repeat, until, while, not, do, done, as, val, inherit, and,
new, module, sig, deriving, datatype, struct, if, then, else, open, private, virtual, include, success, failure,
lazy, assert, true, false, end},
sensitive=true,
commentstyle=\small\itshape\ttfamily,
keywordstyle=\ttfamily\bfseries, %\underbar,
identifierstyle=\ttfamily,
basewidth={0.5em,0.5em},
columns=fixed,
fontadjust=true,
literate={->}{{$\to$}}3 {===}{{$\equiv$}}1 {=/=}{{$\not\equiv$}}1 {|>}{{$\triangleright$}}3 {\\/}{{$\vee$}}2 {/\\}{{$\wedge$}}2 {>=}{{$\ge$}}1 {<=}{{$\le$}} 1,
morecomment=[s]{(*}{*)}
}

\lstset{
mathescape=true,
%basicstyle=\small,
identifierstyle=\ttfamily,
keywordstyle=\bfseries,
commentstyle=\scriptsize\rmfamily,
basewidth={0.5em,0.5em},
fontadjust=true,
language=ocaml
}
\newcommand{\cd}[1]{\texttt{#1}}

\newcommand{\inbr}[1]{\left<#1\right>}


\tolerance=1000

\newcommand{\defi}{\stackrel{\mathrm{def}}{=}}

\numberwithin{equation}{section}
\newtheorem{theorem}{Теорема}

\newtheorem{definition}{Definition}
\newtheorem{theorem_ru}{Теорема}


\journalnumber{?}
\curyear{2021}
\authorlist{КОСАРЕВ И ДР.}
\titlehead{Обобщённое программирование с комбинаторами и объектами}
\headerdef

\udk{004.421.6}
%92+004.94}
\rubrika{ЯЗЫКИ, КОМПИЛЯТОРЫ И СИСТЕМЫ ПРОГРАММИРОВАНИЯ}
\dateinput{\today}

\rusabstr{
%Годовой отчет о работе научно-исследовательского семинара по компьютерной алгебре.
}

\author{
{\bfseries Д.~С.~Косарев*, Д.~Ю.~Булычев*}
%\\ {\itshape *Вычислительный центр им. А. А. Дородницына Федерального исследовательского центра «Информатика и управление» РАН}
%\\ {\slshape 119333} {\itshape FIXME Москва, ул. Вавилова, д.} {\slshape 40}
\\ {\itshape Санкт-Петербургский государственный университет}
\\ {\slshape 199034} {\itshape  Санкт-Петербург, Университетская наб., д.} {\slshape  7–9}
\\ {\itshape E-mail: Dmitrii.Kosarev@pm.me, dboulytchev@math.spbu.ru}}
\title{ОБОБЩЁННОЕ ПРОГРАММИРОВАНИЕ С КОМБИНАТОРАМИ И ОБЪЕКТАМИ}
%\thanks{~}

\date{}

\begin{document}

\maketitle
\setcounter{page}{3}
%%%%%%%%%%%%%%%%%%%%%%%%%%%%%%%%%%%%%%%%%%%%%%%%%%%%%%%
\begin{comment}


\He{О СЕМИНАРЕ}
%%%%%%%%%%%%%%%%%%%%%%%%%%%%%%%%%%%%%%%%%%%%%%%%%%%%%%%

В семинаре рассматриваются
новые резуль\-таты в области компьютерной алгебры
--- символьные алгоритмы и их реализация, соответствующие вопросы
системного программирования.

В 2011--2012 учебном году семинар собирался
%, как правило,
раз в месяц по третьим
средам на факультете вычислительной математики и кибернетики
% и в НИИ ядерной физики
МГУ,
а в  мае 2012~г. в Дубне, в Объединенном институте ядерных исследований \mbox{(ОИЯИ)}  состоялось
традиционное заседание, организованное совместно с Лабораторией
информационных технологий ОИЯИ.

Web--страница семинара
$${\tt  http://www.ccas.ru/sabramov/seminar/doku.php}$$
содержит информацию о планируемых
и состоявшихся ранее докладах.


%%%%%%%%%%%%%%%%%%%%%%%%%%%%%%%%%%%%%%%%%%%%%%%%%%%%%%%
\He{РЕГУЛЯРНЫЕ СОБРАНИЯ СЕМИНАРА}
\label{resob}
%%%%%%%%%%%%%%%%%%%%%%%%%%%%%%%%%%%%%%%%%%%%%%%%%%%%%%%

С сентября по апрель были прочитаны следующие
доклады\footnote{Перечень докладов, прочитанных в 1995--2011 гг.,
опубликован в \cite{AZ97}--\cite{ABR12}.}.

%%%%%%%%%%%%%%%%%%%%%%%%%%%%%%%%%%%%%%%%
\vspace{0.25cm}
В.П.~Гердт~~(ОИЯИ, ~Дубна;~~\mbox{gerdt@jinr.ru}). {\em
Анализ аппроксимируемости систем уравнений в частных производных конечными разностями.}



  %   В докладе
  Рассматриваются конечно-разностные аппроксимации полиномиально-нелинейных систем дифференциальных уравнений в частных производных, коэффициенты которых являются рациональными функциями от независимых переменных. Общепринятое понятие (слабой) аппроксимации исходных уравнений сопоставляется с понятием сильной аппроксимации, введенным ранее автором для линейных дифференциальных систем и обобщенным недавно на нелинейные системы. Для ортогональных и равномерных сеток предлагается алгоритмическая процедура проверки сильной аппроксимируемости методами компьютерной алгебры, основанная на приведении в инволюцию исходной дифференциальной системы и построении разностного стандартного базиса для рассматриваемой аппроксимации. В качестве примеров изучаются конечно-разностные аппроксимации некоторых переопределенных линейных дифференциальных систем и две %различные
  аппроксимации для двумерных нелинейных уравнений Навье-Стокса.
  %Предлагаемая в докладе алгоритмическая
  Предложенная
  процедура устанавливает, что одна из этих двух аппроксимаций является сильной, а другая нет.
 %%%%%%%%%%%%%%%%%%%%%%%%%%%%%%%%%%%%%%%%



\vspace{0.25cm}
В.В.~Корняк~(ОИЯИ, Дубна; \mbox{kornyak@jinr.ru}). {\em
Вычислительная теория групп и квантовая физика.}

Вычислительная теория групп
%-- это подраздел символьной алгебры. Она
изучает конечные (или конечно-представленные) группы конструктивными алгоритмическими методами.
%Имея в виду физические приложения, мы всегда можем предполагать конечность всех элементов описания:
%введение континуума или других бесконечностей в физику приводит только к техническим усложнениям без
%какой-либо необходимости в них для описания эмпирических данных. Более того, имеются веские экспериментальные
%свидетельства (главным образом из нейтринной физики) того, что конечные группы относительно небольших порядков
%лежат в основе некоторых фундаментальных физических процессов. В более общем контексте
%Конечные группы и их представления над конструктивными числовыми системами ведут к
%очень простому и естественному
%переформулированию квантовой механики.
%Мы кратко обсуждаем

Предлагается новая -- ``конечная'' формулировка квантовой механики,
основанная на конечных группах и их представлениях над конструктивными
числовыми системами.
Обсуждаются
алгоритмы, необходимые для построения ``конечной'' квантовой механики.



%%%%%%%%%%%%%%%%%%%%%%%%%%%%%%%%%%%%%%%%%%%%%%%%%%%%%

\vspace{0.25cm}
    А.А.~Михалев~(Мех-мат МГУ, Москва; aamikhalev@mail.ru). {\em
Примитивные элементы свободных алгебр.}

Многообразие линейных алгебр над полем называется шрайеровым, если любая подалгебра свободной алгебры этого многообразия является свободной. Многообразия всех алгебр,
%многообразие всех
коммутативных алгебр,
%многообразие всех
антикоммутативных алгебр,
%многообразие всех
алгебр Ли,
%многообразие всех
супералгебр Ли,
%многообразия всех
$p-$алгебр Ли и
%всех
$p-$супералгебр Ли являются основными типами шрайеровых многообразий алгебр.

Пусть $A(X)$~-- свободная алгебра шрайерового многообразия алгебр с множеством $X$ свободных образующих. Система элементов $u_1,…, u_n$ алгебры $A(X)$ называется примитивной, если существует множество $Y$ свободных образующих алгебры $A(X)$, содержащее элементы $u_1,…, u_n$.

Для свободных алгебр основных типов шрайеровых многообразий алгебр построены и реализованы алгоритмы распознавания примитивных систем элементов, а также алгоритмы дополнения примитивных систем элементов до свободных порождающих множеств.

Доклад основан на совместных работах автора с К.~Шампаньер, А.А.~Чеповским, А.В.~Михалевым, И.П.~Шестаковым, У.У.~Умирбаевым, Дж.~Ю, А.А.~Золотых.


%%%%%%%%%%%%%%%%%%%%%%%%%%%%%%%%%%%%%%%%%%%%%%%%%%%%%


\vspace{0.25cm}
    А.В.~Смирнов~(НИИВЦ МГУ, Москва; asmirnov@gmail.com). {\em
Алгоритмы вычисления фейнмановских интегралов}.


Фейнмановские интегралы являются фундаментальными величинами при построении квантово-полевых амплитуд в рамках теории возмущений, в частности, они возникают при вычислениях в рамках Стандартной Модели физики элементарных частиц. На современном уровне исследований в конкретной задаче может требоваться вычисление миллионов интегралов Фейнмана, что, естественно, невозможно без
%разработки и применения алгоритмов на
современных компьютеров. Согласно классическому подходу,
задача вычисления фейн\-мановских интегралов сводится, во-первых, к их
редукции к относительно небольшому числу мастер-интегралов и, во-вторых,
к вычислению последних.
%задача вычисления фейнмановских интегралов распадается на их редукцию к относительно небольшому числу мастер-интегралов и вычисление последних.
%Автором были разработаны
Предлагаются новые
алгоритмы
%, применяющиеся
%как
%при
редукции
%, так
и вычисления фейнмановских интегралов.
%Алгоритм редукции относится к области решения очень больших рассеянных систем линейных уравнений с полиномиальными коэффициентами. Алгоритм вычисления %представляет собой алгебраическое упрощение выражений, выделение особенностей и последующее численное интегрирование.
%Оба алгоритма будут описаны в докладе.


%%%%%%%%%%%%%%%%%%%%%%%%%%%%%%%%%%%%%%%%%%%%%%%%%%%%%

\vspace{0.25cm}
 С.П.~Царев~(Сибирский Федеральный Уни\-вер\-ситет, Красноярск; sptsarev@mail.ru). \emph{О струк\-ту\-ре решетки правых делителей линейного обыкновенного дифференциального оператора.}


В 1996 г. автором был предложен алгоритм полного перечисления всех возможных факторизаций заданного линейного обыкновенного дифференциального оператора (ЛОДО) на неприводимые множители над полем рациональных функций.
%К сожалению, полное описание всех возможных структур, выдаваемых данным алгоритмом, до сих пор неизвестно.
Полное описание всех возможных структур, выдаваемых данным алгоритмом,
до сих пор неизвестно. Приводятся новые результаты исследования
структуры всех возможных факторизаций заданного ЛОДО над произвольным
дифференциальным полем коэффициентов.
%В докладе будут изложены недавние результаты о структуре всех возможных факторизаций заданного ЛОДО
%над произвольным дифференциальным полем коэффициентов.
Удобным алгебраическим инструментом для этого описания является теория модулярных решеток.






%%%%%%%%%%%%%%%%%%%%%%%%%%%%%%%%%%%%%%%%

\vspace{0.25cm}
 А.Б.~Батхин~~(ИПМ~им.~М.В.~Келдыша, Москва; batkhin@gmail.com). {\em
Точные решения шестого уравнения Пенлеве.}

%Уравнения Пенлеве играют важную \mbox{роль} в математической физике и имеют многочисленные применения.
%Для приложений важным является знание точных решений уравнений Пенлеве для определенных значений параметров, входящих в уравнение.
Предлагается метод построения точных решений шестого уравнения Пенлеве (PVI) в виде конечных сумм степенных функций
с рациональными показателями степени. Метод существенно использует алгоритмы степенной геометрии для построения
степенных разложений решений обыкновенного дифференциального уравнения и алгоритмы компьютерной алгебры.

%%%%%%%%%%%%%%%%%%%%%%%%%%%%%%%%%%%%%%%%
\vspace{0.25cm}

Д.С.~Кулябов, А.В.~Королькова, М.Н.~Геворкян, Л.А.~Севастьянов (ФМиЕН РУДН, Москва; yamadharma@gmail.com, avkorolkova@gmail.com, mngevorkyan@sci.pfu.edu.ru,
 sevast@sci.pfu.edu.ru). {\em
Применение симплектических интеграторов для задачи распространения электромагнитных волн.}

%Предлагается
Для численного решения дифференциальных уравнений предлагается использовать методы, сохраняющие структуру решений, а именно,
%группу методов
методы
геометрических интеграторов. Рассматривается метод симплектического интегратора в применении к задачам распространения электромагнитных волн в волноводе. Подробно рассматриваются вспомогательные задачи: запись уравнений Максвелла в криволинейных голономных координатах; получение гамильтониана для уравнений Максвелла.
 %%%%%%%%%%%%%%%%%%%%%%%%%%%%%%%%%%%%%%%%


\vspace{0.25cm}
    С.А.~Абрамов~(ВЦ~РАН,~ВМК~МГУ,~Москва; sergeyabramov@mail.ru), Д.Е.~Хмельнов (ВЦ РАН,  Москва; dennis{\textunderscore}khmelnov@mail.ru). {\em О валюацях мероморфных решений линейных разностных систем произвольного порядка с полиномиальными коэффициентами.}

Рассматривается несколько алгоритмов получения нижних оценок валюаций (например, порядков полюсов) компонент мероморфных решений разностных линейных систем произвольного порядка с полиномиальными коэффициентами. Наряду с алгоритмами, основанными на идеях, в каком-то виде уже применявшихся в компьютерной алгебре при работе с нормальными разностными системами первого порядка, предлагается новый алгоритм, использующий «тропические» вычисления. Показывается, что последний алгоритм, обеспечивая хорошую точность оценок, является при этом и достаточно быстрым.

%%%%%%%%%%%%%%%%%%%%%%%%%%%%%%%%%%%%%%%%


\vspace{0.25cm}
А.М.~Денисов~(ВМК МГУ, Москва;  \\
\noindent den@cs.msu.ru). {\em
Неединственность решения бесконечной системы дифференциальных уравнений в частных производных в пространстве Шварца.}

Рассматривается задача, возникающая в двумерной доплеровской томографии. Требуется определить две функции $u(x,y)$ и $v(x,y)$ из пространства Шварца в ограниченной области, являющиеся решением бесконечной системы уравнений в частных производных. Система обладает некоторыми свойствами симметрии. Решение сформулированной задачи не единственно.
Предлагается для обсуждения
%Было бы интересно обсудить В
вопрос о выделении некоторых специальных классов единственности решения сформулированной задачи.

%%%%%%%%%%%%%%%%%%%%%%%%%%%%%%%%%%%%%%%%

\vspace{0.25cm}
    М.В.~Зинин~(ООО ``Андер Девелопмент'', Москва; mzinin@gmail.com).~{\em
Символьные алгоритмы и программы вычисления булевых базисов Грёбнера.}

%Базисы Грёбнера являются теоретически и практически значимым математическим объектом с обширной областью применения. Это справедливо и в отношении %базисов Грёбнера в целом, и в отношении булевых базисов. Например, базисы Грёбнера оказываются весьма полезны в случаях, когда решение некоторой %задачи связано с исследованием соответствующей полиномиальной системы уравнений.
%В случае булевых базисов (о которых и пойдет речь в докладе ) наиболее распространенными подобными задачами являются задачи криптоанализа и булевой %выполнимости. Еще одним многообещающим применением булевых базисов Грёбнера является моделирование квантовых вычислений на классическом компьютере.
%
%Таким образом, можно утверждать, что в настоящее время существует потребность в быстрых и эффективных алгоритмах для вычисления булевых базисов %Грёбнера и их реализациях. В докладе
Кратко описываются наиболее распространенные алгоритмы вычисления базисов Грёбнера и обсуждаются аспекты применений этих алгоритмов в задаче построения булевых базисов Грёбнера. Предлагается вариант инволютивного алгоритма для эффективного решения этой задачи.
%Далее в докладе подробно п
Представлена реализация
%вышеупомянутого
алгоритма
%. Реализация существует в виде отдельной утилиты и
в виде пакетов для открытых систем компьютерной алгебры REDUCE и Macaulay2.
%Завершают доклад результаты сравнения быстродействия представленной реализации и с другими системами и пакетами, позволяющими вычислять булевы базисы %Грёбнера.
%%%%%%%%%%%%%%%%%%%%%%%%%%%%%%%%%%%%%%%%



%%%%%%%%%%%%%%%%%%%%%%%%%%%%%%%%%%%%%%%%
\vspace{0.25cm}
\He{ДВУХДНЕВНАЯ КОНФЕРЕНЦИЯ В
ОБЪЕДИНЕННОМ ИНСТИТУТЕ ЯДЕРНЫХ
ИССЛЕДОВАНИЙ (ДУБНА)}
%%%%%%%%%%%%%%%%%%%%%%%%%%%%%%%%%%%%%%%%

По установившейся традиции в  мае 2012 г. в Дубне прошло
совместное
заседание семинаров ``Компьютерная алгебра'' факультета ВМК МГУ и ВЦ РАН и
семинара  Лаборатории информационных
технологий ОИЯИ.  По существу, это была двухдневная конференция по
компьютерной алгебре и ее приложениям\footnote{{\em Примечание С.А. Абрамова и А.А. Боголюбской.}
Эта конференция была посвящена восьмидесятилетию В.А.~Ростовцева, см. поздравительную статью в N$^o$ 3 за 2012 г.}.

Вниманию участников были предложены следующие
выступления.

%%%%%%%%%%%%%%%%%%%%%%%%%%%%%%%%%%%%%%%%
%%%%%%%%%%%%%%%%%%%%%%%%%%%%%%%%%%%%%%%%


\vspace{0.25cm}

В.П.~Иванников (ИСП РАН, Москва;

\noindent ivan@ispras.ru). {\em О верификации программ.

}

Рассматриваются вопросы верификации
больших программных комплексов, таких, например, как операционные
системы.

%%%%%%%%%%%%%%%%%%%%%%%%%%%%%%%%%%%%%%%%
%%%%%%%%%%%%%%%%%%%%%%%%%%%%%%%%%%%%%%%%


\vspace{0.25cm}

С.А.~Абрамов~(ВЦ РАН, ВМК МГУ,~Москва; sergeyabramov@mail.ru), М.~Пет\-ков\-шек (Уни\-вер\-си\-тет Любляны, Словения;

\noindent Marko.Petkovsek@fmf.uni-lj.si). \emph{О полиномиальных решениях линейных уравнений с частными производными и разностями.}


Вопрос о том, имеет ли  данное линейное уравнение с частными производными или разностями  с полиномиальными коэффициентами
ненулевое полиномиальное решение,
в общем случае неразрешим алгоритмически.
Но дифференциальное или разностное уравнение
$L(y)=0$, $y = y(x_1,\ldots,x_m)$, $m>1$, с постоянными коэффициентами
имеет ненулевое полиномиальное решение если и только если коэффициент уравнения при $y$ равен нулю,
причем в последнем случае уравнение имеет полиномиальное решение любой степени.


%%%%%%%%%%%%%%%%%%%%%%%%%%%%%%%%%%%%%%%%

\vspace{0.25cm}
	
 С.Ф.~Адлай (ВЦ~РАН, Москва;

 \noindent SemjonAdlaj@gmail.com).  {\em Высокоэффективная ариф\-метика эллиптических кривых. }

%Современное интенсивное исследование эллиптических кривых явилось естественным продолжением исследований эллиптических функций. Высокоэффективные
Алгоритмы деления точек эллиптической кривой позволят
%высокоэффективно и высокоточно
с высокой точностью вычислять (неполные) эллиптические интегралы, в изобилии возникающие
%(в частности)
среди решений
%основополагающих
задач теоретической механики.
%В данной работе будет
Указывается
%указан
подход, позволивший получить новые алгоритмы быстрых и точных вычислений для
%таких основополагающих
задач,
%порой тогда, когда
в которых
традиционные методы вычисления не позволяют добиваться
%желанной
желаемой
точности за
%какое-либо
разумное время.

%%%%%%%%%%%%%%%%%%%%%%%%%%%%%%%%%%%%%%%%
 \vspace{0.25cm}
	
В.В.~Корняк~(ОИЯИ, Дубна; kornyak@jinr.ru). {\em Квантовая механика и билинейные перестановочные инварианты конечных груп.}

Любая квантово-механическая задача может быть сформулирована в инвариантном под\-пространстве
перестановочного представления некоторой группы (без потери физического содержания достаточно
рассматривать конечные группы). Скалярные произведения в ин\-ва\-ри\-антных подпространствах,
необходимые для формулировки правила Борна,
% --- основного постулата квантовой механики, связывающего математическое описание с наблюдением)
пред\-ставляют собой линейные комбинации некоторого набора независимых билинейных
инвариантных форм перестановочного представления.
%Полный набор таких форм для любой группы перестановок можно легко вычислить с помощью простого алгоритма.
Предлагается алгоритм вычисления полного набора таких форм для любой группы перестановок.

%%%%%%%%%%%%%%%%%%%%%%%%%%%%%%%%%%%%%%%%

\vspace{0.25cm}
	
 А.Н.~Воропаев~(Петрозаводский ГУ;

 \noindent voropaev@psu.karelia.ru).~{\em Подсчёт $k$-угольников в конечных проективных плоскостях по явным формулам для определения количества циклов в графах.}

На примере графов конечных проективных плоскостей демонстрируется техника символьных преобразований
явных выражений для подсчёта циклов фиксированной длины. Циклы длины $2\,k$ в графе плоскости
соответствуют $k$-угольникам в плоскости. Ранее были известны многочлены, представляющие зависимость
чисел $k$-угольников в плоскости от её порядка, при $k = 3, 4, 5, 6$.
%Ожидалось, что неизоморфные
%плоскости одного и того же порядка могут отличаться в количестве десятиугольников.

Благодаря учёту того, что графы конечных проективных плоскостей двудольны и не содержат циклов длины
4, при выводе явных формул для подсчёта циклов удалось продвинуться до значения длины цикла 20.
Путём аналитических преобразований из выведенных явных формул были получены многочлены, выражающие
зависимость количества $k$-угольников в плоскости от её порядка при $k = 3, 4, \ldots, 10$.


%%%%%%%%%%%%%%%%%%%%%%%%%%%%%%%%%%%%%%%%


\vspace{0.25cm}

А.Д.~Брюно~(ИПМ~им.~М.В.~Келдыша,~Мос\-ква;~abruno@keldysh.ru),~В.Ф.~Еднерал~(НИИЯФ \linebreak МГУ, Москва;~~edneral@theory. sinp.msu.ru). \emph{Нормальные формы уравнений Эйлера-Пуассона.}


%  В статье [Брюно~A.D.: Теория нормальных форм уравнений Эйлера--Пуассона. Препринт № 100 ИПМ РАН, 27 стр. (2005)\\
%http://dl.dropbox.com/u/59058738/Preprint100.pdf] рассмотрен однопараметрический  случай уравнений Эйлера--Пуассона,
%описывающих движение тяжелого твердого тела с закрепленной точкой. Для изучения были выбраны два семейства однопараметрических
%решений, лежащих вблизи неподвижных точек системы. Эти семейства соответствуют резонансу $(0, 0, \lambda, -\lambda,
%2\lambda,-2\lambda)$  собственных значений матрицы линейной части.  В процитированной выше статье была  предложена гипотеза об отсутствии
% дополнительных первых интегралов вблизи этих семейств, за исключением  классических случаев глобальной
% интегрируемости. В настоящем сообщении эта гипотеза проверяется путем вычисления  и анализа коэффициентов нормальной формы. Все вычисления были %произведены  при помощи пакета  [Edneral,~V.F.: On Algorithm of the Normal Form Building.
% Proceedings of CASC 2007, ed. by Ganzha et al., LNCS 4770, 134-142 (2007)],   который был разработан для системы MATHEMATICA.


Обсуждаются итоги проверки одной гипотезы о первых интегралах уравнений Эйлера-Пуассона
%. Проверка выполнялась
с помощью программного пакета
построения нормальных форм.


%%%%%%%%%%%%%%%%%%%%%%%%%%%%%%%%%%%%%%%%

\vspace{0.25cm}

Д. Штефанеску (Бухарестский университет, Румыния; stef@rms.unibuc.ro). 	
%D.~Stef\u anescu~(University of Bucharest, Romania; stef@rms.unibuc.ro) {\em Optimization of bounds for polynomial roots.}
\emph{ Оптимизация гра\-ниц корней полиномов.}

Исходя из оценки Лагранжа, одной из лучших известных оценок положительных корней полиномов одной
переменной, выводится оценка абсолютных величин корней полиномов с комплексными коэффициентами. Обсуждаются некоторые
новые
оценки корней полиномов с вещественными коэффициентами.

%We consider the bound $R+\rho$ of Lagrange, which proved to be one
%of the best for estimating positive roots of univariate polynomials.
%We refine to a bound $S(R,\rho) < R+\rho$ and deduce efficient bounds
%for the absolute values of poynomials with complex coefficients.
%We also discuss the absolute positiveness of some bounds for roots
%of polynomials with real coefficients.

%%%%%%%%%%%%%%%%%%%%%%%%%%%%%%%%%%%%%%%%

\vspace{0.25cm}
	
 С.И.~Сердюкова, Ю.М.~Шукринов~(ОИЯИ,\linebreak Дубна; sis@jinr.ru, shukrinv@theor.jinr.ru).  \emph{Оп\-ре\-де\-ление критической точки ВАХ системы джозефсоновских
переходов. Периодические и непереодические с $\gamma=0$ граничные
условия.}

%В рассматриваемых случаях вычисление ВАХ для систем $n$
%джозефсоновских переходов сводится к решению одного и того же
%единственного уравнения $$\ddot \eta(t)=-\beta \dot \eta(t)
%+I-sin(\eta(t)).$$ Решая это уравнение на интервале $[0,T_{max}]$
%при различных $I$, получаем вольтамперную характеристику $V(I)$ в
%виде петли Гистерезиса. Сначала решается задача Коши с нулевыми
%начальными данными $\eta(I_0,0)=\dot \eta(I_0,0)=0$. Для каждого
%следующего $I=I_{k+1}$ найденные $\eta(I_k,T_{max})=\dot
%\eta(I_k,T_{max})$ используются как начальные данные. На обратной
%ветви петли Гистерезиса значение $V(I)$ быстро спадает к нулю в
%окрестности критической точки $I_b$. Был разработан эффективный
%численно-аналитический метод вычисления ВАХ. Нам удалось вывести
%уравнение, определяющее приближенное значение $I_b, \tilde I_b$.
%Это решает проблему выбора точки перехода от аналитического
%расчета к численному: $I=2\tilde I_b$. Предлагаемый
%численно-аналитический метод показал хороший результат при
%вычислении ВАХ системы 9 джозефсоновских переходов. Вычисления
%были выполнены с использованием системы REDUCE 3.8.

Решается проблема выбора точки перехода от аналитического
расчета к численному в задаче нахождения вольт-амперной характеристики
системы 9 джозефсоновских переходов. Для вычислений
%были выполнены с
используется система REDUCE 3.8.

%%%%%%%%%%%%%%%%%%%%%%%%%%%%%%%%%%%%%%%%

\vspace{0.25cm}
 	
С.В.~Парамонов~(ВМК МГУ, Москва;

\noindent s.v.paramonov@yandex.ru). {\em О дробно-ра\-цио\-наль\-ных решениях линейных однородных урав\-не\-ний с частными производными или разностями.}

Статья по теме доклада публикуется в этом номере журнала.

%Доказывается, что в общем случае  задача распознавания существования ненулевых дробно-рациональных решений для
%имеющего полиномиальные коэффициенты линейного  однородного уравнения с частными производными или разностями  алгоритмически неразрешима.

%%%%%%%%%%%%%%%%%%%%%%%%%%%%%%%%%%%%%%%%

\vspace{0.25cm}
	
С.А.~Гутник~(МФТИ, МГИМО, Москва; s.gutnik@inno.mgimo.ru).~{\em Символьно-численные методы исследования динамики осесимметричного
спутника под действием гравитационного и гиростатического моментов.}


Исследуется динамика вращательного движения осесимметричного спутника на круговой орбите
под действием гравитационного и гиростатического моментов. С использованием систем
 компьютерной  алгебры Mathematica  и Maple  получены уравнения стационарных движений
 спутника, численно определены все положения равновесия спутника в орбитальной системе
 координат, получены достаточные условия устойчивости положений равновесия. Проведено
 исследование устойчивости полученных положений равновесия.

%%%%%%%%%%%%%%%%%%%%%%%%%%%%%%%%%%%%%%%%



\vspace{0.25cm}
	
 Д.С.~Кулябов, А.В.~Королькова~~(\mbox{ФМиЕН} РУДН, Москва; yamadharma@gmail.com, avkorolkova@gmail.com).
{\em Тензорные вычисления в системах компьютерной алгебры.}

Рассматриваются возможности тензорных вычислений в свободно распространяемых пакетахкомпьютерной алгебры. Выделяется несколько
типов тензорных расчётов, обсуждается специфика и область применения каждого типа.

В качестве примеров реализации тензорных вычислений приводятся специализированная система Cadabra и универсальная система Maxima
как типичные представители своих категорий. Особенности их применения демонстрируются на примере уравнений Максвелла в криволинейных
координатах.

Статья по теме доклада готовится к публикации в №~3, 2013 г.



%Рассматриваются возможности тензорных вычислений в свободно распространяемых пакетах
%компьютерной алгебры. Выделяется несколько типов тензорных расчётов,
%обсуждается специфика и область применения каждого типа.
%
%В качестве примеров реализации тензорных вычислений приводятся
%специализированная система Cadabra и универсальная система Maxima как
%типичные представители своих категорий.  Особенности их применения
%демонстрируются на примере уравнений Максвелла в криволинейных
%координатах.


%%%%%%%%%%%%%%%%%%%%%%%%%%%%%%%%%%%%%%%%



\vspace{0.25cm}
	
 М.Н.~Геворкян~(ФМиЕН РУДН, Москва; mngevorkyan@sci.pfu.edu.ru). \emph{Особенности сис\-те\-мы компьютерной алгебры Sage.}

%Обзор Sage ---
Предлагается обзор возможностей
свободной системы компьютерной алгебры Sage, разрабатываемой на языке Python с 2005 года (www.sagemath.org), ее
взаимодействие с другими системами компьютерной алгебры (GAP, Maxima, Singular) и пакетами
%для
численных расчетов (GSL, SciPy, NumPy, ATLAS).
Веб-интерфейс Sage Notebook обеспечивает возможность удаленной работы; в качестве системы визуализации используется библиотека Matplotlib (которая, в частности, позволяет использовать \LaTeX код при создании графических изображений).
% Веб-интерфейс Sage Notebook дает возможность удаленной работы; в качестве системы визуализации используется библиотека Matplotlib (которая, в частности, дает возможность использовать \LaTeX код при создании графических изображений).
Отдельно
обсуждается
%стоит отметить
%взаимодействие \LaTeX'а и Sage (
использование команд Sage непосредственно в \LaTeX-документах.

%%%%%%%%%%%%%%%%%%%%%%%%%%%%%%%%%%%%%%%%

\vspace{0.25cm}
	
  Н.Н.~Васильев~(ПОМИ~РАН,~\mbox{Санкт-Петербург}; vasiliev@pdmi.ras.ru). {\em Универсальный инволютивный базис и бордер-базисы Роббиано.}

%Мы описываем в естественной
Описывается в
геометрических терминах класс инволютивных делений,
допускающий конструкцию универсального инволютивного базиса. Обсуждается
связь этой конструкции универсальных инволютивных базисов с бордер-базисами Роббиано.
%%%%%%%%%%%%%%%%%%%%%

\vspace{0.25cm}
 	
 Ю.А.~Блинков, С.В.~Иванов~~(Саратовский ГУ; BlinkovUA@ info.sgu.ru, evilgraywolf@gmail.com),  Л.И.~Могилевич~(ПФ МГУПС, Саратов; mogilevich@sgu.ru).~{\em Математическое и компьютерное моделирование
нелинейных волн деформаций в оболочке, содержащей
вязкую жидкость.}


Обсуждается использование средств компьютерной алгебры в анализе
%Настоящее исследование, с использованием средств компьютерной алгебры, посвящено анализу
распространения нелинейных волн деформаций в физически нелинейной упругой цилиндрической
оболочке, содержащей вязкую несжимаемую жидкость. Волновые процессы в
упругой цилиндрической оболочке без взаимодействия с жидкостью ранее исследованы
с позиций теории солитонов. Наличие жидкости потребовало разработки новой
математической модели и компьютерного моделирования процессов, происходящих в
рассматриваемой системе.
%%%%%%%%%%%%%%%%%%%%%

\vspace{0.25cm}
В.П. Гердт (ОИЯИ, Дубна; gerdt@jinr,ru), А. Хашеми (Технологический университет, Исфахан, Иран; Amir.Hashemi@cc.iut.ac.ir). 	
\emph{Об ис\-поль\-зовании критериев Бухбергера в ал\-го\-рит\-ме G{\large$^2$}V вычисления базисов Гребнера.}

Статья по теме доклада публикуется в этом номере журнала.

%Как было экспериментально продемонстированно Фожером, его алгоритм F$_5$ является самым
%быстрым среди алгоритмов вычисления базисов Гребнера. Вычислительная эффективность F$_5$ обусловлена не только использованием линейной алгебры но и %применением авторского критерия  F$_5$ для выявления бесполезных нулевых редукций. На конференции ISSAC 2010 Гао, Гуан и Вольны представили G$^2$V, %вариант алгоритма   F$_5$, который является более простым по своей структуре, чем оригинальная версия F$_5$.  Однако инкрементальная структура G$^2$V, %используемая в алгоритме для применения критерия F$_5$, является серьезным препятствием для использования второго критерия Бухбергера. В настоящей %работе представлена модификация алгоритма G$^2$V, позволяющая использовать не только первый, но и второй критерий Бухбергера. Для экспериментального %анализа вычислительного эффекта от предложенной модификации мы реализовали модифицированный алгоритм на языке {\scshape Maple}. Приведено сравнение %G$^2$V и его модификации на ряде тестовых примеров.

%%%%%%%%%%%%%%%%%%%%%

\vspace{0.25cm}
 	
Д.А.~Янович~(ОИЯИ, Дубна; yan@jinr.ru).\linebreak
{\em Параллельное модулярное вычисление базисов Грёбнера и инволютивных базисов.}

Статья по теме доклада публикуется в этом номере журнала.


%Представлен алгоритм и реализация параллельного вычисления инволютивного базиса при помощи
%модулярных вычислений. Рассмотрены проблемы и способы корректного восстановления целочисленного базиса по модулярным образам.
%%%%%%%%%%%%%%%%%%%%%%%%%%%%%%%%%%%%%%%%






\vspace{0.25cm}
	
   П.В.~Фокин, Ю.А.~Блинков~(Саратовский ГУ;\linebreak  fokinpv@gmail.com,~BlinkovUA@info.sgu.ru). \emph{ZDD диаграммы с общим кэшем.}

Предлагается использовать ZDD диаграммы с общим кэшем для представления булевых полиномов при построение инволютивных базисов Грёбнера.
% Оценка использования памяти по сравнению со списочным представлением максимального авторедуцированного множества.
%
%Для представления авторедуцированного множества
%имеет место следующая оценка по памяти для самого худшего
%случая при представление полиномов в виде списка мономов
%
%\begin{itemize}
% \item $\Theta(n^{2d-1})$ для представления полиномов в виде списка мономов с упорядочением совместимым со степенью и лексикографическим;
% \item $\Theta(n^{d})$ для представления полиномов в виде ZDD диаграмм с упорядочением совместимым со степенью;
% \item $\Theta(n^{d-1})$ для представления полиномов в виде ZDD диаграмм с лексикографическим упорядочением.
%\end{itemize}
%
%Комбинированные ZDD диаграммы, как представление полинома в виде множества ZDD диаграмм с общим кэшем.
%При
Такое представление ускоряет
выполнение
%основные арифметические
операций над булевыми полиномами.
%при построении базисов Грёбнера.
%
%Представление
Обсуждаются
программные реализации на Python и C++.

%%%%%%%%%%%%%%%%%%%%%%%%%%%%%%%%%%%%%%%%

\vspace{0.25cm}
	
Е.С.~Шемякова~(ВЦ РАН, Москва;

\noindent shemyakova.katya@gmail.com).~~{\em Доказательство того, что вронскианы Дарбу описывают все возможные преобразования Дарбу порядка два.}

Преобразования Дарбу (ПД) используются для точного решения линейных и нелинейных
уравнений с частными производными.
%УрЧПов.
Вронскианы Дарбу --- это формулы, позволяющие строить ПД по некоторому числу частных решений исходного уравнения.
%УрЧП.
Ранее было доказано, что хорошо известные преобразования Лапласа являются единственными ПД порядка один,
которые нельзя построить с помощью вронскианов Дарбу.
Предлагается доказательство того, что
%Здесь же мы докажем, довольно неожиданно, что
любое ПД порядка два можно построить с помощью вронскианов Дарбу.
%%%%%%%%%%%%%%%%%%%%%%%%%%%%%%%%%%%%%%%%x


\vspace{0.25cm}
	
   Н.Н.~Васильев~(ПОМИ~РАН,~\mbox{Санкт-Петербург}; vasiliev@pdmi.ras.ru), А.Б.~Терентьев~(СПбГТУ, \linebreak  Санкт-Петербург; alterterrific@gmail.com). {\em Моделирование марковских процессов с асимптотически центральной мерой на трехмерных диаграммах Юнга.}

Рассматриваются проблемы
%Доклад посвящен
компьютерного моделирования марковских случайных блужданий на трехмерном графе Юнга.
Вершины этого графа соответствуют трехмерным диаграммам Юнга, а
пути из корневой вершины~-- трехмерным таблицам Юнга.
%В случае двумерного графа Юнга процессы
%с центральной мерой соответствуют характерам бесконечной симметрической группы и
%особую роль среди таких мер играет мера Планшереля. В трехмерном случае имеется
%множество открытых вопросов касающихся процессов, порождающих центральные меры и, в частности,
%практически ничего не неизвестно о  трехмерных аналогах меры Планшереля.
%Мы
Используется интерпретация таблиц Юнга как мономиальных упорядочений.
%, а переходных вероятностей
% соответствующего марковского процесса, как
% вероятностных распределений на образующих трехмерных
% мономиальных идеалов.
Это позволяет
%использовать
привлекать
параметризацию Роббиано допустимых упорядочений
для исследования отклонений переходных вероятностей вдоль разных путей,
соединяющих две фиксированные диаграммы Юнга.


%%%%%%%%%%%%%%%%%%%%%%%%%%%%%%%%%%%%%%%%

\vspace{0.25cm}
	
 И.П.~Юдин~(ОИЯИ, Дубна; yudin@jinr.ru). {\em Аналитический алгоритм решения задач нелинейной динамики заряженных частиц
в тороидальном магнитном поле с использованием метода функций влияния.}

%Предлагаются полученные
%методом функций влияния с использованием матричного формализма
%аналитические алгоритмы решения уравнения нелинейной динамики %заряженных частиц в тороидальном магнитном поле.
Предлагаются
аналитические алгоритмы решения уравнения нелинейной динамики заряженных частиц в тороидальном магнитном поле, полученные
методом функций влияния с использованием матричного формализма.
%Впервые получены
Даются
формулы для аберрационных коэффициентов нелинейной оптики до третьего порядка включительно.
%Предлагаемый алгоритм реализован в виде программы на языке Фортран.
В качестве приложения рассмотрен тороидальный спектрометр для физики высоких энергий.


%%%%%%%%%%%%%%%%%%%%%%%%%%%%%%%%%%%%%%%%

\vspace{0.25cm}
	
 С.Д.~Мешвелиани~(ИПС РАН, Переславль-Залесский; mechvel@botik.ru).~~{\em О проекте интерфейса  DoCon-Haskell ---  Axiom.}

Описывается
%разработка программы ---
интерфейс
%двух
систем
%вычислительной
%алгебры (CA): {\tt
DoCon и Axiom.
%Первая есть библиотека СА,
%написанная на функциональном и ``ленивом'' языке {\tt Haskell}. Вторая
%есть более обширная библиотека CA, написанная на не-функциональном и прямом
%языке {\tt Spad}.
Интерфейс основан на обмене строками, именованных трубках (Unix named pipes)
и т.д.
%, \
%в) особом парсере на обеих сторонах интерфейса, запрограммированном в
%   категорном стиле, \
%г) даваемой языком {\tt Spad} некоторой возможности вычислять области (типы).

%%%%%%%%%%%%%%%%%%%%%%%%%%%%%%%%%%%%%%%%

\vspace{0.25cm}
	
 А.М.~Рапортиренко~(ОИЯИ, Дубна;

 \noindent ram@sunct1.jinr.ru). {\em Локальная версия системы AXIOM.}

%В настоящее время существует три версии системы AXIOM -- fricas, axiom и open-axiom.
%Каждая из них эволюционирует в своем напровлении, и все они ориентированы на
%использование Common LISP в стандарте ANSI.
%
%Автору захотелось иметь
Описывается
версия системы AXIOM, в которой
%максимально приближенной к оригинальной,
%где
присутствуют
%такие ее
компоненты
%как
ALDOR и NAG link.
В настоящее время эта версия
%система
работает с использованием gcl версии Common LISP.

%Ведется работа по оживлению прародителя cmucl -- spice-lisp, который и
%предполается использовать в окончательной версии системы.
%%%%%%%%%%%%%%%%%%%%%%%%%%%%%%%%%%%%%%%%

\vspace{0.25cm}
	
 Н.А.~Малашонок,~М.А.~Рыбаков~(ТГУ им. Державина, Там\-бов; namalaschonok@gmail.com, mixail08101987@mail.ru). {\em Символьно-численное решение систем линейных обыкновенных дифференциальных уравнений с требуемой точностью.}

Приводится алгоритм символьного решения системы линейных обыкновенных дифференциальных уравнений с постоянными коэффициентами, основанный на применении преобразования Лапласа. Возможно получение требуемой точности решения исходной системы. Алгоритм входит в состав библиотеки алгоритмов системы Mathpar.

Статья по теме доклада готовится к публикации в №~3, 2013.
%Приводится алгоритм  символьного  решения  системы линейных обыкновенных дифференциальных уравнений с постоянными
%коэффициентами, основанный на применении преобразования Лапласа.
%%Составной частью алгоритма является
%Возможно
%получение  требуемой точности решения исходной системы.  Алгоритм входит в состав библиотеки алгоритмов системы Mathpar.
%%Приводятся примеры решения систем  дифференциальных уравнений в системе Mathpar.
%%%%%%%%%%%%%%%%%%%%%%%%%%%%%%%%%%%%%%%%


\vspace{0.25cm}
Л.~Хай, А.А.~Гусев, С.И.~Виницкий, О.~Чулуунбаатар, В.П.~Гердт, В.А.~Ростовцев \linebreak (ОИЯИ, Дубна; luonglehai\_tcl@yahoo.com.vn, gooseff@jinr.ru, vinitsky@thsun1.jinr.ru,

\noindent chuka@jinr.ru,  gerdt@jinr.ru, rost@jinr.ru). \emph{Сим\-воль\-но-численный алгоритм для расчёта ридберговских состояний и скоростей распада в сильных магнитных полях.}

%Представлен символьно-численный алгоритм решения
Рассматривается
краевая задача для уравнения Шрёдингера в цилиндрических координатах. Эта задача описывает примесные состояния в квантовых проволоках или в водородоподобных атомах  в сильном однородном магнитном поле. Исходная задача редуцируется с помощью метода Канторовича к граничной задаче для системы обыкновенных дифференциальных уравнений относительно продольных переменных.
%Эффективные потенциалы этих уравнений даются интегралами по поперечной переменной.
%Подынтегральные выражения -- произведения поперечных базисных функций, зависящих от продольной переменной как от параметра,
%и их первых производных по параметру. Для решения задачи при больших магнитных квантовых числах $|m|$
%и анализа её решений,
Представлен символьно-численный алгоритм, реализованный в системе Maple, который позволяет получать аналитические выражения для эффективных потенциалов, собственных функций и собственных значений, элементов матрицы дипольных моментов.
%Эффективность и точность алгоритма и  схемы Канторовича подтверждены,  расчётом собственных значений и собственных функций, дипольных моментов и %скоростей распада низколежащих ридберговских состояний при $|m|\sim 200$ атома водорода в лабораторном однородном магнитном поле  $\gamma \sim 2.35 %10^{-5}$ ($B\sim 6$T).

%%%%%%%%%%%%%%%%%%%%%%%%%%%%%%%%%%%%%%%%

\vspace{0.25cm}

Р.Т.~Файзуллин (ОмГТУ, Омск; frt@omgtu.ru). {\em Сведение задачи ВЫПОЛНИМОСТЬ к основным задачам линейной алгебры.}

%Цель работы заключается в установление аналогии между  проблемой ВЫПОЛНИМОСТЬ и основными проблемами линейной алгебры.
Показывается, что задача ВЫПОЛНИМОСТЬ может быть сведена к системе линейных алгебраических уравнений
с положительной определенной симметрической матрицей.
Построена функция от
%носительно
коэффициентов разложения по собственным векторам этой матрицы,
глобальный минимум  которой отвечает решению задачи 3-ВЫПОЛНИМОСТЬ.
Предлагается эвристический тест
определения значимого числа бит в решении проблемы факторизации.

%%%%%%%%%%%%%%%%%%%%%%%%%%%%%%%%%%%%%%%%

\vspace{0.25cm}

В.С.~Рихвицкий~(ОИЯИ,~Дубна;

\noindent rqvtsk@mail.ru)
{\em Цифровая Вселенная: Петлевая Квантовая Космология Абэя Аштекара, логический аспект.}

%Рассматривается тезис, что обсуждаемая Вселенная существует
%постольку, поскольку существует ее непротиворечивая
%логико-математическая модель.

Приводится пример петлевой квантовой космологической модели типа
Бьянки II в виде базы знаний на языке предикатов; предполагается, что реальность
сущностей в рамках базы знаний заключается в том, что при обработке
запросов к базе знаний ответ доставляет стандартный механизм
логического вывода.
Язык
% описания представленной модели является язык
предикатов дополнен некоторыми предикатами,
%реализующими аналитические вычисления в системе Maple.
использующими аналитические вычисления в системе Maple.


%Следует также отметить известный факт, что для каждой
%непротиворечивой теории в логике предикатов существует ее счетная (в
%смысле мощности множества) модель.
%%%%%%%%%%%%%%%%%%%%%%%%%%%%%%%%%%%%%%%%

%%%%%%%%%%%%%%%%%%%%%%%%%%%%%%%%%%%%%%%%

\vspace{0.25cm}
	
 Г.И.~Малашонок~(ТГУ им. Державина, Тамбов; malaschonok@gmail.com).~~{\em Треугольные разложения матриц в коммутативных областях.}

%Треугольные факторизации матриц лежат в основе многих матричных
%алгоритмов. В линейной алгебре важную роль играет разложение матрицы в
%произведение нижней треугольной и верхней треугольной матриц,
%известное под названием $LU$-разложения. Для матриц общего вида
%вычисляют разложение $A=PLUQ$, где $P$ и $Q$ - матрицы перестановок.
%В теории представлений групп важную роль играет разложением Брюа
%(Bruhat)  $A=VwU$, где обе треугольные матрицы $V$ и $U$ являются
%верхними треугольными,  а $w$ - матрица перестановок.

%Доклад посвящен
Предлагаются
два новых рекурсивных алгоритма вычисления
треугольного разложения в коммутативной области для матрицы общего
вида, имеющих сложность матричного умножения.
Оба алгоритма вычисляют разложение вида $A=LdU$, где $L$ --- нижняя
треугольная матрица, $U$ --- верхняя треугольная матрица, $d$ ---
матрица перестановок, домноженная на обратную диагональную матрицу.
Каждая из матриц, стоящих в правой части, имеет такой же ранг, что и
матрица $A$.  Кроме того, в первом алгоритме одновременно вычисляется
присоединенная матрица к блоку матрицы $A$, имеющему максимальный
ранг.


%%%%%%%%%%%%%%%%%%%%%%%%%%%%%%%%%%%%%%%%

\vspace{0.25cm}
	
Г.И.~Малашонок~(ТГУ им. Державина, Тамбов; malaschonok@gmail.com).~~
{\em Интеллектуальная математическая система Mathpar.}

Дается характеристика веб-сервиса Mathpar (http://mathpar.com), предназначенного  для проведения символьных и численных
вычислений в научных исследованиях, инженерных расчетах и образовании.
\end{comment}


% !TeX encoding = windows-1251
\section{��������}


\begin{comment}

������� ����� (Frederic Brooks) � ����� ��������� ����� �� ���������
��������  "���������� ��������-�����" ("The Mythical Man-Month")~\cite{MMM} ��������������� �������� ���������������� ��������� �������:

\blockquote{"�����������, ������� �����, �������� ����� ��������������� � ������ ������. �� ������ ���� ����� � ������� � �� �������, ����� ����� �����������. ������ ����� ������ ��������, ������������ � ����������, ������� ����� �� �����, ����� ��� �������� ��� ����������� � �������� ��� ���������� ����������� ��������. (��� �� ������� ������, ����� ������������ ���� ���� ��������.)"}

�������������, ���������������� �������� � �������� �� �������������  ��������� � ����������������; ���������� ���������� ��������� ����� ����� �������� � ���������������� ������������
(��� ��� ��������� � ���������� ������������� ��������� � �������). 
\end{comment}

����� �� ������� � ����������� ���������� ������������ ����������� �������� ����������� ������������� �����������, ������� ����� �������� � ������ ��������� ������������ ��������. ���������� ���������� ��������� ����� ����� �������� � ���������������� ������������
(��� ��� ��������� � ���������� ������������� ��������� � �������). 
����� �� �������� ���������������� �������� ��������������� ��������� �������� ������������� \emph{����� ������}. ��� ��������� ��������� �������� ������; ��� ����� � ���� �������, � ��� ������; � ����� � ��������� ������� ��������� ��������� �������� ������. ���� ���������� � ����� ������ ������������ �� ����� ������ ���������, �� ���������� ��������� ����������� ����-�������������� ���� ������� ����� (\emph{������������}) ��� ���� �������� ����� ����� ������ �� ���� (\emph{���������}).

������, � ���������� �������������� �������������� ������, ��� �������, ���� ��������� ��������� 
����� ���� ���������� � ����������� �� ����� ����������. ����������� ��������� �������� ��������� ������������� �� ��������� � ������������, ������ ��� ����������
�� ����� �������������� ���� �� ����� ���������� � ������ ���������� ������ ��������� �������� -- ������ ��������� -- �� ���������. � ������ �������, ��������� ��������������, ������� � ������������ ������ ����� ���� ����������� ``��� � ��������'' �� �������� �������� ����� � ������ ���� ��������������� ��� ������� ����������� ���� �� �����������. ���� �� �������� � ����������� ����� ���������� �������� ���������� ����� ������������� ������� �����, ��� ������� ���������� ������� ����� ���� ���������������. �������� ����� ������� ����� ��������� ���������� (\emph{ad hoc} ������������) � ����� \haskell � ���� ������� 
�����~\cite{TypeClasses} � �������� �����~\cite{TypeFamilies}. ������, �� ������� ���������� ����������� � ��������� �������� ����� � ��������������� �������������� �������� �������� �����, ������ ����� ������������ ``�������'' ���������, �������
�� ����� ���� ���������������. ������ �������� �������� \emph{���������� ����������������}~\cite{DGP} (\emph{datatype-generic programming}), ����� �������� �������� ���������� ������� ��� ���������� ����������� ������ �������� �������
��������������� ������, ��������� ��������� ����������� �����. ��������, ���� ����� ���� ������������ �� ���������� �����~\cite{Hinze,InstantGenerics,GenericOCaml}, ���� ����� ���������� � ����� ����� ���� ������� ��������� �� ����� ����������, ��� 
���������� ������� ��� ����������� ���� ������ ����� ���� ������������� �� �����
���������� �������������~\cite{Yallop,PPXLib}. ��� ������, ��������� ����, ��������� ���� �����: ��� ����� ������ �������� ������� �����, ��� ������ ������������ ��� ����������� ���������������� ���� ����� ����������. ��������, ��������������� ����������� ��������� ����������� �������� ������� ��� ���������� ����� ������ ������������ ��������� � �.�.

%�������� ����� �������� ������ ��� ���������  ``Scrap Your Boilerplate''~\cite{SYB,SYB1,SYB2}, ������� ��������� � ��������� �� �������� ��������� �� �����, ��������������� �������. ������, ��� ���� �������, ��� ��������������� � ������������� SYB ������������, � ���� �� ��������������, ������� ������� ������ �� ����, �� ����� ������ ������������. 

� ���������� �������� ����������������� �������������� ����������� ������ ���� �� ����������. ����� �������, ���� 
��������������� �������������� �� ������ ���������� ������������, �� �� ������ ���� ����������� ��������� ������ ���� ��������������, ���� �������� ��� �������. ����� �������, ��� ���������������� ������� ������������ ��� ��������������� ��������������, ����� ���������� ���������� ��������������.

������ ��������� ����� �������� �������������  �������� ���������� (late binding) ��� ����, ����� ������������ ����� ��������������� �������������� ��� ����� ������������. ����������� ��� ���� ������������� � ��������-��������������� �����������������, ����� ���������� � ��������������.


� ������ ������ �� ������������ ������, ����������� � ���� ���� 
�������� ���������� ��� ������������� ��������� ����������� ��������������, � ����������������� ��������������, ��������� ��� ��������������� ����������������. ��� ������������� ������� ������� ���� ����������� ���������� ��� ����������� ���������������� \textsc{GT}\footnote{\url{https://github.com/kakadu/GT/tree/ppx}} (\emph{Generic Transformers}), � ����� ��������� ���������� �������� ��� ��, ����������� ����������� ������������ ����� ��������������.


%������� ��������� � �������� ���������� � 2014 ����. � ��� ������������ ���� �������� ���������� �� ������ �������� � ����� \ocaml, ��� � �~\visitors. ������, ��������� ������������ ������� � \visitors, ������� �� ������� ��������������,   �� ��������� ��� ������������ � � ��������� �������, � ��������� �  ������� \ocanren~\cite{OCanren}.


\section{План}



\subsection{Метод}

\subsection{Введение в обобщённое программирование}
%Указать кратко, что тут будет описываться.\\
%В конце сказщать про GT, строки кода (?) выложена там-то.
%
%В статье на ML Workshop мы писали тут детали реализации, а именно какой код стоит генерировать по типам данных, чтобы получилось сделать то, что хочется сделать. Как это написать, не скатившись в детали реализации -- мне пока не понятно.

\subsection{Результаты}
%\emph{Пока плохо понимаю, что тут писать}.
%
%Был представлен подход по представлению преобразований с помощью объектов языка \ocaml. В данном представлении конструкторы алгебраических типов данных кодируются один к одному в методы, таким образом программист достаточно легко может изменить преобразование для некоторого конструктора алгебраического типа. Для испытания подхода была спроектирована библиотека обобщённого программирования \GT, в рамках которой реализован типичный набор преобразований.


%\subsection{Background}
%
%Описание предметной области и используемые технологии есть во введении
%
%Результаты предыдущих (собственных) исследований отсутствуют.
%
%\textit{Не знаю что тут писать} (Д.Кознов сказал, что можно обойтись)


\subsection{Related works}
\begin{itemize}
\item  Стандартные подходы про обобщённое программирование, в том числе ppx\_deriving, который все сейчас в \ocaml~используют
\item Обощенное программирование с использование "представлений типов"
\item Scrap Your Boilerplate -- специфический вид обобщенного программирования
\item Алгебры объектов -- что-то подобное тому, что мы хотим, но чисто в ОО-стиле для С\#/Java
\item \cd{ppx\_traverse} делает примерно то же, что и мы, только кодирует типы без  нашей идеи "один конструктор -- один метод"
\item \visitors~главный конкурент, так как он использует ту же идею, но реализованную по-другому. Предлагаю здесь их просто упомянуть и содержательно сравниваться с ними в evaluation.
\end{itemize}

\subsection{Evaluation}
%Здесь можно и RQ указать.\\
%
%Здесь планируется \textbf{пример №1}, чтобы показать, что наше видоизменение трансформаций в принципе работает (ответ на RQ 2)
%
%У нас основной конкурент -- это \visitors, поэтому здесь сравниваемся в основном с ней
%\begin{itemize}
%\item[$\star$] У них чисто ОО подход, а у нас совмещенный, поэтому получающийся в  итоге код выглядит привычнее для функционального программирования (+ к ответу на RQ1)
%
%\item[$\star\star$] У них добавляются некоторые искусственные методы в объекты, что позволяет из-за чего "протекает" абстракция и открываются возможности "сломать" преобразования
%
%\item[$\star\star\star$] В \visitors~два немного отличающихся подхода к созданию преобразований, а одном получается выразить одни преобразования, а в другом другие. Важно то, что в  \visitors~мы не можем выразить то, что нам надо для~\cite{OCanren} (про это можно сделать \textbf{пример №2}).
% У нас  же один подход, при котором получается выразить всё что надо. (+ к ответу на RQ2)
%
%\item[$\star\star\star\star$] \visitors~ не поддерживают полиморфные вариантные типы, мы поддерживаем. Про это будет \textbf{пример №3} на тему \emph{expression problem}~\cite{ExpressionProblem}  на основе~\cite{PolyVar,PolyVarReuse}. Этот пример хочется включить, так как expression problem достаточно известная тема в сообществе и много кто предлагал различные подходы к её решению.
%\end{itemize}
%
%Также сюда можно добавить сравнение производительности нашего подхода с наивным подходом, где не подразумевается возможность последующего видоизменения преобразований (мы буквально это делаем для гранта).
%
%Тут можно два  "стиля" evaluation
%\begin{enumerate}
%\item Примеры и сравнение с
%\item С метрикой можно "отключать" некоторые шаги и смотреть что будет (объем кода, быстродействие, сложность использования)
%\end{enumerate}



\subsection{Discussion}
%Обсуждение результатов+полемика: \emph{пока плохо понимаю, что тут писать}.
%
%\begin{enumerate}
%\item Не смотря на то, что библиотека \visitors~не предоставляет привычный комбинаторный интерфейс, доработка её в этом аспекте не должна вызвать серьёзных проблем.
%\item ``Протекание'' абстракции в библиотеке \visitors~ (а также в подходе SYB~\cite{SYB}) позволяет удобно реализовывать некоторые преобразования, типичные для подхода SYB. При нашем подходе реализация таких преобразований вызовет сложности, однако, мы находим преобразования в стиле SYB чересчур специфическими и мало полезными для наших нужд.
%\item Сейчас реализация \visitors~полагается на использование \emph{явного полиморфизма}\footnote{\url{https://caml.inria.fr/pub/docs/manual-ocaml/polymorphism.html\#s\%3Apolymorphic-recursion}} при указании типа методов объектов. При таком подходе не понятно как добавлять поддержку полиморфных вариантных типов. Наш опыт говорит, что ради поддержки полиморфных вариантных типов от явного полиморфизма придется отказаться, но тогда новая реализация \visitors~будет почти в точности повторять подход \GT.
%\end{enumerate}
%



\section{Введение в обобщённое программирование}
\label{sec:tutorial}
% !TeX encoding = UTF-8

В этом разделе мы презентуем подход обобщённого по типам данных программирования (datatype generic programming) но основе стандартной задачи суммирования чисел в различных контейнерах.

Рассмотрим стандартную реализацию связных списков, встроенных в \ocaml{}, и напишем функцию суммирования чисел в  списке. Реализация суммирования стандартная: 0 для пустых списков, а для непустых -- это сумма элемента \lstinline{n} и суммы хвоста списка.

\begin{lstlisting}
type 'a list = 'a List.t = 
  | [] 1
  | ( :: ) of 'a * 'a list

let rec sum_list = function 
  | [] -> 0 
  | n :: xs -> n + sum_list xs
\end{lstlisting}

Заметим, что функция суммирования списка является более частным случаем другого преобразования, так называемой "свертки" или катаморфизма~\cite{DGP}.

\begin{lstlisting}
let rec fold_list f acc = function 
| [] -> acc 
| x::tl -> fold_list f (f acc x) tl 

let sum_list xs = fold_list ( + ) 0 xs
\end{lstlisting}

Теперь рассмотрим ту же самую задачу, но для суммирования значений, хранящихся в листьях бинарного дерева типа \lstinline{btree}. Сумма листовой вершины будет значением, хранящимся в этой вершине, а сумма узла дерева будет суммой левого и правого по деревьев.

\begin{lstlisting}
type 'a btree = 
  | Leaf of 'a 
  | Node of 'a btree * 'a btree

let rec sum_btree = function
  | Leaf n -> n + 0
  | Node (l, r) -> sum_btree l + sum_btree
\end{lstlisting}

И cнова сумма элементов в дереве является частным случаем более общей операции --- свертки для типа \lstinline{btree}.
\begin{lstlisting}
let rec fold_btree f acc = function
  | Leaf _ -> f acc 1
  | Node (l, r) -> 
      fold_btree f (fold_btree f acc l) r

let sum_btree t = fold_btree (+) 0 t
\end{lstlisting}

Заметим, что в обоих случаях искомое преобразование для двух типов выражается с помощью более общего преобразования, которое реализуется отдельно для каждого из типов, но сходным образом. Обобщённое по типам данных программирование позволяет избежать написания программистом такого шаблонного (англ. boilerplate) кода, позволяя на основе соответствующего типа данных породить на стадии компиляции реализации преобразований \lstinline{fold_btree} и \lstinline{fold_list}.

\begin{lstlisting}
type 'a list = 'a List.t = 
  | [] 
  | ( :: ) of 'a * 'a list 
  [@@deriving fold]

type 'a btree = 
  | Leaf of 'a 
  | Node of 'a btree * 'a btree 
  [@@deriving fold]
\end{lstlisting}

\noindent Объявления типов данных проаннотированы указанием вида преобразованя, которое надо построить, а подход \textsc{ppx\_deriving}~\cite{ppxderiving} строит преобразования \lstinline{fold_btree} и \lstinline{fold_list}. Традиционно, библиотеки для обобщенного по типам данных программирования содержат способы автоматического получения не только \lstinline{fold}, но и других видов преобразований: \lstinline{show}, \lstinline{map} и т.д.






%% !TeX encoding = windows-1251
\section{�������� ������}

\begin{comment}
 ������� ����� (Frederic Brooks) � ����� ��������� ����� �� ���������
 ��������  "���������� ��������-�����" ("The Mythical Man-Month")~\cite{MMM} ��������������� �������� ���������������� ��������� �������:
 
 \blockquote{"�����������, ������� �����, �������� ����� ��������������� � ������ ������. �� ������ ���� ����� � ������� � �� �������, ����� ����� �����������. ������ ����� ������ ��������, ������������ � ����������, ������� ����� �� �����, ����� ��� �������� ��� ����������� � �������� ��� ���������� ����������� ��������. (��� �� ������� ������, ����� ������������ ���� ���� ��������.)"}
 
 �������������, ���������������� �������� � �������� �� �������������  ��������� � ����������������; ���������� ���������� ��������� ����� ����� �������� � ���������������� ������������
 (��� ��� ��������� � ���������� ������������� ��������� � �������). 
 \end{comment}
 
 ����� �� ������� � ����������� ���������� ������������ ����������� �������� ����������� ������������� �����������, ������� ����� �������� � ������ ��������� ������������ ��������. ���������� ���������� ��������� ����� ����� �������� � ���������������� ������������
 (��� ��� ��������� � ���������� ������������� ��������� � �������). 
 ����� �� �������� ���������������� �������� ��������������� ��������� �������� ������������� \emph{����� ������}. ��� ��������� ��������� �������� ������; ��� ����� � ���� �������, � ��� ������; � ����� � ��������� ������� ��������� ��������� �������� ������. ���� ���������� � ����� ������ ������������ �� ����� ������ ���������, �� ���������� ��������� ����������� ����-�������������� ���� ������� ����� (\emph{������������}) ��� ���� �������� ����� ����� ������ �� ���� (\emph{���������}).
 
 ������, � ���������� �������������� �������������� ������, ��� �������, ���� ��������� ��������� 
 ����� ���� ���������� � ����������� �� ����� ����������. ����������� ��������� �������� ��������� ������������� �� ��������� � ������������, ������ ��� ����������
 �� ����� �������������� ���� �� ����� ���������� � ������ ���������� ������ ��������� �������� -- ������ ��������� -- �� ���������. � ������ �������, ��������� ��������������, ������� � ������������ ������ ����� ���� ����������� ``��� � ��������'' �� �������� �������� ����� � ������ ���� ��������������� ��� ������� ����������� ���� �� �����������. ���� �� �������� � ����������� ����� ���������� �������� ���������� ����� ������������� ������� �����, ��� ������� ���������� ������� ����� ���� ���������������. �������� ����� ������� ����� ��������� ���������� (\emph{ad hoc} ������������) � ����� \haskell � ���� ������� 
 �����~\cite{TypeClasses} � �������� �����~\cite{TypeFamilies}. ������, �� ������� ���������� ����������� � ��������� �������� ����� � ��������������� �������������� �������� �������� �����, ������ ����� ������������ ``�������'' ���������, �������
 �� ����� ���� ���������������. ������ �������� �������� \emph{���������� ����������������}~\cite{DGP} (\emph{datatype-generic programming}), ����� �������� �������� ���������� ������� ��� ���������� ����������� ������ �������� �������
 ��������������� ������, ��������� ��������� ����������� �����. ��������, ���� ����� ���� ������������ �� ���������� �����~\cite{Hinze,InstantGenerics,GenericOCaml}, ���� ����� ���������� � ����� ����� ���� ������� ��������� �� ����� ����������, ��� 
 ���������� ������� ��� ����������� ���� ������ ����� ���� ������������� �� �����
 ���������� �������������~\cite{Yallop,PPXLib}. ��� ������, ��������� ����, ��������� ���� �����: ��� ����� ������ �������� ������� �����, ��� ������ ������������ ��� ����������� ���������������� ���� ����� ����������. ��������, ��������������� ����������� ��������� ����������� �������� ������� ��� ���������� ����� ������ ������������ ��������� � �.�.
 
 %�������� ����� �������� ������ ��� ���������  ``Scrap Your Boilerplate''~\cite{SYB,SYB1,SYB2}, ������� ��������� � ��������� �� �������� ��������� �� �����, ��������������� �������. ������, ��� ���� �������, ��� ��������������� � ������������� SYB ������������, � ���� �� ��������������, ������� ������� ������ �� ����, �� ����� ������ ������������. 
 
 � ���������� �������� ����������������� �������������� ����������� ������ ���� �� ����������. ����� �������, ���� 
 ��������������� �������������� �� ������ ���������� ������������, �� �� ������ ���� ����������� ��������� ������ ���� ��������������, ���� �������� ��� �������. ����� �������, ��� ���������������� ������� ������������ ��� ��������������� ��������������, ����� ���������� ���������� ��������������.
 
 ������ ��������� ����� �������� �������������  �������� ���������� (late binding) ��� ����, ����� ������������ ����� ��������������� �������������� ��� ����� ������������. ����������� ��� ���� ������������� � ��������-��������������� �����������������, ����� ���������� � ��������������.
 
 
 � ������ ������ �� ������������ ������, ����������� � ���� ���� 
 �������� ���������� ��� ������������� ��������� ����������� ��������������, � ����������������� ��������������, ��������� ��� ��������������� ����������������. ��� ������������� ������� ������� ���� ����������� ���������� ��� ����������� ���������������� \textsc{GT}\footnote{\url{https://github.com/kakadu/GT/tree/ppx}} (\emph{Generic Transformers}), � ����� ��������� ���������� �������� ��� ��, ����������� ����������� ������������ ����� ��������������.
 
 
 %������� ��������� � �������� ���������� � 2014 ����. � ��� ������������ ���� �������� ���������� �� ������ �������� � ����� \ocaml, ��� � �~\visitors. ������, ��������� ������������ ������� � \visitors, ������� �� ������� ��������������,   �� ��������� ��� ������������ � � ��������� �������, � ��������� �  ������� \ocanren~\cite{OCanren}.
 
 
%% !TeX encoding = windows-1251
\section{������ � �������������� �������� ����������}
\label{sec:expo}

� ������ ������� �� ���������� ������, ������������ � ������������� �������� ����������; ����� ���������� ��� ������, ��� ������� ���������� �� ��������������; � ����� ���������� � �������������� ������ �������.

���������� ����������� ������������� ������-��������� � �������� ������ ��������� �������������� ������ ���������, ��������� ��������� ��������� ����������.

\begin{lstlisting}
type lam =
| Var of string
| Abs of string * lam
| App of lam *lam
\end{lstlisting}

\begin{comment}
� ���� ������� �� ���������� ���������� ��� ������ ��������� ��������� ��������. 
���� ��������� �� ������������� ���������� ������� � �� ����� �������������� ��� ������ ������������,
�� ����� ������������� �������� ������������ ������� � ���������, ������� ������� � ���.
����� �� ����� ������������ ��������� ����������: ����� ���������� $\inbr{\dots}$ ������������� ���������� ������� � ���������� ���������� ����� \textsc{OCaml}. ��������, ``$\inbr{f_t}$`` �������� ������������ ���������� ������� ��������������� �����  ``$f$'' ��� ���� ``$t$''. 
� ���������� ���������� ��� ����� ���� �������� ��� ``\lstinline{f_t}'', �� �� ���� ����������� �� �������� ���������� �����.
\end{comment}

%������ � �������� �������. 
%���������� ���������� ���� �������������� ���������:
%
%\begin{lstlisting}
%type expr =
%| Const of int
%| Var   of string
%| Binop of string * expr * expr
%\end{lstlisting}

���������������� ��������� ���������� ������ ��������� (call-by-name, call-by-value, applicative order, normal order) ����������� � ���� ��������: 1) ���������� ��� ������-����������� ��� ���; 2) ��� ������ ��������� ���������� ������ ��������� � �������. ��� ����������� ������� ���������� � ������� ������������� � �������� ��� �������� �� ��������� ���������� ��� ��������� ������������� ������. ������ �� ���������� ������� ���������� � ������� �������� (records) ����� OCaml, ����� ��������� ������ ����.


\begin{lstlisting}
type strategy = 
  { on_var : strategy -> string -> lam 
  ; on_abs : strategy -> string -> 
      lam -> lam 
  ; on_app : strategy -> lam -> 
      lam -> lam 
  }
\end{lstlisting}

��� ����� ������� ������ ����� ���������  ������� ��������� ���������� (������ \texttt{this}), ����� ������������ ������� ��� ������ �������������� ���������.
\begin{lstlisting}
let apply_strat st = function
  | Var name -> st.on_var st name
  | Abs (x, b) -> st.on_abs st x b
  | App (l, r) -> st.on_app st l r
\end{lstlisting}
������� ���������� ��������� ����������� ��� � �������� ��� ���� ���������.

\begin{lstlisting}
let reduce_under_abstraction st x b =
  Abs (x, apply_strat st b)
let no_reduce_under_abstraction _ x b = 
  Abs (x, b)
\end{lstlisting}

������� ��� ������������ ���������� ��� ����������� ����������� ������������ � ������ ����� ����� �� ���, ��� � ���� \lstinline{on_var} � � ��������� \lstinline{strategy}.

\begin{lstlisting}
let no_strat =
  let on_var _ name = Var name in
  let on_abs _ name body = 
    Abs (name, body) in
  let on_app _ l r = App (l, r) in
  { on_var; on_abs; on_app }
\end{lstlisting}
���������~\lstinline{no_strat}, ������� �� ���������� ������� ����������, ����� ������� ���������� ��� ���������� ������ ���������.

\begin{lstlisting}
let subst: string * lam -> lam -> lam = ...

let cbv =
  let on_app st f arg =
    match apply_strat st f with
    | Abs (x, e) -> 
        apply_strat st (subst (x, arg) e)
    | f2 -> App (f2, arg)
  in
  { no_strat with on_app }
\end{lstlisting}

%TODO: subst

��������� \lstinline{cbv} (����� �� ��������) ���������� �� \lstinline{no_strat} ������ ���, ��� ��� ��������� ����������. ��� ������ �����, ���� ����������� ��������� ��������� ������������ \lstinline{App}, ��������� ���� ������ ���� �� �����������.

\begin{lstlisting}
let ao_strat =
  { cbv_strat with 
    on_abs = reduce_under_abstraction }
\end{lstlisting}

������������� ������� ���������� ���������� �� ������ �� �������� ������ �������� ���������� ����������, ������� �� ��������� ��������� ������ � ����� \lstinline{on_abs}.

������ ��� �������� �������� ���������� ������������ � ����������� OCaml ��� ������ ������� ��������� �������������� ������������ ��������������� ������ OCaml. ��� �� ���������� � ������ ��� ������������� ������������ ��������. ����� ��� ����� ��� ������ ������ ��������� � ���������������� ����� ����������������, �������� ��.

���� ��� �������� �������� ���������� ������������. ��-������, ���������� "�������" � ���������  ���������� ������� � ������� ��������� �������������� �������, ������������ ��������������� (��������, \lstinline{subst}), ���������� ������������ ��������. ����� � ������ ������� �������� ������ ���������� �� ������, � ��� �� ������� ���������� ��� ����� �������� � �������� ��������.

������, ������������ � ������ ������, ��������� ��� ��� ���������� ���� ������������� ���������� �������������� \emph{��������} \ocaml, ������ ��������.

\begin{lstlisting}
class no_under_abs = object
  method on_Var v = Var v 
  method on_Abs x b = Abs (x, b)
end
\end{lstlisting}

���������� ������ \lstinline{no_under_abs} ���������� �������������� ��� ���������� � ����������, �� ����� �� �������  ������� ������, ��� ��������� ����������.

\begin{lstlisting}
class cbv_part fself = object(self)
  method private subst 
    : string*lam -> lam -> lam 
    = ...
  method on_App f arg = 
    match fself f  with
    | Abs (x, e) -> 
        fself (subst (x, arg) e)
    | f2 -> App (f2, arg)
end
\end{lstlisting}

���� ������ �� �������� ������� � ������ \lstinline{cbv_part}, ������� ���������� ������ �������� ����������, � ����� ����� \lstinline{subst} ��� ������������� �����������. ���� ����� ����� �� ������� ����������� ������� ��� ���������� ��� ���������� � ������-����������.

\begin{lstlisting}
class cbv fself = object
  inherit no_under_abs
  inherit cbv_part fself
end
\end{lstlisting}

����������� ���������� ������ �� �������� ����� ���� �������� ���� ������������ ���� �������.

\begin{lstlisting}
let gcata tr = function 
| Var v -> tr#on_Var v 
| Abs (x,b) -> tr#on_Abs x b 
| App (l,r) -> tr#on_App l r

let eval_cbv e =
  let rec eval e = 
    gcata (new cbv eval) e 
  in
  eval e
\end{lstlisting}

�������� ������� \lstinline{apply_strat} ����� ��������� \lstinline{gcata} (generic catamorphism). �������� \lstinline{fself} ������ \lstinline{cbv} �������� ��������� �������������� ������-��������� � ���������� ``������������ � ����'' ���������� ������� \lstinline{eval_cbv}. ��������� � ������� �������, � ����� �������, �� ������� ����� \lstinline{subst} ��� ������ ���������, ����� ��������� � ������� ~\cite{related}.

����� ������� ���� ������������� ���������� �������������� ��������, ������� ������������� ������� ��������� ������� ����������, �� �������� ����������� ��������� ����� �������������� ���������������� ��������� �������� ���� �� �����, � ����� ������ ������������� ����� �������������� � ������� ������������.




\section{������ �� ������, �� �� �����:}
\section{���������� ���������}
\section{��������� � visitors}
\section{����������}

����������� ������� ``$\inbr{show_{expr}}$'' (�������� ������������ �������� �� ����������)
����������� ��������� � ������: 

\begin{lstlisting}
let rec $\inbr{show_{expr}}$ = function
| Const  n         -> "Const " ^ string_of_int n
| Var    x         -> "Var " ^ x
| Binop (op, l, r) ->
  Printf.sprintf "Binop (%S, %s, %s)" 
                 op ($\inbr{show_{expr}}$ l) ($\inbr{show_{expr}}$ r)
\end{lstlisting}

�������������, ������������ ``$\inbr{show_{expr}}$'', ��������� ����� �������������. ��� ����� ����
������� ��� ������� ��� ������������. ������, ��� �������, ����� ��������� ����, ``��������''(\emph{pretty-printed}) �������������. 
� ���� ������������� ��������� �������������� � ``������������ ����������'' � �������������� ��������� �������� � ��� ��� 
�������������, ��� ������ ����������� ������ ���, ��� ��� ������������� �����. �� ����� ����������� ��� �������������� 
����� ������:

\begin{lstlisting}
let $\inbr{pretty_{expr}}$ e =
  let rec pretty_prio p = function
  | Const  n        -> string_of_int n
  | Var    x        -> x
  | Binop (o, l, r) ->
    let po = prio o in
    (if po <= p then br else id) @@
    pretty_prio po l ^ " " ^ o ^ " " ^ pretty_prio po r
  in
  pretty_prio min_int e
\end{lstlisting}

����� �� ���������� ��������� ``\lstinline{prio}'', ``\lstinline{br}'' � ``\lstinline{id}'', ���������� �� ���. ������� ``\lstinline{prio}''
���������� ��������� �������� ��������, ``\lstinline{br}'' �������� ���� �������� ��������, � ``\lstinline{id}'' --- ������������� �������.
�������������� ������� ``\lstinline{pretty_prio}'' ��������� �������� ��������, ������� ���������� ��������� ���������� �������� (���� ����� �������). ���� ��������� ������� �������� ������ ��� ����� �����������, ����� ��������� ���������� ��������. ��� �������� �� �������, ��� ��� �������� ��������������, �� ����� �� ������ ���� ����� ���� ����������� ��� ��������� ������������� ��������.
�� ������� ������ �� �������� ���������� ��������� ����� ��� ���������, ����� ���������, ��� �� �� ������� ������, ���������� ��������� ������� 

���������� ���� ���� ������� ����� ����� ���� ������. ��� ��������� ������, �� ������ ��������� �������������� ��������, � 
������ ����� ������������� � �������� ��� ��������������� ������������� �����������. ������������ ����� ������ ��������
������������� � �������� ���� �� ����. �� ����� ������� ��� � ��������� ������� � ��������������� ��� ������� ���������� �������������, 
��������������� �������������:

\begin{lstlisting}
let $\inbr{gcata_{expr}}$ $\omega$ $\iota$ = function
| Const n         -> $\omega$#$\inbr{Const}$ $\iota$ n
| Var   x         -> $\omega$#$\inbr{Var}$   $\iota$ x
| Binop (o, l, r) -> $\omega$#$\inbr{Binop}$ $\iota$ o l r
\end{lstlisting}

����� �� ������������ ��������� ������������ ��������� ������� ��������. ``$\omega$'' -- ��� ������, ��� ������ ������������� �������������
���� � ������. ``$\iota$'' ������������ �������������� ��������, ������� ����� �������������� ��������� ���, ��������, ``$\inbr{pretty_{expr}}$'' (� �������������� ��������� �� ������� ``$\inbr{show_{expr}}$'').

���������� � ������ ������� ``$\inbr{show_{expr}}$'' ����� ���� �������� ��������� �������\footnote{��� ������� ��������� �� �������� ��������� ��������� �����, ������� �������� ����� �������� ���� ������ �������� �����.}:

\begin{lstlisting}
let rec $\inbr{show_{expr}}$ e = $\inbr{gcata_{expr}}$
  object
    method $\inbr{Const}$ _ n   = "Const " ^ string_of_int n
    method $\inbr{Var}$  $\enspace$   _ x   = "Var " ^ x
    method $\inbr{Binop}$ _ o l r =
      Printf.sprintf "Binop (%S, %s, %s)" 
                     o ($\inbr{show_{expr}}$ l) ($\inbr{show_{expr}}$ r)
  end
  ()
  e
\end{lstlisting}

�, ����������, �� �� ��  ����� ��������� � ������� $\inbr{pretty_{expr}}$.

�� ����� ��������, ��� ��� �������, ����������� ��� ���������� ���� �������, ����� ���� ������� � ������� ������ ������������ ������:

\begin{lstlisting}
class virtual [$\iota$, $\sigma$] $\inbr{expr}$ =
object
  method virtual $\inbr{Const}$ : $\iota$ -> int -> $\sigma$
  method virtual $\inbr{Var}\enspace\;\;$ : $\iota$ -> string -> $\sigma$
  method virtual $\inbr{Binop}$ : $\iota$ -> string -> expr -> expr -> $\sigma$  
end
\end{lstlisting}

���������� �����, �������������� �������������� ����� ������������� �� ����� ������ ������. ����� ����� ����������� 
�������� ���������� ������ ��������������, �� ������������� ����� �������� ����������������� ``\lstinline{fself}'' 
(\emph{�������� ��������}). 
��������� � ����� �������� �������� ���������� ��� ����������� ��������� ����������� ���������� ����� � ����������� �����������.
������ �� ������ ����������� ������ ���������� � ������, ������� ��������, � ��������, �������� �� ������� ``��������'' ����������
 (�������� �������� �� ������������� ``\lstinline{fself}''):

\begin{lstlisting}
class $\inbr{pretty_{expr}}$ (fself : $\iota$ -> expr -> $\sigma$) = object 
  inherit [int, string] $\inbr{expr}$ 
  method $\inbr{Const}$ p n = string_of_int n
  method $\inbr{Var}$ p x = x
  method $\inbr{Binop}$ p o l r =
    let po = prio o in
    (if po <= p then fun s -> "(" ^ s ^ ")" else fun s -> s) @@
    fself po l ^ " " ^ o ^ " " ^ fself po r
end
\end{lstlisting}

������� ���������� � ������� �������� ������ ����� ���� ����� ������� � �������������� ������ ���� � ������� ���������� 
�������������\footnote{��� ��� ����� ������� � ������� ��������� � ������ ������������� ���� � \textsc{OCaml}, �� ����� 
������������ ���� � �� �� ��� ��� ������ � ������� �������������.}:

\begin{lstlisting}
let $\inbr{pretty_{expr}}$ e =
  let rec pretty_prio p e = 
    $\inbr{gcata_{expr}}$ (new $\inbr{pretty_{expr}}$ pretty_prio) p e 
  in
  pretty_prio min_int e
\end{lstlisting}

����� �� ����� �������� ���������� ��������� ������� � ������� ����������� ����������� ����� ``\lstinline{fix}'':

\begin{lstlisting}
let $\inbr{pretty_{expr}}$ e =
  fix (fun fself p e -> $\inbr{gcata_{expr}}$ (new $\inbr{pretty_{expr}}$ fself) p e)
      min_int e
\end{lstlisting}

���� �� ������ �������� ��� ����� ����� ��� ���� ����������� ��������� ��������������: ������� ����������� ������
(``$\inbr{gcata_{expr}}$'') � ����� ����������� ����� (``$\inbr{expr}$''), ��� ��� ������������� ����� ����������� ��� ��� ����������.
�� ������ �� ��� ����? � ����������������, � ���� ������� �� �������� �� ����� �������� ����������������� ���� ���� ����������
�������� ���������� ����������. �������� ��� ��������� �� ������� ���� ������ ���������.

�� ����������, ��� �������������� � ������ ���������� ������ ���� ������������ ����������. ����� ��������� ��������� ������,
������� ���������� ����� ������������� ��������. ������ ��������, ��� ������ ������������� ����� ���� ������������ 
(�� �������� ��������) ��� \emph{������������}, �.�. ��� ``������''~\cite{Fold,Bananas,CalculatingFP}. 
���������, ����� ���������� ������������ ����������� ��� ����� �������������� ��� ``\lstinline{expr}'' 
�� ������ ���� � ������� �������, �� ����� �� ������������� ����� ����������� ��������:

\begin{lstlisting}
class [$\iota$] $\inbr{fold_{expr}}$ (fself : $\iota$ -> expr -> $\iota$) = object 
  inherit [$\iota$, $\iota$] $\inbr{expr}$ 
  method $\inbr{Const}$ i n = i
  method $\inbr{Var}$ i x = i
  method $\inbr{Binop}$ i o l r = fself (fself i l) r
end
\end{lstlisting}

��� ���������� ������ �������� ``\lstinline{i}'' ������ ��� ���� ����������������� ��������, ��� �������� �������� ���������� �� ������ ������.
������, ������ ������� ���������, ����� �������� ���-��� ��������:

\begin{lstlisting}
let fv e =
  fix (fun fself i e ->
        $\inbr{gcata_{expr}}$ (object inherit [string list] $\inbr{fold_{expr}}$ fself
                      method $\inbr{Var}$ i x = x :: i
                    end) i e
      ) [] e
\end{lstlisting}

��� ������� ������� ������ ���� ��������� ���������� � ���������, � ��� ��� � ����� ��������� ��� ������� ��������� ����������, 
�� ��� ������ ������ ���� ���������� � �����. ������, ������� �� ���������������, ����������� �� ``������������'' ������ ``$\inbr{fold_{expr}}$'' � �������������� ������ ���� ����� -- ����� ��� ��������� ����������.
���� ��������� ��� ��� �������� ���, ��� ��� �����~--- ``$\inbr{gcata_{expr}}$'' ������� ���������, 
� ��������� ������ ������� ��������� �������� ����������� ������ ������.
����� �������, �� ������ ����������� ���������� �������������� � ������� ����� ����� ����������� ������������� ����, 
���������������� ��� ��������� ������� ``$\inbr{fold_{expr}}$''. ����� �������� �����������, ��� �� ��������� ���������������� �
������������� ������ ����� �������, �� ������� ��� ����:

\begin{lstlisting}
let height e =
  fix (fun fself i e ->
        $\inbr{gcata_{expr}}$ 
          (object 
            inherit [int] $\inbr{fold_{expr}}$ fself
            method $\inbr{Binop}$ i _ l r = 1 + max (fself i l) (fself i r) 
          end) 
          i 
          e
      ) 0 e
\end{lstlisting}

����� �� ��������� ������ ������ ���������, ��������� ��� �� ����� ����� ``$\inbr{fold_{expr}}$'' ��� ������� ��� ������� �������������� �������, �������������� ����� ��� ��������� ���������, ������� ������ ����� ��������� ������ �����������, �������� �� ��� ������������ ������ � ���������� �������.

����� �� ������ ����� ��������� ���������� ������� �������� ``map'':

\begin{lstlisting}
class $\inbr{map_{expr}}$ fself = object 
  inherit [unit, expr] $\inbr{expr}$
  method $\inbr{Var}$ _ x = Var x
  method $\inbr{Const}$ _ n = Const n
  method $\inbr{Binop}$ _ o l r = Binop (o, fself () l, fself () r)
end
\end{lstlisting}

�����, ��� ��� ��� �� ��������, ��� ``\lstinline{expr}'' -- ��� �������, �� ��, ��� �� ����� ������� � ������� ``$\inbr{map_{expr}}$'' --- 
��� ������������. ������, ���������������� �� ����� ������, �� ����� �������� ����� ��� ��������������:

\begin{lstlisting}
class simplify fself = object 
  inherit $\inbr{map_{expr}}$ fself
  method $\inbr{Binop}$ _ o l r =
    match fself () l, fself () r with
    | Const l, Const r -> Const ((op o) l r)
    | l      , r       -> Binop (o, l, r)     
end
\end{lstlisting}

������ ����� �������� ��������� ���������: ���� ��� ��������� �������� �������� ����������� � ��������� ��� �� ����� ��������������, ����� 
�� ����� ���������� �������� �����. ������� ``\lstinline{op}'' ��������� ���-�� ���, ��� ���������� �������, ������� ����� ����������� ���������� ������� ��������� ���������.

��� ��� ���� ������:

\begin{lstlisting}
class substitute fself state = object 
  inherit $\inbr{map_{expr}}$ fself
  method $\inbr{Var}$ _ x = Const (state x)  
end
\end{lstlisting}

����� �� ��������� ����������� ���������� � ��������� �� ��������, ������������ � ��������� ���������, �������������� ������� ``\lstinline{state}''. ��� ������, ����������� ���� ����� ���� �������������� ��� ��������� �������������� ���������:

\begin{lstlisting}
class eval fself state =  object
  inherit substitute fself state
  inherit simplify   fself
end

let eval state e =
  fix (fun fself i e -> $\inbr{gcata_{expr}}$ (new eval fself state) i e) () e  
\end{lstlisting}

�� ���� �������� �� ������� ���������� ����������� �������������� � ����� �������, ��� ����������� �� ���������� ������ ��������,
���� ������� ����� �� ��������� ������������ ��������� ������� � �������� �  \textsc{OCaml}. � ������ ������ ���� ���������� ��������������
������ ���� ����� � ��������������� ��������, ���������� ���������� �� ����. 
� ������ �������, �� �������� � ����� ������ ���������� �����, �� ���� �� ��� �����������, � ��������� ������������ ����� �������� � 
������������ ���������. � ���������� ����� ������ �� �������, ��� ����, �������������� ����, ����� ���� ��������� �� ������� � ����������� ����������������, ��� ��� ���������� ����� ���� ������������� �� ���������� ����. � ���������, ��� ������ ������������� ������ ���������:

\begin{itemize}
\item ������������;
\item ���������� ������� ���������� (type operators);
\item �������� ��������, ��� ��������� �������� \emph{�����������} �������������� ��������� ��������� ������;
\item ����������� ���������� �����, � �������� ����� ���������� ������������ � ������� ���������� ������������ ����������� ��������� � ������������ �������;
\item ���������� ���������� --- �� ����� ������������� ��� �� ������������ ����� �� ���������� ������ �������, �� ������� ������� �������������� ���;
\item ������������, � ������ ��������� �������� �������, ������� ����������� ���� � ��������� �����������. ���������� ������� ��� ����������� ����� ����� �������������� �� ��� ������, �� �� �������� �������������� ��� �������� ���������� ������������ ����.
\end{itemize}

��� �������� �������� ������������������, �� ��� �� ����� ��������, �� ���� �������� �� ��������� ������� ���������� 
\emph{����������} �������� �� ����� �������������� (��� ������ �� ������ ���� ��������� ������). ����� �� ������, ��� � ���� ����� ����������
� ������� ����������. �������, ��� ������ ������������� ������� ��������, ������� ����� ���� ������������ ��� ��������� �������� ���������� ��������������, ���, ��������, ``\lstinline{show}'', ``\lstinline{fold}'' ��� ``\lstinline{map}''. ������� �������� ����������, �.�. ������������ �����  ����������� �� ����������� ������� � ������� ��������� ������, ��� ������� ����� ���������������� �� ������ ��������� ������ ��������������� �����������. 


\blockquote{��������/���������\\
����� �� ������ ����������, ������� �������������� ����������, �������� ��, ��� ������ ���������� ������� ����� ������������� ��� ����������� ��������� ������ ���������� �������. ��� ������, ������� ������������ (�� ������� ������� �� ������ ���������� �����), ����� ��������� �������� ������������� ����� �������� ����� �������������� ����������� ��������� ����� ��� ���������. ��� ����������� ���� ����������� ������������� ���� � ������ � ������ �������, ��� ��������� ���������� ������, ���������� ��������� ��������~\cite{ObjectAlgebras}.

������������� ������������� ������ ������� ��������:

\begin{itemize}
\item ������ �������������� ���������� � ������� \emph{������� ��������������} � \emph{������� ��������������}, ������� �������� � ���� ``����������'' ����� ��������������;
\item ������� �������������� ��������� ��� ������� ���� � ���� �������� ��������������, ������� �������� ��������� ����������� ������;
\item � ������� ��������������, � ����� ������������ �� ���������� ����; �� ������������ ���������� �������������� ���� ������, ���������, ����������� ���������� ���� � ������� ������� ��������� ����;
\item �� ������������� ��������� ��������, ��� ����, ����� ������������ ����������� �������� �������������� � ���� ���������� �������;
\item ������� �������� ���������� -- ����������� ����� ����������� ���� ����������� �������.
\end{itemize}



����������� ����� ������ ������ ���� ���������� ������������ � ��������-���������������� �������: ������ �������� ������ ����������� �������������� ������� �������, � ������ ����������� ������������� �� ���� �������� ���������� (late binding). ���� � ���� ����������� ������� �������������� ���� ������� ������������~\cite{SCICO}, � ����� ����������� � ���� ���������� � ��������������� ����������~\cite{TransformationObjects}. ����������� ����, ��������� � ����������� ����������~\cite{OCanren}, ����� ������ �� ��������� ���������� � ����������. � ������ ������ �������������� ����� ��������� ������������ ����������, ��� ��������� ���������� ���� ����������.
}

���������� ����� ������ ������������ ��������� �������. � �������~\cite{section2} �� ������� ������ �������� �������� ���������� ��� ������ ����������������, ��� ��� ���, � ����� ������� ������� ��� ������������� \GT~�������� ��������. 
� ������� \ref{sec:relatedworks} ������� ����������� ������� � ���������� � ��������� � ����. � ��������� ������� \ref{sec:futurework} ������ ����������� ��� ����������� ��������.
\begin{comment}
 � ��������� ������� \ref{sec:expo} �� ����������� ������ ��� ������ � ������� ��������. ����� \ref{sec:implementation} ������ ���������� � �������, ���������� �������, ������� ������� ������� ��� �����������. ����� ���������� ��������� ��������, ������������� \ref{sec:examples} � ������� ����� ����������. � ������� \ref{sec:relatedworks} ������� ����������� ������� � ���������� � ��������� � ����. � ��������� ������� \ref{sec:futurework} ������ ����������� ��� ����������� ��������.
\end{comment}



\section{Метод}
\label{sec:implementation}
% !TeX encoding = windows-1251
\section{�����}
\label{sec:implementation}

����� ������� ����������� ���������������� �������� ���������� <<�����������>> ���� �� ���� ������, ������� � ������ ������� �� �������� ��� ������ 
%� �������������� �������� ���������� 
� ����� ������� � ��������� ���� �� \ocaml{}.

%��������� ������������ ������ ������� �������� �������������� ���������� (� ��� \cd{camlp5}~\cite{Camlp5}, � ���  \cd{ppxlib}~\cite{PPXLib}), ���������� ������� ���������� � ������� ��������. �������������� ���������� �������� ���������� �����, �������������� �������������, � ���������� ��������� ��������:

���������� ���������������� ����������� � ����������� �����, �������������� �������������, � � ����� ������ ��������� ��������� ��������:

\begin{itemize}
\item ���������� ������� �������������� (\emph{generic catamorphism}, \emph{gcata}), ���� �� ������ ���;
\item ����������� �����, ������� ������������ ��� ����� ������ ��� ���� ��������������, ���� �� ������ ���;
\item ��������� ���������� ���������� �������, �� ������ �� ������ ��� ��������� �������������;
\item ��������� ������ \emph{typeinfo}, ������� �������� � ���� ����������, ����������� ��� ������� ����, � ������ ���������� ������� �������������� � ����� �������-��������������, ������� ������� �������������� � ������� �������; �� ������������ ��� ��������������� ������.
\end{itemize}

�� ������������ ��������� �������� ���������� ����� �� ���������� �������������:

\begin{itemize}
\item ������ ���������� �������������� ���� ������; GADT'� �������������� ��� ������� �������������� ����;
\item ����������� �� ���� (constraints) �� �����������;
\item �������, ������ � ���� � �������� ������ ``\lstinline{nonrec}'' �� ��������������;
\item ����������� ���� ������ (``\lstinline{...}''/``\lstinline{+=}'') �� ��������������.
\end{itemize}

� �������, ���� ��� ���� ``\lstinline{t}'' �������� �������������� ``\lstinline{show}'', �� � ����� ���������� �������� ��������� �������� (� ������� ``$\dots$'' �� ���������� �����, ���������� ������� ���� ��������):

\begin{comment}
\begin{figure}[t]
  \center
  \begin{tabular}{L{6cm}|l}
    \hline
    \multicolumn{2}{c}{� �������������� \cd{camlp5}}\\
    \hline
    \lstinline|@type ... = ... | & �������������� ����������� ��� ���������  \\
    \lstinline|and  ... = ... | & ���� � ��������� $p_1, p_2, \dots$; ������� \\
    \lstinline|   $[$ with  $p_1, p_2, \dots$ $]$| & ����������� ���� ����� ��������������; \\
    \lstinline|@$typ$| & �������� ������������ ������ ��� ���� $typ$; \\
    \lstinline|@$plugin$[$typ$]| & ��� ������ ������� ��� ���� $typ$ � \\
                                 & ������� $plugin$\\
    \hline
        \multicolumn{2}{c}{� �������������� \cd{ppxlib}}\\
    \hline
    \lstinline|type ... = ...|  & �������������� ����������� ��� \\
    \lstinline|and  ... = ...|  & ��������� ����  � ��������� $p_1, p_2, \dots$  \\
    \lstinline|[@@deriving gt | & $ $ \\
    \lstinline|  ~options:{ $p_1, p_2, \dots$}]| & \\
  \end{tabular}
  \caption{����������� ������������ ����������}
  \label{syntax}
\end{figure}
\end{comment}

\begin{lstlisting}
let $\inbr{gcata_t}$ $\dots$ = $\dots$

class virtual [$\dots$] $\inbr{t}$ = object  $\dots$ end

class [$\dots$] $\inbr{show_t}$ $\dots$ = object 
  inherit [$\dots$] $\inbr{t}$ $\dots$
  $\dots$
end

let t = {
  gcata   = $\inbr{gcata_t}$;
  $\dots$
  plugins = object method show = $\dots$ end
}
\end{lstlisting}

� ������� ��������� ``\lstinline{t}'' � ����������� � ���� �� ����� ������������ �������-��������������, ��������������� ������:

%\begin{lstlisting}
%let transform t = t.gcata
%let show      t = t.plugins#show
%\end{lstlisting}
\begin{lstlisting}
let show      t = t.plugins#show
\end{lstlisting}

������� ``\lstinline{transform(t)}'' -- ��� ������� �������� ������ �� ���������� \GT{}, ������� ����� ���� ��������������� ��� ������ ��������������� ����  ``\lstinline{t}''. 

%\begin{comment}
%�� �������~\ref{syntax} �� ��������� ���������� �������������� �����������, ������������� ��� �������������� ����������. �������� ��������, ��� ���������� ������������� ���� ��� ������� � ������� ������������� (�������������� ���� ���$\inbr{...}$) �������� ��������������� ���� ������������ \cd{camlp5}, ��� ��� ��������������� ��������������� �������������� ����������.
%\end{comment}

\subsection{���� ��������������}

������ ������� ������� �� ���� �������� �������������~\cite{Bananas} � ������� ���������� 
���������~\cite{AGKnuth,AGSwierstra,ObjectAlgebrasAttribute}.
�� ������������� ������ ������������� ���������� ����

\[
\iota \to t \to \sigma
\]
��� $t$ -- ��� ���, �������� �������� �� �����������, $\iota$ � $\sigma$~--- ���� \emph{�����������} � \emph{�������������} ���������. 
�� �� ����� ������������ ���������� ����������, ����� ��������� ��������������� ����� �������������, �� ������ �������������� ������������ ��� �������� ����� �����. 

���� ��������������� ��� �������� ���������������, �� �������������� ���� ����� ���������������. ����� �� ����� ���������� � �������
$\left\{...\right\}$ ������������� ��������� �������� � �������. � ������� ����� ������� �� ������ ������� ���������� ����� ��������������, ������������ � ������� ����� ����������, ���

\[
  \left\{\iota_i \to \alpha_i \to \sigma_i\right\}\to\iota \to\left\{\alpha_i\right\}\;t \to \sigma
\]

����� $\iota_i\to\alpha_i\to\sigma_i$ �������� ��������-���\-������������ ��� �������� ��������� $\alpha_i$. � �����, �������-�������������� ��������� ������ ��������� �� ����������� �������� � ���������� �������� � ���������� ������������� �������� ��� ��������� �����. ����� ��� ���� �������������� �����-������ ��� $n$-���������������� ���� ����� $3\cdot(n+1)$ ������� ����������:

\begin{itemize}
\item ������ $\iota_i$, $\alpha_i$, $\sigma_i$ ��� ������� �������� ��������� $\alpha_i$, ��� $\iota_i$ � $\sigma_i$ --- ��� ������� ���������� ������������ � ���������������� ��������� ��� ��������������  $\alpha_i$;
\item ���� �������������� ������� ���������� $\iota$ � $\sigma$ ��� ������������� ������������ � ���������������� ��������� ����������������� ����;
\item �������������� ������� ���������� $\varepsilon$, ������� �������������� � ``\lstinline|$\{\alpha_i\}$ t|'' ��� ����� �������� �� ����������� ����������, � �������������� � \emph{���������} ���� ``\lstinline|[> $\{\alpha_i\}$ t]|'' ��� ����������� ���������� ����� %(��������� � �������~\ref{pv})
.
\end{itemize}

��������, ���� ��� ��� ������������������ ��� \lstinline{($\alpha$, $\beta$) t}, �� ���������� ������ ������-������ ����� 

\begin{lstlisting}
class virtual [$\iota_\alpha$, $\!\alpha$, $\!\sigma_\alpha$, $\!\iota_\beta$, $\!\beta$, $\!\sigma_\beta$, $\!\iota$, $\!\varepsilon$, $\!\sigma$] $\inbr{t}$
\end{lstlisting}

���������� �������������� ����� ������������� �� ����� ������ �, ��������, ���������������� ��������� �� ������� ����������.
������������� ���������� ������ �������� ��������� ����������-�������:

\begin{itemize}
\item $n$ �������, ������������� ������� ���������: \lstinline|f$_{\alpha_i}$ : $\iota_i$ -> $\alpha_i$ -> $\sigma_i$|;
\item ������� ��� ���������� �������� ��������: \lstinline|fself : $\iota$ -> $\varepsilon$ ->  $\sigma$|.
\end{itemize}

��������, ��� ����, ����������� ���� � �������������� ``\lstinline{show}'' ��������� ����������� ����� ����� ��������� ���

\begin{lstlisting}
class [$\alpha$, $\beta$, $\varepsilon$] $\inbr{show_t}$ 
  (f$_\alpha$     : unit -> $\alpha$ -> string)
  (f$_\beta$     : unit -> $\beta$ -> string)
  (fself : unit -> $\varepsilon$ -> string) =
object 
  inherit [ unit, $\alpha$, string
          , unit, $\beta$, string
          , unit, $\varepsilon$, string] $\inbr{t}$
  $\dots$
end 
\end{lstlisting}

�������� ��������, ��� �� ������������ ��� ���������� ��� ���� �����, ���� ��� ��������� ����� ��������� ���������� ����� ���� �������, ��������, ``\lstinline{fself}''
����� ������ ��� ����������� �����. ���������� ����� �������: ���� �� \emph{����������} ��������� ���
�� �� � ����� ������ �� ����� ��� �����������. �������������, ��� ��������� ���������� ���������� ���������� ���� ��������� ������ ����� ����� ���������.

��� ����� ��������� �������� ����� ������������ � �����������. ������������ ������� ���������� ������� ���������� � ������� ����� ����������.
������, ������������� ����������� ����������� � ���� ������ ���� ��� ����� ������������� �������������� \emph{�������} � ���� ���� 
������������ �� ������ ������-������.
� ����������� ������� �������������� ����������� ���� ��������� ������������� ����������� ������� ��� ��������� ������� ��������. 
� ������ ������ ������ ������� ��������� ����� ��� ������������������ (��������, ���  ``\lstinline{show}'' ����������� ������� ���������� ���������������� � ������� ����), �� ������ ������� �������� �������� ������� ���������� ������������� ������� ����������. % (��������� � �������~\ref{plugins}).

��� ����� ���������� ������� ���� ���������� � ������� ������ ������. ����� ��� ������������  ``\lstinline|C of a$_1$ * a$_2$ * ... * a$_k$|'' ����� ��������� ���������:

\begin{lstlisting}
method virtual $\inbr{C}$ 
  : $\iota$ -> $\varepsilon$ -> a$_1$ -> a$_2$ -> ... -> a$_k$ -> $\sigma$
\end{lstlisting}

\noindent ����� ��������� �� ������ ����������� ������� � ���������, ��������������� ������������, �� � ��������, ������� ������ �������������.

�������, �� ������ ��� ���������� ������� ��������������. ��� ������ ���������� ��� ������ ����������� ���������� �����.

��� ����, �� ����������� ����������� ���������� �����, � ������ ``\lstinline|$\{\alpha_i\}$ t|'' \emph{���������� ������� ��������������} ����� ��������� ���:

\begin{lstlisting}
val $\inbr{gcata_t}$ : [$\{\iota_{\alpha_i}$, $\!\!\alpha_i$, $\!\!\sigma_{\alpha_i}\}$, $\!\!\iota$, $\!\!\{\alpha_i\}$ t, $\sigma$] #$\inbr{t}$ 
                -> $\iota$ -> $\{\alpha_i\}$ t -> $\sigma$
\end{lstlisting}

��� ��������� ������, �������������� ��������������, � �������� ������� ���������, ���������� ���� ������������ �� �������� ������, ��������������� ������� ����������������, ����������� �������, ��������, ������� ����� ������������� � ���������� ������������� �������.
�������������� �������� ``$\varepsilon$'' ���������������� � �������������� ���, � 
��� ����������� ���������� ����� ---~� \emph{��������}
������ ����:  ``\lstinline|[> $\!\!\{\alpha_i\}$ t]|''. 
��� ��������� ��������� ������� �������������� � �������, ��������������� �������������� ������������ ���� � �\'������ ���������� �������������.


\subsection{���������� ����������� ����� � ����������}
\label{memofix}

� ���������� ������� �� ������� ������ ���������� ��������, �������������� ��������������. ������ ���������� ��������������� ������ ��� ������������ ���������, ����� ��������� �������-��������������. �� ���������� �������, ����� ��� ������ � �������������� ����������� ���������������� ������������ ������������� ������������ �������� ���������.

�� ����������  �� �������� ��������: �����, ����������� ���������� �������������� ��������� ������� �������������� ������ ���� ��� ��������.
����� ������� ����� ������� ��������� ���������� ����������� �����. �  ���� �������
�� ���������� ������ ������� ����� ����������, � ������ ��� ���������� ���������� ����.
�� ������� ����������� ������ ����������� ���� ����� ������� ���������� (��������� � 
�������~\ref{murec}).

%�� ���������� ��� ������ �� �������~\ref{sec:expo}:

�������������� $tr$ ��� ���� $t$, �������������� � ������� ������� $\inbr{tr_{t}}$, ����������� ��������� ��������:

\begin{lstlisting}
let $\inbr{tr_{t}}$ $\{f_i\}$ $\iota$ x =
  transform $t$.gcata (new $\inbr{tr_{t}}$ $\{f_i\}$) $\iota$ x
\end{lstlisting}
\noindent ��� ������������ � ������� ������  $\inbr{tr_{t}}$, ��������������� �������������� ��� ���� $t$, � ���������� ������� �������������� ���� $t$. 

\begin{lstlisting}
let transform t = transform_gc t.gcata

let transform_gc gcata make_obj $\iota$ x =
  let rec obj = lazy (make_obj fself)
  and fself $\iota$ x = 
    gcata (Lazy.force obj) $\iota$ x in
  fself $\iota$ x
\end{lstlisting}

� ���� ���������� ����� ������������ ���������� ����������� ����� \lstinline{transform_gc}, ����������� ���� ��� ��� ���� �����. ��� ����� ��� ����, ����� ���������� � �����, ��������� � ������� �������� ��������, �������-�������������� ���� $t$, ������� �� ����� �������. � ���������� ������������ ������� �������� �������, ��������������� ��������������, ����� �������� �������� ����� ������� ��� ������ ����������� ������. ��� �������� �������� �� ������� ����, ��� �� ����� ���������� �������������� ������, �������������� ��������������, �� ����������.

%\begin{comment}
%\begin{lstlisting}
%let $\inbr{pretty_{expr}}$ i e = 
%  fix (fun fself -> 
%         $\inbr{gcata_{expr}}$ (new $\inbr{pretty_{expr}}$ fself))
%  i e
%\end{lstlisting}
%\end{comment}


%\begin{comment}
%����� ������������ ������ ����������, ���� ������� ����������� ������ ���, ����� ���������� \lstinline{fself}'' � ������ �������������� (�� ����, ��� ������� ���� � ������ ����������������� ��������). ��� ��� ��� ������� ���������, �� �� �������� ����� ���������������.
%
%�� ����������� �������� �������, ��������������� ��������������, � ������� ������� ����������. ��� ����� �� ������������ �������� ������� � �������, ������� ���������
%�������� ``\lstinline{fself}''. ���������� ����������� ����������� ����� �������� ��������� �������:
%
%\begin{lstlisting}
%let fix gcata make_obj $\iota$ x =
%  let rec obj = lazy (make_obj fself)
%  and fself $\iota$ x = gcata (Lazy.force obj) $\iota$ x in
%  fself $\iota$ x
%\end{lstlisting}
%
%���� ���������� ����� �������������� ��� ���� ����� � �� �������� ������������ �� ���� ������. ������ �� ����� ������� ��������� ���������� ������� ``\lstinline{transform}'':
%
%\begin{lstlisting}
%let transform typeinfo = fix typeinfo.gcata
%\end{lstlisting}
%
%� ������� ����� ����������� ������������ �� ����� ������������ ���������� ����������� ����� ����:

%\begin{lstlisting}
%let $\inbr{show_{expr}}$ e =
%  transform(expr) (fun fself -> new $\inbr{show_{expr}}$ fself) () e
%\end{lstlisting}

%\end{comment}

\subsection{������� ����������� �����������}
\label{murec}

� ������, ���� ���������� ��������� �������������� ��� ������ �������-����������� ����������� �����, �� ����������� ��� ��������� ����������. ��-������ ������ �������������� ����� ������ �������� �������������� ��� ������ �����, ����������� �������-����������, � ��������� ������������� ������������� � �������� ���������. ��-������, ��� ������ ������ ����� ����� ����������� ���������� ����������� �����, ������� ������������ <<����������� � ����>> �������������� ������ ���� ��� ������ ������ �����. � �������, ����� ���������� �������� �������������� ������ ��������������, ������������������ ��� ������ ��� ����������� �������������� ������ �����, ����� ��������� ��������� ������� ��� �������� ��� �������-����������� � ������� �����������.


\subsection{����������}
�� ������ ���������� ������ ������������� �������������� � ������� �������� ���� ��������������� ���������� Generic Transformers\footnote{\url{https://github.com/Kakadu/GT/tree/v0.3.0}} (GT), ����������� � �������������� ����������� ���������������� ��������� �� ����� ������ ����������� ��������������. � ��� �������������� ��� �������� ���������������� ���� �������������� ���������� ��� ����� \ocaml{}: \textsc{PPX}~\cite{PPXLib} � \camlpfive~\cite{camlp5}. ���������� \GT{} �������� �����������: � ��� ����������� ��������� ���������� ���������������� ��������, ����������� ���������� ����� ����� ��������������, � ����� ����� ����������� ��������������, �������������� � ������ �������� �����������.



\begin{comment}


\subsection{������� ��������}
\label{plugins}


\subsection{�������� ��������}
%\label{murec}

%������� 2 �������� ��� ���

������ ��������� ������� ����������� ����������� ����� ������� �������������� ������.
���������, �������� ���� ����������� ��������� ����� ���� ����������� �����, ��� � ��� 
���������� ������, �� ��� ����� �������� ������������� ���������� ��������������.
�� ���������������� ��� ������� � ������� ����. ���������� ����������� ����


\begin{lstlisting}
type expr = $\dots$ | LocalDef of def * expr
and  def  = Def of string * expr
\end{lstlisting}

��� �� �������� �������� ����� (����������, �������� �������� � �.�.) � ���������� ���� ���������. �������� ��������, ��� ���������� ������� �������������� ��� ����� ����� �����  ���� ��������� ��� ��� ����, ��� ��� ��� �� ���� ������ ������������� ������ ��� ���������� �������������� �� ����� ������� ������� � �� ������� �� ������� �������� � ������������ �����.

\begin{lstlisting}
let $\inbr{gcata_{expr}}$ $\omega$ $\iota$ = function
$\dots$
| LocalDef (d, e) as x -> $\omega$#$\inbr{LocalDef}$ $\iota$ x d e

let $\inbr{gcata_{def}}$ $\omega$ $\iota$ = function
| Def (s, e) as x -> $\omega$#$\inbr{Def}$ $\iota$ x s e
\end{lstlisting}

�� �� ����� ����� � ��� ������ ������-������. ������, ���� �� ������ ������������� ���������� ��������������, �� ��� ����������� �������������� �������� 
���� ``\lstinline{expr}'' ������ ������ ��� ``\lstinline{def}'', � ��������. ��� ����� ���� ������� � ������� ������� ����������� ����������� ������� (�� ����� �� �������� �������� ����� ����):

\begin{lstlisting}
class $\inbr{show_{expr}}$ fself = object 
  inherit [unit, _, string] $\inbr{expr}$ fself
  $\dots$
  method $\inbr{LocalDef}$ $\iota$ x d e =
    $\dots$ (fix $\inbr{gcata_{def}}$ (fun fself -> new $\inbr{show_{def}}$ fself) $\dots$) $\dots$
end
and $\inbr{show_{def}}$ fself = object 
  inherit [unit, _, string] $\inbr{def}$ fself
  method $\inbr{Def}$ $\iota$ x s e =
    $\dots$ (fix $\inbr{gcata_{expr}}$ (fun fself -> new $\inbr{show_{expr}}$ fself) $\dots$) $\dots$
end
\end{lstlisting}

��������, ��� � ����� ���������� ``\lstinline{fix}'' �� ������� \emph{����������} ������  (``$\inbr{show_{def}}$'' � ``$\inbr{show_{expr}}$''). �� ������ ������, ��� ������ �������� ��� ����������. ������ ������, ��� \emph{����������} �������������� ������������� ��������.
�� ��� ��������, ���� ��� ����������� �������������� ��������� � ������ 
 ``$\inbr{show_{expr}}$''? �������� �������, ������������� ����, �� ���������� ��������������� �� ``$\inbr{show_{expr}}$'', �������������� ��������� ������ � ��������������� ������� � ������� ����������� ����������� �����:

\begin{lstlisting}
class custom_show fself = object 
  inherit $\inbr{show_{expr}}$ fself
  method $\inbr{Const}$ $\iota$ x n = "a constant"
end

let custom_show e = 
  fix $\inbr{gcata_{expr}}$ (fun fself -> new custom_show fself) () e
\end{lstlisting}

� ��� �� ����� �������� ���, ��� �� �������, ������ �� �� ���������� �����
``$\inbr{LocalDef}$'', ������� ���������� ����� �� ��������� ��� ����  ``\lstinline{def}'', ������� � ���� ������� ���������� ������� �� ��������� ��� ����  ``\lstinline{expr}''.
����������, ��� �� �������������� ��������� ������ ����� ���������� ������� ������������ �������������� �����, � ������ ��� ���� ``\lstinline{expr}''. 
��� ��������� ���� ``\lstinline{expr}'' � ������ ����� �� ��� ������������� ����������� �������. ����� ��������� ��� ���������, ��� �������� ��������� ���������� ������� ����������� ������� \emph{�������}, ��� ������������ ��� ���� �������������.

���� ������� �������� ����� ���������� �� ���� �������� ��������. �������, �� ��������������� ���������� ����� �������������� ��������������� \emph{����} �����, ����������� �� ������� ����������� ����������� �����.
��� ��� ��� �������������� �������� ���������� �� ����������� �������, ��� �������� �������� ��� ������ ��� ��������������. ��� ������ ������� ��� ����� �������� ��� ���:

\begin{lstlisting}
class $\inbr{show\_stub_{expr}}$ $f_{expr}$ $f_{def}$ = object 
  inherit [unit, _, string] $\inbr{expr}$ $f_{expr}$
  $\dots$
  method $\inbr{LocalDef}$ $\iota$ x d e = $\dots$ ($f_{def}$ $\dots$) $\dots$
end

class $\inbr{show\_stub_{def}}$ $f_{expr}$ $f_{def}$ = object 
  inherit [unit, _, string] $\inbr{def}$ $f_{def}$
  method $\inbr{Def}$ $\iota$ x s e = $\dots$ ($f_{expr}$ $\dots$) $\dots$
end
\end{lstlisting}

�������� �������� �� ���������� ����������� �������.

����� �� ����������� ���������� ����������� ����� ��� ����� ������� ������������ �����������:

\begin{lstlisting}
let $\inbr{fix_{expr, def}}$ ($c_{expr}$, $c_{def}$) =
  let rec $t_{expr}$ $\iota$ x = $\inbr{gcata_{expr}}$ ($c_{expr}$ $t_{expr}$ $t_{def}$) $\iota$ x
  and $t_{def}$ $\iota$ x = $\inbr{gcata_{def}}$ ($c_{def}$ $t_{expr}$ $t_{def}$) $\iota$ x in
  ($t_{expr}$, $t_{def}$)
\end{lstlisting}

����� $c_{expr}$ � $c_{def}$ �������� ������������ ��������, ������� ��������� ��� ��������� ������� �������������� ���� �����, ������� ����������� �� ������� ����������� �����������. �������� ��������, ��� ��� �� ����� ���������� ����������� ����� ����� �������������� ��� ����, ����� ��������������� ����� ���������� �������������� ��� ������� ������� ������������ ����������� �����.

� ����� ��������������� �������� �� ����� ��������������� ���������� �� ��������� ��� ������ ����������� ��������������:

\begin{lstlisting}
let $\inbr{show_{expr}}$, $\inbr{show_{def}}$ =
  $\inbr{fix_{expr,def}}$ (new $\inbr{show\_stub_{expr}}$, new $\inbr{show\_stub_{def}}$) 
\end{lstlisting}

��� �������������� �� ���������, ��-������, ������ ����������� �� ���� ���������� � ����������� � ����� ��� ��������������� �����, � ��-������, ������������ ��� �������� ������� ��������������, � ��������� �����������:

\begin{lstlisting}
class $\inbr{show_{expr}}$ fself = object 
  inherit $\inbr{show\_stub_{expr}}$ fself $\inbr{show_{def}}$ 
end
class $\inbr{show_{def}}$ fself = object 
  inherit $\inbr{show\_stub_{def}}$ $\inbr{show_{expr}}$ fself 
end
\end{lstlisting}

����� �� ����� ������� ������� ����������� ���� ������������ �� ������� (� �������� ����������� �������), ��� ��������� ������������� �������� �������������� �������������� ���� ����� � ������, ��� ��� ���� ������������, �� �� ���������.

� ������ �������, ����� ��������� ��������� ��������������, ������ ���������� ������������� �� \emph{��������������} ������� � ������������ ����������� ���������� ����������� �����.
��� ������ ����������� ���������� ������ �������������� �������� ����� ����� ������, ��� � ��� ���������� ���������� ����:

\begin{lstlisting}
let custom_show, _ =
  $\inbr{fix_{expr,def}}$ ((fun $f_{expr}$ $f_{def}$ ->
                  object inherit $\inbr{show\_stub_{expr}}$ $f_{expr}$ $f_{def}$
                    method $\inbr{Const}$ $\iota$ x n = "a constant"
                  end),
                new $\inbr{show\_stub_{def}}$) 
\end{lstlisting}

� ���������� ���������� ���������� �� ���������� ������������� ���������� ����������� �����, ������� ������� ���� �� �������, ������� ��� ������ � ������� ~\ref{memofix}. � ���� ��, �� ��������� ������ ���������� � ��������� � ����������� � ����, ����� ��� 
���� ``\lstinline{t}'' ���� ���������� ��� ���� ����������� � ������� ��������� 
``\lstinline{fix(t)}''. �������������, ������, �������� ������� � ���, ��� ��� �������� ������� �����������, ����� ��������������� ������������ ���������.

������ ������������ ���� ��������� � ���������� �������� ��������: �� ���������� �� �� ��������, ��� ���������� ����� ������� �������������� ��� ����  ����������, ����� ����������� �������� ��������. ������, ������ ������, ��� �� ���. ��������, ���������� ��������� ���������� ����:

\begin{lstlisting}
type ($\alpha$, $\beta$) a = A of $\alpha$ b * $\beta$ b
and  $\alpha$ b = X of ($\alpha$, $\alpha$) a
\end{lstlisting}

� ���������� ������������ ``\lstinline{A}'' �� ����� \emph{���������} �������������� ���� ``\lstinline{b}'', � ������� ��� ����������� \emph{���} �������~--- ���``\lstinline{$\alpha$ b}'' � ��� ``\lstinline{$\beta$ b}''. ������, ��� ``\lstinline{a}'' �� �������� ����������~--- ����� �������������� ���� ``\lstinline{($\alpha$, $\beta$) a}'' �� ����� � ������������� �������������� �������� ����� ``\lstinline{($\alpha$, $\alpha$) a}'' � ``\lstinline{($\beta$, $\beta$) a}''.

�������������, �� ��� ������� ����� ���������� �����. ����������, ��� ������� ����������� ���������� ����� �������� \emph{�������������} � ��� ������, ��� ��� �� ������ ����� ���� ��������� �� ��� �� ������� ����������� �����������, � ������, ����� ������ ���� ����� ������� ���������. ���� �� ������� ������ ���������� ����, ������, ��

\begin{lstlisting}
...
and $\alpha$ b = int
\end{lstlisting}

�� �� ������� ���������� �����, ������� �� �������������� � ���. ������, ��� ��� ���� ``\lstinline{a}'' � ``\lstinline{b}''  \emph{�� ��������}
�� ���� ������� ������������, �� �� ����������� ����� ����� ���� ����������, ��� ��� �������� ��������������� ������ �����������.



\subsection{����������� ���������� ����}
\label{pv}

�� ������� ��������� ����������� ���������� �����~\cite{PolyVar,PolyVarReuse} ������ ����� ����� ������, ��� ��� ��� ��������� ����������� 
������������������ ����������� �������� ������ � ����������� ���������� ����������������� ��������������.
������� �������� ����� ������������ ���������� ������  � ���������������, �������� ����������� 
\emph{����������} ����������� ����� ����������� ���������� ����� ���� ���������� ����� ������������� ��� ��������������� ���������� ����� � ����. 

����� ������� �������� ��������������  \emph{���������} ���������� � ����������� �������������. ����� ��������� ����� ����� ��������������, �� ������ �������� ��� ���������� ����������� ������� ������������ ��������������� �����.

��� �� ������� �����, �������������� ��������  ``$\varepsilon$'' ����������� � �������� ������������� ������������ ����������� ����. �������������, ������ ���� ��������� ������������ �� �� ������� ����������� �������������� �� ��� ����� \emph{��������} ����\footnote{�� �������������� �� ������������� ������� ``������'' ��� ��� � \textsc{OCaml} ��� ���������� ��������������.}. 
��� ����� ���� ���������� ������������� ������ ���������� ������� �������������, ������� ���������� ``��������'':

\begin{lstlisting}
let $\inbr{gcata_t}$ $\omega$ $\iota$ subj =
  match subj with
  $\dots$
  | C $\dots$ -> $\omega$#$\inbr{C}$ $\iota$ (match subj with #t as subj -> subj) $\dots$
  $\dots$
\end{lstlisting}

��� ���������� � ���������� ������� �������, ��������������� ��������������, � �������� ������������� ����, � �� ����� ��� ���������� ������� �������������� ��������� ��������� ���.

���� ��������� ����������� ���������� ����� ������������, �� ���������� ������� �������������� ������������ �������� � ���������-������ � �������� ���������� ��������������� ����������� �������� ��������������.



\end{comment}

\section{Обзор похожих решений}
\label{sec:relatedworks}
\section{Обзор похожих решений}
\label{sec:relatedworks}

В данной работе использованы одновременно и функциональные (комбинаторы), и объектно-ориентированные возможности языка \textsc{OCaml}. Можно найти связанные работы  одновременно и в области типизированного функционального и объектно-ориентированного программирования. Наиболее близкой, использующий язык \textsc{OCaml} и имеющей отношение к этой работе, библиотекой является \textsc{Visitors}~\cite{Visitors}, которая использует те же самые идеи, но принимает существенно другие дизайнерские решения. Детальное сравнение с \textsc{Visitors} вы найдете в конце данного раздела.

Во-первых, существует несколько библиотек для обобщенного программирования для \textsc{OCaml}, которые используют полностью генеративный подход~\cite{Yallop,PPXLib}~--- все необходимые обобщенные функции для всех типов генерируются по отдельности. Этот подход очень практичен до тех пор, пока набор предоставляемых преобразований удовлетворяет всем нуждам. Однако, если это не так, необходимо расширить кодовую базу, реализовав все отсутствующие функции заново
(с потенциально очень малым переиспользовыванием кода). К тому же, те функции,
которые получаются в результате, нерасширяемы. В нашем подходе, во-первых,
большое количество полезных обобщенных функций может быть получено из уже сгенерированных. Во-вторых, чтобы получить полностью новый плагин, достаточно модифицировать только ``интересные'' части, так как функции обхода и класс для объекта преобразования библиотека создает самостоятельно.

Несколько подходов для функционального обобщенного программирования используют 
\emph{представление типов}~\cite{Hinze}. В основе лежит идея разработки универсального представления для произвольного типа, преобразования которого необходимо получить, и предоставления двух функций, выполняющих преобразование в универсальное представление и обратно, и в идеале образующих изоморфизм. Обобщенные функции преобразуют представление исходных типов данных, что позволяет реализовать все необходимые преобразования один раз. Функции трансляции в универсальное представление и обратно могут быть получены (полу)автоматически, используя такие особенности системы типов  как классы типов~\cite{Hinze,ALaCarte} и семейства типов~\cite{InstantGenerics} в языке \textsc{Haskell}, или  используя синтаксические расширения~\cite{GenericOCaml} в языке \textsc{OCaml}. Хотя некоторые из этих подходов позволяют модификацию получаемых преобразований (например, обработка некоторых типов особым образом) и поддерживают расширяемые типы, наш подход более гибок, так как позволяет модификацию на уровне отдельных конструкторов. К тому же, мы позволяем сосуществовать нескольким видам преобразований для одного типа.

Другой подход был задействован в ``Scrap Your Boilerplate''~\cite{SYB} (для краткости SYB), изначально разработанного для языка \textsc{Haskell}. Он делает возможным реализовать преобразования,  которые обнаруживают вхождения конкретного типа в произвольной структуре данных. Поддерживаются два основных вида действий: \emph{запросы}, которые выбирают значения конкретного типа данных на основе критериев, заданных пользователем, и \emph{преобразования}, которые единообразно применяют преобразование, сохраняющее тип, в конкретной структуре данных. В последующих статьях этот подход был расширен для трансформаций, которые обходят пару структур данных одновременно~\cite{SYB1}, а также поддержкой расширения уже существующих преобразований новыми случаями~\cite{SYB2}. Позднее, данный подход был реализован в других языках, включая \textsc{OCaml}~\cite{SYBOCaml,Staged}. В отличие от нашего случая, SYB позволяет применять трансформации к конкретным типам целиком, а не отдельным конструкторам. К тому же, многообразие получающихся преобразований выглядит достаточно ограниченным. Также, потенциально, преобразования в SYB-стиле могут сломать барьер инкапсуляции, так как могут обнаруживать вхождения значений нужно типа в структуре данных \emph{произвольного} типа. Таким образом, поведение зависит от особенностей внутренней реализации структуры данных, даже от тех, что были скрыты при инкапсуляции. Это может привести, во-первых, к возможности нежелаемой обратной разработки (reverse engineering) путём применения различных чувствительных к типу, преобразований и анализа результатов. Во-вторых, к ненадежности интерфейсов: после модификации структуры данных реализация обобщенной функции для \emph{старой} версии всё ещё может быть применена без статических или динамических ошибок, но с неправильным (или нежелательным) результатом.

Существует определенное сходство между нашим подходом и \emph{алгебрами объектов}~\cite{ObjectAlgebras}. Алгебры объектов были предложены как решение проблемы выражения (expression problem) в распространенных объектно-ориентированных языках  (\textsc{Java}, \textsc{C++}, \textsc{C\#}), которые не требуют продвинутых особенностей системы типов кроме наследования и шаблонов. В оригинальном представлении алгебры объектов были преподнесены как шаблон проектирования и реализации; в последующих работах изначальная идея была улучшена различными способами~\cite{ObjectAlgebrasAttribute,ObjectAlgebrasSYB}. При использовании алгебр объектов преобразуемая структура данных также кодируется с использованием идеи ``методы и варианты (конструкторы) один к одному'', которая предоставляет расширяемость в обоих направлениях, а также ретроактивную реализацию. Однако, будучи  разработанной для совершенно другого языкового окружения, решение с использование алгебр объектов существенно отличается от нашего. Во-первых, с использованием алгебр объектов ``форма'' структуры данных должна быть представлена в виде обобщенной функции, которая принимает конкретный экземпляр алгебры объектов как параметр (кодирование Чёрча для типов~\cite{Hinze}). Применяя данную функцию к различным реализациям алгебры объектов можно получать различные преобразования (например, распечатывание). Чтобы инстанциировать саму структуру данных нужно предоставить особый экземпляр алгебры объектов~---~\emph{фабрику}. Однако, после инстанциации структура данных больше не может быть трансформирована обобщенным образом. Следовательно, алгебры объектов заставляют пользователя переключиться на представление данных с помощью функций, которое может быть, а может не быть удобно в зависимости от обстоятельств.  Наш же подход недеструктивно добавляет новую функциональность к уже знакомому миру алгебраических типов данных, сопоставления с образцом и рекурсивных функций. Обобщенные реализации преобразований полностью отделены от представления данных и пользователи могут свободно преобразовывать их структуры данных привычным способом  без потери возможности объявлять (и расширять) обобщенные функции. Другой особенностью OCaml, в отличии от распространенных языков объектно-ориентированного программирования, является то, что для написания расширяемого кода в основном используются полиморфные вариантные типы, а не классы. Поддержка полиморфных вариантных типов для написания расширяемых типов данных требует нового подхода.


Итого, среди уже существующих библиотек для обобщенного программирования для \textsc{OCaml} мы можем называть две, которые напоминают нашу: \cd{ppx\_deriving}/\cd{ppx\_traverse}, последняя версия которых находится в кодовой базе \cd{ppxlib}~\cite{PPXLib}, и \textsc{Visitors}~\cite{Visitors}.

\cd{ppx\_deriving} является наипростейшим подходом: объявления типов данных отображаются один к одному в рекурсивные функции, представляющие конкретный вид преобразования. Это наиболее эффективная реализация, так как функции вызываются напрямую, без позднего связывания, но нерасширяемая. Если пользователю понадобится слегка модифицировать сгенерированную функцию, то он должен будет полностью скопировать реализацию функции и изменить её. Количество работы по программированию нового преобразования может существенно увеличиться, если тип данных будет видоизменяться во время цикла разработки.

В \cd{ppx\_traverse} расширяемые трансформации также представлены как объекты. В отличие от нашего подхода, там не используется кодирование конструкторов и методов один к одному. К тому же \cd{ppx\_traverse} не использует наследуемые атрибуты, следовательно некоторые преобразования, такие как проверка на равенство и сравнение, невыразимы.

\textsc{Visitors}, с другой стороны, использует сходный с нашим подход, в котором были приняты многие решения, отвергнутые нами, и наоборот.
Ниже мы подытожим главные различия:

\begin{itemize}
   \item \textsc{Visitors} полностью объектно-ориентированы. Чтобы воспользоваться преобразованием необходим создать некоторый объект и вызвать нужный метод. В нашем случае, если используются возможности, предусмотренные заранее, то можно использовать более естественный комбинаторный подход.
     
   \item \textsc{Visitors} реализуют некоторое количество преобразований в специфичной ad-hoc манере. В нашем случае все преобразования принадлежат некоторой обобщенной схеме. Различные трансформации можно скомбинировать с помощью наследования, если типы в схеме унифицируются. Мы также заявляем, что в нашей библиотека реализация ползовательски плагинов с трансформациями проще. 
     
   \item Как и  SYB, \textsc{Visitors} поддерживают указание способа трансформации для входящих в структуру данных типов: для каждого типа присутствует метод в объекте, представляющий трансформацию. Хотя такое представление добавляет некоторой гибкости мы осознанно отказывается от него, так как оно позволяет преодолеть инкапсуляционный барьер: изменяя методы преобразования (которые не могут быть скрыты в сигнатуре), можно получить некоторую информацию об внутреннем реализации инкапсулированной структуры данных. Более того, абстрактные структуры данных могут быть изменены способом, не предусмотренным публичным интерфейсом

   \item В нашем случае типовые параметры классов, представляющих трансформацию, должны быть указаны пользователем. В \textsc{Visitors} это работа возлагается на плечи компилятора, с помощью оригинального трюка. Однако, он не позволяет использовать \textsc{Visitors} в сигнатурах модулей. В нашем случае нет никаких проблем: поддерживается работа и с файлами реализации, и с файлами сигнатур.

   \item \textsc{Visitors} на сегодняшний день\footnote{Последней доступной версией на данный момент является 20180513.} не поддерживает полиморфные вариантные типы.
   
   \item \textsc{GT} поддержает произвольные применения конструкторов типов, а  \textsc{Visitors} и в мономорфном, и в полиморфном режиме -- нет.
     Например, данный пример не компилируется:
     
   \begin{lstlisting}
   type ('a,'b) alist = Nil | Cons of 'a * 'b
   [@@deriving visitors { variety = "map"
                        ; polymorphic = true }]

   type 'a list = ('a, 'a list) alist
   [@@deriving visitors { variety = "map"
                        ; polymorphic = false }]
   \end{lstlisting}
   
   Более того, добавление искусственного конструктора не решает проблему:
   
   \begin{lstlisting}
   type 'a list = L of ('a, 'a list) alist [@@unboxed]
   [@@deriving visitors { variety = "map"
                        ; polymorphic = false }]
   \end{lstlisting}
    
    Также присутствуют сложности с переименованиями (aliases) типов в полиморфном режиме (мономорфная часть библиотеки \textsc{Visitors} компилируется успешно):
    
    \begin{lstlisting}
    type ('a,'b) t = Foo of 'a * 'b (* OK *)
    [@@deriving visitors { variety = "map"
                         ; polymorphic = true }]

    type 'a t2 = ('a, int) t
    [@@deriving visitors { variety = "map"; name="somename"
                         ; polymorphic = true }]
    \end{lstlisting}
    
    Сгенерированный код можно исправить вручную, путём удаления типовых аннотаций для явного полиморфизма (explicit polymorphism) у методов, что приведет к коду, который очень напоминает генерируемый  \textsc{GT}. Из этого мы можем заключить, что на \textsc{GT} можно смотреть как перереализацию полиморфного режима библиотеки  \textsc{Visitors}, где большее количество объявлений типов компилируется корректно.
    
\end{itemize}


\section{Реализация и примеры}
\label{sec:Evaluation}

Представленный метод был реализован в библиотеке Generic Transformers\footnote{\url{https://github.com/Kakadu/GT/tree/v0.3.0}} (\GT). Библиотека  поддерживает два наиболее распространенных вида синтаксических расширений языка \ocaml{}: \textsc{PPX}~\cite{PPXLib} и \camlpfive~\cite{camlp5}. Библиотека \GT{} является расширяемой: к ней прилагается интерфейс для добавления пользовательских плагинов, реализующих порождение новых видов преобразований, а также набор стандартных преобразований, использующихся в других подобных библиотеках.


В этом разделе мы представим несколько примеров, реализованных с помощью нашего подхода. В них используются синтаксические расширения \camlpfive{}, но это же может быть реализовано также с использованием \PPX{}. %Данная работа является прямым наследником~\cite{SCICO} и все примеры из той статьи работают и в этой версии. 

\subsection{Рассмотренные примеры}

Сначала (раздел~\ref{sec:lists}) мы продемонстрируем совместимость 
нашего подхода для \emph{типов данных с неограниченной рекурсией}
(ключ компилятора \texttt{-rectypes}), 
реализовав
представление логических значений, использующихся в  библиотеке реляционного программирования \OCanren{}~\cite{OCanren}. %Особенностью данного примера является то, что в нём используется относительно редкий ключа компилятора \texttt{-rectypes} для объявления типов данных. Не всякое представление объектов-преобразований будет работать с типами данных, для которых необходим этот ключ компиляции.

Затем мы решим <<The Expression Problem>>~\cite{ExpressionProblem}
%, с помощью полиморфных вариантных типов и нашего подхода
 (раздел~\ref{sec:nameless}). Эта задача часто используется как ``лакмусовый тест'' для оценки подходов к обобщенному программированию~\cite{ObjectAlgebras,ALaCarte}. В литературе встречается различные подходы к решению этой задачи, но наша реализация данного примера интересна тем, что использует  и обобщенное программирование, и \emph{полиморфные вариантные типы} языка \OCaml{}.

Эти два примера демонстрируют использование предложенного метода с теми типами данных \OCaml{}, которые в данный момент не поддерживаются другими подходами к построению расширяемых преобразований (в частности, \visitors~\cite{Visitors}). 
Также данные примеры помогут обосновать наш дизайн  интерфейса объектов, а именно ответить на вопрос: <<Эффективно ли  на практике кодирование конструкторов один к одному в методы объектов, позволяет ли оно описывать достаточно разнообразные расширяемые преобразования?>>.

%Убрать то, что справа:
%В нём мы будем отдельно описывать преобразования для различных частей языка, а потом объединять их с помощью наследования. Это пример также не может быть переписан с использованием \visitors{}, так как та не поддерживает полиморфные вариантные типы языка \ocaml{}.

В разделе~\ref{sec:irregular} мы обсудим работу с нерегулярными типами данных~\cite{irregular}, которые не поддерживаются нашим методом непосредственно, а требуют некоторого изменения способа объявления типов данных, чтобы наш метод был применим. %но поддерживаются при использовании \visitors{}.

% нерегулярными типами данных~\cite{irregular}
% вместо "нерегулярными~\cite{irregular} типами данных"
% потому что термин разрывается 

%Кроме того, перечисленные выше примеры важны для сравнения с подходом~\cite{Visitors}, который  в данный момент не поддерживает использование ключа \texttt{-rectypes} и полиморфные вариантные типы. 



В разделе \ref{sec:design} мы коснемся вопросов дизайна расширяемых преобразований, которые могут упростить или усложнить использование  разработчиками полученных преобразований. В заключительном разделе \ref{sec:performance} коснёмся вопросов производительности.

%TODO: вообще сравниваться непосредственно с Visitors не хорошо, будет выглядет как мелкое улучшение примеров.

%Может быть другого сорта мотивацию, сказать почему примеры важны (а в конце, "кроме того, примеры важны для сравнения с взиторами, потому что...)

%\subsection{Research Questions}
%Данные примеры помогут ответить на исследовательский вопрос:

%\textcolor{red} {Это не вопрос!!!} Также важно ответить на вопрос: <<Является ли предоставляемый интерфейс достаточно знакомым для практикующего разработчика на \ocaml{}?>>

%TODO: Тут сказать почему они важны

%\parbox{\textwidth}{
%\textcolor{blue} {Запихнуть RQ в предыдущий раздел}
%}

\subsection{Типизированные логические значения}
\label{sec:lists}

Этот пример появился во время работы над строго типизированным встроенным логическим предметно-ориентированным языком на основе \textsc{OCaml}~\cite{OCanren}. 
В нём одной из самых важных конструкций является унификация термов, содержащих свободные логические переменные. Работать с такими структурами данных сложно, а допустить ошибку --- легко. 
Типичным сценарием взаимодействия  
%между логическими и нелогическими (\textcolor{red}{ПЕРЕФРАЗИРОВАТЬ})  
% частями программ 
cо встроенным языком 
является 
создание так называемых \emph{целей вычислений} (goal), содержащих структуры данных со свободными логическими переменными.
Решением логической цели является подстановка переменных, правые части которой в идеальном случае не содержат свободных переменных. 
Чтобы сконструировать цель вычислений необходимо уметь систематически вводить логические переменные в типизированную структуру данных,  а для восстановления ответа -- систематически извлекать из представления, подходящего для работы с \OCanren{}, ответы в обыкновенном
%нелогическом(\textcolor{red}{ПЕРЕФРАЗИРОВАТЬ})  
представлении (т.е. без логических переменных).

Упрощенный тип для логических переменных может быть описан следующим образом:

\begin{lstlisting}
@type 'a logic =
| V     of int
| Value of 'a       with show
\end{lstlisting}
Логическое значение может быть либо свободной логической переменной (``\lstinline{V}'') или каким-то другим значением (``\lstinline{Value}''), которое не является свободной переменной, но потенциально может содержать свободные переменные.
\begin{comment}

Чтобы преобразовывать в и из логических значений, можно воспользоваться следующими функциями:

\begin{lstlisting}
let lift x = Value x

let reify  = function
| V     _ -> invalid_arg "Free variable"
| Value x -> x
\end{lstlisting}

Функция ``\lstinline{reify}'' бросает исключение для свободных переменных, так как в присутствии вхождений свободных переменных
логическое значение нельзя рассматривать как обыкновенную (нелогическую) структуру данных.
\end{comment}


Когда мы работем с логическими структурами данных, нам необходима возможность вставлять логические переменные в произвольные позиции.
Это означает, что мы должны использовать другой тип данных, подходящий для использования 
с точки зрения системы типов. Например,
для списков нам придется абстрагироваться от рекурсии, чтобы иметь возможность описать тип логических списков \lstinline{llist}\footnote{Этот способ применим только при использовании ключа компиляции \texttt{-rectypes}.}:

\begin{lstlisting}
type ('a, 'self) list_like = 
    | Nil 
    | Cons of 'a * 'self
type 'a list = ('a, 'a list) list_like
type 'a llist = 
    ('a, 'a llist) list_like logic
\end{lstlisting}
%которые будут иметь тип ``\lstinline{lexpr}'', объявленный как
%
%\begin{lstlisting}
%type expr' = Var of string logic | Const of int logic 
%           | Binop of lexpr * lexpr
%and  lexpr = expr' logic
%\end{lstlisting}

Если мы захотим, чтобы списки типа \lstinline{llist} без логических значений преобразовывались в строковое представление также, как списки типа \lstinline{list}, необходимо модифицировать преобразование типа \lstinline{logic} в строку, убрав название конструктора \lstinline{Value}:

\begin{lstlisting}
class ['a, 'self] my_show fa fself = 
object
  inherit ['a, 'self] $\inbr{show_{logic}}$ fa fself
  method c_Value () _ x = fa () x
end
\end{lstlisting}
В такой реализации преобразования логических значений, где мы изменили только один конструктор, мы можем объявить тип логических списков заново, и получить для него преобразование в строку, которое на списках без переменных работает так же, как и для типа \lstinline{list}.

Особенностью данного подхода является, во-первых, получение нового преобразования в строку путём изменения одного метода, а, во-вторых, способ объявления типов \lstinline{list} и \lstinline{llist}, который не удается переиспользовать при использовании подхода, предоставляемого \visitors{}.

%Нам также нужно реализовать две функции преобразования. Все эти определения представляют собой типичный пример однотипного (boilerplate) кода.
%
%С изпользованием нашего подхода решение почти полностью декларативно\footnote{При условии включения ключа компиляции \cd{-rectypes}}.
%Во-первых, мы абстрагируемся от интересующего нас типа, заменяя все его вхождения типовой переменной с не встречающимся ранее именем:

%\begin{lstlisting}
%@type ('string, 'int, 'expr) a_expr =
%| Var   of 'string
%| Const of 'int
%| Binop of 'string * 'expr * 'expr with show, gmap
%\end{lstlisting}
%
%Здесь мы абстрагировали тип от всего конкретного, но мы могли обойтись абстрагированием только от самого себя. Заметьте, что 
%мы воспользовались двумя видами обобщенных преобразований~--- ``\lstinline{show}'' и ``\lstinline{gmap}''. 
%Первое будет полезно для отладочных целей, а второе является необходимым для нашего решения.
%
%Теперь мы можем объявить логические и нелогические составляющие как специализации исходного типа:
%
%\begin{lstlisting}
%@type expr  = (string, int, expr) a_expr 
%  with show, gmap
%@type lexpr = (string logic, int logic, lexpr) a_expr logic 
%  with show, gmap
%\end{lstlisting}

%Обратите внимание, что ``новый'' тип ``\lstinline{expr}'' эквивалентен старому, следовательно, такое переписывание типов не нарушает существующий код.
%
%Наконец, определения функций преобразования воспользуются преобразованием, полученным с помощью плагина ``\lstinline{gmap}'', предоставляемого библиотекой:
%
%\begin{lstlisting}
%let rec to_logic   expr = gmap(a_expr) lift  lift  to_logic  expr
%let rec from_logic expr = gmap(a_expr) reify reify from_logic @@ 
%                           reify expr
%\end{lstlisting}
%
%Как вы видите, поддержка типовых операторов существенна для этого примера. В предыдущей реализации~\cite{TransformationObjects} типовые операторы не были поддержаны и их было не так просто добавить.

\subsection{Преобразование в безымянное представление}
\label{sec:nameless}

Полиморфные вариантные типы в языке \ocaml{} позволяют описывать структуры данных композиционально, статически типизировано и в разных модулях компоновки~\cite{PolyVarReuse}.
Целесообразно объявлять преобразования таких структур данных отдельно друг от друга. Задача конструирования преобразований для 
раздельно объявленных и строго типизированных компонент известна как ``проблема выражений'' (``The Expression Problem''~\cite{ExpressionProblem}).
%которая часто используется (\textcolor{red}{Убрать в 5.1}) как ``лакмусовый тест'' для оценки подходов к обобщенному программированию~\cite{ObjectAlgebras,ALaCarte}. 
В этом подразделе мы представим решение этой задачи в рамках нашего подхода. В качестве конкретной задачи мы реализуем преобразование $\lambda$-выражений в безымянное представление.

Во-первых, опишем часть языка выражений без связывающих конструкций:

\begin{lstlisting}
@type ('name, 'lam) lam = 
[ `App of 'lam * 'lam
| `Var of 'name
] with show
\end{lstlisting}

\noindent Выделение этого типа выглядит логично, так как 
кроме указанных двух конструкций, потенциально в языке могут появиться другие, которые будут связывать переменные 
($\lambda$-абстракции, \lstinline{let}-определения и т.д.). Комбинируя различные типы и преобразования этих типов, можно получать различные расширения деревьев абстрактного синтаксиса и преобразований для $\lambda$-выражений.
%, их с несвязывающими конструкциями, а также с ними самими, можно получать различные языки с согласованным поведением \textcolor{red}{ПЕРЕФРАЗИРОВАТЬ}.

Введенный выше тип ``\lstinline{lam}'' является полиморфным: первый параметр используется для представления имен или индексов %(или уровней) 
де Брёйна\footnote{Способ представления лямбда-выражений в безымянном виде предложенный де Брёйном в~\cite{deBruijn}.}, второй необходим для открытой рекурсии (здесь мы следуем  подходу к описанию расширяемых структур данных с помощью полиморфных 
вариантных типов~\cite{PolyVarReuse}).

Для данного типа преобразование в безымянное представление можно определить следующим образом:
%Рассмотрим как для такого типа должны выглядеть преобразование в безымянное представление, а именно, как должен выглядеть класс преобразования.
%Как должно выглядеть преобразование в безымянное представление для такого типа? А именно, как должен выглядеть класс преобразования? Это показано ниже:

\begin{lstlisting}
class ['lam, 'nless] lam_to_nameless
 (flam : string list -> 'lam -> 'nless) =
object
  inherit 
    [ string list, string, int
    , string list, 'lam, 'nless
    , string list, 'lam, 'nless] $\inbr{lam}$
  method $\inbr{App}$ env _ l r = 
    `App (flam env l, flam env r)
  method $\inbr{Var}$ env _ x   = `Var (index env x)
end
\end{lstlisting}

% TODO: Здесь у нас нет call-by-value, поэтому это нифига не интерпретатор

\noindent Здесь мы используем список строк для хранения подстановки переменных и  передаем его как наследуемый атрибут. Затем мы пользуемся функцией 
``\lstinline{index}'' чтобы найти строку в подстановке, т.е.  эта функция преобразует имя в индекс де Брёйна. 
Интересной частью преобразования является типизация общего класса предка ``$\inbr{lam}$''. 
Первая тройка параметров описывает преобразование первого типового параметра. Можно заметить, что мы преобразуем строки в числа используя подстановку.
Здесь типовая переменная ``\lstinline{'lam}'', 
%как мы знаем, 
приравнивается (раздел~\ref{pv}) открытой версии типа ``\lstinline{lam}''. %(ДОИСПРАВИТЬ)
Наконец, результат преобразования типизируется с помощью переменной ``\lstinline{'nless}'', введение которой необходимо для правильной реализации преобразования объединения типов.
%Так происходит именно так потому, что, как будет понятно позднее,  это будет действительно другой тип. (\textcolor{blue}{Сказать прямее, может даже лишнее предложение})
Так как второй типовый параметр обычно ссылается рекурсивно на себя, третья тройка типовых параметров совпадает со второй.

Давайте теперь добавим в язык связывающую конструкцию --- $\lambda$-абстракцию:

\begin{lstlisting}
@type ('name, 'lam) abs = 
  [ `Abs of 'name * 'lam ] with show
\end{lstlisting}

Те же самые рассуждения применимы и тут: мы пользуемся открытой рекурсией и параметризируем представление относительно имени.
Класс для преобразования будет выглядеть похожим образом:

\begin{lstlisting}
class ['lam, 'nless] abs_to_nameless
 (flam : string list -> 'lam -> 'nless) =
object
  inherit [string list, string, int
          , string list, 'lam, 'nless
          , string list, 'lam, 'nless] $\inbr{abs}$
  method $\inbr{Abs}$ env name term = 
    `Abs (flam (name :: env) term)
end
\end{lstlisting}

Заметьте, что метод ``$\inbr{Abs}$'' конструирует значения \emph{другого} типа, чем любая возможная параметризация типа ``\lstinline{abs}''. Действительно, безымянное представление типа не должно содержать никаких суррогатов имён.

Теперь мы можем объединить эти два типа, чтобы получить тип выражений со связывающими конструкциями.

\begin{lstlisting}
@type ('name, 'lam) term = 
  [ ('name, 'lam) lam 
  | ('name, 'lam) abs) ] with show
\end{lstlisting}

Представим два новых типа для именованного и безымянного представления\footnote{Для того чтобы эти определения типов скомпилировались, необходимо использовать ключ компиляции \cd{-rectypes}.}:

\begin{lstlisting}
@type named = (string, named) term 
  with show
@type nameless = 
  [ (int, nameless) lam | `Abs of nameless] 
  with show
\end{lstlisting}

Наконец, мы можем описать преобразование, которое превращает именованные термы в их безымянное представление:

\begin{lstlisting}
class to_nameless
(f : string list -> named -> nameless) = 
object
 inherit 
   [string list, named, nameless] $\inbr{named}$
 inherit 
   [named, nameless] lam_to_nameless f
 inherit 
   [named, nameless] abs_to_nameless f
end
\end{lstlisting}

Это преобразование получается путём наследования поределеннных выше компонент: общего класса для всех преобразований типа ``\lstinline{named}'' 
и двух конкретных преобразований его составляющих: 
``\lstinline{lam_to_nameless}'' и ``\lstinline{abs_to_nameless}''.
Функция-преобразование может быть получена стандартным способом:

\begin{lstlisting}
let to_nameless term =
  transform(named) 
    (fun fself -> new to_nameless fself) 
    [] 
    term
\end{lstlisting}

Только что мы построили реализацию преобразования типа, комбинируя реализации преобразований его составляющих. Эти  реализации могут быть раздельно скомпонованы, но вся система при этом останется строго типизированной. В этом примере демонстрируются возможности подхода по раздельному и модульному представлению преобразований с помощью объектов, а также возможности по использованию полиморфных вариантных типов языка \ocaml{}, которые не доступны в подходе \visitors{}.

\subsection{Нерегулярные типы данных}
\label{sec:irregular}

Основным достоинством подхода \visitors{} является поддержка нерегулярных типов данных с некоторой оговоркой: поддерживаются преобразования в так называемом ``полиморфном режиме''~\cite{Visitors}. Наш метод не позволяет построить преобразования для уже описанных нерегулярных типов данных. Однако, если разработчик проектирует типы с нуля, то у него есть возможность описать их так, чтобы они были регулярными и были совместимы с нашим подходом. 

Рассмотрим объявления нерегулярного типа данных  из работы~\cite{irregular}.

\begin{lstlisting}
type 'a tree = N | C of 'a * ('a * 'a) tree
\end{lstlisting}
\noindent Для этого типа метод на основе \GT{} не сможет построить преобразование, так в языке \ocaml{} не поддерживается нерегулярная типизация объектов. Необходимо переписать это тип, абстрагировавшись от вхождения типа \lstinline{'a * 'a}, и описать тип \lstinline{tree_list}, и уже с помощью него описать  необходимый тип \lstinline{tree} (потребуется использование ключа компилятора \texttt{-rectypes}).
\begin{lstlisting}
type ('a, 'b) t = N | C of 'a * ('a, 'b)
type 'a tree = ('a, 'a * 'a) t 
\end{lstlisting}
\noindent Для этих двух типов метод уже сможет построить требуемые преобразования.

\subsection{Особенности дизайна}
\label{sec:design}

В данном разделе мы рассмотрим пример построения расширяемых преобразований с помощью \visitors~\cite{Visitors}, ещё одного подхода по построению раширяемых преобразований для \OCaml{} с помощью объектов.

\begin{lstlisting}
(struct 
  type 'a menu = ('a * int) list
  [@@deriving visitors 
    { variety = "map"
    ; name = "map_menu"
    ; polymorphic = true }]
  ...
end : sig 
  type 'a menu 
  val add_exn
    : 'a menu -> 'a -> int -> 'a menu
    
  class virtual ['c] map_menu : object ('c)
    constraint 'c = ...
    method private visit_int : 
      'env. 'env -> int -> int
    method visit_menu :
      ('env -> 'a -> 'b) -> 
      'env -> 'a menu -> 'b menu
    ...
end)
\end{lstlisting}

\noindent Вы видите тип данных \lstinline{'a menu}, аннотированный для использования \visitors{}, который реализует ресторанное меню как список блюд и цен. Предположим, разработчик решил скрыть детали реализации меню, объявив тип абстрактным и предоставив функцию добавления в меню, которая, например, в случае указания отрицательной цены, приводит к исключительной ситуации.

Мы хотим обратить внимание на несколько недостатков данной реализации, которые отсутствуют в методе, предложенном в данной работе.

Во-первых, объект-преобразование реализует приватный метод \lstinline{visit_int}, для преобразования чисел, который разрешается переопределять при наследовании. Это позволяет определять преобразования, которые нарушают внутреннюю целостность типа \lstinline{'a menu}, например, которые создают пункты меню с отрицательными ценами. Данный недостаток хорошо известен среди исследователей обобщенного программирования~\cite{SYB} для языка \haskell{}, и порицается~\cite{SafeHaskell}.

Во-вторых, класс-преобразование кодирует свой интерфейс в единственном типовом параметре с помощью ключевого слова \lstinline{constraint}. Преимуществом такого кодирования является некоторое сокращение порождаемого объема кода. Недостатком является невозможность породить тип объекта преобразования в файлах-интерфейсах языка OCaml{}, что заставляет выписывать типы вручную. Стандартные~\cite{ppxderiving} подходы к обобщенному программированию, как и предложенный в данной работе метод, не страдают от этого недостатка.

В-третьих, порожденный с помощью \visitors{} объект-преобразование, является виртуальным классом, и поэтому не готов к немедленному использованию пользователем. Другие подходы к обобщенному программированию сразу предоставляет готовые к использованию функции-преобразования.




\subsection{Производительность}
\label{sec:performance}

При сравнении преобразований, реализованных в традиционном и расширяемом виде,
следует ожидать от первых большей производительности, так как дополнительный слой абстракции вносит некоторые накладные расходы.

При замерах использовались преобразования, реализованные четырьмя способами.
\begin{enumerate}
\item Нерасширяемые преобразования записанные традиционным способом с помощью рекурсивных функций. От них следует ожидать максимальную производительность.
\item Частично расширяемые преобразования, реализованные c помощью записей в стиле библиотеки \cd{ppx\_deriving\_morphism}. Этот  подход вносит накладные расходы на косвенный вызов преобразований подвыражений.% и неприменим, например, для полиморфных вариантых типов. 
\item Расширяемые преобразования в стиле \visitors{}, т.е. с использованием объектов. Вносит некоторые накладные расходы на косвенный вызов из-за таблицы виртуальных методов.
\item Метод из данной работы, сходный с предыдущим, но где обобщенная функция преобразования (раздел~\ref{transtypes}) реализована отдельно от объекта.
\end{enumerate}

Для замеров были выбраны два вида преобразований: копирование выражения и преобразование в текстовый формат.
Данные преобразования применялись к $\lambda$-выражению, которое состоит из некоторого количества $\lambda$-абстракций,  примененных к тождественной функции. Количество таких абстракций определяет размер выражения.

При замерах производительности были реализованы четыре метода, описанные выше. В таблице \ref{tab:caption} в процентах указана производительность методов относительного самого медленного (больше --- лучше), который обозначается прочерком.



\begin{table*}[t]
  \centering
  \begin{tabular}{l ccccc}
    \toprule
    \multirow{2}{*}{Вид преобразования}& \multirow{2}{*}{Размер} & \multicolumn{4}{c}{Метод реализации и улучшение (\%)} \\\cline{3-6}
     & & \GT & \visitors & \PPXMorphism  & Default \\\hline
    \multirow{5}{*}{Копирование}  
      & 300  & --  & 149 & 220  & 311 \\
      & 500  & --  & 146 & 218  & 246 \\
      & 700  & -- & 143 & 212  & 244 \\
      & 900  & --  & 141 & 210  & 243 \\
      & 1000 & --  & 140 & 205  & 233 \\\hline
    \multirow{5}{*}{Форматирование}  
      & 300  & 0  & -- & 1  & 3 \\
      & 500  & 0  & -- & 1  & 4 \\
      & 700  & 1  & -- & 1  & 4 \\
      & 900  & -- & 1  & 2  & 3 \\
      & 1000 & -- & 0  & 1  & 3 \\ 
    \bottomrule
  \end{tabular}
\caption{Caption below table.}
\label{tab:caption}
\end{table*}


\begin{comment}

\subsection{Пример пользовательского плагина}
\label{pluginExample}

Наконец, мы продемонстрируем использование системы плагинов на свежем примере реализации плагина. Для этой цели мы выбрали широко известное преобразование \emph{hash-consing}~\cite{HC}. Это преобразование превращает структуры данных в их максимально компактное представление в памяти, при котором структурно равные части представляются в памяти как один физический объект. Например, синтаксическое дерево выражения

\begin{lstlisting}
let t =
  Binop ("+",
    Binop ("-",
      Var "b",
      Binop ("*", Var "b", Var "a")),
    Binop ("*", Var "b", Var "a"))
\end{lstlisting}
может быть переписано  как

\begin{lstlisting}
let t =
  let b  = Var "b" in
  let ba = Binop ("*", b, Var "a") in
  Binop ("+", Binop ("-", b, ba), ba)  
\end{lstlisting}
где равные подвыражения представляются как равные поддеревья.
 
Наш плагин по типу  ``\lstinline|$\left\{\alpha_i\right\}$ t|'' предоставит функцию для 
hash-consing ``\lstinline{hc(t)}'' с сигнатурой 

\begin{lstlisting}
$\{$ H.t -> $\alpha_i$ -> H.t * $\alpha_i$ $\}$ -> H.t -> $\left\{\alpha_i\right\}$ t -> H.t * $\left\{\alpha_i\right\}$ t
\end{lstlisting}
где ``\lstinline{H.t}''~--- это гетерогенная хэш таблица для произвольных типов. Интерфейс у неё следующий:

\begin{lstlisting}
module H : sig
  type t
  val hc : t -> 'a -> t * 'a
end
\end{lstlisting}

Функция  ``\lstinline{H.hc}'' принимает хэш таблицу и некоторое значение и возвращает потенциально обновленную хэш таблицу и значение, которое структурно эквивалентно поданному на вход. Мы не будет описывать реализацию этого модуля, а приведем пример использования в конструкторе:

\begin{lstlisting}
method $\inbr{Binop}$ h _ op l r =
  let h, op = hc(string) h op in
  let h, l  = fself h l in
  let h, r  = fself h r in
  H.hc h (Binop (op, l, r))
\end{lstlisting}

Этот метод принимает как наследуемый атрибут хэш таблицу ``\lstinline{h}'', преобразуемое целиком, которое здесь не потребуется; три аргумента конструктора:наследуемыех и синтезированных атрибутов:


Здесь мы предполагаем, что тип ``\lstinline{ht_typ}'' объявлен как

\begin{lstlisting}
let ht_typ ~loc =
  Typ.of_longident ~loc (Ldot (Lident "H", "t"))
\end{lstlisting}

Другими словами, мы объявляем, что типом наследуемого атрибут всегда будет 
 ``\lstinline{H.t}'', а типом синтезированного атрибута будет пара
``\lstinline{H.t * t}''.

Следующая группа методов описывает параметры классов плагина:

\begin{lstlisting}
method plugin_class_params tdecl =
  let ps = List.map tdecl.ptype_params 
             ~f:(fun (t, _) -> typ_arg_of_core_type t)
  in
  ps @
  [ named_type_arg ~loc:(loc_from_caml tdecl.ptype_loc) @@
    Naming.make_extra_param tdecl.ptype_name.txt
  ]

method prepare_inherit_typ_params_for_alias ~loc tdecl rhs_args =
  List.map rhs_args ~f:Typ.from_caml
\end{lstlisting}

Первый метод описывает типовые параметры класса плагина: для данного случая это типовые параметры самого объявления типа плюс дополнительный типовый параметр 
``$\varepsilon$''. Второй метод описывает вычисление типовых параметров для применения конструктора типа. В случае, если объявление типа выглядит как 

\begin{lstlisting}
type $\{\alpha_i\}$ t = $\{a_i\}$ tc
\end{lstlisting}

нам необходимо построить реализацию преобразовния для типа ``\lstinline{t}'' 
из реализации оного для типа  ``\lstinline{tc}'', наследуясь от правильного 
инстанциированного соответвующего класса. Для нашего случая класс параметризуется теми же 
типовыми параметрами, что и объяляемый тип, поэтому мы оставляем их как есть.

Последняя группа методов отвечает за генерацию тел методов для тра2нсформаций  конструкторов.
Мы поддерживаем регулярные конструкторы алгебраических типов, где аргументами может быть и кортеж, и запись, а также записи и кортежи на верхнем уроне, преобразование которые, как правило, имеет много общих частией. Всего за это отвечают 4 метода, но здесь мы покажем только один:

\begin{lstlisting}
method on_tuple_constr ~loc ~is_self_rec ~mutual_decls 
                            ~inhe tdecl constr_info ts =
  $\dots$ 
  match ts with
  | [] -> Exp.tuple ~loc [ inhe; c [] ]
  | ts ->
     let res_var_name = sprintf "%s_rez" in
     let argcount = List.length ts in
     let hfhc = Exp.of_longident ~loc (Ldot (Lident "H", "hc")) in
     List.fold_right
       (List.mapi ~f:(fun n x -> (n, x)) ts)
       ~init:$\dots$
       ~f:(fun (i, (name, typ)) acc ->
            Exp.let_one ~loc
              (Pat.tuple ~loc 
                 [ Pat.sprintf ~loc "ht%d" (i+1)
                 ; Pat.sprintf ~loc "%s" @@ res_var_name name])
              (self#app_transformation_expr ~loc
                 (self#do_typ_gen ~loc ~is_self_rec 
                                  ~mutual_decls tdecl typ)
                 (if i = 0 then inhe else Exp.sprintf ~loc "ht%d" i)
                 (Exp.ident ~loc name)
              )
              acc
          )
  $\dots$
\end{lstlisting}

Реализация использует заранее заготовленный метод нашей библиотеки
``\lstinline{self#app_transformation_expr}'', который генерируется применение функции преобразования к соответствующему типу.

Конечной компонентой реализации является сам  модуль ``\lstinline{H}''. Стандартный функтор ``\lstinline{Hashtbl.Make}'' создает хэш таблицы, используя некотрую хэш фукнцию и предикат равенства, предоставленные пользователем. В целом, следуем мы следуем такому соглашению: как хэш фукнцию используем полиморфную ``\lstinline{Hashtbl.hash}'', а качестве равенства используем физическое равенство ``\lstinline{==}''. Однако, присутсвуют две сложности:

\begin{itemize}
\item Так как таблице гетерогенная нам необходимо использовать небезопасное приведение типов ``\lstinline{Obj.magic}''.
\item Наша реализация равенства чуть более сложная, чем обычное ``\lstinline{==}''. Нам необходимо стравнивать верхнеуровневые конструкторы и количества их аргументов  \emph{структурно}, а только затем сравнивать соответствующие аргументы взическим равенством. Технически, мы может считать равными структурно равные значения   \emph{различных} типов.
\end{itemize}

Мы полагаемся здесь на следующее наблюдение: hash-consing корректно использовать тольео для структур данных, которые прозрачны по ссылкам, мы предполагаем что равные структуры данных взаимозаменяемы не смотря на их типы. 

Полную реализацию плагина может быть увидеть в главном репозитории. Она занимает 164 строчки кода, учитывая комментарии и пустые строки.
\end{comment}


\section{Обсуждение результатов}
\label{sec:discussion}
% !TeX encoding = windows-1251
�������� ������� �������� ������ ������� �������� �����, ������������ � ���������� \visitors{}, ������� ����� �������� ������������ �������������� ����� ����-�-������ � ������ ��������, �� ��� ���� ��� ����� ������������ ������ ������.

�������� ������������ �������, ������������� � \visitors{}~\cite{Visitors}, �������� ��������� ������������ ����� ������. ����� � �����������, ��� �������������� ���������� �������� � ����� ������� ����������� \emph{����� ��������� ������������ (explicit polymorphism)\footnote{\url{https://caml.inria.fr/pub/docs/manual-ocaml/polymorphism.html\#s\%3Apolymorphic-recursion}}}. ������, ������ ����������� �� ����������� �������� ������ � ����������� �������� (������ \ref{sec:lists}), � ������� ������ \visitors{} �� �������� ��� \OCanren{}.

��� ������������� ������ ������� ��������� �������������� ��� ������������ ����� �� ���������, �� ���� ��� ���� ����������� ��������� ���, �� �� ����� �������������� ��� �� ������������� ���������. ��� �������� ������� ����������� ���� �� ��������� ����� ������, ��� �������, ��-�����������, ������������ � ������ ������ �����, ����� ��������. �� ����� ����� ��������, ��� � ����� �������� ������������ ���� ������ ����������� �����, ������� �� ������� ��� ���������� ������ ������� �� ����� ������������.

��� ������ ��������� ��������� �������������� ��� ����������� ���������� ����� ����� \ocaml{}, � �� ����� ��� ��� ���� � \visitors{} �� ��������������. �� ������� ���������� \visitors{} � �� ����� �������� ������� �������� ���� ��������� ���� �����.

����������� ������� � ����������� ���������������� �� \ocaml{} ��������� ��������������, �������������� ���������. ��� ������ ������������ �������������� ��� �������, � �� ���� �������� ����� ������ �������-��������������, ������� ��� ������ ������������� ��� �� ��������� ��� ������������, ��� � �����������~\cite{PPXLib,ppxderiving} ������� � ����������� ���������������� �� \ocaml{}. � \visitors{} �������-�������������� �� ���������, � ������� ������������� ����� ������� ����� ����� �������� ��� ������������.

� ����� ������� � �������, �������������� ��������������, ����� ���� ����� ����� ������� ����������, ��� ����������� ������ ������������ ����. � \visitors{} ��� ���������� �������� ����������: ��� �������� ������ ������ �������� ������ �������� ��������� ������ $3\cdot(n+1)$, ��� $n$ -- ��� ���������� ������� ����������. ������, ����� ������� �� ��������� ��������� ��� � ������ ���������� ����� \ocaml{}, ��� ����� ��������� ������������ ����������� ��� ������������ ���������� ������������ �����������.

��� ������������� \visitors{} � ����������� ������� ����� ����� �������������� ������ ��� �������������� ��������� ����� ������, � �� ����� ��� � \GT{} ��� �������� �������� ��� �������� ��������� �������-�������������� �� �������. ����������� ������� � \visitors{} �������� ��������� �������� ����������: ������������ ��������� �������������� �������������� ��������� �����, ��� ������������ ����� �������� ����������, ������� ����������� ��� ���������� �������� ����� ������. � ������ �������, ����� ������ ��������� ����������� ��������������, ��� ������� ������ ���� ������������ ������ \cite{SYB}, ���� �� �������, ��� ���������� ������ ���� �������������� ���������� �����.




%\section{Обзор похожих решений}
\label{sec:relatedworks}

В данной работе использованы одновременно и функциональные (комбинаторы), и объектно-ориентированные возможности языка \textsc{OCaml}. Можно найти связанные работы  одновременно и в области типизированного функционального и объектно-ориентированного программирования. Наиболее близкой, использующий язык \textsc{OCaml} и имеющей отношение к этой работе, библиотекой является \textsc{Visitors}~\cite{Visitors}, которая использует те же самые идеи, но принимает существенно другие дизайнерские решения. Детальное сравнение с \textsc{Visitors} вы найдете в конце данного раздела.

Во-первых, существует несколько библиотек для обобщенного программирования для \textsc{OCaml}, которые используют полностью генеративный подход~\cite{Yallop,PPXLib}~--- все необходимые обобщенные функции для всех типов генерируются по отдельности. Этот подход очень практичен до тех пор, пока набор предоставляемых преобразований удовлетворяет всем нуждам. Однако, если это не так, необходимо расширить кодовую базу, реализовав все отсутствующие функции заново
(с потенциально очень малым переиспользовыванием кода). К тому же, те функции,
которые получаются в результате, нерасширяемы. В нашем подходе, во-первых,
большое количество полезных обобщенных функций может быть получено из уже сгенерированных. Во-вторых, чтобы получить полностью новый плагин, достаточно модифицировать только ``интересные'' части, так как функции обхода и класс для объекта преобразования библиотека создает самостоятельно.

Несколько подходов для функционального обобщенного программирования используют 
\emph{представление типов}~\cite{Hinze}. В основе лежит идея разработки универсального представления для произвольного типа, преобразования которого необходимо получить, и предоставления двух функций, выполняющих преобразование в универсальное представление и обратно, и в идеале образующих изоморфизм. Обобщенные функции преобразуют представление исходных типов данных, что позволяет реализовать все необходимые преобразования один раз. Функции трансляции в универсальное представление и обратно могут быть получены (полу)автоматически, используя такие особенности системы типов  как классы типов~\cite{Hinze,ALaCarte} и семейства типов~\cite{InstantGenerics} в языке \textsc{Haskell}, или  используя синтаксические расширения~\cite{GenericOCaml} в языке \textsc{OCaml}. Хотя некоторые из этих подходов позволяют модификацию получаемых преобразований (например, обработка некоторых типов особым образом) и поддерживают расширяемые типы, наш подход более гибок, так как позволяет модификацию на уровне отдельных конструкторов. К тому же, мы позволяем сосуществовать нескольким видам преобразований для одного типа.

Другой подход был задействован в ``Scrap Your Boilerplate''~\cite{SYB} (для краткости SYB), изначально разработанного для языка \textsc{Haskell}. Он делает возможным реализовать преобразования,  которые обнаруживают вхождения конкретного типа в произвольной структуре данных. Поддерживаются два основных вида действий: \emph{запросы}, которые выбирают значения конкретного типа данных на основе критериев, заданных пользователем, и \emph{преобразования}, которые единообразно применяют преобразование, сохраняющее тип, в конкретной структуре данных. В последующих статьях этот подход был расширен для трансформаций, которые обходят пару структур данных одновременно~\cite{SYB1}, а также поддержкой расширения уже существующих преобразований новыми случаями~\cite{SYB2}. Позднее, данный подход был реализован в других языках, включая \textsc{OCaml}~\cite{SYBOCaml,Staged}. В отличие от нашего случая, SYB позволяет применять трансформации к конкретным типам целиком, а не отдельным конструкторам. К тому же, многообразие получающихся преобразований выглядит достаточно ограниченным. Также, потенциально, преобразования в SYB-стиле могут сломать барьер инкапсуляции, так как могут обнаруживать вхождения значений нужно типа в структуре данных \emph{произвольного} типа. Таким образом, поведение зависит от особенностей внутренней реализации структуры данных, даже от тех, что были скрыты при инкапсуляции. Это может привести, во-первых, к возможности нежелаемой обратной разработки (reverse engineering) путём применения различных чувствительных к типу, преобразований и анализа результатов. Во-вторых, к ненадежности интерфейсов: после модификации структуры данных реализация обобщенной функции для \emph{старой} версии всё ещё может быть применена без статических или динамических ошибок, но с неправильным (или нежелательным) результатом.

Существует определенное сходство между нашим подходом и \emph{алгебрами объектов}~\cite{ObjectAlgebras}. Алгебры объектов были предложены как решение проблемы выражения (expression problem) в распространенных объектно-ориентированных языках  (\textsc{Java}, \textsc{C++}, \textsc{C\#}), которые не требуют продвинутых особенностей системы типов кроме наследования и шаблонов. В оригинальном представлении алгебры объектов были преподнесены как шаблон проектирования и реализации; в последующих работах изначальная идея была улучшена различными способами~\cite{ObjectAlgebrasAttribute,ObjectAlgebrasSYB}. При использовании алгебр объектов преобразуемая структура данных также кодируется с использованием идеи ``методы и варианты (конструкторы) один к одному'', которая предоставляет расширяемость в обоих направлениях, а также ретроактивную реализацию. Однако, будучи  разработанной для совершенно другого языкового окружения, решение с использование алгебр объектов существенно отличается от нашего. Во-первых, с использованием алгебр объектов ``форма'' структуры данных должна быть представлена в виде обобщенной функции, которая принимает конкретный экземпляр алгебры объектов как параметр (кодирование Чёрча для типов~\cite{Hinze}). Применяя данную функцию к различным реализациям алгебры объектов можно получать различные преобразования (например, распечатывание). Чтобы инстанциировать саму структуру данных нужно предоставить особый экземпляр алгебры объектов~---~\emph{фабрику}. Однако, после инстанциации структура данных больше не может быть трансформирована обобщенным образом. Следовательно, алгебры объектов заставляют пользователя переключиться на представление данных с помощью функций, которое может быть, а может не быть удобно в зависимости от обстоятельств.  Наш же подход недеструктивно добавляет новую функциональность к уже знакомому миру алгебраических типов данных, сопоставления с образцом и рекурсивных функций. Обобщенные реализации преобразований полностью отделены от представления данных и пользователи могут свободно преобразовывать их структуры данных привычным способом  без потери возможности объявлять (и расширять) обобщенные функции. Другой особенностью OCaml, в отличии от распространенных языков объектно-ориентированного программирования, является то, что для написания расширяемого кода в основном используются полиморфные вариантные типы, а не классы. Поддержка полиморфных вариантных типов для написания расширяемых типов данных требует нового подхода.


Итого, среди уже существующих библиотек для обобщенного программирования для \textsc{OCaml} мы можем называть две, которые напоминают нашу: \cd{ppx\_deriving}/\cd{ppx\_traverse}, последняя версия которых находится в кодовой базе \cd{ppxlib}~\cite{PPXLib}, и \textsc{Visitors}~\cite{Visitors}.

\cd{ppx\_deriving} является наипростейшим подходом: объявления типов данных отображаются один к одному в рекурсивные функции, представляющие конкретный вид преобразования. Это наиболее эффективная реализация, так как функции вызываются напрямую, без позднего связывания, но нерасширяемая. Если пользователю понадобится слегка модифицировать сгенерированную функцию, то он должен будет полностью скопировать реализацию функции и изменить её. Количество работы по программированию нового преобразования может существенно увеличиться, если тип данных будет видоизменяться во время цикла разработки.

В \cd{ppx\_traverse} расширяемые трансформации также представлены как объекты. В отличие от нашего подхода, там не используется кодирование конструкторов и методов один к одному. К тому же \cd{ppx\_traverse} не использует наследуемые атрибуты, следовательно некоторые преобразования, такие как проверка на равенство и сравнение, невыразимы.

\textsc{Visitors}, с другой стороны, использует сходный с нашим подход, в котором были приняты многие решения, отвергнутые нами, и наоборот.
Ниже мы подытожим главные различия:

\begin{itemize}
   \item \textsc{Visitors} полностью объектно-ориентированы. Чтобы воспользоваться преобразованием необходим создать некоторый объект и вызвать нужный метод. В нашем случае, если используются возможности, предусмотренные заранее, то можно использовать более естественный комбинаторный подход.
     
   \item \textsc{Visitors} реализуют некоторое количество преобразований в специфичной ad-hoc манере. В нашем случае все преобразования принадлежат некоторой обобщенной схеме. Различные трансформации можно скомбинировать с помощью наследования, если типы в схеме унифицируются. Мы также заявляем, что в нашей библиотека реализация ползовательски плагинов с трансформациями проще. 
     
   \item Как и  SYB, \textsc{Visitors} поддерживают указание способа трансформации для входящих в структуру данных типов: для каждого типа присутствует метод в объекте, представляющий трансформацию. Хотя такое представление добавляет некоторой гибкости мы осознанно отказывается от него, так как оно позволяет преодолеть инкапсуляционный барьер: изменяя методы преобразования (которые не могут быть скрыты в сигнатуре), можно получить некоторую информацию об внутреннем реализации инкапсулированной структуры данных. Более того, абстрактные структуры данных могут быть изменены способом, не предусмотренным публичным интерфейсом

   \item В нашем случае типовые параметры классов, представляющих трансформацию, должны быть указаны пользователем. В \textsc{Visitors} это работа возлагается на плечи компилятора, с помощью оригинального трюка. Однако, он не позволяет использовать \textsc{Visitors} в сигнатурах модулей. В нашем случае нет никаких проблем: поддерживается работа и с файлами реализации, и с файлами сигнатур.

   \item \textsc{Visitors} на сегодняшний день\footnote{Последней доступной версией на данный момент является 20180513.} не поддерживает полиморфные вариантные типы.
   
   \item \textsc{GT} поддержает произвольные применения конструкторов типов, а  \textsc{Visitors} и в мономорфном, и в полиморфном режиме -- нет.
     Например, данный пример не компилируется:
     
   \begin{lstlisting}
   type ('a,'b) alist = Nil | Cons of 'a * 'b
   [@@deriving visitors { variety = "map"
                        ; polymorphic = true }]

   type 'a list = ('a, 'a list) alist
   [@@deriving visitors { variety = "map"
                        ; polymorphic = false }]
   \end{lstlisting}
   
   Более того, добавление искусственного конструктора не решает проблему:
   
   \begin{lstlisting}
   type 'a list = L of ('a, 'a list) alist [@@unboxed]
   [@@deriving visitors { variety = "map"
                        ; polymorphic = false }]
   \end{lstlisting}
    
    Также присутствуют сложности с переименованиями (aliases) типов в полиморфном режиме (мономорфная часть библиотеки \textsc{Visitors} компилируется успешно):
    
    \begin{lstlisting}
    type ('a,'b) t = Foo of 'a * 'b (* OK *)
    [@@deriving visitors { variety = "map"
                         ; polymorphic = true }]

    type 'a t2 = ('a, int) t
    [@@deriving visitors { variety = "map"; name="somename"
                         ; polymorphic = true }]
    \end{lstlisting}
    
    Сгенерированный код можно исправить вручную, путём удаления типовых аннотаций для явного полиморфизма (explicit polymorphism) у методов, что приведет к коду, который очень напоминает генерируемый  \textsc{GT}. Из этого мы можем заключить, что на \textsc{GT} можно смотреть как перереализацию полиморфного режима библиотеки  \textsc{Visitors}, где большее количество объявлений типов компилируется корректно.
    
\end{itemize}

\section{Заключение}
\label{sec:futurework}

В данной работе представлен подход на основе обобщенного программирования, который кодирует преобразования значений типов данных с помощью объектов, что позволяет видоизменять построенные преобразования, не описывая их заново.

Существует несколько возможны направлений для дальнейшего развития проекта. Во-первых, можно снижать накладные расходы на реализацию расширяемых преобразований 
%в данной работе мы не касались вопросов производительности. Мы представляем преобразования в очень обобщенном виде, с несколькими слоями косвенности. Очевидно, что преобразования, реализованные с помощью нашей библиотеки будут работать медленнее, чем написанные вручную. Мы предполагаем, что 
%производительность может быть улучшено 
с помощью, так называемого, staging~\cite{Staged} или, возможно, с помощью оптимизаций, специфичных для объектов. 
Также, стоит рассмотреть вариант реализации обобщенной функции-преобразования~\ref{transtypes} как метода, что позволит достигнуть паритета в производительности с \Visitors{}.

Другим важным направлением является поддержка большего разнообразия объявлений типов, а именно GADT и нерегулярных типов. Хотя уже сделаны некоторые наработки, получившиеся решение делает интерфейс всей библиотеки чересчур сложным даже для простых случаев.

Наконец, структура с информацией о типе, которую мы генерируем, может быть использована, чтобы сымитировать \emph{ad-hoc} полиморфизм, так как они содержит реализацию функций, индексированных типами. Это в сумме с недавно предложенными расширениями~\cite{ModularImplicits} может открыть интересные перспективы.







%%%%%%%%%%%%%%%%%%%%%%%%%%%%%%%%%%%%%%%%



\begin{thebibliography}{99}

\small

\bibitem{SCICO}
{\em Boulytchev D.} Combinators and Type-driven Transformers in
Objective Caml // Sci. Comput. Program. 2015.  v. 114. no. C. pp. 57–73.

%\bibitem{brooks}
%{\em Brooks Jr. F. P.}  The Mythical Man-month (Anniversary
%Ed.). — Boston, MA, USA : Addison-Wesley Longman Publishing Co.,
%Inc., 1995. — ISBN: 0-201-83595-9.

\bibitem{SYBOCaml}
{\em Boulytchev D., Mechtaev S.} Efficiently Scrapping Boilerplate
Code in OCaml // Workshop on ML. — ML ’11. — Tokyo, Japan, 2011.

\bibitem{InstantGenerics}
{\em Chakravarty M. M. T., Ditu G., Leshchinskiy R.} In\-stant Generics : Fast and Easy. — 2009.

\bibitem{HC}
{\em Filli\^atre J.-C., Conchon S.} Type-safe Modular Hashconsing // Workshop on ML. — ML ’06. — New York, NY, USA : ACM, 2006. — P. 12–19.

\bibitem{PolyVar}
{\em  Garrigue J.} Programming with Polymorphic Variants // Workshop on ML. — 1998.

\bibitem{PolyVarReuse}
{\em Garrigue J.} Code reuse through polymorphic variants // In
Workshop on Foundations of Software Engineering. — 2000.

\bibitem{CalculatingFP}
{\em  Gibbons J.} Calculating Functional Programs // Algebraic and
Coalgebraic Methods in the Mathematics of Program Construction:
International Summer School and Workshop Oxford, UK, April 10–14, 2000 Revised Lectures / Ed. by Roland Backhouse, Roy Crole,
Jeremy Gibbons. — Berlin, Heidelberg : Springer Berlin Heidelberg,
2002. — P. 151–203. %— ISBN: 978-3-540-47797-6. 
%— Access mode: https://doi.org/10.1007/3-540-47797-7_5.

\bibitem{DGP}
{\em Gibbons J.} Datatype-generic Programming // Proceedings
of the 2006 International Conference on Datatype-generic Program\-ming. — SSDGP’06. — Berlin, Heidelberg : Springer-Verlag, 2007. —
P. 1–71.% — 
%Access mode: http://dl.acm.org/citation.cfm?id=1782894.1782895.

\bibitem{Hinze}
{\em Hinze R.} Generics for the Masses // J. Funct. Program. — 2006. —
Jul. — Vol. 16, no. 4-5. — P. 451–483.
%Access mode: http://dx.doi.org/10.1017/S0956796806006022.

\bibitem{Fold}
{\em Hutton G.} A Tutorial on the Universality and Expressive\-ness of Fold // J. Funct. Program. — 1999. — Jul. — Vol. 9,
no. 4. — P. 355–372. %— Access mode: http://dx.doi.org/10.1017/S0956796899003500.

\bibitem{TypeFamilies}
{\em Kiselyov O., Jones S.P.,  Chung-chieh S. } Fun with
Type Functions // Reflections on the Work of C.A.R. Hoare / Ed. by
A.W. Roscoe, Cliff B. Jones, Kenneth R. Wood. — London : Springer
London, 2010. — P. 301–331. %— ISBN: 978-1-84882-912-1. 
%— Access mode: https://doi.org/10.1007/978-1-84882-912-1_14.

\bibitem{AGKnuth}
{\em Knuth D.E.} Semantics of context-free languages // Mathemati\-cal systems theory. — 1968. — Jun. — Vol. 2, no. 2. — P. 127–145. 
%— Access mode: https://doi.org/10.1007/BF01692511.

\bibitem{OCanren}
{\em Kosarev D, Boulytchev D.} Typed Embedding of a Relational Language in OCaml // Proceedings ML Family Workshop /
OCaml Users and Developers workshops, ML/OCAML 2016, Nara,
Japan, September 22-23, 2016. — 2016. — P. 1–22. 
%— Access mode:https://doi.org/10.4204/EPTCS.285.1.

\bibitem{SYB}
{\em L\"ammel R., Jones S.P.} Scrap Your Boilerplate: A Practical Design Pattern for Generic Programming // SIGPLAN Not. —
2003. — Jan. — Vol. 38, no. 3. — P. 26–37. 
%— Access mode: http://doi.acm.org/10.1145/640136.604179.

\bibitem{SYB1}
{\em L\"ammel R., Jones S.P.} Scrap More Boilerplate: Reflection, Zips, and Generalised Casts // Proceedings of the Ninth ACM
SIGPLAN International Conference on Functional Programming. —
ICFP ’04. — New York, NY, USA : ACM, 2004. — P. 244–255. 
%— Ac-cess mode: http://doi.acm.org/10.1145/1016850.1016883.

\bibitem{SYB2}
{\em L\"ammel R., Jones S. P.} Scrap Your Boilerplate with Class:
Extensible Generic Functions // SIGPLAN Not. — 2005. — Sep. —
Vol. 40, no. 9. — P. 204–215. %— Access mode: http://doi.acm.org/10.1145/1090189.1086391.

\bibitem{Bananas}
{\em Meijer E., Fokkinga M., Paterson R.} Functional Programming with Bananas, Lenses, Envelopes and Barbed Wire. — Springer-Verlag, 1991. — P. 124–144.

\bibitem{ObjectAlgebras}
{\em Oliveira B. C. d. S., Cook W.R.} Extensibility for the
Masses: Practical Extensibility with Object Algebras // Proceedings of the 26th European Conference on Object-Oriented Programming. — ECOOP’12. — Berlin, Heidelberg : Springer-Verlag,
2012. — P. 2–27.

\bibitem{Visitors}
{\em Pottier F.} Visitors Unchained // Proc. ACM Program.
Lang. — 2017. — Aug. — Vol. 1, no. ICFP. — P. 28:1–28:28. 
%— Access mode: http://doi.acm.org/10.1145/3110272.

\bibitem{ObjectAlgebrasAttribute}
{\em Tillmann R., Brachth\"auser J.I., Ostermann K. } From Object Algebras to Attribute Grammars //
 SIGPLAN Not. — 2014. — Oct. — Vol. 49, no. 10. — P. 377–395. 
% Access mode: http://doi.acm.org/10.1145/2714064.2660237.

\bibitem{ObjectAlgebrasSYB}
{\em Zhang H., Chu Z.,  Oliveira B. C.d. S., van der Storm T.}
Scrap Your Boilerplate With Object Algebras  // Proceedings of the Object-oriented Programming, Systems, Languages, and
 Applications (OOPSLA, 2015). — New York, United States, 2015. %—
% Access mode: https://hal.inria.fr/hal-01261477.

\bibitem{ALaCarte}
{\em Swierstra W.} Data Types \`a La Carte // J. Funct. Program. —
 2008. — Jul. — Vol. 18, no. 4. — P. 423–436.
% Access mode: http: //dx.doi.org/10.1017/S0956796808006758.

\bibitem{AGSwierstra}
{\em Viera M., Swierstra S. D., Swierstra W.} Attribute
 Grammars Fly First-class: How to Do Aspect Oriented 
 Program\-ming in Haskell // Proceedings of the 14th ACM SIGPLAN 
 In\-ternational Conference on Functional Programming. — ICFP ’09. —
 New York, NY, USA : ACM, 2009. — P. 245–256. 
% — Access mode: http://doi.acm.org/10.1145/1596550.1596586.

\bibitem{ExpressionProblem}
{\em Wadler P.} The Expression Problem. — 1998. — Dec.
 
\bibitem{TypeClasses}
{\em Wadler P., Blott S.} How to Make Ad-hoc Polymorphism Less Ad
 Hoc // Proceedings of the 16th ACM SIGPLAN-SIGACT Symposium on Principles of Programming Languages. — POPL ’89. — New
 York, NY, USA : ACM, 1989. — P. 60–76. 
% — Access mode: http: //doi.acm.org/10.1145/75277.75283.

 \bibitem{ModularImplicits}
{\em White L., Bour F., Yallop J.} Modular implicits // Electronic Proceedings in Theoretical Computer Science. — 2015. — 12. —
 Vol. 198.
 
 \bibitem{Yallop}
{\em Yallop J.} Practical Generic Programming in OCaml // Proceedings of the 2007 Workshop on Workshop on ML. — ML ’07. —
 New York, NY, USA : ACM, 2007. — P. 83–94. 

\bibitem{Staged} 
{\em  Yallop J.} Staged Generic Programming // Proc. ACM Program. Lang. — 2017. — Aug. — Vol. 1, no. ICFP. — P. 29:1–29:29. 
% — Access mode: http://doi.acm.org/10.1145/3110273.

\bibitem{GenericOCaml} 
{\em  Balestrieri F., Mauny M.} Proceedings ML Family Workshop / OCaml Users and Developers
workshops, Nara, Japan, September 22-23, 2016 / Ed. by Kenichi Asai,
Mark Shinwell. — Vol. 285 of Electronic Proceedings in Theoretical
Computer Science. — Open Publishing Association, 2018. — P. 59–100.

% https://dl.acm.org/doi/proceedings/10.1145/317636
\bibitem{modules-vs-objects} 
{\em       Leroy X.} 
Objects, classes and modules in Objective Caml / Proceedings of International Conference on Functional Programming, Paris /     Association for Computing Machinery, 1999.


\bibitem{camlp5} 
\camlpfive{}. --- \url{https://camlp5.github.io}.
 
\bibitem{PPXLib} 
\textsc{ppxlib}. --- \url{https://github.com/ocaml-ppx/ppxlib}.

\bibitem{ppxderiving} 
ppx\_deriving. --- \url{https://github.com/ocaml-ppx/ppx\_deriving}

\bibitem{dotNetSG}
.NET source Generators. --- \url{https://github.com/dotnet/roslyn/blob/master/docs/features/source-generators.md}

\begin{comment}
\bibitem{AZ97}
{\em Абрамов С.А., Зима Е.В.} Семинар по компьютерной алгебре на
факультете вычислительной математики и кибернетики МГУ в 1995--1996 г.
// Программирование, 1997,
No 1. С. 75--77.

\bibitem{AZ98}
{\em Абрамов С.А., Зима Е.В.} Научно-ис\-сле\-до\-вательский семинар
``Компьютерная алгебра'' в 1996--1997 г.
// Программирование, 1998,
No 1. С. 69--72.

\bibitem{AR99}
{\em Абрамов С.А., Ростовцев В.А.} Семинар по компьютерной алгебре в
1997--1998 г.  // Программирование, 1998, No 6. С. 3--7.

\bibitem{AKR00}
{\em Абрамов С.А., Крюков А.П., Ростовцев В.А.} Семинар по компьютерной
алгебре в
1998--1999 г.  // Программирование, 2000, No 1. С. 8--12.

\bibitem{AKR01}
{\em Абрамов С.А., Крюков А.П., Ростовцев В.А.} Семинар по компьютерной
алгебре
в 1999--2000 г.  // Программирование, 2001, No 1. С. 3--7.

\bibitem{AKR02}
{\em Абрамов С.А., Крюков А.П., Ростовцев В.А.} Семинар по компьютерной
алгебре в 2000--2001 г.  // Программирование, 2002, No 2. С. 6--9.

\bibitem{AKR03}
{\em Абрамов~С.А., Крюков~А.П., Ростовцев~В.А.} Семинар по компьютерной
алгебре в 2001--2002 г. // Программирование, 2003, No 2. С. 3--7.
\bibitem{AER04}
{\em Абрамов~С.А., Еднерал~В.Ф., Ростовцев~В.А.} Семинар по
компьютерной алгебре в 2002--2003~г.  // Программирование, 2004, No 2.
С. 3--7.
\bibitem{ABRE05}
{\em Абрамов~С.А., Боголюбская~А.А., Ростовцев~В.А., Еднерал~В.Ф.} Семинар по
компьютерной алгебре в 2003--2004 г.  // Программирование, 2005, No 2.
С. 3--9.
\bibitem{ABRE06}
{\em Абрамов~С.А., Боголюбская~А.А., Ростовцев~В.А., Еднерал~В.Ф.} Семинар по
компьютерной алгебре в 2004--2005 г.  // Программирование, 2006, No 2.
С. 3--7.
\bibitem{ABRE07}
{\em Абрамов~С.А., Боголюбская~А.А., Ростовцев~В.А., Еднерал~В.Ф.} Семинар по
компьютерной алгебре в 2005--2006 г.  // Программирование, 2007, No 2.
С. 3--8.
\bibitem{ABRE08}
{\em Абрамов~С.А., Боголюбская~А.А., Ростовцев~В.А., Еднерал~В.Ф.} Семинар по
компьютерной алгебре в 2006--2007 г.  // Программирование, 2008, No 2.
С. 3--8.
\bibitem{ABRE09}
{\em Абрамов~С.А., Боголюбская~А.А., Ростовцев~В.А., Еднерал~В.Ф.} Семинар по
компьютерной алгебре в 2007--2008 г.  // Программирование, 2009, No 2.
С. 3--9.
\newpage
\bibitem{mmcp09}
``Mathematical Modeling and Computational Physics (CAAP'2009)''. Book of abstracts of the internationl conference. Dubna, July 7-11, 2009.
Dubna, 2009.
\bibitem{ABRE10}
{\em Абрамов~С.А., Боголюбская~А.А., Ростовцев~В.А., Еднерал~В.Ф.} Семинар по
компьютерной алгебре в 2008-2009 г.  // Программирование, 2010, No 2. С. 3--8.
\bibitem{ABER11}
{\em Абрамов~С.А., Боголюбская~А.А., Еднерал~В.Ф., Ростовцев~В.А.} Семинар по
компьютерной алгебре в 2009-2010 г. // Программирование, 2011, No 2. С. 3--8.
\bibitem{ABR12}
{\em Абрамов~С.А., Боголюбская~А.А.,  Ростовцев~В.А.} Семинар по
компьютерной алгебре в 2010-2011 г.  // Программирование, 2012, No 2. С. 3--10.

%
\end{comment}
\end{thebibliography}

\label{lastpage}
\end{document}






