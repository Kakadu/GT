% !TeX program = pdflatex
% !TeX spellcheck = ru_RU

\documentclass[a4paper,twoside,11pt]{article}
\usepackage[utf8]{inputenc}
\usepackage[english,russian]{babel}
\usepackage{fancyhdr}
\usepackage{comment}
\usepackage{newprog1e}
\usepackage{amsfonts,amsmath,amssymb,amsthm}
\usepackage{graphicx}
\usepackage{hyperref}
\usepackage{csquotes}

\usepackage[final]{listings}
\usepackage{cmap}   % for copy-pastable text
\usepackage{xcolor}
\usepackage[normalem]{ulem} % \sout
%multi-row
\usepackage{multirow}
\usepackage{booktabs}
\usepackage[shortlabels]{enumitem}  % \begin{enumerate}[(a)]



\begin{comment}
\usepackage{tabularx}
\newcolumntype{Y}{>{\centering\arraybackslash}X}
\newcolumntype{L}[1]{>{\raggedright\let\newline\\\arraybackslash\hspace{0pt}}m{#1}}
\newcolumntype{C}[1]{>{\centering\let\newline\\\arraybackslash\hspace{0pt}}m{#1}}
\newcolumntype{R}[1]{>{\raggedleft\let\newline\\\arraybackslash\hspace{0pt}}m{#1}}
\end{comment}


\newcommand{\ocaml}{\textsc{OCaml}}
\newcommand{\OCaml}{\ocaml}
\newcommand{\haskell}{\textsc{Haskell}}
\newcommand{\visitors}{\textsc{Visitors}}
\newcommand{\Visitors}{\visitors}
\newcommand{\ocanren}{\textsc{OCanren}}
\newcommand{\OCanren}{\ocanren}
\newcommand{\scheme}{\textsc{Scheme}}
\newcommand{\Scala}{\textsc{Scala}}
\newcommand{\GT}{\textsc{GT}}
\newcommand{\PPX}{\textsc{PPX}}
\newcommand{\camlpfive}{\textsc{Camlp5}}
\newcommand{\PPXMorphism}{\textsc{ppx\!\_\!deriving\!\_\!morphism}}


\lstdefinelanguage{ocaml}{
keywords={@type, function, fun, let, in, match, with, when, class, type,
nonrec, object, method, of, rec, repeat, until, while, not, do, done, as, val, inherit, and,
new, module, sig, deriving, datatype, struct, if, then, else, open, private, virtual, include, success, failure,
lazy, assert, true, false, end},
sensitive=true,
commentstyle=\small\itshape\ttfamily,
keywordstyle=\ttfamily\bfseries, %\underbar,
identifierstyle=\ttfamily,
basewidth={0.5em,0.5em},
columns=fixed,
fontadjust=true,
literate={->}{{$\to$}}3 {===}{{$\equiv$}}1 {=/=}{{$\not\equiv$}}1 {|>}{{$\triangleright$}}3 {\\/}{{$\vee$}}2 {/\\}{{$\wedge$}}2 {>=}{{$\ge$}}1 {<=}{{$\leqslant$}} 1,
morecomment=[s]{(*}{*)}
}

\lstset{
mathescape=true,
%basicstyle=\small,
identifierstyle=\ttfamily,
keywordstyle=\bfseries,
commentstyle=\scriptsize\rmfamily,
basewidth={0.5em,0.5em},
fontadjust=true,
language=ocaml
}
\newcommand{\cd}[1]{\texttt{#1}}

\newcommand{\inbr}[1]{\left<#1\right>}


\tolerance=1000

\newcommand{\defi}{\stackrel{\mathrm{def}}{=}}

\numberwithin{equation}{section}
\newtheorem{theorem}{Теорема}

\newtheorem{definition}{Definition}
\newtheorem{theorem_ru}{Теорема}


\journalnumber{?}
\curyear{2021}
\authorlist{КОСАРЕВ И ДР.}
\titlehead{Обобщённое программирование с комбинаторами и объектами}
\headerdef

\udk{004.421.6}
%92+004.94}
\rubrika{ЯЗЫКИ, КОМПИЛЯТОРЫ И СИСТЕМЫ ПРОГРАММИРОВАНИЯ}
\dateinput{\today}

\rusabstr{}

\author{
{\bfseries Д.~С.~Косарев*, Д.~Ю.~Булычев*}
\\ {\itshape Санкт-Петербургский государственный университет}
\\ {\slshape 199034} {\itshape  Санкт-Петербург, Университетская наб., д.} {\slshape  7–9}
\\ {\itshape E-mail: Dmitrii.Kosarev@pm.me, dboulytchev@math.spbu.ru}}
\title{ОБОБЩЁННОЕ ПРОГРАММИРОВАНИЕ С КОМБИНАТОРАМИ И ОБЪЕКТАМИ}
%\thanks{~}

\date{}

\newcommand\blfootnote[1]{%
  \begingroup
  \renewcommand\thefootnote{}\footnote{#1}%
  \addtocounter{footnote}{-1}%
  \endgroup
}

\begin{document}

\maketitle
\setcounter{page}{3}
%%%%%%%%%%%%%%%%%%%%%%%%%%%%%%%%%%%%%%%%%%%%%%%%%%%%%%%

% !TeX encoding = windows-1251
\section{��������}


\begin{comment}

������� ����� (Frederic Brooks) � ����� ��������� ����� �� ���������
��������  "���������� ��������-�����" ("The Mythical Man-Month")~\cite{MMM} ��������������� �������� ���������������� ��������� �������:

\blockquote{"�����������, ������� �����, �������� ����� ��������������� � ������ ������. �� ������ ���� ����� � ������� � �� �������, ����� ����� �����������. ������ ����� ������ ��������, ������������ � ����������, ������� ����� �� �����, ����� ��� �������� ��� ����������� � �������� ��� ���������� ����������� ��������. (��� �� ������� ������, ����� ������������ ���� ���� ��������.)"}

�������������, ���������������� �������� � �������� �� �������������  ��������� � ����������������; ���������� ���������� ��������� ����� ����� �������� � ���������������� ������������
(��� ��� ��������� � ���������� ������������� ��������� � �������). 
\end{comment}

����� �� ������� � ����������� ���������� ������������ ����������� �������� ����������� ������������� �����������, ������� ����� �������� � ������ ��������� ������������ ��������. ���������� ���������� ��������� ����� ����� �������� � ���������������� ������������
(��� ��� ��������� � ���������� ������������� ��������� � �������). 
����� �� �������� ���������������� �������� ��������������� ��������� �������� ������������� \emph{����� ������}. ��� ��������� ��������� �������� ������; ��� ����� � ���� �������, � ��� ������; � ����� � ��������� ������� ��������� ��������� �������� ������. ���� ���������� � ����� ������ ������������ �� ����� ������ ���������, �� ���������� ��������� ����������� ����-�������������� ���� ������� ����� (\emph{������������}) ��� ���� �������� ����� ����� ������ �� ���� (\emph{���������}).

������, � ���������� �������������� �������������� ������, ��� �������, ���� ��������� ��������� 
����� ���� ���������� � ����������� �� ����� ����������. ����������� ��������� �������� ��������� ������������� �� ��������� � ������������, ������ ��� ����������
�� ����� �������������� ���� �� ����� ���������� � ������ ���������� ������ ��������� �������� -- ������ ��������� -- �� ���������. � ������ �������, ��������� ��������������, ������� � ������������ ������ ����� ���� ����������� ``��� � ��������'' �� �������� �������� ����� � ������ ���� ��������������� ��� ������� ����������� ���� �� �����������. ���� �� �������� � ����������� ����� ���������� �������� ���������� ����� ������������� ������� �����, ��� ������� ���������� ������� ����� ���� ���������������. �������� ����� ������� ����� ��������� ���������� (\emph{ad hoc} ������������) � ����� \haskell � ���� ������� 
�����~\cite{TypeClasses} � �������� �����~\cite{TypeFamilies}. ������, �� ������� ���������� ����������� � ��������� �������� ����� � ��������������� �������������� �������� �������� �����, ������ ����� ������������ ``�������'' ���������, �������
�� ����� ���� ���������������. ������ �������� �������� \emph{���������� ����������������}~\cite{DGP} (\emph{datatype-generic programming}), ����� �������� �������� ���������� ������� ��� ���������� ����������� ������ �������� �������
��������������� ������, ��������� ��������� ����������� �����. ��������, ���� ����� ���� ������������ �� ���������� �����~\cite{Hinze,InstantGenerics,GenericOCaml}, ���� ����� ���������� � ����� ����� ���� ������� ��������� �� ����� ����������, ��� 
���������� ������� ��� ����������� ���� ������ ����� ���� ������������� �� �����
���������� �������������~\cite{Yallop,PPXLib}. ��� ������, ��������� ����, ��������� ���� �����: ��� ����� ������ �������� ������� �����, ��� ������ ������������ ��� ����������� ���������������� ���� ����� ����������. ��������, ��������������� ����������� ��������� ����������� �������� ������� ��� ���������� ����� ������ ������������ ��������� � �.�.

%�������� ����� �������� ������ ��� ���������  ``Scrap Your Boilerplate''~\cite{SYB,SYB1,SYB2}, ������� ��������� � ��������� �� �������� ��������� �� �����, ��������������� �������. ������, ��� ���� �������, ��� ��������������� � ������������� SYB ������������, � ���� �� ��������������, ������� ������� ������ �� ����, �� ����� ������ ������������. 

� ���������� �������� ����������������� �������������� ����������� ������ ���� �� ����������. ����� �������, ���� 
��������������� �������������� �� ������ ���������� ������������, �� �� ������ ���� ����������� ��������� ������ ���� ��������������, ���� �������� ��� �������. ����� �������, ��� ���������������� ������� ������������ ��� ��������������� ��������������, ����� ���������� ���������� ��������������.

������ ��������� ����� �������� �������������  �������� ���������� (late binding) ��� ����, ����� ������������ ����� ��������������� �������������� ��� ����� ������������. ����������� ��� ���� ������������� � ��������-��������������� �����������������, ����� ���������� � ��������������.


� ������ ������ �� ������������ ������, ����������� � ���� ���� 
�������� ���������� ��� ������������� ��������� ����������� ��������������, � ����������������� ��������������, ��������� ��� ��������������� ����������������. ��� ������������� ������� ������� ���� ����������� ���������� ��� ����������� ���������������� \textsc{GT}\footnote{\url{https://github.com/kakadu/GT/tree/ppx}} (\emph{Generic Transformers}), � ����� ��������� ���������� �������� ��� ��, ����������� ����������� ������������ ����� ��������������.


%������� ��������� � �������� ���������� � 2014 ����. � ��� ������������ ���� �������� ���������� �� ������ �������� � ����� \ocaml, ��� � �~\visitors. ������, ��������� ������������ ������� � \visitors, ������� �� ������� ��������������,   �� ��������� ��� ������������ � � ��������� �������, � ��������� �  ������� \ocanren~\cite{OCanren}.


%\section{План}



\subsection{Метод}

\subsection{Введение в обобщённое программирование}
%Указать кратко, что тут будет описываться.\\
%В конце сказщать про GT, строки кода (?) выложена там-то.
%
%В статье на ML Workshop мы писали тут детали реализации, а именно какой код стоит генерировать по типам данных, чтобы получилось сделать то, что хочется сделать. Как это написать, не скатившись в детали реализации -- мне пока не понятно.

\subsection{Результаты}
%\emph{Пока плохо понимаю, что тут писать}.
%
%Был представлен подход по представлению преобразований с помощью объектов языка \ocaml. В данном представлении конструкторы алгебраических типов данных кодируются один к одному в методы, таким образом программист достаточно легко может изменить преобразование для некоторого конструктора алгебраического типа. Для испытания подхода была спроектирована библиотека обобщённого программирования \GT, в рамках которой реализован типичный набор преобразований.


%\subsection{Background}
%
%Описание предметной области и используемые технологии есть во введении
%
%Результаты предыдущих (собственных) исследований отсутствуют.
%
%\textit{Не знаю что тут писать} (Д.Кознов сказал, что можно обойтись)


\subsection{Related works}
\begin{itemize}
\item  Стандартные подходы про обобщённое программирование, в том числе ppx\_deriving, который все сейчас в \ocaml~используют
\item Обощенное программирование с использование "представлений типов"
\item Scrap Your Boilerplate -- специфический вид обобщенного программирования
\item Алгебры объектов -- что-то подобное тому, что мы хотим, но чисто в ОО-стиле для С\#/Java
\item \cd{ppx\_traverse} делает примерно то же, что и мы, только кодирует типы без  нашей идеи "один конструктор -- один метод"
\item \visitors~главный конкурент, так как он использует ту же идею, но реализованную по-другому. Предлагаю здесь их просто упомянуть и содержательно сравниваться с ними в evaluation.
\end{itemize}

\subsection{Evaluation}
%Здесь можно и RQ указать.\\
%
%Здесь планируется \textbf{пример №1}, чтобы показать, что наше видоизменение трансформаций в принципе работает (ответ на RQ 2)
%
%У нас основной конкурент -- это \visitors, поэтому здесь сравниваемся в основном с ней
%\begin{itemize}
%\item[$\star$] У них чисто ОО подход, а у нас совмещенный, поэтому получающийся в  итоге код выглядит привычнее для функционального программирования (+ к ответу на RQ1)
%
%\item[$\star\star$] У них добавляются некоторые искусственные методы в объекты, что позволяет из-за чего "протекает" абстракция и открываются возможности "сломать" преобразования
%
%\item[$\star\star\star$] В \visitors~два немного отличающихся подхода к созданию преобразований, а одном получается выразить одни преобразования, а в другом другие. Важно то, что в  \visitors~мы не можем выразить то, что нам надо для~\cite{OCanren} (про это можно сделать \textbf{пример №2}).
% У нас  же один подход, при котором получается выразить всё что надо. (+ к ответу на RQ2)
%
%\item[$\star\star\star\star$] \visitors~ не поддерживают полиморфные вариантные типы, мы поддерживаем. Про это будет \textbf{пример №3} на тему \emph{expression problem}~\cite{ExpressionProblem}  на основе~\cite{PolyVar,PolyVarReuse}. Этот пример хочется включить, так как expression problem достаточно известная тема в сообществе и много кто предлагал различные подходы к её решению.
%\end{itemize}
%
%Также сюда можно добавить сравнение производительности нашего подхода с наивным подходом, где не подразумевается возможность последующего видоизменения преобразований (мы буквально это делаем для гранта).
%
%Тут можно два  "стиля" evaluation
%\begin{enumerate}
%\item Примеры и сравнение с
%\item С метрикой можно "отключать" некоторые шаги и смотреть что будет (объем кода, быстродействие, сложность использования)
%\end{enumerate}



\subsection{Discussion}
%Обсуждение результатов+полемика: \emph{пока плохо понимаю, что тут писать}.
%
%\begin{enumerate}
%\item Не смотря на то, что библиотека \visitors~не предоставляет привычный комбинаторный интерфейс, доработка её в этом аспекте не должна вызвать серьёзных проблем.
%\item ``Протекание'' абстракции в библиотеке \visitors~ (а также в подходе SYB~\cite{SYB}) позволяет удобно реализовывать некоторые преобразования, типичные для подхода SYB. При нашем подходе реализация таких преобразований вызовет сложности, однако, мы находим преобразования в стиле SYB чересчур специфическими и мало полезными для наших нужд.
%\item Сейчас реализация \visitors~полагается на использование \emph{явного полиморфизма}\footnote{\url{https://caml.inria.fr/pub/docs/manual-ocaml/polymorphism.html\#s\%3Apolymorphic-recursion}} при указании типа методов объектов. При таком подходе не понятно как добавлять поддержку полиморфных вариантных типов. Наш опыт говорит, что ради поддержки полиморфных вариантных типов от явного полиморфизма придется отказаться, но тогда новая реализация \visitors~будет почти в точности повторять подход \GT.
%\end{enumerate}
%



\section{Введение в обобщенное программирование }
\label{sec:tutorial}

В этом разделе мы кратко представляем  подход обобщенного программирования по типам данных (datatype generic programming). В качестве примеров мы рассмотрим преобразования арифметических выражений.  Сначала -- самое простое преобразование в текстовое представление для демонстрации возможностей непосредственно обобщённого программирования. Затем рассмотрим другие преобразования, где мы будем представлять их с помощью объектов, а также покажем полезность расширяемых.


%В этом разделе мы постепенно представим наш подход используя несколько примеров. Хотя изложение не предоставляет конкретных деталей и не может использоваться как точная спецификация, мы здесь предоставляем основные составляющие решения и мотивацию, которая привела к ним.

В этой работе мы используем следующее соглашение: будем обозначать $\inbr{\dots}$ представление некоторого понятия в конкретном синтаксисе языка \textsc{OCaml}. Например, ``$\inbr{f_t}$`` является обозначением конкретной функции индексированной типом  ``$f$'' для типа ``$t$''. 
В конкретном синтаксисе оно может быть выражено как ``\lstinline{f_t}'', но мы пока воздержимся от указания конкретной формы.

%TODO: Подробнее написать где есть что, для людей, которые не очень 

\subsection{Преобразование ``\texttt{show}''}

Мы будем работать только с типом данных для арифметических выражений, которые состоят из констант, переменных и бинарных операций.

\begin{lstlisting}
type expr =
| Const of int
| Var   of string
| Binop of string * expr * expr
\end{lstlisting}

%TODO: Попытаться заострить "первый" уровень

Для него можно реализовать преобразование в строковое представление ``\lstinline{show}''
%``$\inbr{show_{expr}}$'' 
(наиболее естественный кандидат на реализацию),
которое преобразует выражение в строку: 

\begin{lstlisting}
let rec show = function
| Const  n         -> sprintf "Const %d" n
| Var    x         -> sprintf "Var %s" x
| Binop (op, l, r) ->
    sprintf "Binop (%S, %s, %s)" 
      op (show l) (show r)
\end{lstlisting}

Преобразования данного вида не очень сложно реализовывать вручную, но воспользоваться обобщённым программированием предпочтительнее. Чтобы это сделать (например, с использованием~\cite{ppxderiving})  необходимо проаннотировать тип данных и подключить специальный препроцессор времени компиляции. 

\begin{lstlisting}
type expr =
| Const of int
| Var   of string
| Binop of string * expr * expr
[@@deriving show]
\end{lstlisting}

\noindent В итоге, во время компиляции по типу данных ``\lstinline{expr}'' построится функция-преобразование ``\lstinline{show}'', которое можно будет использовать в остальной кодовой базе.

Преобразование 
``\lstinline{show}'',
%``$\inbr{show_{expr}}$'', 
сохраняет имена конструкторов, оно может быть
полезно при отладке или сериализации. Однако, как правило, также требуется иное преобразование, более ``красивое'' (\emph{pretty-printed}) и удобное для чтения человеком. 
В нём выражение представляется в ``естественном синтаксисе'': с использованием инфиксных операций, без имён конструкторов, и скобки расставлены только там, где они действительно нужны. Это преобразование может быть реализовано следующим образом:

\begin{lstlisting}
let pretty e =
  let rec helper p = function
    | Const n -> string_of_int n
    | Var x -> x
    | Binop (o, l, r) ->
        let po = prio o in
        sprintf "%s %s %s" 
          (helper po l) o (helper po r)
        |> if po <= p then br else Fun.id 
  in
  helper min_int e
\end{lstlisting}

\noindent Здесь мы пользуемся функциями ``\lstinline{prio}'', ``\lstinline{br}'' и ``\lstinline{id}'', доступными из вне. Функция ``\lstinline{prio}''
возвращает приоритет бинарной операции, ``\lstinline{br}'' окружает свой аргумент скобками, а ``\lstinline{id}'' --- тождественная функция.
Дополнительная функция ``\lstinline{helper}'' принимает числовой параметр, который обозначает приоритет окружающей операции (если такая имеется). Если приоритет текущей операции меньше или равен переданному, тогда выражение окружается скобками. На верхнем уровне мы передаем наименьшее возможное число как приоритет, чтобы убедиться, что мы не получим скобок, окружающих выражение целиком. Для простоты мы считаем, что все операции неассоциативны, но такой же шаблон кода может быть использован для поддержки ассоциативных операций.

Преобразование ``\lstinline{pretty}'' можно было бы реализовать тоже с помощью обобщённого программирования, но в имеющихся библиотеках~\cite{PPXLib,ppxderiving} \OCaml{} такой вид преобразований не поддерживается.

В оставшейся части главы мы расскажем, как можно представлять преобразования по-другому, а именно с помощью объектов, что позволит получать расширяемые преобразования. Более строгое изложения того, как должны быть устроены классы и объекты, можно найти в разделе~\ref{sec:implementation}.

Реализации этих двух функций имеют очень мало общего. Обе возвращают строки, но вторая принимает дополнительный аргумент, и 
правые части сопоставления с образцом для соответствующих конструкторов различаются. Единственной общей частью является
сопоставление с образцом само по себе. Мы можем реализовать его как отдельную функцию (\emph{обобщённый катаморфизм}, раздел~\ref{transtypes}) и параметризовать эту функцию набором  преобразований, 
соответствующих конструкторам:

% $\inbr{gcata_{expr}}$
\begin{lstlisting}
let gcata $\omega$ $\iota$ = function
| Const n         -> $\omega$#$\inbr{Const}$ $\iota$ n
| Var   x         -> $\omega$#$\inbr{Var}$   $\iota$ x
| Binop (o, l, r) -> $\omega$#$\inbr{Binop}$ $\iota$ o l r
\end{lstlisting}

Здесь мы представляем множество семантически связанных функций объектом ``$\omega$'', где методы соответствуют конструкторам
один к одному. \emph{Наследуемый атрибут} (раздел~\ref{transtypes}) ``$\iota$'' представляет дополнительный параметр, который может как использоваться функциями (например, ``pretty''), так и игнорироваться функциями (``show'').

Упомянутая выше функция ``show'' может быть реализована следующим образом\footnote{Для ясности понимания мы опустили некоторые аннотации типов, которые помогают этому листингу кода пройти проверку типов.}:

\begin{lstlisting}
let rec show e = gcata
  (object
    method $\inbr{Const}$ _ n = 
      sprintf "Const %d" n
    method $\inbr{Var}$  $\enspace$ _ x = sprinf "Var %s" x
    method $\inbr{Binop}$ _ o l r =
      sprintf "Binop (%S, %s, %s)" 
        op (show l) (show r)
  end)
  ()
  e
\end{lstlisting}

\noindent Функция ``\lstinline{pretty}'' может быть реализована аналогично.

%И, разумеется, всё то же  самое применимо к 

Заметим, что оба объекта, необходимые для реализации этих функций, могут быть экземплярами общего виртуального класса:

\begin{lstlisting}
class virtual [$\iota$, $\sigma$] $\inbr{expr}$ = object
  method virtual $\inbr{Const}$ : $\iota$ -> int -> $\sigma$
  method virtual $\inbr{Var}\enspace\;\;$ : $\iota$ -> string -> $\sigma$
  method virtual $\inbr{Binop}$ :
    $\iota$ -> string -> expr -> expr -> $\sigma$  
end
\end{lstlisting}

Конкретный класс, представляющий преобразование, будет наследоваться от этого общего предка. Чтобы иметь возможность 
вызывать рекурсивно данное преобразование, мы параметризуем класс функцией самотрансформации ``\lstinline{fself}'' 
(\emph{открытая рекурсия}). 
Написание в стиле открытой рекурсии необходимо для возможности поддержки полиморфных вариантных типов и рекурсивных определений.
Теперь преобразование выражений в удобный человеку формат может быть реализовано не в виде функции, а в виде расширяемого объекта  (обратите внимание на использование ``\lstinline{fself}''):

\begin{lstlisting}
class pretty (fself : $\iota$ -> expr -> $\sigma$) = 
object 
  inherit [int, string] $\inbr{expr}$ 
  method $\inbr{Const}$ _ n = sprintf "%d" n
  method $\inbr{Var}$ _ x = x
  method $\inbr{Binop}$ p o l r =
    let po = prio o in
    sprintf "%s %s %s" 
      (fself po l) o (fself po r)
    |> if po <= p then br else Fun.id
end
\end{lstlisting}

Функция-преобразование выражения  в строковый  формат может быть  описана с использованием класса, представленного выше, и обобщенной  функции-преобразования\footnote{Так как имена функций и классов находятся в разных пространствах имен в \textsc{OCaml}, мы можем использовать одно и то же имя для класса и функции-преобразования.}.

\begin{lstlisting}
let pretty e =
  let rec pretty_prio p = 
    gcata (new pretty pretty_prio) p 
  in
  pretty_prio min_int e
\end{lstlisting}

Также мы можем избежать объявления вложенной функции с помощью комбинатора неподвижной точки\footnote{Обычно, комбинатор неподвижной точки используется для определения рекурсивных функций, если не возможно вызывать определяемую функцию явно. В данном конкретном случае он также применяется для оптимизации, а именно экономии на создании объектов-преобразований.} ``\lstinline{fix}'' (раздел \ref{memofix}): 

\begin{lstlisting}
let pretty e = fix 
  (fun fself -> gcata (new pretty fself))
  min_int e
\end{lstlisting}

\subsection{Преобразование ``\texttt{fold}''} 

Выше мы смогли выделить две общие части для двух существенно различных преобразований: функцию обобщенного обхода
( ``\lstinline{gcata}'') и такой виртуальный класс (``\lstinline{expr}''), что все преобразования можно представить как его экземпляры.
Но стоило ли это того? В действительности, в этом примере мы добились не очень большого переиспользования кода путём добавления
большого количества абстракций. Итоговый код получился по размеру даже больше исходного.

Мы утверждаем, что преобразования в данном конкретном случае были недостаточно обобщенные. Чтобы оправдать описанный подход,
давайте рассмотрим более оптимистичный сценарий. Широко известно, что многие трансформации могут быть представлены 
(по понятным причинам) как \emph{катаморфизмы}, т.е. как ``свёртки''~\cite{CalculatingFP,Fold,Bananas}. 
Формально, чтобы определить канонический катаморфизм нам нужно абстрагировать тип ``\lstinline{expr}'' 
от самого себя\footnote{Подробнее см. в~\cite{CalculatingFP}}, но здесь мы воспользуемся более легковесным решением:


% ссылаться на разделы цитируемых у нас не принято
\begin{lstlisting}
class [$\iota$] fold (fself : $\iota$ -> expr -> $\iota$) = 
object 
  inherit [$\iota$, $\iota$] $\inbr{expr}$ 
  method $\inbr{Const}$ i n = i
  method $\inbr{Var}$ i x = i
  method $\inbr{Binop}$ i o l r = 
    fself (fself i l) r
end
\end{lstlisting}

\noindent Эта реализация просто передает ``$\iota$'' сквозь все узлы трансформируемого значения, что на первый взгляд выглядит довольно бесполезно.
Однако, слегка изменив поведение, можно получить кое-что полезное:

\begin{lstlisting}
let fv e = fix (fun fself ->
  gcata (object 
    inherit [string list] fold fself
    method $\inbr{Var}$ i x = x :: i
  end)) 
  [] e
\end{lstlisting}

\noindent Эта функция создает список всех свободных переменных в выражении, а так как в языке выражений нет способа связывать переменные, 
то это просто список всех переменных. Объект, который мы сконструировали, наследуется от ``бесполезного'' класса 
``\lstinline{fold}'' 
%``$\inbr{fold_{expr}}$'' 
и переопределяет только один метод -- метод для обработки переменных.
Весь остальной код уже работает так, как нам нужно~---
``\lstinline{gcata}''
%``$\inbr{gcata_{expr}}$'' 
обходит выражение, 
а остальные метода объекта аккуратно передают построенный список дальше.
Таким образом, мы смогли реализовать интересное преобразование с помощью очень малой модификации существующего кода, 
предоставленного уже имеющимся классом ``\lstinline{fold}''. 
%``$\inbr{fold_{expr}}$''. 

Чтобы избежать впечатления, что мы аккуратно подготавливались к
представлению именно этого примера, мы покажем ещё один:

\begin{lstlisting}
let height e = fix (fun fself ->
  gcata (object 
    inherit [int] fold fself
    method $\inbr{Binop}$ i _ l r = 
        1 + max (fself i l) (fself i r) 
  end)) 
  0 
  e
\end{lstlisting}

\noindent Здесь мы вычисляем высоту дерева выражения, используя  для другого самостоятельно объекта тот же самый базовый класс ``\lstinline{fold}''.
%``$\inbr{fold_{expr}}$'' 
Затем мы переопределяем метод для бинарного оператора, который теперь будет вычислять высоты поддеревьев, выбирать из них максимальную высоту и прибавлять единицу.

\subsection{Преобразование ``\lstinline{map}''} 

Одной из других всеми известных обобщенных функций является ``map'', которую мы строим в предположении, что тип ``\lstinline{expr}'' является функтором:

\begin{lstlisting}
class map fself = object 
  inherit [unit, expr] $\inbr{expr}$
  method $\inbr{Var}$ _ x = Var x
  method $\inbr{Const}$ _ n = Const n
  method $\inbr{Binop}$ _ o l r = 
    Binop (o, fself () l, fself () r)
end
\end{lstlisting}

\noindent У данного типа нет типовых параметров, поэтому для данного случая преобразование ``\lstinline{map}'' ---~это тривиальное копирование. Но это преобразование может служить основой для некоторых других преобразований, например, 
новый вид преобразования может быть получен, как результат наследования от этого класса и переопределения метода $\inbr{Binop}$:

%Так как нам не известно, что ``\lstinline{expr}'' -- это функтор, то всё, что мы можем сделать в функции ``\lstinline{map}'' --- 
%это копирование. \footnote{Тут надо подробнее почему функциторность важна и почему мы не можем что-то сделать} \textcolor{red}{ Однако, отнаследовавшись от этого класса, мы может получить новый вид преобразований}\footnote{Новый вид преобразования может быть получен, как результат наследования от этого класса или типа того}:

\begin{lstlisting}
class simplify fself = object 
  inherit map fself
  method $\inbr{Binop}$ _ o l r =
    match fself () l, fself () r with
    | Const l, Const r -> 
        Const ((op o) l r)
    | l, r -> 
        Binop (o, l, r)
end
\end{lstlisting}

\noindent Данный класс производит упрощение выражения: если оба аргумента бинарной операции вычисляются в константу тем же самым преобразованием, тогда 
мы можем произвести операцию сразу. Функция ``\lstinline{op}'' объявлена где-то ещё, она возвращает функцию, которая будет производить вычисление данного бинарного оператора.

Вот ещё один пример:

\begin{lstlisting}
class substitute fself state = object 
  inherit map fself
  method $\inbr{Var}$ _ x = Const (state x)  
end
\end{lstlisting}

\noindent Здесь мы выполняем подстановку переменных в выражении на значения, определенные в некотором состоянии, представленном функций ``\lstinline{state}''. Два класса, объявленных выше могут быть скомбинированы для получения интерпретатора выражений:

\begin{lstlisting}
class eval fself state =  object
  inherit substitute fself state
  inherit simplify   fself
end

let eval state e =
  fix (fun fself ->
    gcata (new eval fself state)) () e  
\end{lstlisting}

Во всех примерах мы выбрали достаточно стандартные преобразования и можно сказать, что реализовали всё достаточно малыми усилиями,
если закрыть глаза на несколько многословный синтаксис классов и объектов в  \textsc{OCaml}. В каждом случае было необходимо переопределить
только один метод и воспользоваться функцией, однозначно получаемой по типу. 
С другой стороны, мы работали с очень просто устроенным типом, он даже не был полиморфным, а поддержка полиморфизма может привести к 
специфическим проблемам. В оставшейся части работы мы покажем, что идеи, представленные выше, могут быть реализованы с помощью обобщенного программирования, где все компоненты могут быть синтезированы из объявления типа. 

\begin{comment}



В частности, наш подход предоставляет полную поддержку:

\begin{itemize}
\item полиморфизма;
\item применения типовых операторов (type operators);
\item взаимной рекурсии, где поддержка воистину \emph{расширяемых} преобразований потребует некоторых усилий;
\item полиморфных вариантных типов, с которыми будет необходимо позаботиться о гладкой интеграции возможностей полиморфных вариантом и наследования классов;
\item раздельной компиляции --- мы можем сгенерировать код по определениям типов не заглядывая внутрь модулей, от которых зависит обрабатываемый тип;
\item инкапсуляции, а именно поддержки сигнатур модулей, включая абстрактные типы и приватные определения. Обобщенные функции для абстрактных типов могут использоваться из вне модуля, но не позволят инспектировать или изменять содержимое абстрактного типа.
\end{itemize}

Что касается вопросов производительности, то как Вы могли заметить, во всех примерах мы создавали большое количество 
\emph{идентичных} объектов во время преобразования (под одному на каждый узел структуры данных). Далее мы увидим, что с этим можно побороться
с помощью мемоизации. Наконец, наш подход предоставляет система плагинов, которые могут быть использованы для генерации большого количества преобразований, как, например, ``\lstinline{show}'', ``\lstinline{fold}'' или ``\lstinline{map}''. Система плагинов расширяема, т.е. пользователи могут  реализовать их собственные плагины с помощью небольших усилий, так большая часть функциональности по обходу структуры данных предоставляется библиотекой. 
\end{comment}
% !TeX encoding = windows-1251
\section{�����}
\label{sec:implementation}

����� ������� ����������� ���������������� �������� ���������� <<�����������>> ���� �� ���� ������, ������� � ������ ������� �� �������� ��� ������ 
%� �������������� �������� ���������� 
� ����� ������� � ��������� ���� �� \ocaml{}.

%��������� ������������ ������ ������� �������� �������������� ���������� (� ��� \cd{camlp5}~\cite{Camlp5}, � ���  \cd{ppxlib}~\cite{PPXLib}), ���������� ������� ���������� � ������� ��������. �������������� ���������� �������� ���������� �����, �������������� �������������, � ���������� ��������� ��������:

���������� ���������������� ����������� � ����������� �����, �������������� �������������, � � ����� ������ ��������� ��������� ��������:

\begin{itemize}
\item ���������� ������� �������������� (\emph{generic catamorphism}, \emph{gcata}), ���� �� ������ ���;
\item ����������� �����, ������� ������������ ��� ����� ������ ��� ���� ��������������, ���� �� ������ ���;
\item ��������� ���������� ���������� �������, �� ������ �� ������ ��� ��������� �������������;
\item ��������� ������ \emph{typeinfo}, ������� �������� � ���� ����������, ����������� ��� ������� ����, � ������ ���������� ������� �������������� � ����� �������-��������������, ������� ������� �������������� � ������� �������; �� ������������ ��� ��������������� ������.
\end{itemize}

�� ������������ ��������� �������� ���������� ����� �� ���������� �������������:

\begin{itemize}
\item ������ ���������� �������������� ���� ������; GADT'� �������������� ��� ������� �������������� ����;
\item ����������� �� ���� (constraints) �� �����������;
\item �������, ������ � ���� � �������� ������ ``\lstinline{nonrec}'' �� ��������������;
\item ����������� ���� ������ (``\lstinline{...}''/``\lstinline{+=}'') �� ��������������.
\end{itemize}

� �������, ���� ��� ���� ``\lstinline{t}'' �������� �������������� ``\lstinline{show}'', �� � ����� ���������� �������� ��������� �������� (� ������� ``$\dots$'' �� ���������� �����, ���������� ������� ���� ��������):

\begin{comment}
\begin{figure}[t]
  \center
  \begin{tabular}{L{6cm}|l}
    \hline
    \multicolumn{2}{c}{� �������������� \cd{camlp5}}\\
    \hline
    \lstinline|@type ... = ... | & �������������� ����������� ��� ���������  \\
    \lstinline|and  ... = ... | & ���� � ��������� $p_1, p_2, \dots$; ������� \\
    \lstinline|   $[$ with  $p_1, p_2, \dots$ $]$| & ����������� ���� ����� ��������������; \\
    \lstinline|@$typ$| & �������� ������������ ������ ��� ���� $typ$; \\
    \lstinline|@$plugin$[$typ$]| & ��� ������ ������� ��� ���� $typ$ � \\
                                 & ������� $plugin$\\
    \hline
        \multicolumn{2}{c}{� �������������� \cd{ppxlib}}\\
    \hline
    \lstinline|type ... = ...|  & �������������� ����������� ��� \\
    \lstinline|and  ... = ...|  & ��������� ����  � ��������� $p_1, p_2, \dots$  \\
    \lstinline|[@@deriving gt | & $ $ \\
    \lstinline|  ~options:{ $p_1, p_2, \dots$}]| & \\
  \end{tabular}
  \caption{����������� ������������ ����������}
  \label{syntax}
\end{figure}
\end{comment}

\begin{lstlisting}
let $\inbr{gcata_t}$ $\dots$ = $\dots$

class virtual [$\dots$] $\inbr{t}$ = object  $\dots$ end

class [$\dots$] $\inbr{show_t}$ $\dots$ = object 
  inherit [$\dots$] $\inbr{t}$ $\dots$
  $\dots$
end

let t = {
  gcata   = $\inbr{gcata_t}$;
  $\dots$
  plugins = object method show = $\dots$ end
}
\end{lstlisting}

� ������� ��������� ``\lstinline{t}'' � ����������� � ���� �� ����� ������������ �������-��������������, ��������������� ������:

%\begin{lstlisting}
%let transform t = t.gcata
%let show      t = t.plugins#show
%\end{lstlisting}
\begin{lstlisting}
let show      t = t.plugins#show
\end{lstlisting}

������� ``\lstinline{transform(t)}'' -- ��� ������� �������� ������ �� ���������� \GT{}, ������� ����� ���� ��������������� ��� ������ ��������������� ����  ``\lstinline{t}''. 

%\begin{comment}
%�� �������~\ref{syntax} �� ��������� ���������� �������������� �����������, ������������� ��� �������������� ����������. �������� ��������, ��� ���������� ������������� ���� ��� ������� � ������� ������������� (�������������� ���� ���$\inbr{...}$) �������� ��������������� ���� ������������ \cd{camlp5}, ��� ��� ��������������� ��������������� �������������� ����������.
%\end{comment}

\subsection{���� ��������������}

������ ������� ������� �� ���� �������� �������������~\cite{Bananas} � ������� ���������� 
���������~\cite{AGKnuth,AGSwierstra,ObjectAlgebrasAttribute}.
�� ������������� ������ ������������� ���������� ����

\[
\iota \to t \to \sigma
\]
��� $t$ -- ��� ���, �������� �������� �� �����������, $\iota$ � $\sigma$~--- ���� \emph{�����������} � \emph{�������������} ���������. 
�� �� ����� ������������ ���������� ����������, ����� ��������� ��������������� ����� �������������, �� ������ �������������� ������������ ��� �������� ����� �����. 

���� ��������������� ��� �������� ���������������, �� �������������� ���� ����� ���������������. ����� �� ����� ���������� � �������
$\left\{...\right\}$ ������������� ��������� �������� � �������. � ������� ����� ������� �� ������ ������� ���������� ����� ��������������, ������������ � ������� ����� ����������, ���

\[
  \left\{\iota_i \to \alpha_i \to \sigma_i\right\}\to\iota \to\left\{\alpha_i\right\}\;t \to \sigma
\]

����� $\iota_i\to\alpha_i\to\sigma_i$ �������� ��������-���\-������������ ��� �������� ��������� $\alpha_i$. � �����, �������-�������������� ��������� ������ ��������� �� ����������� �������� � ���������� �������� � ���������� ������������� �������� ��� ��������� �����. ����� ��� ���� �������������� �����-������ ��� $n$-���������������� ���� ����� $3\cdot(n+1)$ ������� ����������:

\begin{itemize}
\item ������ $\iota_i$, $\alpha_i$, $\sigma_i$ ��� ������� �������� ��������� $\alpha_i$, ��� $\iota_i$ � $\sigma_i$ --- ��� ������� ���������� ������������ � ���������������� ��������� ��� ��������������  $\alpha_i$;
\item ���� �������������� ������� ���������� $\iota$ � $\sigma$ ��� ������������� ������������ � ���������������� ��������� ����������������� ����;
\item �������������� ������� ���������� $\varepsilon$, ������� �������������� � ``\lstinline|$\{\alpha_i\}$ t|'' ��� ����� �������� �� ����������� ����������, � �������������� � \emph{���������} ���� ``\lstinline|[> $\{\alpha_i\}$ t]|'' ��� ����������� ���������� ����� %(��������� � �������~\ref{pv})
.
\end{itemize}

��������, ���� ��� ��� ������������������ ��� \lstinline{($\alpha$, $\beta$) t}, �� ���������� ������ ������-������ ����� 

\begin{lstlisting}
class virtual [$\iota_\alpha$, $\!\alpha$, $\!\sigma_\alpha$, $\!\iota_\beta$, $\!\beta$, $\!\sigma_\beta$, $\!\iota$, $\!\varepsilon$, $\!\sigma$] $\inbr{t}$
\end{lstlisting}

���������� �������������� ����� ������������� �� ����� ������ �, ��������, ���������������� ��������� �� ������� ����������.
������������� ���������� ������ �������� ��������� ����������-�������:

\begin{itemize}
\item $n$ �������, ������������� ������� ���������: \lstinline|f$_{\alpha_i}$ : $\iota_i$ -> $\alpha_i$ -> $\sigma_i$|;
\item ������� ��� ���������� �������� ��������: \lstinline|fself : $\iota$ -> $\varepsilon$ ->  $\sigma$|.
\end{itemize}

��������, ��� ����, ����������� ���� � �������������� ``\lstinline{show}'' ��������� ����������� ����� ����� ��������� ���

\begin{lstlisting}
class [$\alpha$, $\beta$, $\varepsilon$] $\inbr{show_t}$ 
  (f$_\alpha$     : unit -> $\alpha$ -> string)
  (f$_\beta$     : unit -> $\beta$ -> string)
  (fself : unit -> $\varepsilon$ -> string) =
object 
  inherit [ unit, $\alpha$, string
          , unit, $\beta$, string
          , unit, $\varepsilon$, string] $\inbr{t}$
  $\dots$
end 
\end{lstlisting}

�������� ��������, ��� �� ������������ ��� ���������� ��� ���� �����, ���� ��� ��������� ����� ��������� ���������� ����� ���� �������, ��������, ``\lstinline{fself}''
����� ������ ��� ����������� �����. ���������� ����� �������: ���� �� \emph{����������} ��������� ���
�� �� � ����� ������ �� ����� ��� �����������. �������������, ��� ��������� ���������� ���������� ���������� ���� ��������� ������ ����� ����� ���������.

��� ����� ��������� �������� ����� ������������ � �����������. ������������ ������� ���������� ������� ���������� � ������� ����� ����������.
������, ������������� ����������� ����������� � ���� ������ ���� ��� ����� ������������� �������������� \emph{�������} � ���� ���� 
������������ �� ������ ������-������.
� ����������� ������� �������������� ����������� ���� ��������� ������������� ����������� ������� ��� ��������� ������� ��������. 
� ������ ������ ������ ������� ��������� ����� ��� ������������������ (��������, ���  ``\lstinline{show}'' ����������� ������� ���������� ���������������� � ������� ����), �� ������ ������� �������� �������� ������� ���������� ������������� ������� ����������. % (��������� � �������~\ref{plugins}).

��� ����� ���������� ������� ���� ���������� � ������� ������ ������. ����� ��� ������������  ``\lstinline|C of a$_1$ * a$_2$ * ... * a$_k$|'' ����� ��������� ���������:

\begin{lstlisting}
method virtual $\inbr{C}$ 
  : $\iota$ -> $\varepsilon$ -> a$_1$ -> a$_2$ -> ... -> a$_k$ -> $\sigma$
\end{lstlisting}

\noindent ����� ��������� �� ������ ����������� ������� � ���������, ��������������� ������������, �� � ��������, ������� ������ �������������.

�������, �� ������ ��� ���������� ������� ��������������. ��� ������ ���������� ��� ������ ����������� ���������� �����.

��� ����, �� ����������� ����������� ���������� �����, � ������ ``\lstinline|$\{\alpha_i\}$ t|'' \emph{���������� ������� ��������������} ����� ��������� ���:

\begin{lstlisting}
val $\inbr{gcata_t}$ : [$\{\iota_{\alpha_i}$, $\!\!\alpha_i$, $\!\!\sigma_{\alpha_i}\}$, $\!\!\iota$, $\!\!\{\alpha_i\}$ t, $\sigma$] #$\inbr{t}$ 
                -> $\iota$ -> $\{\alpha_i\}$ t -> $\sigma$
\end{lstlisting}

��� ��������� ������, �������������� ��������������, � �������� ������� ���������, ���������� ���� ������������ �� �������� ������, ��������������� ������� ����������������, ����������� �������, ��������, ������� ����� ������������� � ���������� ������������� �������.
�������������� �������� ``$\varepsilon$'' ���������������� � �������������� ���, � 
��� ����������� ���������� ����� ---~� \emph{��������}
������ ����:  ``\lstinline|[> $\!\!\{\alpha_i\}$ t]|''. 
��� ��������� ��������� ������� �������������� � �������, ��������������� �������������� ������������ ���� � �\'������ ���������� �������������.


\subsection{���������� ����������� ����� � ����������}
\label{memofix}

� ���������� ������� �� ������� ������ ���������� ��������, �������������� ��������������. ������ ���������� ��������������� ������ ��� ������������ ���������, ����� ��������� �������-��������������. �� ���������� �������, ����� ��� ������ � �������������� ����������� ���������������� ������������ ������������� ������������ �������� ���������.

�� ����������  �� �������� ��������: �����, ����������� ���������� �������������� ��������� ������� �������������� ������ ���� ��� ��������.
����� ������� ����� ������� ��������� ���������� ����������� �����. �  ���� �������
�� ���������� ������ ������� ����� ����������, � ������ ��� ���������� ���������� ����.
�� ������� ����������� ������ ����������� ���� ����� ������� ���������� (��������� � 
�������~\ref{murec}).

%�� ���������� ��� ������ �� �������~\ref{sec:expo}:

�������������� $tr$ ��� ���� $t$, �������������� � ������� ������� $\inbr{tr_{t}}$, ����������� ��������� ��������:

\begin{lstlisting}
let $\inbr{tr_{t}}$ $\{f_i\}$ $\iota$ x =
  transform $t$.gcata (new $\inbr{tr_{t}}$ $\{f_i\}$) $\iota$ x
\end{lstlisting}
\noindent ��� ������������ � ������� ������  $\inbr{tr_{t}}$, ��������������� �������������� ��� ���� $t$, � ���������� ������� �������������� ���� $t$. 

\begin{lstlisting}
let transform t = transform_gc t.gcata

let transform_gc gcata make_obj $\iota$ x =
  let rec obj = lazy (make_obj fself)
  and fself $\iota$ x = 
    gcata (Lazy.force obj) $\iota$ x in
  fself $\iota$ x
\end{lstlisting}

� ���� ���������� ����� ������������ ���������� ����������� ����� \lstinline{transform_gc}, ����������� ���� ��� ��� ���� �����. ��� ����� ��� ����, ����� ���������� � �����, ��������� � ������� �������� ��������, �������-�������������� ���� $t$, ������� �� ����� �������. � ���������� ������������ ������� �������� �������, ��������������� ��������������, ����� �������� �������� ����� ������� ��� ������ ����������� ������. ��� �������� �������� �� ������� ����, ��� �� ����� ���������� �������������� ������, �������������� ��������������, �� ����������.

%\begin{comment}
%\begin{lstlisting}
%let $\inbr{pretty_{expr}}$ i e = 
%  fix (fun fself -> 
%         $\inbr{gcata_{expr}}$ (new $\inbr{pretty_{expr}}$ fself))
%  i e
%\end{lstlisting}
%\end{comment}


%\begin{comment}
%����� ������������ ������ ����������, ���� ������� ����������� ������ ���, ����� ���������� \lstinline{fself}'' � ������ �������������� (�� ����, ��� ������� ���� � ������ ����������������� ��������). ��� ��� ��� ������� ���������, �� �� �������� ����� ���������������.
%
%�� ����������� �������� �������, ��������������� ��������������, � ������� ������� ����������. ��� ����� �� ������������ �������� ������� � �������, ������� ���������
%�������� ``\lstinline{fself}''. ���������� ����������� ����������� ����� �������� ��������� �������:
%
%\begin{lstlisting}
%let fix gcata make_obj $\iota$ x =
%  let rec obj = lazy (make_obj fself)
%  and fself $\iota$ x = gcata (Lazy.force obj) $\iota$ x in
%  fself $\iota$ x
%\end{lstlisting}
%
%���� ���������� ����� �������������� ��� ���� ����� � �� �������� ������������ �� ���� ������. ������ �� ����� ������� ��������� ���������� ������� ``\lstinline{transform}'':
%
%\begin{lstlisting}
%let transform typeinfo = fix typeinfo.gcata
%\end{lstlisting}
%
%� ������� ����� ����������� ������������ �� ����� ������������ ���������� ����������� ����� ����:

%\begin{lstlisting}
%let $\inbr{show_{expr}}$ e =
%  transform(expr) (fun fself -> new $\inbr{show_{expr}}$ fself) () e
%\end{lstlisting}

%\end{comment}

\subsection{������� ����������� �����������}
\label{murec}

� ������, ���� ���������� ��������� �������������� ��� ������ �������-����������� ����������� �����, �� ����������� ��� ��������� ����������. ��-������ ������ �������������� ����� ������ �������� �������������� ��� ������ �����, ����������� �������-����������, � ��������� ������������� ������������� � �������� ���������. ��-������, ��� ������ ������ ����� ����� ����������� ���������� ����������� �����, ������� ������������ <<����������� � ����>> �������������� ������ ���� ��� ������ ������ �����. � �������, ����� ���������� �������� �������������� ������ ��������������, ������������������ ��� ������ ��� ����������� �������������� ������ �����, ����� ��������� ��������� ������� ��� �������� ��� �������-����������� � ������� �����������.


\subsection{����������}
�� ������ ���������� ������ ������������� �������������� � ������� �������� ���� ��������������� ���������� Generic Transformers\footnote{\url{https://github.com/Kakadu/GT/tree/v0.3.0}} (GT), ����������� � �������������� ����������� ���������������� ��������� �� ����� ������ ����������� ��������������. � ��� �������������� ��� �������� ���������������� ���� �������������� ���������� ��� ����� \ocaml{}: \textsc{PPX}~\cite{PPXLib} � \camlpfive~\cite{camlp5}. ���������� \GT{} �������� �����������: � ��� ����������� ��������� ���������� ���������������� ��������, ����������� ���������� ����� ����� ��������������, � ����� ����� ����������� ��������������, �������������� � ������ �������� �����������.



\begin{comment}


\subsection{������� ��������}
\label{plugins}


\subsection{�������� ��������}
%\label{murec}

%������� 2 �������� ��� ���

������ ��������� ������� ����������� ����������� ����� ������� �������������� ������.
���������, �������� ���� ����������� ��������� ����� ���� ����������� �����, ��� � ��� 
���������� ������, �� ��� ����� �������� ������������� ���������� ��������������.
�� ���������������� ��� ������� � ������� ����. ���������� ����������� ����


\begin{lstlisting}
type expr = $\dots$ | LocalDef of def * expr
and  def  = Def of string * expr
\end{lstlisting}

��� �� �������� �������� ����� (����������, �������� �������� � �.�.) � ���������� ���� ���������. �������� ��������, ��� ���������� ������� �������������� ��� ����� ����� �����  ���� ��������� ��� ��� ����, ��� ��� ��� �� ���� ������ ������������� ������ ��� ���������� �������������� �� ����� ������� ������� � �� ������� �� ������� �������� � ������������ �����.

\begin{lstlisting}
let $\inbr{gcata_{expr}}$ $\omega$ $\iota$ = function
$\dots$
| LocalDef (d, e) as x -> $\omega$#$\inbr{LocalDef}$ $\iota$ x d e

let $\inbr{gcata_{def}}$ $\omega$ $\iota$ = function
| Def (s, e) as x -> $\omega$#$\inbr{Def}$ $\iota$ x s e
\end{lstlisting}

�� �� ����� ����� � ��� ������ ������-������. ������, ���� �� ������ ������������� ���������� ��������������, �� ��� ����������� �������������� �������� 
���� ``\lstinline{expr}'' ������ ������ ��� ``\lstinline{def}'', � ��������. ��� ����� ���� ������� � ������� ������� ����������� ����������� ������� (�� ����� �� �������� �������� ����� ����):

\begin{lstlisting}
class $\inbr{show_{expr}}$ fself = object 
  inherit [unit, _, string] $\inbr{expr}$ fself
  $\dots$
  method $\inbr{LocalDef}$ $\iota$ x d e =
    $\dots$ (fix $\inbr{gcata_{def}}$ (fun fself -> new $\inbr{show_{def}}$ fself) $\dots$) $\dots$
end
and $\inbr{show_{def}}$ fself = object 
  inherit [unit, _, string] $\inbr{def}$ fself
  method $\inbr{Def}$ $\iota$ x s e =
    $\dots$ (fix $\inbr{gcata_{expr}}$ (fun fself -> new $\inbr{show_{expr}}$ fself) $\dots$) $\dots$
end
\end{lstlisting}

��������, ��� � ����� ���������� ``\lstinline{fix}'' �� ������� \emph{����������} ������  (``$\inbr{show_{def}}$'' � ``$\inbr{show_{expr}}$''). �� ������ ������, ��� ������ �������� ��� ����������. ������ ������, ��� \emph{����������} �������������� ������������� ��������.
�� ��� ��������, ���� ��� ����������� �������������� ��������� � ������ 
 ``$\inbr{show_{expr}}$''? �������� �������, ������������� ����, �� ���������� ��������������� �� ``$\inbr{show_{expr}}$'', �������������� ��������� ������ � ��������������� ������� � ������� ����������� ����������� �����:

\begin{lstlisting}
class custom_show fself = object 
  inherit $\inbr{show_{expr}}$ fself
  method $\inbr{Const}$ $\iota$ x n = "a constant"
end

let custom_show e = 
  fix $\inbr{gcata_{expr}}$ (fun fself -> new custom_show fself) () e
\end{lstlisting}

� ��� �� ����� �������� ���, ��� �� �������, ������ �� �� ���������� �����
``$\inbr{LocalDef}$'', ������� ���������� ����� �� ��������� ��� ����  ``\lstinline{def}'', ������� � ���� ������� ���������� ������� �� ��������� ��� ����  ``\lstinline{expr}''.
����������, ��� �� �������������� ��������� ������ ����� ���������� ������� ������������ �������������� �����, � ������ ��� ���� ``\lstinline{expr}''. 
��� ��������� ���� ``\lstinline{expr}'' � ������ ����� �� ��� ������������� ����������� �������. ����� ��������� ��� ���������, ��� �������� ��������� ���������� ������� ����������� ������� \emph{�������}, ��� ������������ ��� ���� �������������.

���� ������� �������� ����� ���������� �� ���� �������� ��������. �������, �� ��������������� ���������� ����� �������������� ��������������� \emph{����} �����, ����������� �� ������� ����������� ����������� �����.
��� ��� ��� �������������� �������� ���������� �� ����������� �������, ��� �������� �������� ��� ������ ��� ��������������. ��� ������ ������� ��� ����� �������� ��� ���:

\begin{lstlisting}
class $\inbr{show\_stub_{expr}}$ $f_{expr}$ $f_{def}$ = object 
  inherit [unit, _, string] $\inbr{expr}$ $f_{expr}$
  $\dots$
  method $\inbr{LocalDef}$ $\iota$ x d e = $\dots$ ($f_{def}$ $\dots$) $\dots$
end

class $\inbr{show\_stub_{def}}$ $f_{expr}$ $f_{def}$ = object 
  inherit [unit, _, string] $\inbr{def}$ $f_{def}$
  method $\inbr{Def}$ $\iota$ x s e = $\dots$ ($f_{expr}$ $\dots$) $\dots$
end
\end{lstlisting}

�������� �������� �� ���������� ����������� �������.

����� �� ����������� ���������� ����������� ����� ��� ����� ������� ������������ �����������:

\begin{lstlisting}
let $\inbr{fix_{expr, def}}$ ($c_{expr}$, $c_{def}$) =
  let rec $t_{expr}$ $\iota$ x = $\inbr{gcata_{expr}}$ ($c_{expr}$ $t_{expr}$ $t_{def}$) $\iota$ x
  and $t_{def}$ $\iota$ x = $\inbr{gcata_{def}}$ ($c_{def}$ $t_{expr}$ $t_{def}$) $\iota$ x in
  ($t_{expr}$, $t_{def}$)
\end{lstlisting}

����� $c_{expr}$ � $c_{def}$ �������� ������������ ��������, ������� ��������� ��� ��������� ������� �������������� ���� �����, ������� ����������� �� ������� ����������� �����������. �������� ��������, ��� ��� �� ����� ���������� ����������� ����� ����� �������������� ��� ����, ����� ��������������� ����� ���������� �������������� ��� ������� ������� ������������ ����������� �����.

� ����� ��������������� �������� �� ����� ��������������� ���������� �� ��������� ��� ������ ����������� ��������������:

\begin{lstlisting}
let $\inbr{show_{expr}}$, $\inbr{show_{def}}$ =
  $\inbr{fix_{expr,def}}$ (new $\inbr{show\_stub_{expr}}$, new $\inbr{show\_stub_{def}}$) 
\end{lstlisting}

��� �������������� �� ���������, ��-������, ������ ����������� �� ���� ���������� � ����������� � ����� ��� ��������������� �����, � ��-������, ������������ ��� �������� ������� ��������������, � ��������� �����������:

\begin{lstlisting}
class $\inbr{show_{expr}}$ fself = object 
  inherit $\inbr{show\_stub_{expr}}$ fself $\inbr{show_{def}}$ 
end
class $\inbr{show_{def}}$ fself = object 
  inherit $\inbr{show\_stub_{def}}$ $\inbr{show_{expr}}$ fself 
end
\end{lstlisting}

����� �� ����� ������� ������� ����������� ���� ������������ �� ������� (� �������� ����������� �������), ��� ��������� ������������� �������� �������������� �������������� ���� ����� � ������, ��� ��� ���� ������������, �� �� ���������.

� ������ �������, ����� ��������� ��������� ��������������, ������ ���������� ������������� �� \emph{��������������} ������� � ������������ ����������� ���������� ����������� �����.
��� ������ ����������� ���������� ������ �������������� �������� ����� ����� ������, ��� � ��� ���������� ���������� ����:

\begin{lstlisting}
let custom_show, _ =
  $\inbr{fix_{expr,def}}$ ((fun $f_{expr}$ $f_{def}$ ->
                  object inherit $\inbr{show\_stub_{expr}}$ $f_{expr}$ $f_{def}$
                    method $\inbr{Const}$ $\iota$ x n = "a constant"
                  end),
                new $\inbr{show\_stub_{def}}$) 
\end{lstlisting}

� ���������� ���������� ���������� �� ���������� ������������� ���������� ����������� �����, ������� ������� ���� �� �������, ������� ��� ������ � ������� ~\ref{memofix}. � ���� ��, �� ��������� ������ ���������� � ��������� � ����������� � ����, ����� ��� 
���� ``\lstinline{t}'' ���� ���������� ��� ���� ����������� � ������� ��������� 
``\lstinline{fix(t)}''. �������������, ������, �������� ������� � ���, ��� ��� �������� ������� �����������, ����� ��������������� ������������ ���������.

������ ������������ ���� ��������� � ���������� �������� ��������: �� ���������� �� �� ��������, ��� ���������� ����� ������� �������������� ��� ����  ����������, ����� ����������� �������� ��������. ������, ������ ������, ��� �� ���. ��������, ���������� ��������� ���������� ����:

\begin{lstlisting}
type ($\alpha$, $\beta$) a = A of $\alpha$ b * $\beta$ b
and  $\alpha$ b = X of ($\alpha$, $\alpha$) a
\end{lstlisting}

� ���������� ������������ ``\lstinline{A}'' �� ����� \emph{���������} �������������� ���� ``\lstinline{b}'', � ������� ��� ����������� \emph{���} �������~--- ���``\lstinline{$\alpha$ b}'' � ��� ``\lstinline{$\beta$ b}''. ������, ��� ``\lstinline{a}'' �� �������� ����������~--- ����� �������������� ���� ``\lstinline{($\alpha$, $\beta$) a}'' �� ����� � ������������� �������������� �������� ����� ``\lstinline{($\alpha$, $\alpha$) a}'' � ``\lstinline{($\beta$, $\beta$) a}''.

�������������, �� ��� ������� ����� ���������� �����. ����������, ��� ������� ����������� ���������� ����� �������� \emph{�������������} � ��� ������, ��� ��� �� ������ ����� ���� ��������� �� ��� �� ������� ����������� �����������, � ������, ����� ������ ���� ����� ������� ���������. ���� �� ������� ������ ���������� ����, ������, ��

\begin{lstlisting}
...
and $\alpha$ b = int
\end{lstlisting}

�� �� ������� ���������� �����, ������� �� �������������� � ���. ������, ��� ��� ���� ``\lstinline{a}'' � ``\lstinline{b}''  \emph{�� ��������}
�� ���� ������� ������������, �� �� ����������� ����� ����� ���� ����������, ��� ��� �������� ��������������� ������ �����������.



\subsection{����������� ���������� ����}
\label{pv}

�� ������� ��������� ����������� ���������� �����~\cite{PolyVar,PolyVarReuse} ������ ����� ����� ������, ��� ��� ��� ��������� ����������� 
������������������ ����������� �������� ������ � ����������� ���������� ����������������� ��������������.
������� �������� ����� ������������ ���������� ������  � ���������������, �������� ����������� 
\emph{����������} ����������� ����� ����������� ���������� ����� ���� ���������� ����� ������������� ��� ��������������� ���������� ����� � ����. 

����� ������� �������� ��������������  \emph{���������} ���������� � ����������� �������������. ����� ��������� ����� ����� ��������������, �� ������ �������� ��� ���������� ����������� ������� ������������ ��������������� �����.

��� �� ������� �����, �������������� ��������  ``$\varepsilon$'' ����������� � �������� ������������� ������������ ����������� ����. �������������, ������ ���� ��������� ������������ �� �� ������� ����������� �������������� �� ��� ����� \emph{��������} ����\footnote{�� �������������� �� ������������� ������� ``������'' ��� ��� � \textsc{OCaml} ��� ���������� ��������������.}. 
��� ����� ���� ���������� ������������� ������ ���������� ������� �������������, ������� ���������� ``��������'':

\begin{lstlisting}
let $\inbr{gcata_t}$ $\omega$ $\iota$ subj =
  match subj with
  $\dots$
  | C $\dots$ -> $\omega$#$\inbr{C}$ $\iota$ (match subj with #t as subj -> subj) $\dots$
  $\dots$
\end{lstlisting}

��� ���������� � ���������� ������� �������, ��������������� ��������������, � �������� ������������� ����, � �� ����� ��� ���������� ������� �������������� ��������� ��������� ���.

���� ��������� ����������� ���������� ����� ������������, �� ���������� ������� �������������� ������������ �������� � ���������-������ � �������� ���������� ��������������� ����������� �������� ��������������.



\end{comment}
\section{Обзор похожих решений}
\label{sec:relatedworks}

В данной работе использованы одновременно и функциональные (комбинаторы), и объектно-ориентированные возможности языка \textsc{OCaml}. Можно найти связанные работы  одновременно и в области типизированного функционального и объектно-ориентированного программирования. Наиболее близкой, использующий язык \textsc{OCaml} и имеющей отношение к этой работе, библиотекой является \textsc{Visitors}~\cite{Visitors}, которая использует те же самые идеи, но принимает существенно другие дизайнерские решения. Детальное сравнение с \textsc{Visitors} вы найдете в конце данного раздела.

Во-первых, существует несколько библиотек для обобщенного программирования для \textsc{OCaml}, которые используют полностью генеративный подход~\cite{Yallop,PPXLib}~--- все необходимые обобщенные функции для всех типов генерируются по отдельности. Этот подход очень практичен до тех пор, пока набор предоставляемых преобразований удовлетворяет всем нуждам. Однако, если это не так, необходимо расширить кодовую базу, реализовав все отсутствующие функции заново
(с потенциально очень малым переиспользовыванием кода). К тому же, те функции,
которые получаются в результате, нерасширяемы. В нашем подходе, во-первых,
большое количество полезных обобщенных функций может быть получено из уже сгенерированных. Во-вторых, чтобы получить полностью новый плагин, достаточно модифицировать только ``интересные'' части, так как функции обхода и класс для объекта преобразования библиотека создает самостоятельно.

Несколько подходов для функционального обобщенного программирования используют 
\emph{представление типов}~\cite{Hinze}. В основе лежит идея разработки универсального представления для произвольного типа, преобразования которого необходимо получить, и предоставления двух функций, выполняющих преобразование в универсальное представление и обратно, и в идеале образующих изоморфизм. Обобщенные функции преобразуют представление исходных типов данных, что позволяет реализовать все необходимые преобразования один раз. Функции трансляции в универсальное представление и обратно могут быть получены (полу)автоматически, используя такие особенности системы типов  как классы типов~\cite{Hinze,ALaCarte} и семейства типов~\cite{InstantGenerics} в языке \textsc{Haskell}, или  используя синтаксические расширения~\cite{GenericOCaml} в языке \textsc{OCaml}. Хотя некоторые из этих подходов позволяют модификацию получаемых преобразований (например, обработка некоторых типов особым образом) и поддерживают расширяемые типы, наш подход более гибок, так как позволяет модификацию на уровне отдельных конструкторов. К тому же, мы позволяем сосуществовать нескольким видам преобразований для одного типа.

Другой подход был задействован в ``Scrap Your Boilerplate''~\cite{SYB} (для краткости SYB), изначально разработанного для языка \textsc{Haskell}. Он делает возможным реализовать преобразования,  которые обнаруживают вхождения конкретного типа в произвольной структуре данных. Поддерживаются два основных вида действий: \emph{запросы}, которые выбирают значения конкретного типа данных на основе критериев, заданных пользователем, и \emph{преобразования}, которые единообразно применяют преобразование, сохраняющее тип, в конкретной структуре данных. В последующих статьях этот подход был расширен для трансформаций, которые обходят пару структур данных одновременно~\cite{SYB1}, а также поддержкой расширения уже существующих преобразований новыми случаями~\cite{SYB2}. Позднее, данный подход был реализован в других языках, включая \textsc{OCaml}~\cite{SYBOCaml,Staged}. В отличие от нашего случая, SYB позволяет применять трансформации к конкретным типам целиком, а не отдельным конструкторам. К тому же, многообразие получающихся преобразований выглядит достаточно ограниченным. Также, потенциально, преобразования в SYB-стиле могут сломать барьер инкапсуляции, так как могут обнаруживать вхождения значений нужно типа в структуре данных \emph{произвольного} типа. Таким образом, поведение зависит от особенностей внутренней реализации структуры данных, даже от тех, что были скрыты при инкапсуляции. Это может привести, во-первых, к возможности нежелаемой обратной разработки (reverse engineering) путём применения различных чувствительных к типу, преобразований и анализа результатов. Во-вторых, к ненадежности интерфейсов: после модификации структуры данных реализация обобщенной функции для \emph{старой} версии всё ещё может быть применена без статических или динамических ошибок, но с неправильным (или нежелательным) результатом.

Существует определенное сходство между нашим подходом и \emph{алгебрами объектов}~\cite{ObjectAlgebras}. Алгебры объектов были предложены как решение проблемы выражения (expression problem) в распространенных объектно-ориентированных языках  (\textsc{Java}, \textsc{C++}, \textsc{C\#}), которые не требуют продвинутых особенностей системы типов кроме наследования и шаблонов. В оригинальном представлении алгебры объектов были преподнесены как шаблон проектирования и реализации; в последующих работах изначальная идея была улучшена различными способами~\cite{ObjectAlgebrasAttribute,ObjectAlgebrasSYB}. При использовании алгебр объектов преобразуемая структура данных также кодируется с использованием идеи ``методы и варианты (конструкторы) один к одному'', которая предоставляет расширяемость в обоих направлениях, а также ретроактивную реализацию. Однако, будучи  разработанной для совершенно другого языкового окружения, решение с использование алгебр объектов существенно отличается от нашего. Во-первых, с использованием алгебр объектов ``форма'' структуры данных должна быть представлена в виде обобщенной функции, которая принимает конкретный экземпляр алгебры объектов как параметр (кодирование Чёрча для типов~\cite{Hinze}). Применяя данную функцию к различным реализациям алгебры объектов можно получать различные преобразования (например, распечатывание). Чтобы инстанциировать саму структуру данных нужно предоставить особый экземпляр алгебры объектов~---~\emph{фабрику}. Однако, после инстанциации структура данных больше не может быть трансформирована обобщенным образом. Следовательно, алгебры объектов заставляют пользователя переключиться на представление данных с помощью функций, которое может быть, а может не быть удобно в зависимости от обстоятельств.  Наш же подход недеструктивно добавляет новую функциональность к уже знакомому миру алгебраических типов данных, сопоставления с образцом и рекурсивных функций. Обобщенные реализации преобразований полностью отделены от представления данных и пользователи могут свободно преобразовывать их структуры данных привычным способом  без потери возможности объявлять (и расширять) обобщенные функции. Другой особенностью OCaml, в отличии от распространенных языков объектно-ориентированного программирования, является то, что для написания расширяемого кода в основном используются полиморфные вариантные типы, а не классы. Поддержка полиморфных вариантных типов для написания расширяемых типов данных требует нового подхода.


Итого, среди уже существующих библиотек для обобщенного программирования для \textsc{OCaml} мы можем называть две, которые напоминают нашу: \cd{ppx\_deriving}/\cd{ppx\_traverse}, последняя версия которых находится в кодовой базе \cd{ppxlib}~\cite{PPXLib}, и \textsc{Visitors}~\cite{Visitors}.

\cd{ppx\_deriving} является наипростейшим подходом: объявления типов данных отображаются один к одному в рекурсивные функции, представляющие конкретный вид преобразования. Это наиболее эффективная реализация, так как функции вызываются напрямую, без позднего связывания, но нерасширяемая. Если пользователю понадобится слегка модифицировать сгенерированную функцию, то он должен будет полностью скопировать реализацию функции и изменить её. Количество работы по программированию нового преобразования может существенно увеличиться, если тип данных будет видоизменяться во время цикла разработки.

В \cd{ppx\_traverse} расширяемые трансформации также представлены как объекты. В отличие от нашего подхода, там не используется кодирование конструкторов и методов один к одному. К тому же \cd{ppx\_traverse} не использует наследуемые атрибуты, следовательно некоторые преобразования, такие как проверка на равенство и сравнение, невыразимы.

\textsc{Visitors}, с другой стороны, использует сходный с нашим подход, в котором были приняты многие решения, отвергнутые нами, и наоборот.
Ниже мы подытожим главные различия:

\begin{itemize}
   \item \textsc{Visitors} полностью объектно-ориентированы. Чтобы воспользоваться преобразованием необходим создать некоторый объект и вызвать нужный метод. В нашем случае, если используются возможности, предусмотренные заранее, то можно использовать более естественный комбинаторный подход.
     
   \item \textsc{Visitors} реализуют некоторое количество преобразований в специфичной ad-hoc манере. В нашем случае все преобразования принадлежат некоторой обобщенной схеме. Различные трансформации можно скомбинировать с помощью наследования, если типы в схеме унифицируются. Мы также заявляем, что в нашей библиотека реализация ползовательски плагинов с трансформациями проще. 
     
   \item Как и  SYB, \textsc{Visitors} поддерживают указание способа трансформации для входящих в структуру данных типов: для каждого типа присутствует метод в объекте, представляющий трансформацию. Хотя такое представление добавляет некоторой гибкости мы осознанно отказывается от него, так как оно позволяет преодолеть инкапсуляционный барьер: изменяя методы преобразования (которые не могут быть скрыты в сигнатуре), можно получить некоторую информацию об внутреннем реализации инкапсулированной структуры данных. Более того, абстрактные структуры данных могут быть изменены способом, не предусмотренным публичным интерфейсом

   \item В нашем случае типовые параметры классов, представляющих трансформацию, должны быть указаны пользователем. В \textsc{Visitors} это работа возлагается на плечи компилятора, с помощью оригинального трюка. Однако, он не позволяет использовать \textsc{Visitors} в сигнатурах модулей. В нашем случае нет никаких проблем: поддерживается работа и с файлами реализации, и с файлами сигнатур.

   \item \textsc{Visitors} на сегодняшний день\footnote{Последней доступной версией на данный момент является 20180513.} не поддерживает полиморфные вариантные типы.
   
   \item \textsc{GT} поддержает произвольные применения конструкторов типов, а  \textsc{Visitors} и в мономорфном, и в полиморфном режиме -- нет.
     Например, данный пример не компилируется:
     
   \begin{lstlisting}
   type ('a,'b) alist = Nil | Cons of 'a * 'b
   [@@deriving visitors { variety = "map"
                        ; polymorphic = true }]

   type 'a list = ('a, 'a list) alist
   [@@deriving visitors { variety = "map"
                        ; polymorphic = false }]
   \end{lstlisting}
   
   Более того, добавление искусственного конструктора не решает проблему:
   
   \begin{lstlisting}
   type 'a list = L of ('a, 'a list) alist [@@unboxed]
   [@@deriving visitors { variety = "map"
                        ; polymorphic = false }]
   \end{lstlisting}
    
    Также присутствуют сложности с переименованиями (aliases) типов в полиморфном режиме (мономорфная часть библиотеки \textsc{Visitors} компилируется успешно):
    
    \begin{lstlisting}
    type ('a,'b) t = Foo of 'a * 'b (* OK *)
    [@@deriving visitors { variety = "map"
                         ; polymorphic = true }]

    type 'a t2 = ('a, int) t
    [@@deriving visitors { variety = "map"; name="somename"
                         ; polymorphic = true }]
    \end{lstlisting}
    
    Сгенерированный код можно исправить вручную, путём удаления типовых аннотаций для явного полиморфизма (explicit polymorphism) у методов, что приведет к коду, который очень напоминает генерируемый  \textsc{GT}. Из этого мы можем заключить, что на \textsc{GT} можно смотреть как перереализацию полиморфного режима библиотеки  \textsc{Visitors}, где большее количество объявлений типов компилируется корректно.
    
\end{itemize}

\section{Реализация и примеры}
\label{sec:Evaluation}

Представленный метод был реализован в библиотеке Generic Transformers\footnote{\url{https://github.com/Kakadu/GT/tree/v0.3.0}} (\GT). Библиотека  поддерживает два наиболее распространенных вида синтаксических расширений языка \ocaml{}: \textsc{PPX}~\cite{PPXLib} и \camlpfive~\cite{camlp5}. Библиотека \GT{} является расширяемой: к ней прилагается интерфейс для добавления пользовательских плагинов, реализующих порождение новых видов преобразований, а также набор стандартных преобразований, использующихся в других подобных библиотеках.


В этом разделе мы представим несколько примеров, реализованных с помощью нашего подхода. В них используются синтаксические расширения \camlpfive{}, но это же может быть реализовано также с использованием \PPX{}. %Данная работа является прямым наследником~\cite{SCICO} и все примеры из той статьи работают и в этой версии. 

\subsection{Рассмотренные примеры}

Сначала (раздел~\ref{sec:lists}) мы продемонстрируем совместимость 
нашего подхода для \emph{типов данных с неограниченной рекурсией}
(ключ компилятора \texttt{-rectypes}), 
реализовав
представление логических значений, использующихся в  библиотеке реляционного программирования \OCanren{}~\cite{OCanren}. %Особенностью данного примера является то, что в нём используется относительно редкий ключа компилятора \texttt{-rectypes} для объявления типов данных. Не всякое представление объектов-преобразований будет работать с типами данных, для которых необходим этот ключ компиляции.

Затем мы решим <<The Expression Problem>>~\cite{ExpressionProblem}
%, с помощью полиморфных вариантных типов и нашего подхода
 (раздел~\ref{sec:nameless}). Эта задача часто используется как ``лакмусовый тест'' для оценки подходов к обобщенному программированию~\cite{ObjectAlgebras,ALaCarte}. В литературе встречается различные подходы к решению этой задачи, но наша реализация данного примера интересна тем, что использует  и обобщенное программирование, и \emph{полиморфные вариантные типы} языка \OCaml{}.

Эти два примера демонстрируют использование предложенного метода с теми типами данных \OCaml{}, которые в данный момент не поддерживаются другими подходами к построению расширяемых преобразований (в частности, \visitors~\cite{Visitors}). 
Также данные примеры помогут обосновать наш дизайн  интерфейса объектов, а именно ответить на вопрос: <<Эффективно ли  на практике кодирование конструкторов один к одному в методы объектов, позволяет ли оно описывать достаточно разнообразные расширяемые преобразования?>>.

%Убрать то, что справа:
%В нём мы будем отдельно описывать преобразования для различных частей языка, а потом объединять их с помощью наследования. Это пример также не может быть переписан с использованием \visitors{}, так как та не поддерживает полиморфные вариантные типы языка \ocaml{}.

В разделе~\ref{sec:irregular} мы обсудим работу с нерегулярными типами данных~\cite{irregular}, которые не поддерживаются нашим методом непосредственно, а требуют некоторого изменения способа объявления типов данных, чтобы наш метод был применим. %но поддерживаются при использовании \visitors{}.

% нерегулярными типами данных~\cite{irregular}
% вместо "нерегулярными~\cite{irregular} типами данных"
% потому что термин разрывается 

%Кроме того, перечисленные выше примеры важны для сравнения с подходом~\cite{Visitors}, который  в данный момент не поддерживает использование ключа \texttt{-rectypes} и полиморфные вариантные типы. 



В разделе \ref{sec:design} мы коснемся вопросов дизайна расширяемых преобразований, которые могут упростить или усложнить использование  разработчиками полученных преобразований. В заключительном разделе \ref{sec:performance} коснёмся вопросов производительности.

%TODO: вообще сравниваться непосредственно с Visitors не хорошо, будет выглядет как мелкое улучшение примеров.

%Может быть другого сорта мотивацию, сказать почему примеры важны (а в конце, "кроме того, примеры важны для сравнения с взиторами, потому что...)

%\subsection{Research Questions}
%Данные примеры помогут ответить на исследовательский вопрос:

%\textcolor{red} {Это не вопрос!!!} Также важно ответить на вопрос: <<Является ли предоставляемый интерфейс достаточно знакомым для практикующего разработчика на \ocaml{}?>>

%TODO: Тут сказать почему они важны

%\parbox{\textwidth}{
%\textcolor{blue} {Запихнуть RQ в предыдущий раздел}
%}

\subsection{Типизированные логические значения}
\label{sec:lists}

Этот пример появился во время работы над строго типизированным встроенным логическим предметно-ориентированным языком на основе \textsc{OCaml}~\cite{OCanren}. 
В нём одной из самых важных конструкций является унификация термов, содержащих свободные логические переменные. Работать с такими структурами данных сложно, а допустить ошибку --- легко. 
Типичным сценарием взаимодействия  
%между логическими и нелогическими (\textcolor{red}{ПЕРЕФРАЗИРОВАТЬ})  
% частями программ 
cо встроенным языком 
является 
создание так называемых \emph{целей вычислений} (goal), содержащих структуры данных со свободными логическими переменными.
Решением логической цели является подстановка переменных, правые части которой в идеальном случае не содержат свободных переменных. 
Чтобы сконструировать цель вычислений необходимо уметь систематически вводить логические переменные в типизированную структуру данных,  а для восстановления ответа -- систематически извлекать из представления, подходящего для работы с \OCanren{}, ответы в обыкновенном
%нелогическом(\textcolor{red}{ПЕРЕФРАЗИРОВАТЬ})  
представлении (т.е. без логических переменных).

Упрощенный тип для логических переменных может быть описан следующим образом:

\begin{lstlisting}
@type 'a logic =
| V     of int
| Value of 'a       with show
\end{lstlisting}
Логическое значение может быть либо свободной логической переменной (``\lstinline{V}'') или каким-то другим значением (``\lstinline{Value}''), которое не является свободной переменной, но потенциально может содержать свободные переменные.
\begin{comment}

Чтобы преобразовывать в и из логических значений, можно воспользоваться следующими функциями:

\begin{lstlisting}
let lift x = Value x

let reify  = function
| V     _ -> invalid_arg "Free variable"
| Value x -> x
\end{lstlisting}

Функция ``\lstinline{reify}'' бросает исключение для свободных переменных, так как в присутствии вхождений свободных переменных
логическое значение нельзя рассматривать как обыкновенную (нелогическую) структуру данных.
\end{comment}


Когда мы работем с логическими структурами данных, нам необходима возможность вставлять логические переменные в произвольные позиции.
Это означает, что мы должны использовать другой тип данных, подходящий для использования 
с точки зрения системы типов. Например,
для списков нам придется абстрагироваться от рекурсии, чтобы иметь возможность описать тип логических списков \lstinline{llist}\footnote{Этот способ применим только при использовании ключа компиляции \texttt{-rectypes}.}:

\begin{lstlisting}
type ('a, 'self) list_like = 
    | Nil 
    | Cons of 'a * 'self
type 'a list = ('a, 'a list) list_like
type 'a llist = 
    ('a, 'a llist) list_like logic
\end{lstlisting}
%которые будут иметь тип ``\lstinline{lexpr}'', объявленный как
%
%\begin{lstlisting}
%type expr' = Var of string logic | Const of int logic 
%           | Binop of lexpr * lexpr
%and  lexpr = expr' logic
%\end{lstlisting}

Если мы захотим, чтобы списки типа \lstinline{llist} без логических значений преобразовывались в строковое представление также, как списки типа \lstinline{list}, необходимо модифицировать преобразование типа \lstinline{logic} в строку, убрав название конструктора \lstinline{Value}:

\begin{lstlisting}
class ['a, 'self] my_show fa fself = 
object
  inherit ['a, 'self] $\inbr{show_{logic}}$ fa fself
  method c_Value () _ x = fa () x
end
\end{lstlisting}
В такой реализации преобразования логических значений, где мы изменили только один конструктор, мы можем объявить тип логических списков заново, и получить для него преобразование в строку, которое на списках без переменных работает так же, как и для типа \lstinline{list}.

Особенностью данного подхода является, во-первых, получение нового преобразования в строку путём изменения одного метода, а, во-вторых, способ объявления типов \lstinline{list} и \lstinline{llist}, который не удается переиспользовать при использовании подхода, предоставляемого \visitors{}.

%Нам также нужно реализовать две функции преобразования. Все эти определения представляют собой типичный пример однотипного (boilerplate) кода.
%
%С изпользованием нашего подхода решение почти полностью декларативно\footnote{При условии включения ключа компиляции \cd{-rectypes}}.
%Во-первых, мы абстрагируемся от интересующего нас типа, заменяя все его вхождения типовой переменной с не встречающимся ранее именем:

%\begin{lstlisting}
%@type ('string, 'int, 'expr) a_expr =
%| Var   of 'string
%| Const of 'int
%| Binop of 'string * 'expr * 'expr with show, gmap
%\end{lstlisting}
%
%Здесь мы абстрагировали тип от всего конкретного, но мы могли обойтись абстрагированием только от самого себя. Заметьте, что 
%мы воспользовались двумя видами обобщенных преобразований~--- ``\lstinline{show}'' и ``\lstinline{gmap}''. 
%Первое будет полезно для отладочных целей, а второе является необходимым для нашего решения.
%
%Теперь мы можем объявить логические и нелогические составляющие как специализации исходного типа:
%
%\begin{lstlisting}
%@type expr  = (string, int, expr) a_expr 
%  with show, gmap
%@type lexpr = (string logic, int logic, lexpr) a_expr logic 
%  with show, gmap
%\end{lstlisting}

%Обратите внимание, что ``новый'' тип ``\lstinline{expr}'' эквивалентен старому, следовательно, такое переписывание типов не нарушает существующий код.
%
%Наконец, определения функций преобразования воспользуются преобразованием, полученным с помощью плагина ``\lstinline{gmap}'', предоставляемого библиотекой:
%
%\begin{lstlisting}
%let rec to_logic   expr = gmap(a_expr) lift  lift  to_logic  expr
%let rec from_logic expr = gmap(a_expr) reify reify from_logic @@ 
%                           reify expr
%\end{lstlisting}
%
%Как вы видите, поддержка типовых операторов существенна для этого примера. В предыдущей реализации~\cite{TransformationObjects} типовые операторы не были поддержаны и их было не так просто добавить.

\subsection{Преобразование в безымянное представление}
\label{sec:nameless}

Полиморфные вариантные типы в языке \ocaml{} позволяют описывать структуры данных композиционально, статически типизировано и в разных модулях компоновки~\cite{PolyVarReuse}.
Целесообразно объявлять преобразования таких структур данных отдельно друг от друга. Задача конструирования преобразований для 
раздельно объявленных и строго типизированных компонент известна как ``проблема выражений'' (``The Expression Problem''~\cite{ExpressionProblem}).
%которая часто используется (\textcolor{red}{Убрать в 5.1}) как ``лакмусовый тест'' для оценки подходов к обобщенному программированию~\cite{ObjectAlgebras,ALaCarte}. 
В этом подразделе мы представим решение этой задачи в рамках нашего подхода. В качестве конкретной задачи мы реализуем преобразование $\lambda$-выражений в безымянное представление.

Во-первых, опишем часть языка выражений без связывающих конструкций:

\begin{lstlisting}
@type ('name, 'lam) lam = 
[ `App of 'lam * 'lam
| `Var of 'name
] with show
\end{lstlisting}

\noindent Выделение этого типа выглядит логично, так как 
кроме указанных двух конструкций, потенциально в языке могут появиться другие, которые будут связывать переменные 
($\lambda$-абстракции, \lstinline{let}-определения и т.д.). Комбинируя различные типы и преобразования этих типов, можно получать различные расширения деревьев абстрактного синтаксиса и преобразований для $\lambda$-выражений.
%, их с несвязывающими конструкциями, а также с ними самими, можно получать различные языки с согласованным поведением \textcolor{red}{ПЕРЕФРАЗИРОВАТЬ}.

Введенный выше тип ``\lstinline{lam}'' является полиморфным: первый параметр используется для представления имен или индексов %(или уровней) 
де Брёйна\footnote{Способ представления лямбда-выражений в безымянном виде предложенный де Брёйном в~\cite{deBruijn}.}, второй необходим для открытой рекурсии (здесь мы следуем  подходу к описанию расширяемых структур данных с помощью полиморфных 
вариантных типов~\cite{PolyVarReuse}).

Для данного типа преобразование в безымянное представление можно определить следующим образом:
%Рассмотрим как для такого типа должны выглядеть преобразование в безымянное представление, а именно, как должен выглядеть класс преобразования.
%Как должно выглядеть преобразование в безымянное представление для такого типа? А именно, как должен выглядеть класс преобразования? Это показано ниже:

\begin{lstlisting}
class ['lam, 'nless] lam_to_nameless
 (flam : string list -> 'lam -> 'nless) =
object
  inherit 
    [ string list, string, int
    , string list, 'lam, 'nless
    , string list, 'lam, 'nless] $\inbr{lam}$
  method $\inbr{App}$ env _ l r = 
    `App (flam env l, flam env r)
  method $\inbr{Var}$ env _ x   = `Var (index env x)
end
\end{lstlisting}

% TODO: Здесь у нас нет call-by-value, поэтому это нифига не интерпретатор

\noindent Здесь мы используем список строк для хранения подстановки переменных и  передаем его как наследуемый атрибут. Затем мы пользуемся функцией 
``\lstinline{index}'' чтобы найти строку в подстановке, т.е.  эта функция преобразует имя в индекс де Брёйна. 
Интересной частью преобразования является типизация общего класса предка ``$\inbr{lam}$''. 
Первая тройка параметров описывает преобразование первого типового параметра. Можно заметить, что мы преобразуем строки в числа используя подстановку.
Здесь типовая переменная ``\lstinline{'lam}'', 
%как мы знаем, 
приравнивается (раздел~\ref{pv}) открытой версии типа ``\lstinline{lam}''. %(ДОИСПРАВИТЬ)
Наконец, результат преобразования типизируется с помощью переменной ``\lstinline{'nless}'', введение которой необходимо для правильной реализации преобразования объединения типов.
%Так происходит именно так потому, что, как будет понятно позднее,  это будет действительно другой тип. (\textcolor{blue}{Сказать прямее, может даже лишнее предложение})
Так как второй типовый параметр обычно ссылается рекурсивно на себя, третья тройка типовых параметров совпадает со второй.

Давайте теперь добавим в язык связывающую конструкцию --- $\lambda$-абстракцию:

\begin{lstlisting}
@type ('name, 'lam) abs = 
  [ `Abs of 'name * 'lam ] with show
\end{lstlisting}

Те же самые рассуждения применимы и тут: мы пользуемся открытой рекурсией и параметризируем представление относительно имени.
Класс для преобразования будет выглядеть похожим образом:

\begin{lstlisting}
class ['lam, 'nless] abs_to_nameless
 (flam : string list -> 'lam -> 'nless) =
object
  inherit [string list, string, int
          , string list, 'lam, 'nless
          , string list, 'lam, 'nless] $\inbr{abs}$
  method $\inbr{Abs}$ env name term = 
    `Abs (flam (name :: env) term)
end
\end{lstlisting}

Заметьте, что метод ``$\inbr{Abs}$'' конструирует значения \emph{другого} типа, чем любая возможная параметризация типа ``\lstinline{abs}''. Действительно, безымянное представление типа не должно содержать никаких суррогатов имён.

Теперь мы можем объединить эти два типа, чтобы получить тип выражений со связывающими конструкциями.

\begin{lstlisting}
@type ('name, 'lam) term = 
  [ ('name, 'lam) lam 
  | ('name, 'lam) abs) ] with show
\end{lstlisting}

Представим два новых типа для именованного и безымянного представления\footnote{Для того чтобы эти определения типов скомпилировались, необходимо использовать ключ компиляции \cd{-rectypes}.}:

\begin{lstlisting}
@type named = (string, named) term 
  with show
@type nameless = 
  [ (int, nameless) lam | `Abs of nameless] 
  with show
\end{lstlisting}

Наконец, мы можем описать преобразование, которое превращает именованные термы в их безымянное представление:

\begin{lstlisting}
class to_nameless
(f : string list -> named -> nameless) = 
object
 inherit 
   [string list, named, nameless] $\inbr{named}$
 inherit 
   [named, nameless] lam_to_nameless f
 inherit 
   [named, nameless] abs_to_nameless f
end
\end{lstlisting}

Это преобразование получается путём наследования поределеннных выше компонент: общего класса для всех преобразований типа ``\lstinline{named}'' 
и двух конкретных преобразований его составляющих: 
``\lstinline{lam_to_nameless}'' и ``\lstinline{abs_to_nameless}''.
Функция-преобразование может быть получена стандартным способом:

\begin{lstlisting}
let to_nameless term =
  transform(named) 
    (fun fself -> new to_nameless fself) 
    [] 
    term
\end{lstlisting}

Только что мы построили реализацию преобразования типа, комбинируя реализации преобразований его составляющих. Эти  реализации могут быть раздельно скомпонованы, но вся система при этом останется строго типизированной. В этом примере демонстрируются возможности подхода по раздельному и модульному представлению преобразований с помощью объектов, а также возможности по использованию полиморфных вариантных типов языка \ocaml{}, которые не доступны в подходе \visitors{}.

\subsection{Нерегулярные типы данных}
\label{sec:irregular}

Основным достоинством подхода \visitors{} является поддержка нерегулярных типов данных с некоторой оговоркой: поддерживаются преобразования в так называемом ``полиморфном режиме''~\cite{Visitors}. Наш метод не позволяет построить преобразования для уже описанных нерегулярных типов данных. Однако, если разработчик проектирует типы с нуля, то у него есть возможность описать их так, чтобы они были регулярными и были совместимы с нашим подходом. 

Рассмотрим объявления нерегулярного типа данных  из работы~\cite{irregular}.

\begin{lstlisting}
type 'a tree = N | C of 'a * ('a * 'a) tree
\end{lstlisting}
\noindent Для этого типа метод на основе \GT{} не сможет построить преобразование, так в языке \ocaml{} не поддерживается нерегулярная типизация объектов. Необходимо переписать это тип, абстрагировавшись от вхождения типа \lstinline{'a * 'a}, и описать тип \lstinline{tree_list}, и уже с помощью него описать  необходимый тип \lstinline{tree} (потребуется использование ключа компилятора \texttt{-rectypes}).
\begin{lstlisting}
type ('a, 'b) t = N | C of 'a * ('a, 'b)
type 'a tree = ('a, 'a * 'a) t 
\end{lstlisting}
\noindent Для этих двух типов метод уже сможет построить требуемые преобразования.

\subsection{Особенности дизайна}
\label{sec:design}

В данном разделе мы рассмотрим пример построения расширяемых преобразований с помощью \visitors~\cite{Visitors}, ещё одного подхода по построению раширяемых преобразований для \OCaml{} с помощью объектов.

\begin{lstlisting}
(struct 
  type 'a menu = ('a * int) list
  [@@deriving visitors 
    { variety = "map"
    ; name = "map_menu"
    ; polymorphic = true }]
  ...
end : sig 
  type 'a menu 
  val add_exn
    : 'a menu -> 'a -> int -> 'a menu
    
  class virtual ['c] map_menu : object ('c)
    constraint 'c = ...
    method private visit_int : 
      'env. 'env -> int -> int
    method visit_menu :
      ('env -> 'a -> 'b) -> 
      'env -> 'a menu -> 'b menu
    ...
end)
\end{lstlisting}

\noindent Вы видите тип данных \lstinline{'a menu}, аннотированный для использования \visitors{}, который реализует ресторанное меню как список блюд и цен. Предположим, разработчик решил скрыть детали реализации меню, объявив тип абстрактным и предоставив функцию добавления в меню, которая, например, в случае указания отрицательной цены, приводит к исключительной ситуации.

Мы хотим обратить внимание на несколько недостатков данной реализации, которые отсутствуют в методе, предложенном в данной работе.

Во-первых, объект-преобразование реализует приватный метод \lstinline{visit_int}, для преобразования чисел, который разрешается переопределять при наследовании. Это позволяет определять преобразования, которые нарушают внутреннюю целостность типа \lstinline{'a menu}, например, которые создают пункты меню с отрицательными ценами. Данный недостаток хорошо известен среди исследователей обобщенного программирования~\cite{SYB} для языка \haskell{}, и порицается~\cite{SafeHaskell}.

Во-вторых, класс-преобразование кодирует свой интерфейс в единственном типовом параметре с помощью ключевого слова \lstinline{constraint}. Преимуществом такого кодирования является некоторое сокращение порождаемого объема кода. Недостатком является невозможность породить тип объекта преобразования в файлах-интерфейсах языка OCaml{}, что заставляет выписывать типы вручную. Стандартные~\cite{ppxderiving} подходы к обобщенному программированию, как и предложенный в данной работе метод, не страдают от этого недостатка.

В-третьих, порожденный с помощью \visitors{} объект-преобразование, является виртуальным классом, и поэтому не готов к немедленному использованию пользователем. Другие подходы к обобщенному программированию сразу предоставляет готовые к использованию функции-преобразования.




\subsection{Производительность}
\label{sec:performance}

При сравнении преобразований, реализованных в традиционном и расширяемом виде,
следует ожидать от первых большей производительности, так как дополнительный слой абстракции вносит некоторые накладные расходы.

При замерах использовались преобразования, реализованные четырьмя способами.
\begin{enumerate}
\item Нерасширяемые преобразования записанные традиционным способом с помощью рекурсивных функций. От них следует ожидать максимальную производительность.
\item Частично расширяемые преобразования, реализованные c помощью записей в стиле библиотеки \cd{ppx\_deriving\_morphism}. Этот  подход вносит накладные расходы на косвенный вызов преобразований подвыражений.% и неприменим, например, для полиморфных вариантых типов. 
\item Расширяемые преобразования в стиле \visitors{}, т.е. с использованием объектов. Вносит некоторые накладные расходы на косвенный вызов из-за таблицы виртуальных методов.
\item Метод из данной работы, сходный с предыдущим, но где обобщенная функция преобразования (раздел~\ref{transtypes}) реализована отдельно от объекта.
\end{enumerate}

Для замеров были выбраны два вида преобразований: копирование выражения и преобразование в текстовый формат.
Данные преобразования применялись к $\lambda$-выражению, которое состоит из некоторого количества $\lambda$-абстракций,  примененных к тождественной функции. Количество таких абстракций определяет размер выражения.

При замерах производительности были реализованы четыре метода, описанные выше. В таблице \ref{tab:caption} в процентах указана производительность методов относительного самого медленного (больше --- лучше), который обозначается прочерком.



\begin{table*}[t]
  \centering
  \begin{tabular}{l ccccc}
    \toprule
    \multirow{2}{*}{Вид преобразования}& \multirow{2}{*}{Размер} & \multicolumn{4}{c}{Метод реализации и улучшение (\%)} \\\cline{3-6}
     & & \GT & \visitors & \PPXMorphism  & Default \\\hline
    \multirow{5}{*}{Копирование}  
      & 300  & --  & 149 & 220  & 311 \\
      & 500  & --  & 146 & 218  & 246 \\
      & 700  & -- & 143 & 212  & 244 \\
      & 900  & --  & 141 & 210  & 243 \\
      & 1000 & --  & 140 & 205  & 233 \\\hline
    \multirow{5}{*}{Форматирование}  
      & 300  & 0  & -- & 1  & 3 \\
      & 500  & 0  & -- & 1  & 4 \\
      & 700  & 1  & -- & 1  & 4 \\
      & 900  & -- & 1  & 2  & 3 \\
      & 1000 & -- & 0  & 1  & 3 \\ 
    \bottomrule
  \end{tabular}
\caption{Caption below table.}
\label{tab:caption}
\end{table*}


\begin{comment}

\subsection{Пример пользовательского плагина}
\label{pluginExample}

Наконец, мы продемонстрируем использование системы плагинов на свежем примере реализации плагина. Для этой цели мы выбрали широко известное преобразование \emph{hash-consing}~\cite{HC}. Это преобразование превращает структуры данных в их максимально компактное представление в памяти, при котором структурно равные части представляются в памяти как один физический объект. Например, синтаксическое дерево выражения

\begin{lstlisting}
let t =
  Binop ("+",
    Binop ("-",
      Var "b",
      Binop ("*", Var "b", Var "a")),
    Binop ("*", Var "b", Var "a"))
\end{lstlisting}
может быть переписано  как

\begin{lstlisting}
let t =
  let b  = Var "b" in
  let ba = Binop ("*", b, Var "a") in
  Binop ("+", Binop ("-", b, ba), ba)  
\end{lstlisting}
где равные подвыражения представляются как равные поддеревья.
 
Наш плагин по типу  ``\lstinline|$\left\{\alpha_i\right\}$ t|'' предоставит функцию для 
hash-consing ``\lstinline{hc(t)}'' с сигнатурой 

\begin{lstlisting}
$\{$ H.t -> $\alpha_i$ -> H.t * $\alpha_i$ $\}$ -> H.t -> $\left\{\alpha_i\right\}$ t -> H.t * $\left\{\alpha_i\right\}$ t
\end{lstlisting}
где ``\lstinline{H.t}''~--- это гетерогенная хэш таблица для произвольных типов. Интерфейс у неё следующий:

\begin{lstlisting}
module H : sig
  type t
  val hc : t -> 'a -> t * 'a
end
\end{lstlisting}

Функция  ``\lstinline{H.hc}'' принимает хэш таблицу и некоторое значение и возвращает потенциально обновленную хэш таблицу и значение, которое структурно эквивалентно поданному на вход. Мы не будет описывать реализацию этого модуля, а приведем пример использования в конструкторе:

\begin{lstlisting}
method $\inbr{Binop}$ h _ op l r =
  let h, op = hc(string) h op in
  let h, l  = fself h l in
  let h, r  = fself h r in
  H.hc h (Binop (op, l, r))
\end{lstlisting}

Этот метод принимает как наследуемый атрибут хэш таблицу ``\lstinline{h}'', преобразуемое целиком, которое здесь не потребуется; три аргумента конструктора:наследуемыех и синтезированных атрибутов:


Здесь мы предполагаем, что тип ``\lstinline{ht_typ}'' объявлен как

\begin{lstlisting}
let ht_typ ~loc =
  Typ.of_longident ~loc (Ldot (Lident "H", "t"))
\end{lstlisting}

Другими словами, мы объявляем, что типом наследуемого атрибут всегда будет 
 ``\lstinline{H.t}'', а типом синтезированного атрибута будет пара
``\lstinline{H.t * t}''.

Следующая группа методов описывает параметры классов плагина:

\begin{lstlisting}
method plugin_class_params tdecl =
  let ps = List.map tdecl.ptype_params 
             ~f:(fun (t, _) -> typ_arg_of_core_type t)
  in
  ps @
  [ named_type_arg ~loc:(loc_from_caml tdecl.ptype_loc) @@
    Naming.make_extra_param tdecl.ptype_name.txt
  ]

method prepare_inherit_typ_params_for_alias ~loc tdecl rhs_args =
  List.map rhs_args ~f:Typ.from_caml
\end{lstlisting}

Первый метод описывает типовые параметры класса плагина: для данного случая это типовые параметры самого объявления типа плюс дополнительный типовый параметр 
``$\varepsilon$''. Второй метод описывает вычисление типовых параметров для применения конструктора типа. В случае, если объявление типа выглядит как 

\begin{lstlisting}
type $\{\alpha_i\}$ t = $\{a_i\}$ tc
\end{lstlisting}

нам необходимо построить реализацию преобразовния для типа ``\lstinline{t}'' 
из реализации оного для типа  ``\lstinline{tc}'', наследуясь от правильного 
инстанциированного соответвующего класса. Для нашего случая класс параметризуется теми же 
типовыми параметрами, что и объяляемый тип, поэтому мы оставляем их как есть.

Последняя группа методов отвечает за генерацию тел методов для тра2нсформаций  конструкторов.
Мы поддерживаем регулярные конструкторы алгебраических типов, где аргументами может быть и кортеж, и запись, а также записи и кортежи на верхнем уроне, преобразование которые, как правило, имеет много общих частией. Всего за это отвечают 4 метода, но здесь мы покажем только один:

\begin{lstlisting}
method on_tuple_constr ~loc ~is_self_rec ~mutual_decls 
                            ~inhe tdecl constr_info ts =
  $\dots$ 
  match ts with
  | [] -> Exp.tuple ~loc [ inhe; c [] ]
  | ts ->
     let res_var_name = sprintf "%s_rez" in
     let argcount = List.length ts in
     let hfhc = Exp.of_longident ~loc (Ldot (Lident "H", "hc")) in
     List.fold_right
       (List.mapi ~f:(fun n x -> (n, x)) ts)
       ~init:$\dots$
       ~f:(fun (i, (name, typ)) acc ->
            Exp.let_one ~loc
              (Pat.tuple ~loc 
                 [ Pat.sprintf ~loc "ht%d" (i+1)
                 ; Pat.sprintf ~loc "%s" @@ res_var_name name])
              (self#app_transformation_expr ~loc
                 (self#do_typ_gen ~loc ~is_self_rec 
                                  ~mutual_decls tdecl typ)
                 (if i = 0 then inhe else Exp.sprintf ~loc "ht%d" i)
                 (Exp.ident ~loc name)
              )
              acc
          )
  $\dots$
\end{lstlisting}

Реализация использует заранее заготовленный метод нашей библиотеки
``\lstinline{self#app_transformation_expr}'', который генерируется применение функции преобразования к соответствующему типу.

Конечной компонентой реализации является сам  модуль ``\lstinline{H}''. Стандартный функтор ``\lstinline{Hashtbl.Make}'' создает хэш таблицы, используя некотрую хэш фукнцию и предикат равенства, предоставленные пользователем. В целом, следуем мы следуем такому соглашению: как хэш фукнцию используем полиморфную ``\lstinline{Hashtbl.hash}'', а качестве равенства используем физическое равенство ``\lstinline{==}''. Однако, присутсвуют две сложности:

\begin{itemize}
\item Так как таблице гетерогенная нам необходимо использовать небезопасное приведение типов ``\lstinline{Obj.magic}''.
\item Наша реализация равенства чуть более сложная, чем обычное ``\lstinline{==}''. Нам необходимо стравнивать верхнеуровневые конструкторы и количества их аргументов  \emph{структурно}, а только затем сравнивать соответствующие аргументы взическим равенством. Технически, мы может считать равными структурно равные значения   \emph{различных} типов.
\end{itemize}

Мы полагаемся здесь на следующее наблюдение: hash-consing корректно использовать тольео для структур данных, которые прозрачны по ссылкам, мы предполагаем что равные структуры данных взаимозаменяемы не смотря на их типы. 

Полную реализацию плагина может быть увидеть в главном репозитории. Она занимает 164 строчки кода, учитывая комментарии и пустые строки.
\end{comment}

% !TeX encoding = windows-1251
�������� ������� �������� ������ ������� �������� �����, ������������ � ���������� \visitors{}, ������� ����� �������� ������������ �������������� ����� ����-�-������ � ������ ��������, �� ��� ���� ��� ����� ������������ ������ ������.

�������� ������������ �������, ������������� � \visitors{}~\cite{Visitors}, �������� ��������� ������������ ����� ������. ����� � �����������, ��� �������������� ���������� �������� � ����� ������� ����������� \emph{����� ��������� ������������ (explicit polymorphism)\footnote{\url{https://caml.inria.fr/pub/docs/manual-ocaml/polymorphism.html\#s\%3Apolymorphic-recursion}}}. ������, ������ ����������� �� ����������� �������� ������ � ����������� �������� (������ \ref{sec:lists}), � ������� ������ \visitors{} �� �������� ��� \OCanren{}.

��� ������������� ������ ������� ��������� �������������� ��� ������������ ����� �� ���������, �� ���� ��� ���� ����������� ��������� ���, �� �� ����� �������������� ��� �� ������������� ���������. ��� �������� ������� ����������� ���� �� ��������� ����� ������, ��� �������, ��-�����������, ������������ � ������ ������ �����, ����� ��������. �� ����� ����� ��������, ��� � ����� �������� ������������ ���� ������ ����������� �����, ������� �� ������� ��� ���������� ������ ������� �� ����� ������������.

��� ������ ��������� ��������� �������������� ��� ����������� ���������� ����� ����� \ocaml{}, � �� ����� ��� ��� ���� � \visitors{} �� ��������������. �� ������� ���������� \visitors{} � �� ����� �������� ������� �������� ���� ��������� ���� �����.

����������� ������� � ����������� ���������������� �� \ocaml{} ��������� ��������������, �������������� ���������. ��� ������ ������������ �������������� ��� �������, � �� ���� �������� ����� ������ �������-��������������, ������� ��� ������ ������������� ��� �� ��������� ��� ������������, ��� � �����������~\cite{PPXLib,ppxderiving} ������� � ����������� ���������������� �� \ocaml{}. � \visitors{} �������-�������������� �� ���������, � ������� ������������� ����� ������� ����� ����� �������� ��� ������������.

� ����� ������� � �������, �������������� ��������������, ����� ���� ����� ����� ������� ����������, ��� ����������� ������ ������������ ����. � \visitors{} ��� ���������� �������� ����������: ��� �������� ������ ������ �������� ������ �������� ��������� ������ $3\cdot(n+1)$, ��� $n$ -- ��� ���������� ������� ����������. ������, ����� ������� �� ��������� ��������� ��� � ������ ���������� ����� \ocaml{}, ��� ����� ��������� ������������ ����������� ��� ������������ ���������� ������������ �����������.

��� ������������� \visitors{} � ����������� ������� ����� ����� �������������� ������ ��� �������������� ��������� ����� ������, � �� ����� ��� � \GT{} ��� �������� �������� ��� �������� ��������� �������-�������������� �� �������. ����������� ������� � \visitors{} �������� ��������� �������� ����������: ������������ ��������� �������������� �������������� ��������� �����, ��� ������������ ����� �������� ����������, ������� ����������� ��� ���������� �������� ����� ������. � ������ �������, ����� ������ ��������� ����������� ��������������, ��� ������� ������ ���� ������������ ������ \cite{SYB}, ���� �� �������, ��� ���������� ������ ���� �������������� ���������� �����.


\section{Заключение}
\label{sec:futurework}

В данной работе представлен подход на основе обобщенного программирования, который кодирует преобразования значений типов данных с помощью объектов, что позволяет видоизменять построенные преобразования, не описывая их заново.

Существует несколько возможны направлений для дальнейшего развития проекта. Во-первых, можно снижать накладные расходы на реализацию расширяемых преобразований 
%в данной работе мы не касались вопросов производительности. Мы представляем преобразования в очень обобщенном виде, с несколькими слоями косвенности. Очевидно, что преобразования, реализованные с помощью нашей библиотеки будут работать медленнее, чем написанные вручную. Мы предполагаем, что 
%производительность может быть улучшено 
с помощью, так называемого, staging~\cite{Staged} или, возможно, с помощью оптимизаций, специфичных для объектов. 
Также, стоит рассмотреть вариант реализации обобщенной функции-преобразования~\ref{transtypes} как метода, что позволит достигнуть паритета в производительности с \Visitors{}.

Другим важным направлением является поддержка большего разнообразия объявлений типов, а именно GADT и нерегулярных типов. Хотя уже сделаны некоторые наработки, получившиеся решение делает интерфейс всей библиотеки чересчур сложным даже для простых случаев.

Наконец, структура с информацией о типе, которую мы генерируем, может быть использована, чтобы сымитировать \emph{ad-hoc} полиморфизм, так как они содержит реализацию функций, индексированных типами. Это в сумме с недавно предложенными расширениями~\cite{ModularImplicits} может открыть интересные перспективы.



\begin{thebibliography}{99}

\small


\bibitem{SCICO}
{\em Boulytchev D.} Combinators and Type-driven Transformers in
Objective Caml // Sci. Comput. Program. 2015.  v. 114. no. C. pp. 57–73.

%\bibitem{brooks}
%{\em Brooks Jr. F. P.}  The Mythical Man-month (Anniversary
%Ed.). — Boston, MA, USA : Addison-Wesley Longman Publishing Co.,
%Inc., 1995. — ISBN: 0-201-83595-9.

\bibitem{SYBOCaml}
{\em Boulytchev D., Mechtaev S.} Efficiently Scrapping Boilerplate
Code in OCaml // Workshop on ML. — ML ’11. — Tokyo, Japan, 2011.

\bibitem{InstantGenerics}
{\em Chakravarty M. M. T., Ditu G., Leshchinskiy R.} In\-stant Generics : Fast and Easy. — 2009.

\bibitem{HC}
{\em Filli\^atre J.-C., Conchon S.} Type-safe Modular Hashconsing // Workshop on ML. — ML ’06. — New York, NY, USA : ACM, 2006. — P. 12–19.

\bibitem{PolyVar}
{\em  Garrigue J.} Programming with Polymorphic Variants // Workshop on ML. — 1998.

\bibitem{PolyVarReuse}
{\em Garrigue J.} Code reuse through polymorphic variants // In
Workshop on Foundations of Software Engineering. — 2000.

\bibitem{CalculatingFP}
{\em  Gibbons J.} Calculating Functional Programs // Algebraic and
Coalgebraic Methods in the Mathematics of Program Construction:
International Summer School and Workshop Oxford, UK, April 10–14, 2000 Revised Lectures / Ed. by Roland Backhouse, Roy Crole,
Jeremy Gibbons. — Berlin, Heidelberg : Springer Berlin Heidelberg,
2002. — P. 151–203. %— ISBN: 978-3-540-47797-6. 
%— Access mode: https://doi.org/10.1007/3-540-47797-7_5.

\bibitem{DGP}
{\em Gibbons J.} Datatype-generic Programming // Proceedings
of the 2006 International Conference on Datatype-generic Program\-ming. — SSDGP’06. — Berlin, Heidelberg : Springer-Verlag, 2007. —
P. 1–71.% — 
%Access mode: http://dl.acm.org/citation.cfm?id=1782894.1782895.

\bibitem{Hinze}
{\em Hinze R.} Generics for the Masses // J. Funct. Program. — 2006. —
Jul. — Vol. 16, no. 4-5. — P. 451–483.
%Access mode: http://dx.doi.org/10.1017/S0956796806006022.

\bibitem{Fold}
{\em Hutton G.} A Tutorial on the Universality and Expressive\-ness of Fold // J. Funct. Program. — 1999. — Jul. — Vol. 9,
no. 4. — P. 355–372. %— Access mode: http://dx.doi.org/10.1017/S0956796899003500.

\bibitem{TypeFamilies}
{\em Kiselyov O., Jones S.P.,  Chung-chieh S. } Fun with
Type Functions // Reflections on the Work of C.A.R. Hoare / Ed. by
A.W. Roscoe, Cliff B. Jones, Kenneth R. Wood. — London : Springer
London, 2010. — P. 301–331. %— ISBN: 978-1-84882-912-1. 
%— Access mode: https://doi.org/10.1007/978-1-84882-912-1_14.

\bibitem{AGKnuth}
{\em Knuth D.E.} Semantics of context-free languages // Mathemati\-cal systems theory. — 1968. — Jun. — Vol. 2, no. 2. — P. 127–145. 
%— Access mode: https://doi.org/10.1007/BF01692511.

\bibitem{deBruijn}
{\em de Bruijn, N. G.} Lambda Calculus Notation with Nameless Dummies: A Tool for Automatic Formula Manipulation, with Application to the Church-Rosser Theorem. - 1972


\bibitem{OCanren}
{\em Kosarev D, Boulytchev D.} Typed Embedding of a Relational Language in OCaml // Proceedings ML Family Workshop /
OCaml Users and Developers workshops, ML/OCAML 2016, Nara,
Japan, September 22-23, 2016. — 2016. — P. 1–22. 
%— Access mode:https://doi.org/10.4204/EPTCS.285.1.

\bibitem{SYB}
{\em L\"ammel R., Jones S.P.} Scrap Your Boilerplate: A Practical Design Pattern for Generic Programming // SIGPLAN Not. —
2003. — Jan. — Vol. 38, no. 3. — P. 26–37. 
%— Access mode: http://doi.acm.org/10.1145/640136.604179.

\bibitem{SYB1}
{\em L\"ammel R., Jones S.P.} Scrap More Boilerplate: Reflection, Zips, and Generalised Casts // Proceedings of the Ninth ACM
SIGPLAN International Conference on Functional Programming. —
ICFP ’04. — New York, NY, USA : ACM, 2004. — P. 244–255. 
%— Ac-cess mode: http://doi.acm.org/10.1145/1016850.1016883.

\bibitem{SYB2}
{\em L\"ammel R., Jones S. P.} Scrap Your Boilerplate with Class:
Extensible Generic Functions // SIGPLAN Not. — 2005. — Sep. —
Vol. 40, no. 9. — P. 204–215. %— Access mode: http://doi.acm.org/10.1145/1090189.1086391.

\bibitem{Bananas}
{\em Meijer E., Fokkinga M., Paterson R.} Functional Programming with Bananas, Lenses, Envelopes and Barbed Wire. — Springer-Verlag, 1991. — P. 124–144.

\bibitem{ObjectAlgebras}
{\em Oliveira B. C. d. S., Cook W.R.} Extensibility for the
Masses: Practical Extensibility with Object Algebras // Proceedings of the 26th European Conference on Object-Oriented Programming. — ECOOP’12. — Berlin, Heidelberg : Springer-Verlag,
2012. — P. 2–27.

\bibitem{ObjectAlgebrasAttribute}
{\em Tillmann R., Brachth\"auser J.I., Ostermann K. } From Object Algebras to Attribute Grammars //
 SIGPLAN Not. — 2014. — Oct. — Vol. 49, no. 10. — P. 377–395. 
% Access mode: http://doi.acm.org/10.1145/2714064.2660237.

\bibitem{ObjectAlgebrasSYB}
{\em Zhang H., Chu Z.,  Oliveira B. C.d. S., van der Storm T.}
Scrap Your Boilerplate With Object Algebras  // Proceedings of the Object-oriented Programming, Systems, Languages, and
 Applications (OOPSLA, 2015). — New York, United States, 2015. %—
% Access mode: https://hal.inria.fr/hal-01261477.

\bibitem{ALaCarte}
{\em Swierstra W.} Data Types \`a La Carte // J. Funct. Program. —
 2008. — Jul. — Vol. 18, no. 4. — P. 423–436.
% Access mode: http: //dx.doi.org/10.1017/S0956796808006758.

\bibitem{AGSwierstra}
{\em Viera M., Swierstra S. D., Swierstra W.} Attribute
 Grammars Fly First-class: How to Do Aspect Oriented 
 Program\-ming in Haskell // Proceedings of the 14th ACM SIGPLAN 
 In\-ternational Conference on Functional Programming. — ICFP ’09. —
 New York, NY, USA : ACM, 2009. — P. 245–256. 
% — Access mode: http://doi.acm.org/10.1145/1596550.1596586.

\bibitem{ExpressionProblem}
{\em Wadler P.} The Expression Problem. — 1998. — Dec.
 
\bibitem{TypeClasses}
{\em Wadler P., Blott S.} How to Make Ad-hoc Polymorphism Less Ad
 Hoc // Proceedings of the 16th ACM SIGPLAN-SIGACT Symposium on Principles of Programming Languages. — POPL ’89. — New
 York, NY, USA : ACM, 1989. — P. 60–76. 
% — Access mode: http: //doi.acm.org/10.1145/75277.75283.

 \bibitem{ModularImplicits}
{\em White L., Bour F., Yallop J.} Modular implicits // Electronic Proceedings in Theoretical Computer Science. — 2015. — 12. —
 Vol. 198.
 
 \bibitem{Yallop}
{\em Yallop J.} Practical Generic Programming in OCaml // Proceedings of the 2007 Workshop on Workshop on ML. — ML ’07. —
 New York, NY, USA : ACM, 2007. — P. 83–94. 

\bibitem{Staged} 
{\em  Yallop J.} Staged Generic Programming // Proc. ACM Program. Lang. — 2017. — Aug. — Vol. 1, no. ICFP. — P. 29:1–29:29. 
% — Access mode: http://doi.acm.org/10.1145/3110273.

\bibitem{GenericOCaml} 
{\em  Balestrieri F., Mauny M.} Proceedings ML Family Workshop / OCaml Users and Developers
workshops, Nara, Japan, September 22-23, 2016 / Ed. by Kenichi Asai,
Mark Shinwell. — Vol. 285 of Electronic Proceedings in Theoretical
Computer Science. — Open Publishing Association, 2018. — P. 59–100.

% https://dl.acm.org/doi/proceedings/10.1145/317636
\bibitem{modules-vs-objects} 
{\em       Leroy X.} 
Objects, classes and modules in Objective Caml / Proceedings of International Conference on Functional Programming, Paris /     Association for Computing Machinery, 1999.

\bibitem{irregular} 
{\em   Bird R.,  Meertens L.} 
 Nested datatypes.  /  In Mathematics of Program Construction
(MPC), volume 1422 of Lecture Notes in Computer Science, pages 52–67. Springer, 1998.

\bibitem{SafeHaskell} 
Terei, David and Marlow, Simon and Peyton Jones, Simon and Mazières, David. Safe Haskell. ACM SIGPLAN Notices 2012

\bibitem{camlp5} 
\camlpfive{}. --- \url{https://camlp5.github.io}.
 
\bibitem{PPXLib} 
\textsc{ppxlib}. --- \url{https://github.com/ocaml-ppx/ppxlib}.

\bibitem{ppxderiving} 
ppx\_deriving. --- \url{https://github.com/ocaml-ppx/ppx\_deriving}

\bibitem{derivingMorphism} 
ppx\_deriving\_morphism. --- \url{https://github.com/ocaml-ppx/ppx\_deriving\_morphism}


\bibitem{Visitors}
{\em Pottier F.} Visitors Unchained // Proc. ACM Program.
Lang. — 2017. — Aug. — Vol. 1, no. ICFP. — P. 28:1–28:28. 
%— Access mode: http://doi.acm.org/10.1145/3110272.

\bibitem{dotNetSG}
.NET source Generators. --- \url{https://github.com/dotnet/roslyn/blob/master/docs/features/source-generators.md}


\bibitem{cardelli}
{\em Cardelli L., Wegner P.} On Understanding Types,
Data Abstraction, and Polymorphism



\bibitem{GADT}
{\em Dimitrios Vytiniotis, Stephanie Weirich, Simon Peyton Jones } /
.NET source Generators. 

%Simple unification-based type inference for GADTs -- 
% https://www.microsoft.com/en-us/research/publication/simple-unification-based-type-inference-for-gadts/?from=http%3A%2F%2Fresearch.microsoft.com%2F~simonpj%2Fpapers%2Fgadt%2Findex.htm


\begin{comment}
\bibitem{AZ97}
{\em Абрамов С.А., Зима Е.В.} Семинар по компьютерной алгебре на
факультете вычислительной математики и кибернетики МГУ в 1995--1996 г.
// Программирование, 1997,
No 1. С. 75--77.

\bibitem{AZ98}
{\em Абрамов С.А., Зима Е.В.} Научно-ис\-сле\-до\-вательский семинар
``Компьютерная алгебра'' в 1996--1997 г.
// Программирование, 1998,
No 1. С. 69--72.

\bibitem{AR99}
{\em Абрамов С.А., Ростовцев В.А.} Семинар по компьютерной алгебре в
1997--1998 г.  // Программирование, 1998, No 6. С. 3--7.

\bibitem{AKR00}
{\em Абрамов С.А., Крюков А.П., Ростовцев В.А.} Семинар по компьютерной
алгебре в
1998--1999 г.  // Программирование, 2000, No 1. С. 8--12.

\bibitem{AKR01}
{\em Абрамов С.А., Крюков А.П., Ростовцев В.А.} Семинар по компьютерной
алгебре
в 1999--2000 г.  // Программирование, 2001, No 1. С. 3--7.

\bibitem{AKR02}
{\em Абрамов С.А., Крюков А.П., Ростовцев В.А.} Семинар по компьютерной
алгебре в 2000--2001 г.  // Программирование, 2002, No 2. С. 6--9.

\bibitem{AKR03}
{\em Абрамов~С.А., Крюков~А.П., Ростовцев~В.А.} Семинар по компьютерной
алгебре в 2001--2002 г. // Программирование, 2003, No 2. С. 3--7.
\bibitem{AER04}
{\em Абрамов~С.А., Еднерал~В.Ф., Ростовцев~В.А.} Семинар по
компьютерной алгебре в 2002--2003~г.  // Программирование, 2004, No 2.
С. 3--7.
\bibitem{ABRE05}
{\em Абрамов~С.А., Боголюбская~А.А., Ростовцев~В.А., Еднерал~В.Ф.} Семинар по
компьютерной алгебре в 2003--2004 г.  // Программирование, 2005, No 2.
С. 3--9.
\bibitem{ABRE06}
{\em Абрамов~С.А., Боголюбская~А.А., Ростовцев~В.А., Еднерал~В.Ф.} Семинар по
компьютерной алгебре в 2004--2005 г.  // Программирование, 2006, No 2.
С. 3--7.
\bibitem{ABRE07}
{\em Абрамов~С.А., Боголюбская~А.А., Ростовцев~В.А., Еднерал~В.Ф.} Семинар по
компьютерной алгебре в 2005--2006 г.  // Программирование, 2007, No 2.
С. 3--8.
\bibitem{ABRE08}
{\em Абрамов~С.А., Боголюбская~А.А., Ростовцев~В.А., Еднерал~В.Ф.} Семинар по
компьютерной алгебре в 2006--2007 г.  // Программирование, 2008, No 2.
С. 3--8.
\bibitem{ABRE09}
{\em Абрамов~С.А., Боголюбская~А.А., Ростовцев~В.А., Еднерал~В.Ф.} Семинар по
компьютерной алгебре в 2007--2008 г.  // Программирование, 2009, No 2.
С. 3--9.
\newpage
\bibitem{mmcp09}
``Mathematical Modeling and Computational Physics (CAAP'2009)''. Book of abstracts of the internationl conference. Dubna, July 7-11, 2009.
Dubna, 2009.
\bibitem{ABRE10}
{\em Абрамов~С.А., Боголюбская~А.А., Ростовцев~В.А., Еднерал~В.Ф.} Семинар по
компьютерной алгебре в 2008-2009 г.  // Программирование, 2010, No 2. С. 3--8.
\bibitem{ABER11}
{\em Абрамов~С.А., Боголюбская~А.А., Еднерал~В.Ф., Ростовцев~В.А.} Семинар по
компьютерной алгебре в 2009-2010 г. // Программирование, 2011, No 2. С. 3--8.
\bibitem{ABR12}
{\em Абрамов~С.А., Боголюбская~А.А.,  Ростовцев~В.А.} Семинар по
компьютерной алгебре в 2010-2011 г.  // Программирование, 2012, No 2. С. 3--10.

%
\end{comment}
\end{thebibliography}


\label{lastpage}
\end{document}






