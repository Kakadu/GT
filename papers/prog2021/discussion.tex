% !TeX encoding = UTF-8
\section{Обсуждение}
\label{sec:discussion}

%Наиболее близком аналогом нашего подхода является подход
% \visitors{}, который также кодирует конструкторы алгебраических типов один к одному в методы объектов. %, но при этом в этих объектах также присутствуют другие методы.

TODO: обсуждение в заключение сдвинуть


\textcolor{red}{Чутка переписать, чтобы не казалось что статья про Visitors}

Основным достоинством подхода, предложенного в \visitors{}~\cite{Visitors}, является поддержка нерегулярных типов данных. Чтобы её реализовать, при автоматическом построении объектов к типам методов добавляются \emph{явные аннотации полиморфизма (explicit polymorphism)\footnote{\url{https://caml.inria.fr/pub/docs/manual-ocaml/polymorphism.html\#s\%3Apolymorphic-recursion}}}. Однако, данная особенность не позволяется выразить пример с логическими списками (раздел \ref{sec:lists}), и поэтому подход \visitors{} не применим для \OCanren{}.
Чтобы восстановить возможность использования типов с произвольной рекурсией  придется избавиться от явных аннотаций полиморфизма, но это может привести к потере возможности использования нерегулярных типов данных.

Преимуществом \visitors{} можно считать возможность описывать преобразования в стиле SYB\cite{SYB} за счет нарушения абстракции данных, но мы считаем, что такого рода преобразования имеют малую практическую ценность.


