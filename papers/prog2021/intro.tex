\section{Введение}

%Решительнее заявить про важную область обобщенного программирования

Обобщённое программирование ---~это подход, нацеленный на то, чтобы сделать языки программирования более гибкими, не теряя типовую безопасность. 
%В литературе встречаются различные трактовки этого термина, в данной работе мы под ним будем понимать типобезопасную параметризацию по типам данных~\cite{DGP} (polytypism).
Стандартной разновидностью этого подхода является порождение на стадии компиляции по типам данных реализаций требуемых преобразований для соответствующих типов данных. Он является общепринятым для классических типизированных функциональных языков программирования (\ocaml{}, \haskell{}), а также постепенно проникает в объектно-ориентированные языки и платформы, например, \textsc{.NET Source Generators}~\cite{dotNetSG}.


\begin{comment}
При разработке программного обеспечения на функциональных языках программирования, зачастую бывает необходимо реализовывать однотипные преобразования для различных структур данных. Для динамически типизированных языков (например, \scheme{} и его диалекты) иногда можно реализовать одну функцию-метапреобразование, подходящую для всех структур данных. В языках, среды исполнения которых содержат информацию о типах (например, \Scala{}), порою возможно построить метапреобразования во время исполнения, полагаясь на анализ типов во время исполнения (интроспеция) и создание новых типов данных на лету (рефлексия). 
 Для статически типизированных языков программирования, в средах исполнения которых информация о типах обычно не доступна во время исполнения (например, \ocaml{} и \haskell{}), применяется подход под названием 
\emph{обобщенное (generic) программирование}~\cite{Yallop,PPXLib}: 
на стадии компиляции  по описанию типа данных автоматически генерируется нужные преобразования, специализированные для указанных типов данных. 
Обобщённое программирование является стандартным подходом для этих языков, поэтому различным его модификациям крайне желательно, как минимум, повторять его функциональность, а также предоставлять такой же интерфейс использования, чтобы выглядеть знакомыми и понятными практикующему разработчику.

\end{comment}

Обычно при использовании обобщённого программирования в функциональных языках, получающиеся преобразования предоставляются в ``законченном'' виде как функции. 
%Это означает, что 
Разработчик вполне может пользоваться полученными преобразованиями заранее запланированным способом, но у него нет возможности тем или иным образом изменить поведение уже построенных преобразований. В случаях, когда имеющиеся преобразования не подходят под задачу, подход на основе обобщённого программирования по типам данных не вполне справляется со своей задачей.

%При использовании обобщённого программирования переиспользование кода достигается путём композиции порождённых на стадии компиляции преобразований с другими преобразованиями и прочими функциями. 
%Этот подход хорошо себя показывает в большинстве случаев, однако, иногда его может быть недостаточно. 

%Тут надо как-то решительнее и научнее заявлять

\begin{comment}
В некоторых случаях этого подхода не достаточно, и
если разработчика не вполне устраивает порождённое преобразование, %(например, если ему необходимо немного видоизменить обработку одного алгебраического конструктора из десяти),
то он может действовать двумя способами. Во-первых, он может реализовать новый вид преобразования для своих нужд.

В этом случае у разработчика могут возникнуть трудности, так как создание новых преобразований требует специфической экспертизы,
мало применимой при повседневной разработке программного обеспечения. Другим способом решения проблемы является реализация необходимого преобразования вручную без использования подхода обобщённого программирования. В этом случае увеличится размер кодовой базы и усложнится поддержка программного продукта в долгосрочной перспективе. 
\end{comment}

Эта проблема не является таковой в традиционных объектно-ориентированных языках программирования, где разработчику может воспользоваться \emph{поздним связыванием}. 
Этот приём в функциональных языках 
%программирования 
используется редко либо по причине отсутствия языковой поддержки объектно-ориентированного программирования (\haskell{}), либо потому, что объектно-ориентированный подход нарочно избегается и используется только в тех случаях, когда он существенно удобнее~\cite{modules-vs-objects}.
Мы предлагаем представление преобразований с помощью объектов таким образом, что минимальной видоизменяемой единицей является преобразование отдельного конструктора алгебраического типа.
%\vspace{-1em}

В работе представлен подход, кодирующий преобразования типов данных с помощью объектов, что зачастую свойственно объектно-ориентированному программированию, благодаря чему открываются возможности по видоизменению построенных преобразований, не описывая их целиком заново. Сама идея не нова, но наша работа отличается в двух аспектах. Во-первых, мы реализуем в объектах минимальное количество методов, чтобы не давать возможностей программисту преодолевать барьер абстракции.
Во-вторых, на основе объектов мы также строим функции, представляющие преобразования, чтобы интерфейс выглядел знакомым практикующему программисту на языке \ocaml{}.
%При этом сохраняются основные особенности обобщённого программирования, что позволяет получить преобразования, которые выглядят для практикующего программиста на \ocaml{} знакомо и естественно. 
Для испытания подхода была спроектирована библиотека \GT{}\footnote{\url{https://github.com/kakadu/GT/tree/ppx}} (\emph{Generic Transformers}) для языка \ocaml{}, где реализованы методы получения стандартных преобразований, аналогичных тем, что предоставляются популярными библиотеками обобщенного программирования, а также методы получения других преобразований.

